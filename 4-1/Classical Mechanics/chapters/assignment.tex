% \documentclass[../main-sheet.tex]{subfiles}
\documentclass[12pt]{article}
\usepackage{../style}
\graphicspath{ {../img/} }
\backgroundsetup{contents={}}
\begin{document}
% \chapter{First Chapter}
\section{Assignment I}
\begin{prob}
    Define: Generalized coordinates, Degrees of freedom, Constraints (Holonomic and non-holonomic constraints), Generalized velocity, Generalized acceleration, Generalized force, Generalized momentum.
\end{prob}
\begin{prob}
    Derive De'Alembert's principle and hence deduce Lagrange's equation from De'Alembert principle.
\end{prob}
\begin{prob}
    Define and derive Hamilton principle.
\end{prob}
\begin{prob}
    Discuss integral principle.
\end{prob}
\begin{prob}
    Which postulate is suitable for working with monogenic system? Discuss with proper logic.
\end{prob}
\begin{prob}
    Distinguish between the concept of configuration space and the physical three-dimensional space.
\end{prob}
\begin{prob}
    Derive Lagrange's equation from Hamilton's principle.
\end{prob}
\begin{prob}
    Use mathematical concept of calculus of variations elaborately to discuss the following properties:
    \begin{enumerate}[label=(\roman*)]
        \item Shortest distance between two points in a plane
        \item Minimum surface of revolution
        \item The Branchistochrone problem.
    \end{enumerate}
\end{prob}
\begin{prob}
    Discuss the advantage of a variation of principle formulation.
\end{prob}
\begin{prob}
    Find Newton's 2nd law of motion from Hamilton principle.
\end{prob}
\begin{prob}
    Define cyclic coordinates.
\end{prob}
\begin{prob}
    Guptas's Book: P144 (Q. 2,3), P147 (Q. 1,2,3), P150 (Q1)
\end{prob}
\begin{prob}
    Explain the principle of least action with the required diagram features.
\end{prob}
\begin{prob}
    Show the proof of principle of least action.
\end{prob}
\begin{prob}
    Discuss the application of Hamilton equation of motion \begin{enumerate}[label=(\roman*)]
        \item simple pendulum
        \item compound pendulum
        \item polar coordinates
    \end{enumerate}
\end{prob}
\begin{prob}
    Properties (I-IV) P157 (Gupta)
\end{prob}
\begin{prob}
    Derive Hamilton equation of motion.
\end{prob}
\begin{prob}
    Poisson bracket using by first form of generating function.
\end{prob}
\begin{prob}
    How can we get 4 generating function from \(4n+1\)?
\end{prob}
\begin{prob}
    Why is Lagrangian form convergent to obtain equation of motion?
\end{prob}
\begin{prob}
    Show that the kinetic energy of a system can be written as the sum of three homogeneous function of the generalized velocities.
\end{prob}
\begin{prob}
    Analyze the following with the help of Lagrangian form 
    \begin{enumerate}[label=(\roman*)]
        \item Single particle in space
        \item Cartesian coordinates
        \item plane polar coordinates.
    \end{enumerate}
\end{prob}
\end{document}