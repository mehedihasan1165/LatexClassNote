\documentclass[../main-sheet.tex]{subfiles}
\usepackage{../style}
\graphicspath{ {../img/} }
\backgroundsetup{contents={}}
\begin{document}
\chapter{First Chapter}
\section{Difference between mechanics and classical mechanics}
The notion of mechanics mainly denotes the combined overview of statics and dynamics. It refers to the motion of the object itself. On the other hand, classical mechanics is mainly a part of theoretical physics. Classical mechanics is the study of macroscopic bodies. The movement and statics of macroscopic bodies are discussed under classical mechanics.

Classical mechanics has three different branches. They are namely, Newtonian mechanics, Lagrangian mechanics and Hamiltonian mechanics. These three branches are based on the mathematical methods and quantities used to study the motion. The proper method is selected depending on the problem to the solved. Classical mechanics is applied in places such as planetary motions, projectiles and most of the events in daily lives. In classical mechanics, energy is treated as a continuous quantity. A system can take any amount of energy in classical mechanics.

\begin{prob}
    Defined degrees of freedom, generalized coordinate and constraints with example
\end{prob}
\end{document}