\documentclass[../main-sheet.tex]{subfiles}
\usepackage{../style}
\graphicspath{ {../img/} }
\backgroundsetup{contents={}}
% \everymath{\displaystyle}
\begin{document}
\chapter{Non-Linear Partial Differential Equation of First Order}
Suppose the non-linear partial differential equation is 
\[
    f(x,y,z,p,q)=0
\]
Type of solution of PDE of first order
\begin{enumerate}
    \item \emph{Complete solution/integral}: \(z=(x+a)(y+b)\) is a complete solution/integral of \(z=pq\).
    \item \emph{Particular solution:} \(z=(x+3)(y-4)\) is a particular solution of \(z=pq\).
    \item \emph{General solution:} The complete solution/integral \(\Phi(x,y,z,a,b)\) of PDE \(f(x,y,z,p,q)=0\) is said to be general solution if we can write as \(\Phi(x,y,z,a,b)=0\)
    \item \emph{Singular solution:} Let \(z=ax+by+ab\) is a complete solution of \(z=xp+qy+pq\).\\
    partial differential with respect to \(a\) and \(b\) gives\\
    \noindent\begin{minipage}{.45\linewidth}
        \begin{align*}
            &\pd{z}{a}=x+b\\
            \Rightarrow\;&0=x+b\\
            \Rightarrow\;&b=-x
        \end{align*}
        \end{minipage}%
        \hfill\vline\hfill%
        \begin{minipage}{.45\linewidth}
        \begin{align*}
            &\pd{z}{b}=0+y+a\\
            \Rightarrow\;&a=-y
        \end{align*}
        \end{minipage}\\
        Putting in complete solution \(z=-yx-\cancel{xy}+\cancel{xy}\)\\
        \(z=-xy\) is a singular solution.
\end{enumerate}
\begin{enumerate}[label=\underline{Type - \Roman*}:]
    \item First standard form \(f(p,q)=0\) which does not contain \(x\), \(y\), \(z\) explicitly.\\
    \emph{Working rule:}
    \begin{enumerate}[label=\arabic*]
        \item Write \begin{equation}
            f(p,q)=0\label{eq:type1.1}
        \end{equation}
        Assume the solution of \eqref{eq:type1.1} is 
        \begin{equation}
            z=ax+by+c\label{eq:type1.2}
        \end{equation}
        \item Differentiate \eqref{eq:type1.2} with respect to \(x\) and \(y\) partially, we get
        \begin{align*}
            \pdx{z}&=a\quad\Rightarrow\;p=a\\
            \pdy{z}&=b\quad\Rightarrow\;q=b
        \end{align*}
        \item Putting the values of \(p\), \(q\) in equation \eqref{eq:type1.1}, we get 
        \begin{align*}
            &f(a,b)=0\\
            \Rightarrow\;& b=\Phi(a)
        \end{align*}
        \item Putting the values in \eqref{eq:type1.2}
        \[z=ax+\Phi(a)y+c\]
        This is the required complete solution.
    \end{enumerate}
        \begin{prob}
            Find the complete solution of PDE \(p^2+q^2=m^2\)
        \end{prob}
        \begin{soln}
            The given equation 
            \begin{equation}
                p^2+q^2=m^2\label{eq:t1.1.1}
            \end{equation}
            is the first standard form.\\
            Suppose the solution is 
            \begin{equation}
                z=ax+by+c\label{eq:t1.1.2}
            \end{equation}
            partial Differentiate with respect to \(x\) and \(y\) the equation \eqref{eq:t1.1.2} gives,
            \begin{align*}
                \pdx{z}&=a\quad\Rightarrow\;p=a\\
            \pdy{z}&=b\quad\Rightarrow\;q=b
            \end{align*}
            put in \eqref{eq:t1.1.1}, gives
            \begin{align*}
                &a^2+b^2=m^2\\
                \Rightarrow\;&b=\pm \sqrt{m^2-a^2}
            \end{align*}
            Hence from \eqref{eq:t1.1.2}, we get
            \[z=ax\pm \sqrt{m^2-a^2} y+c\qquad\text{(contains 2 arbitrary constants)}\]
            This is the required complete solution.
        \end{soln}
        \begin{prob}
            Find the complete solution of \(p^2+q^2=npq\)
        \end{prob}
        \begin{soln}
            The is
            \begin{equation}
                p^2+q^2=m^2\label{eq:t1.2.1}
            \end{equation}
            first standard form.\\
            Suppose the solution of \eqref{eq:t1.2.1} is 
            \begin{equation}
                z=ax+by+c\label{eq:t1.2.2}
            \end{equation}
            partial Differentiate with respect to \(x\) and \(y\) the equation \eqref{eq:t1.2.2} gives,
            \begin{align*}
                \pdx{z}&=a\quad\Rightarrow\;p=a\\
            \pdy{z}&=b\quad\Rightarrow\;q=b
            \end{align*}
            put in \eqref{eq:t1.2.1}, gives
            \begin{align*}
                &a^2+b^2=nab\\
                \Rightarrow\;& b^2-nab+a^2=0\\
                \Rightarrow\;&b=\frac{na\pm \sqrt{n^2a^2-4a^2}}{2}
            \end{align*}
            Hence from \eqref{eq:t1.2.2}, we get
            \[z=ax+\left[\frac{na\pm \sqrt{n^2a^2-4a^2}}{2}\right]y+c\]
            This is the required complete solution.
        \end{soln}
        \item Second standard form 
        \begin{equation}
            f(z,p,q)=0\label{eq:t2.1}
        \end{equation}
        which does not contain \(x\), \(y\) explicitly.\\
        \emph{Working rule:}
        \begin{enumerate}[label=\arabic*]
            \item Suppose \begin{equation}
                z=f(x+ay)\label{eq:t2.2}
            \end{equation}
            is a solution of \eqref{eq:t2.1}
            \item Let \(u=x+ay\) so that\footnote{Note to self: check other resources to confirm} 
            \begin{align*}
                &z=f(u) &\left[\pdx{u}=1\quad\pdy{u}=a\right]\\
                \Rightarrow\; &\pdx{z}=\dd{z}{u}\cdot\pdx{u}\\
                \Rightarrow\; &p=\dd{z}{u}\cdot 1\\
                \Rightarrow\; &p=\dd{z}{u}\\
                \intertext{and}
                \Rightarrow\; &q=\pdy{z}=\dd{z}{u}\cdot \pdy{u}\\
                \Rightarrow\; &q=a\cdot\dd{z}{u}\\
                \Rightarrow\; &q=a\;\dd{z}{u}
            \end{align*}
            \item Put in \eqref{eq:t2.1} \[f\left(z,\;\dd{z}{u},\;a\dd{z}{u}\right)=0\]
            \item Solve the above equation for \(z\). Which will be the required complete solution.
        \end{enumerate}
        \begin{prob}
            Solve \(z=p^2+q^2\) (Find the complete solution)
        \end{prob}
        \begin{soln}
            Given
            \begin{equation}
                z=p^2+q^2 \label{eq:t2.1.1}
            \end{equation}
            This is second standard form.\\
            We assume the solution
            \[z=f(x+ay)\]
            and \(u=x+ay\), \(p=\displaystyle \dd{z}{u}\), \(q=\displaystyle a\dd{z}{u}\)
            Put in \eqref{eq:t2.1.1}
            \begin{align*}
                &z=\left(\dd{z}{u}\right)^2+a^2\left(\dd{z}{u}\right)^2\\
                \Rightarrow\;&z=\left(\dd{z}{u}\right)^2\left(1+a^2\right)\\
                \Rightarrow\;&\sqrt{z}=\dd{z}{u}\sqrt{1+a^2}\\%partial or total check on book/online
                \Rightarrow\;&\frac{\dz}{\sqrt{z}}=\frac{\D u}{\sqrt{1+a^2}}\\
                \intertext{integrating we get,}
                \Rightarrow\;&2\sqrt{z}=\frac{u}{\sqrt{1+a^2}}+b\\
                \Rightarrow\;&2\sqrt{z}=\frac{x+ay}{\sqrt{1+a^2}}
            \end{align*}
            This is the required solution.
        \end{soln}
        \begin{prob}
            Solve \(p(1+q^2=q(z-a))\)
        \end{prob}
        \begin{soln}
            Given
            \begin{equation}
                p(1+q^2=q(z-a)) \label{eq:t2.2.1}
            \end{equation}
            This is second standard form.\\
            We assume the solution
            \[z=f(x+ay)\]
            and \(u=x+ay\), \(p=\displaystyle \dd{z}{u}\), \(q=\displaystyle a\dd{z}{u}\)\\
            Put in \eqref{eq:t2.2.1}
            \begin{align*}
                &\cancel{\dd{z}{u}}\left[1+a^2\left(\dd{z}{u}\right)^2\right]=b\cancel{\dd{z}{u}}(z-a)\\
                \Rightarrow\;&\left(\dd{z}{u}\right)^2=\frac{b(z-a)-1}{b^2}\\
                \Rightarrow\;&\frac{b\dz}{\sqrt{b(z-a)}}=\D u\\
                \intertext{integrating we get,}
                \Rightarrow\;&b\int\frac{\dz}{\sqrt{b(z-a)}}=u+c\\
                \intertext{put \(b(z-a)=t,\;b\dz=\D t\)}
                \Rightarrow\;&\int\frac{\D t}{\sqrt{t}}=u+c\\
                \Rightarrow\;&2\sqrt{t}=u+c\\
                \Rightarrow\;&2\sqrt{b(z-a)}=x+by+c
            \end{align*}
            This is the required solution.
        \end{soln}
        \begin{prob}
            Solve \(x^2p^2+y^2q^2=z^2\) (which is reducible to 2nd...)
        \end{prob}
        \begin{soln}
            Given
            \begin{equation}
                (xp)^2+(yp)^2=z^2\label{eq:t2.3.1}
            \end{equation}
            Put \(X=\ln x\);\quad \(Y=\ln y\)\\
            \[\pd{X}{x}=\frac{1}{x}\qquad\pd{Y}{y}=\frac{1}{y}\]
            \begin{align*}
                &p=\pd{z}{x}\\
                \Rightarrow\;&p=\pd{z}{X}\pd{X}{x}\\
                \Rightarrow\;&p=\frac{1}{x}\pd{z}{X}\\
                \Rightarrow\;&xp=\pd{z}{X}\\
                \Rightarrow\;&xp=P
            \end{align*}
            \begin{align*}
                &q=\pd{z}{y}\\
                \Rightarrow\;&q=\pd{z}{Y}\pd{Y}{y}\\
                \Rightarrow\;&q=\frac{1}{y}\pd{z}{Y}\\
                \Rightarrow\;&yq=\pd{z}{Y}\\
                \Rightarrow\;&yp=Q
            \end{align*}
            From \eqref{eq:t2.3.1} we get,
            \begin{equation}
                P^2+Q^2=z^2\label{eq:t2.3.2}
            \end{equation}
            This is 2nd standard form.\\
            Assume the solution is\[z=f(X+aY)\] and \(u=X+aY\), \(P=\displaystyle \dd{z}{u}\), \(Q=\displaystyle a\dd{z}{u}\)\\
            Put in \eqref{eq:t2.3.2}
            \begin{align*}
                &\left( \dd{z}{u} \right)^2+a^2\left( \dd{z}{u} \right)^2=z^2\\
                \Rightarrow\;&\left( \dd{z}{u} \right)^2(1+a^2)=z^2\\
                \Rightarrow\;&\dd{z}{u}=\frac{z}{\sqrt{1+a^2}}\\
                \Rightarrow\;&\frac{\dz}{z}=\frac{\D u}{\sqrt{1+a^2}}\\
                \intertext{Integrating,}
                &\ln z=\frac{u}{\sqrt{1+a^2}}+b\\
                \Rightarrow\;&\ln z=\frac{\ln x+a\ln y}{\sqrt{1+a^2}}+b
            \end{align*}
            This is the required complete solution.
        \end{soln}
        \begin{prob}
            Solve \(xp^2+yq^2=z^2\)
        \end{prob}
        \begin{soln}
            Given
            \begin{equation}
                (\sqrt{x}p)^2+(\sqrt{y}p)^2=z^2\label{eq:t2.4.1}
            \end{equation}
            Put \(X=2x^{\frac{1}{2}};\quad Y=2y^{\frac{1}{2}}\)\\
            \[\pd{X}{x}=2\cdot\frac{1}{2}\cdot x^{\frac{1}{2}-1}=\frac{1}{\sqrt{x}}\qquad\pd{Y}{y}=\frac{1}{\sqrt{y}}\]
            Now,
            \begin{align*}
                &p=\pd{z}{x}\\
                \Rightarrow\;&p=\pd{z}{X}\pd{X}{x}\\
                \Rightarrow\;&p=\frac{1}{\sqrt{x}}\pd{z}{X}\\
                \Rightarrow\;&\sqrt{x}p=\pd{z}{X}\\
                \Rightarrow\;&\sqrt{x}p=P
            \end{align*}
            \begin{align*}
                &q=\pd{z}{y}\\
                \Rightarrow\;&q=\pd{z}{Y}\pd{Y}{y}\\
                \Rightarrow\;&q=\frac{1}{\sqrt{y}}\pd{z}{Y}\\
                \Rightarrow\;&\sqrt{y}q=\pd{z}{Y}\\
                \Rightarrow\;&\sqrt{y}p=Q
            \end{align*}
            continue...
        \end{soln}
        \begin{prob}
            Solve \(\displaystyle \frac{p^2}{x^2}+\frac{q^2}{y^2}=z^2\)
        \end{prob}
        \begin{soln}
            Given
            \begin{equation}
                \left( \frac{p}{x} \right)^2+\left( \frac{q}{y} \right)^2=z^2\label{eq:t2.5.1}
            \end{equation}
            Put \(X=\frac{1}{2}x^2\) \(Y=\frac{1}{2}y^2\)\\
            continue...
        \end{soln}
    \item  Third standard form \(f_1(x,p)=f_2(y,q)\)\\
    \emph{Working rule:}
    \begin{enumerate}[label=\arabic*]
        \item Let
            \begin{align*}
                &f_1(x,p)=f_2(y,q)=a\\
                &f_1(x,p)=a\quad \text{and}\quad f_2(y,q)=a\\
                \Rightarrow\;&p=\Phi_1(x,a)\quad \text{and}\quad q=\Phi_2(y,a)
            \end{align*}
        \item Solve 
        \begin{align*}
            \dz&=p\D u+q\D u\\
            &=\Phi_1(x,a)\D u+\Phi_2(y,a)\D u
        \end{align*}
        Integrate on both side
        \[z=\int \Phi_1(x,a)\D u+\Phi_2(y,a)\D u+c\]
        This is the required complete solution.
    \end{enumerate}
    \begin{prob}
        Solve \(p^2-q^2=x-y\)
    \end{prob}
    \begin{soln}
        Given,
        \begin{equation}
            p^2-x=q^2-y\label{eq:t3.1.1}
        \end{equation}
        This is 3rd standard form.\\
        Let 
        \begin{align*}
            &p^2-x=q^2-y=a\\
            \Rightarrow\;&p=\sqrt{a+x}\quad \text{and}\quad q=\sqrt{a+y}\\
        \end{align*}
        Since,
        \begin{align*}
            \dz&=p \dx+q\dy\\
            \dz&=\sqrt{a+x}\dx+\sqrt{a+y}\dy\\
            \intertext{Integrating,}
            z&=\int\sqrt{a+x}\dx+\int\sqrt{a+y}\dy\\
            z&=\frac{2}{3}(a+x)^{\frac{3}{2}}+\frac{2}{3}(a+y)^{\frac{3}{2}}+c
        \end{align*}
        This is the required complete solution.
    \end{soln}
    \begin{prob}
        Solve \(yp+xq+pq=0\)
    \end{prob}
    \begin{soln}
        Given,
        \begin{align*}
            &yp+xq=-pq\\
            \Rightarrow\;&\frac{yp}{-pq}+\frac{xq}{-pq}=1\\
            \Rightarrow\;&\frac{x}{p}=1+\frac{y}{q}
        \end{align*}
        This is 3rd standard form.\\
        Let 
        \begin{align*}
            &\frac{x}{p}=1+\frac{y}{q}=a\\
            \Rightarrow\;&p=\frac{x}{a}\quad \text{and}\quad q=\frac{y}{a-1}\\
        \end{align*}
        Since,
        \begin{align*}
            \dz&=p \dx+q\dy\\
            \dz&=\frac{x}{a}\dx+\frac{y}{a-1}\dy\\
            \intertext{Integrating,}
            z&=\frac{1}{2a}x^2+\frac{1}{2(a-1)}y^2+c
        \end{align*}
        This is the required complete solution.
    \end{soln}
    \begin{prob}
        Solve \(z^2(p^2+q^2)=x^2+y^2\)
    \end{prob}
    \begin{soln}
        Given,
        \begin{equation}
            (zp)^2+(zq)^2=x^2+y^2\label{eq:t3.11.1}
        \end{equation}
        Taking \(Z=\frac{z^2}{2}\) and \(\pdz{Z}=z\)\\
        Now,
        \begin{align*}
            p&=\pdx{z}=\pd{z}{Z}\pdx{Z}\\
            p&=\frac{1}{z}P\\
            \intertext{and,}
            q&=\pdy{z}=\pd{z}{Z}\pdy{Z}\\
            zq&=Q\qquad\left[\text{Where } P=\pdx{Z},\;Q=\pdy{Z}\right]
        \end{align*}
        From \eqref{eq:t3.11.1}
        \begin{align*}
            &P^2+Q^2=x^2+y^2\\
            \Rightarrow\;&P^2-x^2=y^2-Q^2
        \end{align*}
        This is 3rd standard form.\\
        Let 
        \begin{align*}
            &P^2-x^2=y^2-Q^2=a^2\\
            \Rightarrow\;&P^2=a^2+x^2\quad \text{and}\quad Q^2=y^2-a^2
        \end{align*}
        Since,
        \begin{align*}
            \D Z&=P \dx+Q\dy\\
            \Rightarrow\;\D Z&=(a^2+x^2)\dx+(y^2-a^2)\dy\\
            \intertext{Integrating,}
            Z&=\frac{1}{2}\left[x\sqrt{x^2+a^2}+a^2\log (x+\sqrt{x^2+a^2})\right]+\frac{1}{2}\left[y\sqrt{y^2-a^2}-a^2\log (y+\sqrt{y^2-a^2})\right]+c\\
            \Rightarrow\;\frac{z^2}{2}&=\frac{1}{2}\left[x\sqrt{x^2+a^2}+a^2\log (x+\sqrt{x^2+a^2})\right]+\frac{1}{2}\left[y\sqrt{y^2-a^2}-a^2\log (y+\sqrt{y^2-a^2})\right]+c\\
            \Rightarrow\;z^2&=\left[x\sqrt{x^2+a^2}+a^2\log (x+\sqrt{x^2+a^2})\right]+\left[y\sqrt{y^2-a^2}-a^2\log (y+\sqrt{y^2-a^2})\right]+c
        \end{align*}
        This is the required complete solution.
    \end{soln}
    \begin{prob}
        Solve \(p^2-q^2=z(x-y)\)
    \end{prob}
    \begin{soln}
        Given,
        \begin{equation}
            \left(\frac{p}{\sqrt{z}}\right)^2-\left(\frac{q}{\sqrt{z}}\right)^2=x-y\label{eq:t3.12.1}
        \end{equation}
        Taking \(Z=2\sqrt{z}\) and \(\pdz{Z}=\frac{1}{\sqrt{z}}\)\\
        Now,
        \begin{align*}
            p&=\pdx{z}=\pd{z}{Z}\pdx{Z}\\
            \Rightarrow\;p&=\sqrt{z}\pdx{Z}\\
            \Rightarrow\; \frac{p}{\sqrt{z}}=P\\
            \intertext{and,}
            q&=\pdy{z}=\pd{z}{Z}\pdy{Z}\\
            &=\sqrt{z}\pdy{Z}\\
            \Rightarrow\;\frac{q}{\sqrt{z}}&=Q\qquad\left[\text{Where } P=\pdx{Z},\;Q=\pdy{Z}\right]
        \end{align*}
        From \eqref{eq:t3.12.1}
        \begin{align*}
            &P^2-Q^2=x-y\\
            \Rightarrow\;&P^2-x=Q^2-y
        \end{align*}
        This is 3rd standard form.\\
        Let 
        \begin{align*}
            &P^2-x=Q^2-y=a\\
            \Rightarrow\;&P^2=x+a\quad \text{and}\quad Q^2=a+y\\
            \Rightarrow\;&P=\sqrt{x+a}\quad \text{and}\quad Q=\sqrt{a+y}
        \end{align*}
        Since,
        \begin{align*}
            \D Z&=P \dx+Q\dy\\
            \Rightarrow\;\D Z&=\sqrt{x+a}\dx+\sqrt{y+a}\dy\\
            \intertext{Integrating,}
            Z&=\frac{2}{3}(x+a)^{\frac{3}{2}}+\frac{2}{3}(y+a)^{\frac{3}{2}}+c\\
            \Rightarrow\;2\sqrt{z}&=\frac{2}{3}(x+a)^{\frac{3}{2}}+\frac{2}{3}(y+a)^{\frac{3}{2}}+c
        \end{align*}
        This is the required complete solution.
    \end{soln}
    \begin{prob}
        Solve \(z(p^2-q^2)=x-y\)
    \end{prob}
    \begin{soln}
        Given,
        \begin{equation}
            \left(\sqrt{z}p\right)^2-\left(\sqrt{z}q\right)^2=x-y
        \end{equation}
        Put \(Z=\frac{2}{3}z^{\frac{3}{2}}\) and \(\pdz{Z}=\sqrt{z}\)
    \end{soln}
    \begin{prob}
        Solve \(p^2+q^2=z^2(x^2+y^2)\)
    \end{prob}
    \begin{soln}
        Given,
        \begin{equation}
            \left(\frac{p}{z}\right)^2+\left(\frac{q}{z}\right)^2=x^2+y^2
        \end{equation}
        Put \(Z=\log z\) and \(\pdz{Z}=\frac{1}{z}\)
    \end{soln}
    \item Forth standard form
    \[z=px+qy+f(p,q)\]
    The complete solution is
    \begin{equation}
        z=ax+by+f(a,b)\label{eq:t4.1}
    \end{equation}
    Singular solution\\
    Partial differentiate \eqref{eq:t4.1} with respect to \(a\) and \(b\)
    \begin{align}
        0&=x+0+f'(a,b)\label{eq:t4.2}\\
        0&=0+y+f'(a,b)\label{eq:t4.3}
    \end{align}
    Solve \eqref{eq:t4.2} and \eqref{eq:t4.3} find the value of \(a\), \(b\) put in \eqref{eq:t4.1}, which gives the required singular solution.
\end{enumerate}
\begin{prob}
    Find the complete and singular solution of 
    \[z=px+qy+p^2+q^2\]
\end{prob}
\begin{soln}
    Given,
    \begin{equation}
        z=px+qy+p^2+q^2\label{eq:t4.1.1}
    \end{equation}
    This is fourth standard form.\\
    The complete solution is
    \begin{equation}
        z=ax+by+a^2+b^2\label{eq:t4.1.2}
    \end{equation}
    Singular solution\\
    Partial differentiate \eqref{eq:t4.1.2} with respect to \(a\) and \(b\)
    \begin{align*}
        0&=x+0+2a+0\qquad\Rightarrow\;a=\frac{-x}{2}\\
        0&=0+y+0+2b\qquad\Rightarrow\;b=\frac{-y}{2}
    \end{align*}
    Putting in \eqref{eq:t4.1.2}
    \begin{align*}
        z&=-\frac{x^2}{2}-\frac{y^2}{2}+\frac{y^2}{4}+\frac{x^2}{4}\\
        z&=-\frac{x^2}{4}-\frac{y^2}{4}
    \end{align*}
    \(x^2+y^2+4z=0\) is singular solution.
\end{soln}%
\end{document}%