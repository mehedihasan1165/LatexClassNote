\documentclass[../main-sheet.tex]{subfiles}
\usepackage{../style}
\graphicspath{ {../img/} }
\backgroundsetup{contents={}}
\begin{document}
\chapter{Dual-Simplex Method}
\begin{prob}
    Solve the LPP by the dual-simplex method:
    \begin{maxi*}
        {}{Z=-(2x_1+x_2+x_3)}{}{}
        \addConstraint{4x_1+6x_2+3x_3}{\leq 8}
        \addConstraint{x_1-9x_2+x_3}{\leq -3}
        \addConstraint{2x_1+3x_2-5x_3}{\geq 4}
        \addConstraint{x_1,\,x_2,\,x_3}{\geq 0}
    \end{maxi*}
\end{prob}
\begin{soln}
    Here, the 3rd constraint is of \(\geq\) type. To convert into \(\leq\) type, let us multiply it by \(-1\). Adding \(s_1,\,s_2,\,s_3\) as the slack variables to the 1st, 2nd and 3rd constraints respectively, the given LPP is expressed as
    \begin{maxi*}
        {}{Z=-(2x_1+x_2+x_3)}{}{}
        \addConstraint{4x_1+6x_2+3x_3+s_1}{= 8}
        \addConstraint{x_1-9x_2+x_3+s_2}{= -3}
        \addConstraint{-2x_1-3x_2+5x_3+s_3}{= -4}
        \addConstraint{x_1,\,x_2,\,x_3,\,s_1,\,s_2}{\geq 0}
    \end{maxi*}
    Putting \(x_1=x_2=x_3=0\), the initial basic solution is \(s_1=8\), \(x_2=-3\), \(s_3=-4\) which is infeasible. The above information is expressed in tab-I called staring dual simplex table.
    \begin{table}[H]
        \centering
      \begin{tabular}{cccccccccc}
        \toprule
        &  &  & $-2$ & $-1$ & $-1$ & 0 & 0 &0 &Constant/  \\ \cmidrule(lr){4-9}
        \multirow{-2}{*}{Tab} & \multirow{-2}{*}{\(C_B\)} & \multirow{-2}{*}{\diagbox{Basis}{\(c_j \to\)}} & \(x_1\) & \(x_2\) &\(x_3\)& \(s_1\) & \(s_2\) & \(s_3\) & Solution \\ \midrule
      \multirow{4}[3]{*}{I} & 0     & $ s_1 $ & 4     & 6     & 3     & 1     & 0     & 0     & 8 \\
      & 0     & $ s_2 $ & $-1$    & $-9$    & 1     & 0     & 1     & 0     & $-3$ \\
            & 0     & $ s_3 $ & $-2$    & $-3$    & 5     & 0     & 0     & 1     & $-4$ \\
            \cmidrule{2-10}          & \multicolumn{2}{c}{$ \bar{c_j} $ row} & $-2$    & $-1$    & $-1$    & 0     & 0     & 0     & $Z=0$ \\
            \midrule
            \multirow{4}[4]{*}{II} & 0     & $s_1$ & 0     & 0     & 13    & 1     & 0     & 2     & 0 \\
            & 0     & $s_2$ & 7     & 0     & $-14$   & 0     & 1     & $-3$    & 9 \\
            & $-1$    & $x_2$ & 2/3   & 1     & $-5/3$  & 0     & 0     & $-1/3$  & 4/3 \\
            \cmidrule{2-10}          & \multicolumn{2}{c}{$ \bar{c_j} $ row} & $-4/3$  & 0     & $-8/3$  & 0     & 0     & $-1/3$  & $Z=-4/3$ \\
            \bottomrule
      \end{tabular}%
    \label{tab:addlabel}%
  \end{table}%
  In these tableaux, basis refers to the basic variables in the basic solution. The values of the basic variables are given under the column solutions. Here, \(c_j\) denotes the coefficients of the variables in the objective function. \(C_B\) denotes the coefficient of the basic variables only and \(\bar{c_j}\) denotes the relative cost coefficients of the variables which is given by
  \begin{align*}
    \bar{C_j}&=C_j-\text{[inner product of \(C_B\) and the column corresponding the \(j-\)th variable in the canonical system]}\\
    &=C_j-C_BA_j,\text{where \(A_j\) is the \(j-\)th column of the matrix \(A=(a_{ij})\), which is found by the coefficients of the} \\
    & \quad\text{ basis and non-basis variables of constraints equation.}
\end{align*}
In tableau-I, all \(\bar{c}_j\) row entry are either negative of zero and the `solution' column shows that \(s_2\) and \(s_3\) are negative; this solution is optimal but infeasible.\\
Since the basic variable \(s_3\) has the most negative value, so it will be chosen to leave the basis. Since the variable \(x_1\) and \(x_2\) have negative coefficient in row-3, so we take the ratios of these with corresponding relative cost row in \(\bar{c}_j\), which are \(\frac{-2}{-2}\), \(\frac{-1}{-3}\). \(\left(\frac{-2}{-2}\text{ i.e., } 1 \text{ and }\frac{-1}{-3}\text{ i.e., }\frac{1}{3} \right)\)\\

The minimum ratio occurs corresponding to the non-basic variable \(x_2\). Thus, \(s_3\) will be replaced by \(x_2\) in the basis. Hence, the pivot element is \(a_{32}=-3\). We construct tab-II, using pivot operation.\\

Now we see that tableau-II is optimal and the solution is feasible. The optimal solution is
\[x_1=0,\qquad x_2=\frac{40}{3},\qquad x_3=0\]
and the optimum value is \(Z=-\dfrac{4}{3}\).

From tab-II, we see that the none of the non-basic variables has zero relative cost factors in the \(\bar{c}_j\) row. So, there is no alternative optimal of the given LPP. Hence, the optimal solution is unique at \((x_1,x_2,x_3)=(0,\frac{4}{3},0)\)
\end{soln}
\begin{prob}
    Solve by Dual-Simplex method
    \begin{mini*}
        {}{Z=x_1+4x_2+3x_4}{}{}
        \addConstraint{x_1+2x_2-x_3+x_4}{\geq 3}
        \addConstraint{-2x_1-x_2+4x_3+x_4}{\geq 2}
        \addConstraint{x_i}{\geq 0}
    \end{mini*}
\end{prob}
\begin{soln}
    Multiplying both constraints by \(-1\) and then adding \(x_5\) and \(x_6\) as the slack variables to the 1st and 2nd constraints respectively, we get
    \begin{mini*}
        {}{Z=x_1+4x_2+3x_4}{}{}
        \addConstraint{-x_1-2x_2+x_3-x_4+x_5}{= -3}
        \addConstraint{2x_1+x_2-4x_3-x_4+x_6}{=-2}
        \addConstraint{x_i}{\geq 0}
    \end{mini*}
    Putting \(x_1=x_2=x_3=x_4=0\), the initial basic solution is \(x_5=-3\), \(x_6=-2\); which is infeasible. The above information is expressed in tab-I called starting dual simplex table.
    % Table generated by Excel2LaTeX from sheet 'Sheet1'
\begin{table}[H]
    \centering
      \begin{tabular}{cccccccccc}
        \toprule
        &  &  & 1 & 4 & 0 & 3 & 0 &0 &Constant/  \\ \cmidrule(lr){4-9}
        \multirow{-2}{*}{Tab} & \multirow{-2}{*}{\(C_B\)} & \multirow{-2}{*}{\diagbox{Basis}{\(c_j \to\)}} & \(x_1\) & \(x_2\) &\(x_3\)& \(x_4\) & \(x_5\) & \(x_6\) & Solution \\ \midrule
      \multirow{3}[3]{*}{I} & 0     & $ x_5 $ & $-1$    & $-2$    & 1     & $-1$    & 1     & 0     & $-3$ \\
            & 0     & $ x_6 $ & 2     & 1     & $-4$    & $-1$    & 0     & 1     & $-2$ \\
  \cmidrule{2-10}          & \multicolumn{2}{c}{$ \bar{c_j} $ row} & 1     & 4     & 0     & 3     & 0     & 0     & $Z=0$ \\
      \midrule
      \multirow{3}[4]{*}{II} & 1     & $x_1$ & 1     & 2     & $-1$    & 1     & $-1$    & 0     & 3 \\
            & 0     & $x_6$ & 0     & $-3$    & $-2$    & $-3$    & 2     & 1     & $-8$ \\
  \cmidrule{2-10}          & \multicolumn{2}{c}{$ \bar{c_j} $ row} & 0     & 2     & 1     & 2     & 1     & 0     & $Z=3$ \\
      \midrule
      \multirow{3}[4]{*}{III} & 1     & $x_1$ & 1     & 7/2   & 0     & 5/2   & $-2$    & $-1/2$  & 7 \\
            & 0     & $x_3$ & 0     & 3/2   & 1     & 3/2   & $-1$    & $-1/2$  & 4 \\
  \cmidrule{2-10}          & \multicolumn{2}{c}{$ \bar{c_j} $ row} & 0     & 1/2   & 0     & 1/2   & 2     & 1/2   & $Z=7$ \\
      \bottomrule
      \end{tabular}%
  \end{table}%
  In these tableaux, basis refers to the basic variables in the basic solution. The values of the basic variables are given under the column solutions. Here, \(c_j\) denotes the coefficients of the variables in the objective function. \(C_B\) denotes the coefficient of the basic variables only and \(\bar{c_j}\) denotes the relative cost coefficients of the variables which is given by
  \begin{align*}
    \bar{C_j}&=C_j-\text{[inner product of \(C_B\) and the column corresponding the \(j-\)th variable in the canonical system]}\\
    &=C_j-C_BA_j,\text{where \(A_j\) is the \(j-\)th column of the matrix \(A=(a_{ij})\), which is found by the coefficients of the} \\
    & \quad\text{ basis and non-basis variables of constraints equation.}
\end{align*}
In tab-I, the basic solution is given by \(x_1=x_2=x_3=x_4=0\), \(x_5=-3\), \(x_6=-2\). This is infeasible though it satisfies the optimality condition.\\

Since the basic variable \(x_5\) has the most negative value, so it will be chosen to leave the basis. Since the variable \(x_1\), \(x_2\) and \(x_4\) have negative coefficient in row-1, so we take the ratios of these with corresponding relative cost row in \(\bar{c}_j\), which are
\[
    \left\vert\frac{1}{-1}\right\vert=1,\qquad \left\vert\frac{4}{-2}\right\vert=2,\qquad \left\vert\frac{3}{-1}\right\vert=3
\]

The minimum ratio occurs corresponding to the non-basic variable \(x_1\). Thus, \(x_1\) will be replaced by \(x_5\) in the basis. Hence, the pivot element is \(a_{11}=-1\). We construct tab-II, using pivot operation.\\

Tab-II is optimal but the basic variable \(x_6\) has negative value. So, \(x_6\) will leave the basis. By the same procedure, we see that \(x_3\) will replace \(x_6\) in the basis; we construct tab-III.\\

Tab-III is optimal and the solution is feasible. The optimal solution is
\[
    x_1=7,\qquad x_2=0,\qquad x_3=4,\qquad x_4=0
\]
and the optimum value is \(Z=7\).

Since in \(\bar{c}_j\) row there is no zero values corresponding to non-basic variable, hence the solution is unique.
\end{soln}
\end{document}