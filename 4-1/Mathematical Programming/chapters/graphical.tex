\documentclass[../main-sheet.tex]{subfiles}
\usepackage{../style}
\graphicspath{ {../img/} }
\backgroundsetup{contents={}}
\begin{document}
\chapter{Linear Programming: The Graphical Method}

For LP problems that have only two variables, it is possible that the entire set of feasible solutions can be displayed graphically by plotting linear constraints on a graph paper to locate the best (optimal) solution. Although most real-world problems have more than two decision variables, and hence can not be solved graphically, this solution approach provides valuable understanding of how to solve LP problems involving more than two variables algebraically.\\


Here, we shall discuss two graphical solution methods or approaches:
\begin{enumerate}[label=(\roman*)]
    \item Extreme point solution method
    \item Iso-profit (cost) function line method
\end{enumerate}
to find the optimal solution to an LP problem.\\

\textbf{To solve an LPP by using graphical method}:
\begin{enumerate}
    \item Consider the constraints as equalities.
    \item Draw the lines in plane corresponding to each equation and non-negative restriction.
    \item Find the feasible region for the values of the variables, which is the region, bounded by the lines.
    \item Find the feasible point in the region, which gives the optimal value of \(Z\).
\end{enumerate}

\section{Extreme point solution method}
\begin{prob}
    \begin{mini*}
        {}{Z=40x_1+36x_2}{}{}
        \addConstraint{x_1}{\leq 8}
        \addConstraint{x_2}{\leq 10}
        \addConstraint{5x_1+3x_2}{\geq 45}
        \addConstraint{x_1,x_2}{\geq 0}
    \end{mini*}
\end{prob}
\begin{soln}
    Let us consider \(x_1\) along the horizontal axis and \(x_2\) along the vertical axis. Now, each constraint have to be plotted on the graph by treating it as a linear equation appropriate and then inequality conditions have to use to make the area of feasible solutions.\\
    \begin{figure}[H]
        \centering
        \import{../tikz/}{mp1.tikz}
        % \caption{A finite state machine for binary addition}
    \end{figure}
    
    From the graph, we have the shaded region which is a feasible region or solution space. The optimal value of the objective function occurs at one of the corner points of the feasible regions. We now compute the \(Z\)-values corresponding to each corner points.
    \begin{align*}
        \text{For } A\left(8,\frac{5}{3}\right),&\qquad Z=380\\
        \text{For } B(8,10),&\qquad Z=680\\
        \text{For } C(3,10),&\qquad Z=480\\
    \end{align*}
    Since \(A\) is minimum at point \(A\left(8,\frac{5}{3}\right)\), hence the optimal solution is \(x_1=8\), \(x_2=\frac{5}{3}\) and optimal value is \(Z=380\).
\end{soln}
\begin{note}
    \begin{enumerate}[label=(\roman*)]
        \item The optimum solution is identified always by one of the corner point.
        \item To determine which side of a constraint equation is in the feasible region, examine whether the origin (0,0) satisfies the constraints. If it does, then all points on the constraint equation and all points on the side of the constraint at which origin is situated are feasible points. If it does not, then all points on the constraint equation and all points on the opposite side of the origin are feasible points.
    \end{enumerate}
\end{note}

\section{Iso-profit (cost) function line method}
\begin{prob}[Reddy Mikks Model]
    \begin{maxi*}
        {}{Z = 5x_1 + 4x_2}{}{}
        \addConstraint{6x_1+4x_2}{\leq 24}
        \addConstraint{x_1+2x_2}{\leq 6}
        \addConstraint{-x_1+x_2}{\leq 1}
        \addConstraint{x_2}{\leq 2}
        \addConstraint{x_1,x_2}{\geq 0}
    \end{maxi*}
\end{prob}
\begin{soln}
    Let us consider \(x_1\) along the horizontal axis and \(x_2\) along the vertical axis. Now, each constraint have to be plotted on the graph by treating it as a linear equation and then appropriate inequality conditions have to use to make the area of feasible solutions.\\
    
    From the graph we have the shaded region, which is a feasible region or solution space.\\
    To determine the optimum solution, we will identify the direction in which the profit function \(Z = 5x_1 +4x_2\) increase (since we are maximizing \(Z\)). We can do so by assigning \(Z\) the (arbitrary) increasing values of 10 and 15, which would be equivalent to plotting the lines
    \[
        5x_1 +4x_2 = 10 \qquad \text{and}\qquad 5x_1 +4x_2 = 15
    \]
    Figure superimposes these two lines on the solution space of the model. The profit \(Z\) thus can be increased in the direction shown in the figure until we reach the point in the solution space beyond which any further increase will put us outside the boundaries of \(ABCDEF\). Such a point is the optimum.\\

    From the figure, the optimum solution is given by point \(C\). The values of \(x_1\) and \(x_2\) are thus determined by solving the equations associated with lines 1 and 2, i.e.,
    \[
        6x_1 +4x_2 = 24 \qquad \text{and}\qquad x_1 +2x_2 = 6
    \]
    The solution yields \(x_1=3\) and \(x_2 = 1.5\) with \(Z= (5\times 3)+(4\times 1.5) = 21\)\\
    This means that the optimum daily product mix of 3 tons of exterior paint and 1.5 tons of interior paint will yield a daily profit of 21 unit.
\end{soln}
\end{document}