\documentclass[../main-sheet.tex]{subfiles}
\usepackage{../style}
\graphicspath{ {../img/} }
\backgroundsetup{contents={}}
\everymath{\displaystyle}
\begin{document}
\chapter{Topological spaces and metric spaces}
The topology on \(\R^n\) is defined in terms of open balls, which in turn are defined in terms of distance between points. There are many other spaces whose topology can be defined in a similar way in terms of a suitable notion of distance between points in the space.
\begin{defn}
    A metric on a set \(X\) is a function \(d:X\times X\to \R\) such that
    \begin{enumerate}
        \item \(d(x,y)\geq 0\) for all \(x,y\in X\) and \(d(x,y)=0\) iff \(x=y\)
        \item \(d(x,y)=d(y,x)\) for all \(x,y\in X\).
        \item \(d(x,y)\leq d(x,z)+d(z,y)\) for all \(x,y,z\in X\).
    \end{enumerate}
    A metric space \((X,d)\) is a set equipped with a metric \(d\) on \(X\). We denote a metric space simply by \(X\).
\end{defn}
The \textbf{open ball} of radius \(r \) centered at \(x\) in a metric space \((X,d)\) is defined as\[B_r(x)=\set{y\in X|d(x,y)<r}\]
that is, the points in \(X\) within \(r\) in distance from \(x\). It is also known as \textbf{open sphere}, or \textbf{\(r-\) neighborhood} of \(x\).
\begin{defn}
    A subset \(U \) of a metric space \((X,d)\) is open if for any \(x\in U\) there is an \(r>0\) so that \(B_r(x)\subseteq U\). We note the following properties of open subsets of metric spaces.
    \begin{enumerate}
        \item An open ball \(B_r(x)\) is an open set in \((X,d)\)
        \item An arbitrary union of open subsets is open.
        \item The finite intersection of open subsets is open.
    \end{enumerate}
\end{defn}
\begin{thm}
    Let \((X,d)\) be a metric space. For any \(x\in X\) and \(r>0\), let 
    \[B_r(x_=\set{y\in X|d(x,y)<r})\]
    Define \(\tau_d=\set{A\subseteq X:\forall X\in A \exists r>0 \text{ such that }B_r\subseteq A}\cup \varnothing\). Then \(\tau_d\) is a topology on \(X\).
\end{thm}
\begin{proof}
    \begin{enumerate}[label=(\roman*)]
        \item By definition, \(\varnothing \in \tau_d\). Also, for any \(x\in X \), there exist an \(r>0\) such that \(B_r(x)\subseteq X    \). Hence \(X\in \tau_d\).
        \item Let \(A,B\in \tau_d\). If \(A\cap B=\varnothing\), then clearly \(A\cap B\in \tau_d\). If \(A\cap B\neq \varnothing\), then for \(x\in A\cap B\) we get \(x\in A\) and \(x\in B\). Now, \(x\in A\) implies there exists \(r_1>0\) such that \(B_{r_1}(x)\subseteq A\) and \(x\in B\) implies there exists \(r_2>0\) such that \(B_{r_2}(x)\subseteq B\). Let \(r=\min(r_1,r_2)\). Then \(B_r(x)\subseteq B_{r_1}(x)\) and \(B_r(x)\subseteq B_{r_2}(x)\). Thus \(B_r(x)\subseteq A\cap B\). Hence \(A\cap B\in\tau_d\).
        \item Let \(\set{A_\alpha}\) be a family of members of \(\tau_d\). Let \(x\in \cup_{\alpha\in\Omega}A_\alpha\). Then \(x\in A_{\alpha_0}\) for some \(\alpha_0\in \tau_d\), there exist \(r>0\) such that \(B_r(x)\subseteq A_{\alpha_0}\) and hence \(B_r(x)\subseteq \cup_{\alpha\in \Omega}A_\alpha\). Therefore \(\cup_{\alpha\in \Omega}A_\alpha\in \tau_d\).
    \end{enumerate}
    From (i), (ii) and (iii) \(\tau_d\) is a topology on \(X\).
\end{proof}
This topology \(\tau_d\) is called the topology induced by the metric \(d\) on \(X\).

Let \((X,d)\) be a metric space and \(\tau\) be the collection of all open sets in \((X,d)\). Then \(\tau\) is a topology on \(X\), called a \textbf{metric topology} generated by (induced by) the metric \(d\) and the open balls of all points are a basis for this topology.
\begin{ex}
    Let \(d\) be a usual metric on the real line \(\R\), i.e., \(d(x,y)=\abs{x-y}\), then the open balls in \(\R\) are precisely the finite open intervals and these open intervals forms a topology on \(\R\) called usual topology. Hence the usual metric on \(\R\) induces the usual topology on \(\R\). Similarly, the usual metric on the plane \(\R^2\) induces the usual topology on \(\R^2\).
\end{ex}
\begin{prop}
    The collection of all open balls \(B_r(x)\) for \(r>0\) and \(x\in X\) forms a base for a topology on   \(X\).
\end{prop}
\begin{proof}
    First a preliminary observation: For a point \(y\in B_r(X)\) the ball \(B_s(y)\) is contained in \(B_r(x)\) if \(s\leq r-d(x,y)\), since for \(z\in B_s(y)\), we have \(d(z,y)<s\) and hence
    \[d(z,x)\leq d(z,y)+d(y,z)<s+d(x,y)\leq r\]
    Now to show the condition to have a basis is satisfied, suppose \(y\in B_{r_1}(x_1)\cap B_{r_2}(x_2)\). Then the observation in the preceding paragraph implies that \(B_s(y)\subseteq B_{r_1}(x_1)\cap B_{r_2}(x_2)\), for any 
    \(s\leq \min\set{r_1-d(x_1,y),r_2-d(x_2,y)}\). Therefore the collection of all open balls \(B_r(x)\) is a base for a topology on \(X\).
\end{proof}
A topological space \(\xt\) together with a metric \(d\) that induces the topology \(\tau \) is called  a \textbf{metric topological space} or a \textbf{metric space} and it is denoted by \((X,d)\).
\begin{defn}
    A topological space \(\xt \) is said to be \textbf{metrizable} if there is a metric \(d\) on \(X\) which induces the topology \(\tau\).
\end{defn}
\begin{ex}
    \((\R,\tau_u)\) is a metrizable space.
\end{ex}
\begin{ex}
    Discrete topological space is a metrizable space.
\end{ex}
\begin{ex}
     Let \(X=\set{x,y}\) and \(\tau\) be the indiscrete topology. Then \(\tau\) is not metrizable. Indeed, assume that \(\tau\) is a metric topology for some metric \(d\). Let \(r=d(x,y)\). Then  \(B_r(x)=\set{x}\) is an open set. But \(\set{x}\) is not an element of \(\tau\). A contradiction.
\end{ex}
\begin{ex}
    Let \(X\) be an arbitrary set and let \(\tau\) be a discrete topology on \(X\). Let \(d\) be a metric on \(X\) defined by 
\[d(x,y)=\begin{cases}
    0,&\text{ for }x=y\\
    1,&\text{ for }x\neq y
\end{cases}\]
Then \(B_{\frac{1}{2}}(x)=\set{x}   \); so, singleton subsets are open and hence \(d\) induces the discrete topology on     \(X\). Thus, we find a trivial metric \(d\) on \(X\) which induces the given topology \(\tau\). Accordingly, \(\xt\) is metrizable.
\end{ex}
\subsection{Distance between Sets, Diameters}
Let \(d\) be a metric on a set \(X\). The \textbf{distance} between two non-empty sets \(A\) and \(B\) is denoted and defined by
\[d(A,B)=\inf\set{d(a,b):a\in A\text{ and }b\in B}\]
The distance between a point \(p\in X\) and a non-empty subset \(B\) of \(X\) is denoted and defined by
\[d(p,B)=\inf\set{d(p,b):b\in B}\]
The \textbf{diameter} of a non-empty subset \(E\) of \(X\) is denoted and defined by 
\[d(E)=\sup\set{d(a,b):a,b\in E}\]
If the diameter of a non-empty subset \(E\) of \(X\) is finite, i.e., \(d(E)<\infty\), then \(E\) is said to be bounded. If \(d(E)=\infty\), then \(E\) is said to be unbounded. Clearly a set has diameter 0 iff it is a singleton set.
\begin{ex}
    Let \(d\) be a trivial metric on \(X\) defined by 
    \[d(x,y)=\begin{cases}
        0,&\text{ for }x=y\\
        1,&\text{ for }x\neq y
    \end{cases}\]
    Then for any \(p\in X\) and \(A,B\subseteq X\).
    \[d(p,A)=\begin{cases}
        0,&\text{ for }p \in A\\
        1,&\text{ for }p\notin A
    \end{cases}\qquad d(A,B)=\begin{cases}
        0,&\text{ if }A\cap B=\varnothing\\
        1,&\text{ if }A\cap B \neq \varnothing
    \end{cases}\]
\end{ex}
\begin{thm}
    Let \(d\) be a metric on a set \(X\). For any nonempty subset \(E\) of \(X\), \(d(\bar{E})=d(E)\).
\end{thm}
\begin{proof}
    We know that \(E\subset \bar{E}\). Now, if \(d(\bar{E})\) is infinite, then there is nothing to prove. So, let \(d(E)=r\) which is finite. If \(d(\bar{E})=r'\), then \(r'\geq r\). Suppose \(r'\neq r\) and let \(r'-r=s>0\). Then any point \(x_0 \in \bar{E}-E\) must be a limit point of \(E\) and any open sphere centered at \(x_0\) contains some points of \(E\). But the open sphere \(N_{\frac{s}{2}}(x_0)\) does not contain any point of \(E\).

    Hence our assumption that \(r'> r\) is wrong and therefore \(r'=r\), i.e., \(d(\bar{E})=d(E)\).
\end{proof}
\begin{thm}
    For any non-empty set \(A\) of a metric space \(X\), the closure \(\bar{A}\) of \(A\) is the set of points whose distance from \(A\) is 0.
\end{thm}
This theorem can be stated as:\\
Let \(A\) be a non-empty subset of a metric space \(X\). Then \(d(x,A)=0\) iff \(x\in \bar{A}\).
\begin{proof}
    Let \(d(x,A)=0\). Then every open sphere with center at \(x\) contains at least one point of \(A\) and therefore every open set \(G\) containing  \(x\) also contains at least one point of \(A\). Hence, \(x\) is a limit point of \(A\) and so     \(x\in \bar{A}\).

    Conversely, let \(x\in\bar{A}\). Suppose that \(d(x,A)\neq 0\) and \(d(x,A)=\varepsilon >0\). Then the open sphere \(S_{\frac{\varepsilon}{2}}(x)\) with center \(x\) contains no points of \(A\) and so \(x\) is an exterior point of \(A\); i.e., \(x\notin \bar{A}\), a contradiction. Hence, \(\bar{A}=\set{x:d(x,A)=0}\).
\end{proof}
\begin{thm}
    Let \(A\) and \(B\) be closed disjoint subset of a metric space \(X\). Then there exist disjoint open subsets \(G\) and \(H\) in \(X\) such that \(A\subseteq G\) and \(B\subseteq H\).
\end{thm}
\begin{proof}
    If either \(A\) or \(B\) is empty, say \(A=\varnothing\), the \(\varnothing\) and \(X\) are disjoint open sets such that \(A\subseteq \varnothing\) and \(B\subseteq X\). Hence, we may assume that \(A\) and \(B\) are non empty.\\
    Let \(a\in A\). THen since \(A\) and \(B\) are disjoint, \(a\notin B\) and so \(d(a,B)>0\). Similarly, if \(b\in B\), then \(d(b,A)>0\). Set 
    \[S_a=S_{\frac{\delta}{3}}(a)\quad \text{ and }\quad S_b=S_{\frac{\delta}{3}}(b)\]
    Clearly, \(a\in S_a\) and \(b\in S_b\).

    Let \(G=\set{S_a:a\in A}\) and \(H=\set{s_b:b\in B}\). Then clearly \(G\) and \(H\) are open because they are the union of open spheres and \(A\subseteq G\) and \(B\subseteq H\). We now have to show that \(G\cap H=\varnothing\).\\
    Suppose \(G\cap H\neq \varnothing\)  and let \(p\in G\cap H\). Then \(p\in G \) and \(p\in H\) implies \(p\in S_{a_0}\) and \(p\in S_{b_0}\) for some \(a_0\in A\) and \(b_0 \in B\) respectively. Let \(d(a_0,b_0)=\varepsilon >0\). Then \(d(a_0,B)<\varepsilon\) and \(d(b_0,A)<\varepsilon\). But \(d(a_0,p)<\frac{\delta}{3}\) and \(d(b_0,p)<\frac{\delta}{3}\). Therefore, by triangle inequality,
    \[\varepsilon=d(a_0,b_0)\leq d(a_0,p)+d(p,b_0)<\frac{\delta}{3}+\frac{\delta}{3}\leq\frac{\varepsilon}{3}+\frac{\varepsilon}{3}=\frac{2\varepsilon}{3}\]
    which is impossible. Hence, \(G\) and \(H\) are disjoint.
\end{proof}
\section{Euclidean \(n-\)dimensional space}
In \(\R^n\) space, the function \(d\) defined by
\[d(x,y)=\sqrt{(x_1-y_1)^2+(x_2-y_2)^2+\dots+(x_n-y_n)^2}\]
where \(x=(x_1,x_2,\dots,x_n)\) and \(y=(y_1,y_2,\dots, y_n)\) with \(x_i, y_i \in \R\) is a metric called the \textbf{Euclidean} metric on \(\R^n\) and the space \((\R^n,d)\) is known as \textbf{Euclidean \(n-\)dimensional space}.
\begin{thm}
    Euclidean \(n-\)dimensional space is a metric space.
\end{thm}
\section{Hilbert Space}
\textbf{Hilbert Space} is an immediate generalization    of Euclidean \(n-\)dimensional space \(\R^n\) arises when we replace \(n-\)tuples \(x=(x_1,x_2,\dots,x_n)\) with sequences \(x=\langle x_1,x_2,\dots \rangle\).\\
Let \(\ell_2\) denote the set of all sequences of real numbers such that
\[\sum_1^\infty (x_k)^2<\infty\]
i.e., such that the series \(x_1^2+x_2^2+\dots\) converges and define 
\[d(x,y)=\sqrt{\sum_1^\infty (x_k-y_k)^2}\]
Then \(d\) is a metric on \(\ell_2\). The resulting metric space \(\ell_2\) is usually called \(\ell_2\) space or \textbf{Hilbert} space, named after one of the most important and influential mathematician  of his time, David Hilbert (1862-1943).
\begin{thm}
    Hilbert space or \(\ell_2\) space is a metric space.
\end{thm}
\section{Normed Space}
A \textbf{norm} on a linear space is a function that gives a notion of the `length' of a vector. The formal definition of a norm on a linear space is given below:\\
A norm on a linear space \(X\) is a function \(\norm{\cdot}:X\to\R\) with the following properties:
\begin{enumerate}
    \item \(\norm{x}\geq 0\), for all \(x\in X\) and \(\norm{x}=0\) implies \(x=0\)
    \item \(\norm{\lambda x}=\abs{\lambda}\,\norm{x}\), for all \(x\in X\) and \(\lambda \in \R\)
    \item \(\norm{x+y}\leq \norm{x}+\norm{y}\) for all \(x,y\in X\)
\end{enumerate}
A linear space \(X\) together with a norm is called a \textbf{normed linear space}.\\
A normed linear space \(X\) is metric space with the metric
\[d(x,y)=\norm{x-y}\]
And it is known as induced metric on \(X\).\\
The set of real numbers \(\R\) with the absolute value norm \(\norm{x}=\abs{x}\) is a one-dimensional real normed linear space. More generally, \(\R^n\), where \(n=1,2,\dots,\) is an \(n-\)dimension linear space. We define \textbf{Euclidean norm} of a point \(x=(x_1,x_2,\dots,x_n)\in \R\) by 
\[\norm{x}=\sqrt{x_1^2+x_2^2+\dots+x_n^2}\]
A normed linear space \(X\) that is complete (every Cauchy sequence on \(X\) converges in \(X\)) with respect to the metric \(d\) is called a \textbf{Banach space}.
\end{document}