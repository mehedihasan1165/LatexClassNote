\documentclass[../main-sheet.tex]{subfiles}
\usepackage{../style}

\graphicspath{ {../img/} }
\backgroundsetup{contents={}}
\begin{document}
\# The principle of Relativity leads to the results that ``time'' can not be absolute. In different reference systems time runs differently. Therefore, if two events are simultaneous in one inertial system, they will not be simultaneous in some other system.
\section{Simultaneity}
Suppose that two events \(A \) and \(B \) occur at the same place in one particular frame of reference. Let us have a clock at that place which registers the time of occurrence of each event. If the reading is the same for each event, then we regard the events as simultaneous.
\section{Clock Synchronization}
Two clocks (of same nature) are said to be synchronized if they always read the same time when we set them at the same place in a particular frame of reference.

\# Let two events \(A \) and \(B \) occur at different locations in one particular reference frame. We set two clocks of same nature at each event.

If observer \(A \) sees that two clocks to read always the same time, observer \(B \) will find still that the clocks are not synchronized. This is due to the fact that it takes time for light to travel from \(B \) to \(A \) and vice versa.

Let us put a light source, that can be turned on and off (e.g., a flash bulb), at the exact mid-point of the straight line connecting \(A \) and \(B \). 

Let us inform each observer to put his clock at \(t=0\) when the turned-on light signal reaches him.

The light will take an equal amount of time to reach \(A \) and \(B \) from the mid-point, so that this procedure does indeed synchronize the clocks that are at \(A \) and \(B \).

The time of an event is measured by the clock whose location coincides with that of the event. Events occurring at \(A \) and \(B \) must be called simultaneous when the clocks at the respective places record the same time for them.


\subsection{Simultaneity is genuinely a relative concept, not an absolute one}
Suppose that two inertial frames \(S \) and \(S'\) are in coincident form. Two points \(A \) and \(B \) are equidistant from \(O \) in \(S -\) frame. Points \(A,\,O \) and \(B \) coincides with respectively with \(A',\,O'\) and \(B'\) in \(S'-\)frame.

Suppose, two lightning bolts occur at \(A \) and \(B \) simultaneously with respect to observer \(O \) in \(S-\)frame. The two events (i.e., lightning bolts) will also be simultaneous to observer \(O'\) in \(S'-\)frame, because the light signals from \(A \) and \(B \) reach \(O' \) at the same time, since \(AO'=O'B \).

Now, let \(S'-\)frame moves to the right with velocity \(\underbar{v } \) and two lightning bolts occur at \(A \) and \(B\) simultaneously as before. Then due to the relative velocity of \(S'-\)frame, the light signal from \(B \) will reach \(O'\) before reaching the light signal from \(A \).

As a result, observer \(O'\) will see that the lightning bolts are not simultaneous, though they are simultaneous to observer \(O \).

Hence, the simultaneously is not independent of reference frame.


\# The concept of simultaneity enters into the measurement of length. To measure the length of a rod, one must ..... simultaneously the positions of its end points. Therefore, the length of a rod is not an absolute one.
\end{document}