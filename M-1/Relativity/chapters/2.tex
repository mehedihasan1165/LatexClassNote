\documentclass[../main-sheet.tex]{subfiles}
\usepackage{../style}

\graphicspath{ {../img/} }
\backgroundsetup{contents={}}
\begin{document}
\subsubsection{Three possibilities}
\begin{enumerate}
    \item A relativity principle exists for mechanics, but not for electrodynamics. In electrodynamics there is a preferred inertial frame, i.e., the ether frame.
    \item A relativity principle exists both for mechanics and for electrodynamics, but the laws of electrodynamics as given by Maxwell are not correct.
    \item A relativity principle exists both for mechanics and for electrodynamics, but the laws of mechanics as given by Newton are not correct.
\end{enumerate}
Alternatives 1 and 2 are rejected on the basis of experiments. Alternative 3 is then correct one, and we already know that Newtonian mechanics breaks down at a high speed. For the new relativity principle the laws of mechanics are reformulated, and the transformation laws are corrected. The Galilean transformations are rejected, because Maxwell's equations are not invariant under Galilean transformations.

\begin{itemize}
    \item Michelson-Morley experiment gives us the result for the existence of an absolute frame(i.e., the ether frame of reference).
    \item Electrodynamics needs no modification. Experiments show that the velocity of electromagnetic radiation is independent of the velocity of the source.
\end{itemize}
\section{Ether concept}
It seemed inconceivable to 19th century physicists that the light and other electromagnetic waves could be propagated without a medium. It seemed to be logical step to postulate a medium called the ether for the propagation of electromagnetic waves. To account for its undetectability, it was necessary to assume unusual properties for it, such as zero density and perfect transparency. The ether was assumed to fill all space and to be the medium with respect to which is speed of light is \(c=\frac{1}{\sqrt{\epsilon_o\mu_0}}=2.997925\times10^8\,m/s \).

An observer moving through the ether with velocity \(\underbar{v}\) would measure a velocity \(\underbar{c }'\) for a light beam where \(\underbar{c}'=\underbar{c}+\underbar{v}\).

If an ether exists, the spinning and rotating earth should be moving through it. An observer on earth would sense an ``ether wind'' whose velocity is \(\underbar{v}\) relative to the earth.
\section{Michelson-Morley Experiment}
\footnotesize{(An attempt to locate the ether frame)}\\
\normalsize
The Michelson interferometer is fixed on the earth.

We imagine that the ``ether'' is fixed with respect to the sun. Then the earth (and interferometer) moves through ether at a speed of 30 km/sec, in different directions in different season.
For the moment, we neglect the earth's spinning motion. The beam of light from the laboratory source \(S \) (fixed with respect to the interferometer) is split by the partially silvered mirror \(M \) into two coherent beams:\\
beam 1 is transmitted through \(M \) and\\beam 2 is reflected off \(M \).

Beam 1 is reflected back to \(M \) by mirror \(M_1\) and beam 2 by mirror \(M_2\). Then the returning beam 1 is partially reflected and the returning beam 2 is partially transmitted by \(M \) back to a telescope at \(T \) where they interfere.\\
The interference is constructive or destructive depending on the phase difference of the beams.

The partially silvered mirror surface \(M \) is inclined at \(45^\circ \) to the beam direction. The time for beam 1 to travel from \(M \) to \(M_1\) and back is\footnote{earth moves with speed \(v \) towards the right. \((v)\)\\\(\Rightarrow\)ether moves with \(v \) towards the left. \((-v )\) }
\begin{align*}
    t_1&=\frac{l_1}{c-v}+\frac{l_1}{c+v}\\
    &=l_1\left(\frac{2c}{c^2-v^2}\right)\\
    &=\frac{2l_1}{c}\left(\frac{1}{1-\frac{v^2}{c^2}}\right)
\end{align*}
where \(c \) is the speed of light in the ether and \(l_1\) the length traveled by the beam 1 of light.\\
Here, \(\begin{aligned}
    c-v & \text{ is an ``up stream'' speed of the beam and}\\
    c+v & \text{ is a ``down stream'' speed with respect to the apparatus.}\\
\end{aligned}\)\\
The path of beam 2, traveling from \(M \) to \(M_2 \) and back is a cross-stream path through the ether as shown in the following figure, enabling the beam to return to the (advancing) mirror \(M \).


The transit time \((t_2)\) of beam 2 is given by
\begin{align*}
    &ct_2=MM_2+M_2M\\
    \Rightarrow\;&ct_2=2MM_2\\
    \Rightarrow\;&ct_2=2MM_2\left[ l_2^2+\left( \frac{vt_2}{2} \right)^2 \right]^{\frac{1}{2}}\\
    \Rightarrow\;&ct_2=2MM_2\left[ l_2^2+\left( \frac{vt_2}{2} \right)^2 \right]^{\frac{1}{2}}\\
    \Rightarrow\;&\frac{c^2t_2^2 }{4}=l_2^2+\frac{v^2t_2^2}{4}\\
    \Rightarrow\;&\frac{1}{4}(c^2-v^2)t_2^2=l_2^2\\
    \Rightarrow\;&t_2=\frac{2l_2}{\sqrt{c^2-v^2}}=\frac{2l_2}{c }\frac{1}{\sqrt{1-\frac{v^2 }{c^2 }}}
\end{align*}
This difference in transit time is
\[\Delta t=t_2-t_1=\frac{2}{c }\left[ \frac{l_2 }{\sqrt{1-\frac{v^2 }{c^2 }}}-\frac{l_1 }{1-\frac{v^2 }{c^2 }} \right]\]
Suppose that the instrument is rotated through \(90^\circ\), thereby making \(l_1\) the cross-stream length and \(l_2\) the down stream length. The corresponding transit time difference is 
\[\Delta t'=t_2'-t_1'=\frac{2}{c }\left[ \frac{l_2 }{1-\frac{v^2 }{c^2 }}-\frac{l_1 }{\sqrt{1-\frac{v^2 }{c^2 }}} \right]\]
Hence the rotation changes the differences by
\begin{align*}
    \Delta t'-\Delta t&=\frac{2}{c }\left[ \frac{l_2+l_1}{1-\frac{v^2}{c^2}}-\frac{l_2+l_1}{\sqrt{1-\frac{v^2}{c^2}}} \right]\\
    &=\frac{2}{c }(l_1+l_2)\left[ \left(1-\frac{v^2}{c^2}\right)^{-1}-\left(1-\frac{v^2}{c^2}\right)^{-\frac{1}{2}} \right]\\
    \intertext{neglecting terms higher than the second order,}
    &\approx \frac{2}{c }(l_1+l_2)\left[ 1-\frac{v^2}{c^2}-1-\frac{v^2}{2c^2}\right]\\
    &= \frac{2}{c }(l_1+l_2)\frac{v^2}{2c^2}\\
    &= \left(\frac{l_1+l_2}{c}\right)\frac{v^2}{c^2}
\end{align*}
Therefore, the rotation should cause a shift in the fringe pattern, because it changes the phase relationship between beams 1 and 2.\\
Let \(\Delta N \) be the number of fringe-shift due to rotation of the apparatus. If \(\lambda\) be the wave length of light used, so that the period of one vibration is \(T=\frac{1}{\nu}=\frac{\lambda}{c }\), then
\begin{align*}
    \Delta N&=\frac{\Delta t'-\Delta t }{T }\\
    &\approx \frac{l_1+l_2}{cT}\cdot\frac{v^2}{c^2}\\
    &= \frac{l_1+l_2}{\lambda}\cdot\frac{v^2}{c^2}
\end{align*}
Michelson and Morley, in their experiment, used \(l_2\approx l_1 \) and \(l_1+l_2=22 m \), so that,
\[\Delta N=\frac{22m }{5.5\times 10^{-7} m }\cdot10^{-8}=0.4\]
where \(\lambda=5.5\times 10^{-7}\) m, \(\frac{v }{c }=10^{-4}\), \(v=30\) km/sec, \(c=3\E{8}\) m/sec.\\
i.e., a shift of four-tenths of a fringe.\\
But in experiment, the expected fringe shift was not observed. The experimental conclusion was that there was no fringe shift at all. Hence, the experiment to locate an absolute frame (i.e., the ether frame) of reference gives null result \((\Delta N=0)\).
\section{Attempts to Preserve the concept of a preferred Ether Frame - The Lorentz-Fitzgerald contraction hypothesis}
The hypothesis was that all bodies are contracted in the direction of motion relative to the stationary ether by a factor \(\sqrt{1-\frac{v^2}{c^2}}\).

If \(l^\circ\) be the length of a body at rest with respect to the ether and \(l\) its length when in motion with respect to the ether, then
\[l=l^\circ\sqrt{1-\frac{v^2}{c^2}}\]
In Michelson-Morley experiment,
\begin{align*}
    l_1&=l_1^\circ\sqrt{1-\frac{v^2}{c^2}}\\
    \intertext{and}
    l_2&=l_2^\circ
\end{align*}
\[\therefore\,\Delta t=\frac{2 }{c}\left[ \frac{l_2 }{\sqrt{1-\frac{v^2}{c^2}}} -\frac{l_1}{1-\frac{v^2}{c^2}}\right]=\frac{2}{c }\frac{1}{\sqrt{1-\frac{v^2}{c^2}} }\left( l_2^\circ-l_1^\circ \right)\]
on \(90^\circ\) rotation,
\begin{align*}
    l_2&=l_2^\circ\sqrt{1-\frac{v^2}{c^2}}\\
    l_1&=l_1^\circ
\end{align*}
\[\therefore\,\Delta t'=\frac{2 }{c}\left[ \frac{l_2 }{1-\frac{v^2}{c^2}} -\frac{l_1}{\sqrt{1-\frac{v^2}{c^2}}}\right]=\frac{2}{c }\frac{1}{\sqrt{1-\frac{v^2}{c^2}} }\left( l_2^\circ-l_1^\circ \right)\]
\[\therefore\,\Delta t'-\Delta t=0\qquad\text{i.e., }\,\,\Delta N=0\]
Hence, no fringe shift should be expected on the rotation of the interferometer.

But if the velocity of the interferometer changes with respect to ether from \(v \) to \(v'\), we should expect a fringe shift. The predicted shift in fringes is
We have,
\[t_1=\frac{2l_1}{c }\left( \frac{1}{1-\frac{v^2}{c^2}} \right)\]
\[\therefore\,t_1=\frac{2l_1^\circ \sqrt{1-\frac{v^2}{c^2 }}}{c(1-v^2/c^2)}\qquad\text{ using length contraction}\]
If \(v \) changes to \(v'\), then
\begin{align*}
    t_1'&=\frac{2l_1^\circ }{c }\left( \frac{1}{\sqrt{1-v^2/c^2}} \right)\\
    \therefore \Delta t_1&=\frac{2l_1^\circ }{c }\left( \frac{1}{\sqrt{1-v^2/c^2}}-\frac{1}{\sqrt{1-v'^2/c^2}} \right)\\
    \intertext{Also}
    \Delta t_2&=\frac{2l_2^\circ }{c }\left( \frac{1}{\sqrt{1-v^2/c^2}}-\frac{1}{\sqrt{1-v'^2/c^2}} \right)\\
    \therefore \Delta t&=\Delta t_2-\Delta t_1\\
    &\approx \frac{l_2^\circ-l_1^\circ }{c }\left( \frac{v^2 }{c^2}-\frac{v'^2}{c^2} \right)
\end{align*}
\begin{align*}
    \therefore \Delta N&=\frac{\Delta t }{T }=\frac{c }{\lambda}\frac{l_2^\circ -l_1^\circ }{c}\left( \frac{v^2 }{c^2}-\frac{v'^2}{c^2} \right)\\
    &=\frac{l_2^\circ-l_1^\circ }{\lambda}\left( \frac{v^2}{c^2 }-\frac{v'^2}{c^2} \right) 
\end{align*}
\[\Delta N=\frac{l_2^\circ-l_1^\circ }{\lambda}\left( \frac{v^2}{c^2 }-\frac{v'^2}{c^2} \right) \]
Although the difference \(\frac{v^2-v'^2}{c^2}\) should change as a result of earth's spin (the biggest change occurring in twelve hours) and the earth's rotation (the biggest change occurring in six months), neither effect was observed (i.e., \(\Delta N=0\)) in direct contradiction to the contraction hypothesis.
\section{The Ether-Drag Hypothesis }
\footnotesize{(Attempt to preserve the concept of a preferred Ether frame)}\\
\normalsize
This hypothesis assumed that the ether frame was attached to all bodies of finite mass, i.e., dragged along with such bodies.

This assumption of such a ``local'' ether would automatically give a null result in the Michelson-Morley experiment.

However, there were two well-established effects which contradicted the ether-drag hypothesis: stellar aberration and Fizeau convection coefficient.
\subsection{Fizeau Experiment}
The set-up of the Fizeau experiment is shown diagrammatically in the following figure.




\(\begin{aligned}[t]
    S&\to\text{ light source }\\
    M&\to\text{ partially silvered mirror }\\
    M_1,M_2,M_3&\to\text{ Mirrors }\\
    T&\to\text{ Telescope }\\
    v_w&\to\text{ Water velocity }
\end{aligned}\)


Let the apparatus be \(S-\)frame. In this laboratory frame, [the velocity of light in still water is \(\frac{c }{\eta}\) and the velocity of the water is \(v_w \), \(\eta\) being the refraction index of water and \(c \) the free-space velocity of light.]\\
\(\begin{aligned}[t]
    c&\to\text{ velocity of light in free space }\\
    \eta&\to\text{ refractive index of water }\\
    \frac{c }{\eta}&\to\text{ velocity of light in still water }\\
    v_w&\to\text{ velocity of water }
\end{aligned}\)


Then, the velocity of light in the moving water, as was given by Fresnel, is 
\begin{equation}
    v=\frac{c }{\eta}\pm v_w\left( 1-\frac{1}{\eta^2} \right)\label{eq:fiz1}
\end{equation}
where the factor \(\left( 1-\frac{1}{\eta^2} \right)\) is called the Fresnel drag coefficient.\\
Since \(\left( 1-\frac{1}{\eta^2} \right)<1,\qquad\) \(\therefore\, v_2\left( 1-\frac{1}{\eta^2} \right)<v_w\)\\
i.e., the increase or decrease of light speed is less than the speed of medium. This predicts that light is dragged partially along by a moving medium.


In Fizeau experiment, the approximate values of the parameters were as follows: \(l=1.5\)m, \(\eta=1.33\), \(\Lambda=5.3\E{-7}\)m and \(v_w=7\)m/sec. A shift of 0.33 fringe was observed from the case \(v_w=0\) calculated the drag coefficient and compare it with the predicted value.\\


We have from \eqref{eq:fiz1},
\[v=\frac{c }{\eta}\pm v_w d,\qquad d=1-\frac{1}{\eta^2}\]
The time for beam 1 to pass through the water is
\[t_1=\frac{2l}{\frac{2}{n }-v_w d }\]
and for beam 2
\[t_2=\frac{2l}{\frac{c}{\eta}+v_w d }\]
Where \(2l \,\to\) length of whole path on which light travels in water.\\
Hence,
\[\Delta t=t_2-t_1=\frac{4lv_w d }{\left( \frac{c }{\eta} \right)^2-v_w^2d^2}\approx \frac{4l \eta^2 v_w d }{c^2}\]
The period of vibration of the light is \(T=\frac{\lambda}{c }\)
\[\therefore \,\Delta N=\frac{\Delta t }{T }\approx \frac{4l\eta^2v_w d }{\lambda c }\]
Hence, the drag coefficient is
\[d=\frac{\lambda c \Delta N }{4l \eta^2 v_w }=0.49\]
The Fresnel prediction is
\[d=1-\frac{1}{\eta^2}=0.44\]

Fizeau's experiment confirms the Fresnel prediction.\\
If the ether was dragged with the water, then the velocity of light in the laboratory frame, using the Galilean ideas, would have been \(\begin{aligned}[t]
    \frac{c }{\eta} -v_w& \text{ in one tube}\\
    \text{and }\frac{c }{\eta} +v_w& \text{ in the other tube}
\end{aligned}\)

The Fizeau experiment is interpreted in terms of no ether at all, ether by the apparatus or the water moving through it. The observed partial drag is due to the motion of the refractive medium.

Hence, the ether-drag hypothesis is contradicted.
\section{Attempts to Modify Electrodynamics }
A possible interpretation of the Michelson-Morley result (null result) is that the velocity of the light \(c \) has the same value in all inertial frames.

If this is so, then the velocity of light surely cannot depend on the velocity of the light source relative to the observer. Now, if we avoid the principle of the invariance of the velocity modification of electromagnetic is to assume that the velocity of a light wave is connected with the motion of the source rather than with an ether. If the source velocity affected the velocity of the light, we should observe complicated fringe pattern changes in the Michelson-Morley experiment. No such effects were observed in the experiment. Thus, we are forced by experiment to conclude that the laws of electrodynamics are correct and do not need modification.

\# Clearly since the speed of light (i.e., the speed of electromagnetic radiation) is independent of the relative motion of source and observer, we cannot use the Galilean transformation. We conclude that the Galilean transformation must be replaced, and therefore, the laws of mechanics which were consistent with those transformations need to be modified.

\section{The postulates of Special Relativity Theory}
In 1905, Albert Einstein (1879-1955) provided a solution to the dilemma facing physics. He gave two postulates are as follows:
\begin{enumerate}
    \item The laws of physics are the same in all inertial systems. No preferred inertial system exists. (the principle of Relativity)
    \item The speed of light in free space has the same value \(c \) in all inertial systems. (The principle of the constancy of the speed of light).
\end{enumerate}
The special relativity is based on these two principles.\\


\end{document}