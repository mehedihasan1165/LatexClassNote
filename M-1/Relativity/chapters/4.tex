\documentclass[../main-sheet.tex]{subfiles}
\usepackage{../style}

\graphicspath{ {../img/} }
\backgroundsetup{contents={}}
\begin{document}
\section{Lorentz Transformation}
\begin{figure}[H]
    \centering
    \import{../tikz/}{lorentz.tikz}
\end{figure}
we observe an event in one inertial frame \(S \). The location and time of the event are described by the coordinates \((x,\,y,\,z,\,t )\).

In a second inertial frame \(S'\), the same event is recorded as the time-space coordinates \((x',\,y',\,z',\,t' )\).

Let 
\begin{align*}
    x'&=x'(x,\,y,\,z,\,t)\\
    y'&=y'(x,\,y,\,z,\,t)\\
    z'&=z'(x,\,y,\,z,\,t)\\
    t'&=t'(x,\,y,\,z,\,t)
\end{align*}
We use the assumptions:
\begin{enumerate}[label=(\roman*)]
    \item Space is isotropic, i.e., all spatial direction are equivalent.
    \item Space and time are homogenous, i.e., all points in space and time are equivalent.
    \item \(S \) and \(S' \) coincide at \(t=0\), \(t'=0\).
\end{enumerate}
Let \(S'-\)frame moves with relative velocity \(v \) along the common \(x-x'\) axis.\\
The homogeneity of space and time implies that the transformation equations must be linear:
\begin{align*}
    x'&=a_{11 } x+a_{12 }y+a_{13 }z+a_{14 }t\\
    y'&=a_{21 } x+a_{22 }y+a_{23 }z+a_{24 }t\\
    z '&=a_{31 } x+a_{32 }y+a_{33 }z+a_{34 }t\\
    t '&=a_{41 } x+a_{42 }y+a_{43 }z+a_{44 }t
\end{align*}
where the subscripted coefficients are constant.
\begin{note}
    If \(x'=a_{11}x^2\), then \(x_2'-x_1'=a_{11}(x_2^2-x_1^2 )\);\\
    For a rod of unit length in \(S \) with end points at
    \begin{enumerate}[label=(\roman*)]
        \item \(x_1=1\) and \(x_2=2\), we get \(x_2'-x_1'=3a_{11}\);
        \item \(x_1=4\) and \(x_2=5\), we get \(x_2'-x_1'=9a_{11}\);
    \end{enumerate}
    i.e., the measured length of the rod depends on there it is in space. Similar is the situation for \(t \).
\end{note}



If \(v=0\), then \(a_{11}=a_{22}=a_{33}=a_{44}=1\), all other coefficients being zero. The \(x-\) axis coincides continuously with \(x'-\)axis. This gives \(y'=0\), \(z'=0\) for \(y=0\), \(z=0\). Then we have,
\begin{align*}
    y'&=a_{22 }y+a_{23 }z\\
    z '&=a_{32 }y+a_{33 }z\\
    \text{i.e., }\, a_{21}&=a_{24}=a_{31}=a_{34}=0
\end{align*}
Again, the plane \(z=0\) should transform to \(z'=0\) and the plane \(y=0\) to \(y'=0\). Hence,
\begin{align*}
    y'&=a_{22 }y\\
    z '&=a_{33 }z\\
    \text{i.e., }\, a_{23}&=0=a_{32}
\end{align*}
Consider a rod at rest of unit length lying along the \(y-\)axis in \(S \). According to the \(S'\) observer, the rod's length will be\footnote{\(y'=a_{22}y \), \(y_2'-y_1'=a_{22}(y_2-y_1)\), \(y'=a_{22} \times 1\)}
\[y'=a_{22}\times 1=a_{22}\]
Consider the same rod at rest along the \(y'\) axis in S'. To the S observer, the rod's length will be\footnote{\(y_2'-y_1'=a_{22}(y_2-y_1)=a_{22} y\), \(y=\frac{1}{a_{22}}\)}
\[y=\frac{1}{a_{22}}y'=\frac{1}{a_{22}}\times 1=\frac{1}{a_{22}}\]
The first postulate of special relativity implies that these measurements are identical. Therefore,
\[\frac{1}{a_{22}}=a_{22}\qquad \Rightarrow\,\,a_{22}=1\]
With the similar argument, \(a_{33}=1\).\\
Thus,
\begin{align*}
    y'&=y\\
    z'&=z
\end{align*}
Other two transformation equations are
\begin{align*}
    x'&=a_{11 } x+a_{12 }y+a_{13 }z+a_{14 }t\\
    t '&=a_{41 } x+a_{42 }y+a_{43 }z+a_{44 }t
\end{align*}

Since space is isotropic, we get that \(t'\) does not depend on \(y \) and \(z\).\footnote{\begin{note}
    Otherwise, if we place clocks at \(+y \), \(-y \), then \(t'=a_{41 }x+a_{42 }y +a_{43  }z+a_{44}t \neq a_{11 }x-a_{12 }y +a_{13  }z+a_{14}t \)\\
    Similar is the case at \(+z \), \(-z \). That is, clocks placed symmetrically in the \(y-z\) plane about the \(x-\)axis would appear to disagree as observed from \(S' \), which contradicts the isotropy of space.
\end{note}}\\
Hence, \(a_{42}=0=a_{43}\).\\
Also, a point with \(x'=0\) appears to move in the positive \(x-\)axis with speed \(v \). \\
So, \(x'=0\) corresponds to \(x=vt \), and we expect
\begin{align*}
    x'&=a_{11}(x-vt)\\
    &=a_{11}x-a_{11}vt\\
    &=a_{11}x+a_{14}t\\
    \text{i.e., }\quad a_{14}&=-v a_{11}
\end{align*}
Therefore, the transformation equations reduce to 
\begin{equation}
    \begin{rcases}
        x'=a_{11 }(x-vt)\qquad\\
        y'=y\\
        z'=z\\
        t'=a_{41}x+a_{44}t\qquad
    \end{rcases}
    \label{eq:lor1}
\end{equation}
We now recall the second postulate of special relativity i.e., the speed of light in free space has the same value \(c \) in all inertial frames.

Consider a spherical electromagnetic wave leaving the origin at \(t=0\). The wave propagation is described by
\begin{align}
    x^2+y^2+z^2&=c^2t^2\quad\text{for }S \label{eq:lor2}\\
    x'^2+y'^2+z'^2&=c^2t'^2\quad\text{for }S' \label{eq:lor3}
\end{align}
Substituting \eqref{eq:lor1} into \eqref{eq:lor3}, we get
\begin{align*}
    & a_{11}^2(x-vt)^2+y^2+z^2=c^2\left( a_{41}x+a_{44}t \right)^2\\
    \Rightarrow\;& \left(a_{11}^2-c^2a_{41}^2\right)x^2+y^2+z^2-(2 a_{11}^2 v+2c^2a_{41}a_{44})xt=(c^2a_{44}^2-a_{11}^2v^2)t^2
\end{align*}
This must be the same as \eqref{eq:lor2}

\end{document}