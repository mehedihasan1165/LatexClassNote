\documentclass[../main-sheet.tex]{subfiles}
\usepackage{../style}

\graphicspath{ {../img/} }
\backgroundsetup{contents={}}
\begin{document}
\section{Lorentz Transformation}
\begin{figure}[H]
    \centering
    \import{../tikz/}{lorentz.tikz}
\end{figure}
we observe an event in one inertial frame \(S \). The location and time of the event are described by the coordinates \((x,\,y,\,z,\,t )\).

In a second inertial frame \(S'\), the same event is recorded as the time-space coordinates \((x',\,y',\,z',\,t' )\).

Let 
\begin{align*}
    x'&=x'(x,\,y,\,z,\,t)\\
    y'&=y'(x,\,y,\,z,\,t)\\
    z'&=z'(x,\,y,\,z,\,t)\\
    t'&=t'(x,\,y,\,z,\,t)
\end{align*}
We use the assumptions:
\begin{enumerate}[label=(\roman*)]
    \item Space is isotropic, i.e., all spatial direction are equivalent.
    \item Space and time are homogenous, i.e., all points in space and time are equivalent.
    \item \(S \) and \(S' \) coincide at \(t=0\), \(t'=0\).
\end{enumerate}
Let \(S'-\)frame moves with relative velocity \(v \) along the common \(x-x'\) axis.\\
The homogeneity of space and time implies that the transformation equations must be linear:
\begin{align*}
    x'&=a_{11 } x+a_{12 }y+a_{13 }z+a_{14 }t\\
    y'&=a_{21 } x+a_{22 }y+a_{23 }z+a_{24 }t\\
    z '&=a_{31 } x+a_{32 }y+a_{33 }z+a_{34 }t\\
    t '&=a_{41 } x+a_{42 }y+a_{43 }z+a_{44 }t
\end{align*}
where the subscripted coefficients are constant.
\begin{note}
    If \(x'=a_{11}x^2\), then \(x_2'-x_1'=a_{11}(x_2^2-x_1^2 )\);\\
    For a rod of unit length in \(S \) with end points at
    \begin{enumerate}[label=(\roman*)]
        \item \(x_1=1\) and \(x_2=2\), we get \(x_2'-x_1'=3a_{11}\);
        \item \(x_1=4\) and \(x_2=5\), we get \(x_2'-x_1'=9a_{11}\);
    \end{enumerate}
    i.e., the measured length of the rod depends on there it is in space. Similar is the situation for \(t \).
\end{note}



If \(v=0\), then \(a_{11}=a_{22}=a_{33}=a_{44}=1\), all other coefficients being zero. The \(x-\) axis coincides continuously with \(x'-\)axis. This gives \(y'=0\), \(z'=0\) for \(y=0\), \(z=0\). Then we have,
\begin{align*}
    y'&=a_{22 }y+a_{23 }z\\
    z '&=a_{32 }y+a_{33 }z\\
    \text{i.e., }\, a_{21}&=a_{24}=a_{31}=a_{34}=0
\end{align*}
Again, the plane \(z=0\) should transform to \(z'=0\) and the plane \(y=0\) to \(y'=0\). Hence,
\begin{align*}
    y'&=a_{22 }y\\
    z '&=a_{33 }z\\
    \text{i.e., }\, a_{23}&=0=a_{32}
\end{align*}
Consider a rod at rest of unit length lying along the \(y-\)axis in \(S \). According to the \(S'\) observer, the rod's length will be\footnote{\(y'=a_{22}y \), \(y_2'-y_1'=a_{22}(y_2-y_1)\), \(y'=a_{22} \times 1\)}
\[y'=a_{22}\times 1=a_{22}\]
Consider the same rod at rest along the \(y'\) axis in S'. To the S observer, the rod's length will be\footnote{\(y_2'-y_1'=a_{22}(y_2-y_1)=a_{22} y\), \(y=\frac{1}{a_{22}}\)}
\[y=\frac{1}{a_{22}}y'=\frac{1}{a_{22}}\times 1=\frac{1}{a_{22}}\]
The first postulate of special relativity implies that these measurements are identical. Therefore,
\[\frac{1}{a_{22}}=a_{22}\qquad \Rightarrow\,\,a_{22}=1\]
With the similar argument, \(a_{33}=1\).\\
Thus,
\begin{align*}
    y'&=y\\
    z'&=z
\end{align*}
Other two transformation equations are
\begin{align*}
    x'&=a_{11 } x+a_{12 }y+a_{13 }z+a_{14 }t\\
    t '&=a_{41 } x+a_{42 }y+a_{43 }z+a_{44 }t
\end{align*}

Since space is isotropic, we get that \(t'\) does not depend on \(y \) and \(z\).\footnote{\begin{note}
    Otherwise, if we place clocks at \(+y \), \(-y \), then \(t'=a_{41 }x+a_{42 }y +a_{43  }z+a_{44}t \neq a_{11 }x-a_{12 }y +a_{13  }z+a_{14}t \)\\
    Similar is the case at \(+z \), \(-z \). That is, clocks placed symmetrically in the \(y-z\) plane about the \(x-\)axis would appear to disagree as observed from \(S' \), which contradicts the isotropy of space.
\end{note}}\\
Hence, \(a_{42}=0=a_{43}\).\\
Also, a point with \(x'=0\) appears to move in the positive \(x-\)axis with speed \(v \). \\
So, \(x'=0\) corresponds to \(x=vt \), and we expect
\begin{align*}
    x'&=a_{11}(x-vt)\\
    &=a_{11}x-a_{11}vt\\
    &=a_{11}x+a_{14}t\\
    \text{i.e., }\quad a_{14}&=-v a_{11}
\end{align*}
Therefore, the transformation equations reduce to 
\begin{equation}
    \begin{rcases}
        x'=a_{11 }(x-vt)\qquad\\
        y'=y\\
        z'=z\\
        t'=a_{41}x+a_{44}t\qquad
    \end{rcases}
    \label{eq:lor1}
\end{equation}
We now recall the second postulate of special relativity i.e., the speed of light in free space has the same value \(c \) in all inertial frames.

Consider a spherical electromagnetic wave leaving the origin at \(t=0\). The wave propagation is described by
\begin{align}
    x^2+y^2+z^2&=c^2t^2\quad\text{for }S \label{eq:lor2}\\
    x'^2+y'^2+z'^2&=c^2t'^2\quad\text{for }S' \label{eq:lor3}
\end{align}
Substituting \eqref{eq:lor1} into \eqref{eq:lor3}, we get
\begin{align*}
    & a_{11}^2(x-vt)^2+y^2+z^2=c^2\left( a_{41}x+a_{44}t \right)^2\\
    \Rightarrow\;& \left(a_{11}^2-c^2a_{41}^2\right)x^2+y^2+z^2-(2 a_{11}^2 v+2c^2a_{41}a_{44})xt=(c^2a_{44}^2-a_{11}^2v^2)t^2
\end{align*}
This must be the same as \eqref{eq:lor2}

\section{Lorentz Transformation}
Consider, two frames of reference \(S \) and \(S'\), \(S'\) is moving with uniform velocity \(v \) along the \(x'-\)axis relative to \(S\). Let the observers at the origin \(O \) and \(O'\) be observing the same event at any point \(P \) whose coordinates are \((x,y,z,t )\) and \((x',y',z',t')\) in \(S \) and \(S' \) respectively.

Let us consider that the \(x-\)axis of two systems coincides permanently. The event \(P \) is a light signal and is produced when both \(t \) and \(t'\) are zero.

Let us assume that at time \(t=0\), a spherical wave of light signal leaves \(O \) which coincides with \(O'\) at the moment. Since the velocity of light in both systems is the same and so the speed of propagation is the same in all directions and equal to \(c \) in terms of either set of coordinates. Its progress is therefore described by either of the two equations,
\begin{equation}
    x^2+y^2+z^2=c^2t^2\qquad\Rightarrow\,x^2+y^2+z^2-c^2t^2=0\label{eq:lorentz1} 
\end{equation}
and
\begin{equation}
    x'^2+y'^2+z'^2=c^2t'^2\qquad\Rightarrow\,x'^2+y'^2+z'^2-c^2t'^2=0\label{eq:lorentz2} 
\end{equation}
As velocity of \(S' \) is only along \(x-\)axis. Thus, from symmetry,
\begin{equation}
    y=y'\qquad \text{and}\qquad z=z' \label{eq:lorentz3}
\end{equation}
Then from equations \eqref{eq:lorentz1} and \eqref{eq:lorentz2},
\begin{equation}
    x^2-c^2t^2=\lambda(x'^2-c^2t'^2)\label{eq:lorentz4}
\end{equation}
where \(\lambda\) is any undetermined constant.

Now, for the transformation equation relating to \(x \) and \(x'\). Let us put,
\begin{equation}
    x'=\gamma(x-vt)\label{eq:lorentz5}
\end{equation}
\(\gamma\) being independent of \(x \) and \(t\).\\
The reasons for trying the above relation are
\begin{enumerate}[label=(\roman*)]
    \item The transformation must reduce to Galilean transformation for low speed. i.e., for speed \(\frac{v }{c }\to 0\).
    \item The transformation must be linear and simple.
\end{enumerate}
Since the motion is relative, we may assume that, \(S \) is moving relative to \(S' \) with velocity \((-v )\) along the direction of \(x-\)axis. Therefore,
\begin{equation}
    x=\gamma(x'+vt')\label{eq:lorentz6}
\end{equation}
Now, putting the value of \(x' \) from \eqref{eq:lorentz5} in \eqref{eq:lorentz6}, we get,
\begin{align}
    &x=\gamma\left[ \gamma(x-vt)+vt' \right]\notag\\
    \Rightarrow\;&\gamma vt'=x-\gamma^2x+\gamma^2 vt\notag\\
    \Rightarrow\;&t'=\gamma\left[t-\frac{x }{v }\left( 1-\frac{1}{\gamma^2} \right)\right]\label{eq:lorentz7}
\end{align}
Now, substituting the value of \(x' \) from \eqref{eq:lorentz5} and \(t'\) from \eqref{eq:lorentz7} in \eqref{eq:lorentz4}, we get,
\begin{align}
    &x^2-c^2t^2=\lambda\left[ \gamma^2(x-vt)^2-c^2\gamma^2\left\{ t-\frac{x }{v }\left( 1-\frac{1}{\gamma^2} \right) \right\}^2\right]\notag\\
    \Rightarrow\;&x^2-c^2t^2-\lambda\gamma^2\left(x^2-2vxt+v^2t^2 \right)+\lambda c^2\gamma^2\left\{ t^2-2\frac{x }{v }\left( 1-\frac{1}{\gamma^2} \right)t+\frac{x^2 }{v^2}\left( 1-\frac{1}{\gamma^2} \right)^2 \right\}=0\label{eq:lorentz8}
\end{align}
Since, this equation is identity, the coefficients of various powers of \(x \) and \(t \) must vanish separately.\\
Equating coefficients of \(xt \) to zero, we have,
\begin{align}
    &2\lambda \gamma^2v-\frac{2\lambda c^2\gamma^2}{v}\left( 1-\frac{1}{\gamma^2} \right)=0\notag\\
    \Rightarrow\;&2\gamma^4 v^2-2c^2\gamma^4+2c^2\gamma^2=0\notag\\
    \Rightarrow\;&\gamma^2 (v^2-c^2)+c^2=0\label{eq:lorentz9}
\end{align}
Equating coefficients of \(t^2\) to zero, we have,
\begin{align}
    &-c^2-\lambda\gamma^2v^2+\lambda c^2\gamma^2=0\notag\\
    \Rightarrow\;&\lambda(v^2-c^2)\gamma^2+c^2=0\label{eq:lorentz10}
\end{align}
From equation \eqref{eq:lorentz9} and \eqref{eq:lorentz10}, we have \(\lambda=1\).\\
Equation \eqref{eq:lorentz9} gives,
\begin{equation}
    \gamma=\frac{1}{\sqrt{1-\frac{v^2}{c^2}}}\label{eq:lorentz11}
\end{equation}
From \eqref{eq:lorentz7} we have,
\begin{align}
    t'&=\gamma\left[ t-\frac{x }{v }\left( 1-\frac{1 }{\gamma^2} \right) \right]\notag\\
    t'&=\gamma\left[ t-\frac{x }{v }\left( 1-\frac{c^2-v^2}{c^2} \right) \right]\notag\\
    t'&=\gamma\left[ t-\frac{xv^2 }{c^2v }\right]\notag\\
    \therefore\,t'&=\gamma\left( t-\frac{vx }{c^2 }\right)\label{eq:lorentz12}
\end{align}
Thus, using the values of \(\gamma\) from \eqref{eq:lorentz11} we get, from \eqref{eq:lorentz3}, \eqref{eq:lorentz5}, \eqref{eq:lorentz12}
\begin{align*}
    x'&=\frac{x-vt }{\sqrt{1-\frac{v^2}{c^2}}}\\
    y'&=y\\
    z'&=z\\
    t'&=\frac{t-\frac{vx}{c^2} }{\sqrt{1-\frac{v^2}{c^2}}}
\end{align*}
These are called Lorentz transformation equations.
\section{Mass Transformation Relation}
Let \(S,S'\) be two frames of reference, where \(S' \) moves with velocity \(v \) along \(x-\)axis with respect to \(S \) frame.\\
In \(S-\)frame, an object of mass \(m_1\) is moving with velocity \(u\) along \(x-\)axis.\\
In \(S'-\)frame, the mass and velocity is \(m_1'\) and \(u_1'\) respectively, then
\begin{align*}
    & u_1=\frac{u_1'+v}{1+\frac{v }{c^2 }u_1'}\\
    \Rightarrow\;& u_1\left({1+\frac{v }{c^2 }u_1'}\right)={u_1'+v}\\
    \Rightarrow\;& u_1'\left({1-\frac{v }{c^2 }u_1}\right)={u_1-v}\\
    \therefore\;& u_1'=\frac{u_1-v}{1-\frac{v }{c^2 }u_1}
\end{align*}
Now,
\begin{align*}
    &1-\frac{{u_1'}^2}{c^2}=1-\frac{(u_1-v)^2}{c^2\left(1-\frac{v }{c^2 }u_1\right)^2}\\
    \Rightarrow\;&1-\frac{{u_1'}^2}{c^2}=\frac{c^2\left(1-\frac{v }{c^2 }u_1\right)^2- (u_1-v)^2}{c^2\left(1-\frac{v }{c^2 }u_1\right)^2}\\
    \Rightarrow\;&c^2\left(1-\frac{{u_1'}^2}{c^2}\right)\left(1-\frac{v }{c^2}\right)^2=c^2\left(1+\frac{v^2 }{c^4}{u_1}^2-2\frac{v }{c^2}u_1\right)-(u_1^2+v^2-2u_1v)\\
    \Rightarrow\;&c^2\left(1-\frac{{u_1'}^2}{c^2}\right)\left(1-\frac{v }{c^2}\right)^2=c^2\left(1-\frac{u_1^2+v^2 }{c^2}+\frac{u_1^2 v^2 }{c^4}\right)\\
    \therefore\;&\left(1-\frac{{u_1'}^2}{c^2}\right)\left(1-\frac{v }{c^2}\right)^2=\left(1-\frac{u_1^2+v^2 }{c^2}+\frac{u_1^2 v^2 }{c^4}\right)
\end{align*}
Again,
\begin{align*}
    \gamma_1'u_1'&=\frac{u_1-v}{\sqrt{1-\frac{{u_1'}^2}{c^2}}\left(1-\frac{v }{c^2}u_1\right)}\\
    &=\frac{u_1-v}{\sqrt{1-\frac{u_1^2+v^2 }{c^2}+\frac{u_1^2 v^2 }{c^4}}}\\
    &=\frac{u_1-v}{\sqrt{\left(1-\frac{v^2 }{c^2}\right)\,\left(1-\frac{u_1^2}{c^2}\right)}}\\
    &=\gamma\cdot\gamma_1(u_1-v)
\end{align*}
\begin{equation}
    \therefore\;\frac{\gamma_1'u_1'}{\gamma_1}=\gamma(u_1-v ) \label{eq:mass1}
\end{equation}
where
\begin{align*}
    \gamma&=\frac{1}{1-\frac{v^2}{c^2}}\\
    \gamma_1&=\frac{1}{1-\frac{u_1^2}{c^2}}\\
    \gamma_1'&=\frac{1}{1-\frac{{u_1'}^2}{c^2}}
\end{align*}
Consider a particle is moving along the \(x -\)axis with constant velocity in the \(S-\)frame such that their mass and momentum remain unchanged.
\[\therefore\,\sum m_1=\text{constant }\quad\text{ and }\quad \sum m_1u_1=\text{ constant}\]
Since, the values of \(\gamma\) and \(u \) is same for all objects, so we get,
\begin{align*}
    &\sum m_1\gamma\,v=\text{ constant},\quad\text{ and }\quad \sum m_1u_1\gamma=\text{ constant}\\
    \Rightarrow\;&\sum m_1\gamma(u_1-v)=\text{constant }\quad\text{ [subtracting] }\\
    \Rightarrow\;&\sum \left(m_1\frac{\gamma_1'u_1'}{\gamma_1}\right)=\text{constant }\quad\text{ [from \eqref{eq:mass1}] }\\
\end{align*}
Applying the law of conservation of momentum in \(S'\) frame to get,
\begin{equation}
    \sum m_1'u_1'=\text{constant    }\label{eq:mass2}
\end{equation}
Comparing \eqref{eq:mass1} and \eqref{eq:mass2},
\begin{align*}
    & \frac{m_1\gamma_1'}{\gamma_1}=m_1'\\
    \Rightarrow\;& \frac{m_1}{\gamma_1}=\frac{m_1'}{\gamma_1'}=m_0\text{ (say)}
\end{align*}
\[\therefore\; m_1=\frac{m_0}{\sqrt{1-\frac{u_1^2}{c^2}}},\quad m_1'=\frac{m_0}{\sqrt{1-\frac{{u_1'}^2}{c^2}}}\]
This proves that if an object of mass \(m \) is moving in a frame of reference with velocity \(u \),
\[m=\frac{m_0}{\sqrt{q-\frac{u^2}{c^2}}}\]
If \(u=c \), the rest mass of a particle is, \(m_0=0 \).\\
If \(u=0\), \(m=m_0\); where \(m_0\) is the mass of the object at rest.
\end{document}