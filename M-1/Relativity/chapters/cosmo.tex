\documentclass[../main-sheet.tex]{subfiles}
\usepackage{../style}

\graphicspath{ {../img/} }
\backgroundsetup{contents={}}
\begin{document}
\chapter{Cosmology}
\begin{prob}
    State the principle of Cosmology. Derive the Robertson-Walker metric.
\end{prob}
\begin{soln}
    \underline{Cosmological Principle}: On a large-scale \(( 200Mpc)\), the universe appears to be homogeneous and
    isotropic. This is known as the Cosmological Principle. By homogeneity, we mean
    that the universe is the same at all points in space and by isotropy the universe is
    the same in all spatial direction about any point. This means that there is no
    preferred direction or a preferred location in the universe.\\


    \underline{Robertson Walker metric}\\
    Let \(x\), \(y\), \(z\) and \(w\) be the Cartesian co-ordinates in \(E_4\) with \(x\), \(y\), \(z\) being the usual spatial co-ordinates of \(E_3\). Then the hyper surface has the equation when \(k = 1\).
    \begin{align}
        &x^2 + y^2 + z^2 + w^2 = R^2(t)\notag\\
        \Rightarrow\,\,&r^2 + w^2 = R^2(t)\label{eq:robert1}
    \end{align}
 
where \(r\), \(\theta\), \(\phi\) be the spherical polar co-ordinates in \(E_3\).\\
Differentiating the above equation, we get
\begin{align}
    &r\D r=-w\D w\notag\\
    \Rightarrow\,\,&r^2 \D r^2=w^2\D w^2\label{eq:robert2}
\end{align}
The special metric on the hyper surface is given by
\begin{align*}
    \D l^2&=\D r^2 + r^2\D \theta^2 + r^2 \sin^2 \theta \D \varphi^2 + \D w^2\\
    &=\D r^2 + \frac{r^2\D r^2}{R^2-r^2} + r^2 \left(\D \theta^2+ \sin^2 \theta \D \varphi^2\right)\\
    &=\frac{R^2\D r^2}{R^2-r^2} + r^2 \D\Omega^2\qquad \text{Where, } \D\Omega^2=\D\theta^2+\sin^2\theta\D \varphi^2
\end{align*} 
A  change  in angle \(\theta\)	produce a displacement \(r\D\theta\) while a change in \(r\) in any
direction gives a displacement of \(\frac{R\D r }{(R^2-r^2)^{1/2}}\).\\
When the variable \(\sigma=\frac{\pi}{r }\)
is used, the metric equation reduce to
\[
    \D l^2=R^2(t)\left[\frac{\D \sigma^2}{1-\sigma^2} + \sigma^2 \D\Omega^2\right]
\]
Now we can write the complete metric equation in co-ordinating time
\[
    \D s^2=c^2\D t^2-R^2(t)\left[\frac{\D \sigma^2}{1-\sigma^2} + \sigma^2 \D\Omega^2\right]
    \]
    This is Robertson-Walker metric for \(k = 1\).\\
    More generally, the curvature \(k\) could be negative or zero.\\
    General form,
    \[
        \D s^2=c^2\D t^2-R^2(t)\left[\frac{\D \sigma^2}{1-k\sigma^2} + \sigma^2 \D\Omega^2\right]
        \]
        This is the Robertson-Walker metric for isotropic and homogeneous space time.
\end{soln}
\begin{prob}
    What is black hole? Write down and discuss the most general black hole solution. How do you reduce this to Kerr and Reissnar-Nordstrom black hole solutions?
\end{prob}
\begin{soln}
    A black hole is a region of space time where gravity is so strong that nothing - no particles or, even electromagnetic radiation such as light can escape from it. The theory of general relativity predicts that a sufficiently compact mass can deform spacetime to form a black hole.\\


    The Kerr-Newmann black hole solution is the most general black hole solution. In 1963, Kerr had obtained a metric for the space time of mass, which is convenient form for this line element is,
    \begin{equation}
        \D s^2=\frac{\Delta }{\rho^2}(\D T-h\sin^2\theta \D \varphi)^2-\frac{\sin^2 \theta}{\rho}\left[(R^2+h^2)\D\varphi-h\D T\right]^2-\frac{\rho^2}{\Delta}\D R^2-\rho^2\D \varphi^2 \label{eq:kerr1}
    \end{equation}
    where,
    \begin{align*}
        h&=\frac{H }{M }=\text{ angular momentum per unit mass}\\
        \Delta&=R^2-2FMR+h^2\\
        \rho^2&=R^2+h^2\cos^2\theta\text{ and } (T,\,R,\,\theta,\,\varphi) \text{
        co-ordinate; \(h\), \(M\) are parameter.}
    \end{align*} 
The Kerr-Newmann Black hole space time is described by the metric
\begin{align*}
    \D s^2&=(r^2+a^2\cos^2\theta)\left(\frac{\D r^2 }{r^2-2mr+e^2+a^2}+\D\theta\right)+\sin^2 \theta\left\{r^2+a^2+\frac{a^2 \sin^2 \theta(2mr-e^2)}{r^2+a^2\cos^2\theta}\right\}\D \varphi^2\\
    &\phantom{=} -\left(1-\frac{2mr-e^2}{r^2+a^2\cos^2\theta}\right)\D t^2+\frac{2a\sin^2\theta(2mr-e^2)}{r^2+a^2\cos^2\theta}\D t\D \varphi
\end{align*}
where, \(m\) is the mass,
\(a\) is angular momentum per unit mass, \(e\) is electric charge.\\



This metric includes:
\begin{enumerate}[label=(\roman*)]
    \item Kerr-Black hole space time when \(e = 0\).
    \item Reisner-Nordstrom black hole space time if \(a = 0\).
    \item Schwarzschild black hole space time for \(e = a = 0\).
\end{enumerate}
\end{soln}
% \begin{prob}
%     Deduce Friedmann equations:
%     \begin{enumerate}[label=(\roman*)]
%         \item \(\displaystyle\dot{R}^2+k=\frac{8\pi G }{3}\rho R^2\)
%         \item \(\displaystyle\dot{\rho}+3(P+\rho)\frac{\dot{R}}{R }=0\)
%     \end{enumerate}
% \end{prob}
% \begin{soln}
%     We know,
%     \begin{align*}
%         R_{00}&=\frac{3\ddot{R }}{R }\\
%         R_{11}&=\frac{-R\ddot{R }+2\dot{R}^2+2k}{1-kr^2 }\\
%         R_{22}&=-(R\ddot{R }+2\dot{R}^2+2k)r^2\\
%         R_{33}&=-(R\ddot{R }+2\dot{R}^2+2k)r^2\sin^2\theta\\
%         R_{\mn}&=0,\qquad \mu\neq\nu
%     \end{align*}
%     Also we know,
%     \begin{align*}
%         &T_\mn=(P+\rho)u_\mu u_\nu-Pg_\mn\\
%         \Rightarrow\,\,&T_\mn g^\mn=(P+\rho)u_\mu u_\nu g^\mn-Pg_\mn g^\mn\\
%         \Rightarrow\,\,&T=(P+\rho)-4P\,=\,\rho-3P \quad[\because g_\mn g^\mn=4]
%     \end{align*}
%     Again, \[T_\mn=(P+\rho)\delta_\mu^0\,\delta_\nu^0\,-Pg_mn\]
%     Now,
%     \[
%         T_\mn -\frac{1}{2}Tg_\mn=(P+\rho)\delta_\mu^0\,\delta_\nu^0\,-Pg_mn -\frac{1}{2}(\rho-3P)g_\mn =(P+\rho)\delta_\mu^0\,\delta_\nu^0\,-\frac{1}{2}(\rho-P)g_\mn
%     \]
%     So,
%     \begin{align*}
%         T_{00}-\frac{1}{2}Tg_{00}&=(P+\rho)\delta_0^0\,\delta_0^0\,-\frac{1}{2}(\rho-P)g_{00}\;=\; P+\rho-\frac{1}{2}(\rho-P )\;=\;\frac{1}{3}(3p+\rho)\\
%         T_{11}-\frac{1}{2}Tg_{11}&=(P+\rho)\delta_1^0\,\delta_1^0\,-\frac{1}{2}(\rho-P)g_{11}\;=\; \frac{1}{2}(\rho-P )\cdot\frac{R^2 }{1-kr^2}\\
%         T_{22}-\frac{1}{2}Tg_{22}&=(P+\rho)\delta_2^0\,\delta_2^0\,-\frac{1}{2}(\rho-P)g_{22}\;=\; \frac{1}{2}(\rho-P )\cdot r^2{R^2 }\\
%         T_{33}-\frac{1}{2}Tg_{33}&=(P+\rho)\delta_3^0\,\delta_3^0\,-\frac{1}{2}(\rho-P)g_{33}\;=\; \frac{1}{2}(\rho-P )\cdot r^2{R^2 }\sin^2\theta\\
%         \text{ and }T_{\mn}-\frac{1}{2}g_{\mn}&=0\qquad\text{ when, }\quad \mu\neq\nu
%     \end{align*}
%     From Einstein field equations we know that, \(R_\mn=k(T_\mn-\frac{1}{2}T g_\mn )\)
%     \begin{align}
%         \therefore\;\;& R_{00}=k\left( T_{00}-\frac{1}{2}T g_{00}  \right)=\frac{1}{2} k(\rho+3P)\notag\\
%         \Rightarrow\;\;& \frac{3\ddot{R}}{R }=\frac{1}{2}k\left( \rho+3P \right)\notag\\
%         \Rightarrow\;\;& R\ddot{R}=\frac{1}{6}kR^2 \left( \rho+3P \right)\label{eq:fried1}
%     \end{align}
%     Again,
%     \begin{align}
%         & R_{22}=k\left( T_{22}-\frac{1}{2}T g_{22}  \right)\notag\\
%         \Rightarrow\;\;& -(R\ddot{R}+2\dot{R}^2+2k)r^2=\frac{1}{2}k\left( \rho-P \right)r^2R^2\notag\\
%         \Rightarrow\;\;& R\ddot{R}+2\dot{R}^2+2k=-\frac{1}{2}k\left( \rho-P \right)R^2\label{eq:fried2}
%     \end{align}
%     From \eqref{eq:fried1} and \eqref{eq:fried2}
%     \begin{align}
%         & \frac{1}{6}kR^2 \left( \rho+3P \right)+2\dot{R}^2+2k=-\frac{1}{2}k\left( \rho-P \right)R^2\notag\\
%         \Rightarrow\;\;& 2\dot{R}^2+2k=-\frac{1}{2}k\left( \rho-P \right)R^2-\frac{1}{6}kR^2 \left( \rho+3P \right)R^2\notag\\
%         \Rightarrow\;\;& \dot{R}^2+k=-\frac{1}{3}\rho R^2k\label{eq:fried3}
%     \end{align}
%     Put \(k=-8\pi G \) in right side of \eqref{eq:fried3}, we get,
%     \begin{equation}
%         \dot{R}^2+k=\frac{8\pi G}{3}\rho R^2
%         \label{eq:fried4}
%     \end{equation}
%     Again, we know,
%     \begin{equation}
%         {T^\mn}_{,\mu}=0\quad \text{ and }\quad \left( \rho u^\mu \right)_{,\mu}+P {u^\mu}_{,\mu}=0 \label{eq:fried5}
%     \end{equation}
%     From \eqref{eq:fried5},
%     \[
%         \rho_{,\mu}u^\mu+(\rho+P)\left( {u^\mu}_{,\mu}+\Gamma_{0\mu}^\mu u^\nu \right)=0\]
%         Simplifying by \(u^\mu=\delta_0^\mu\), we get
%     \begin{equation}
%         \dot{\rho}+3(\rho+P)\frac{\dot{R}}{R }=0 \label{eq:fried6}
%     \end{equation}
%     Equations \eqref{eq:fried5} and \eqref{eq:fried6} are the required equations.
% \end{soln}
\begin{prob}
    Deduce the Friedmann model of the universe for \(P=0 \) and find out its graph for different values of \(k \).
\end{prob}
\begin{soln}
    The dynamical equations of cosmology that describe the evaluation of the scale factor \(R(t)\) follows from the Einstein's field equations are
    \begin{align}
        &\dot{R}^2+k\;=\;\frac{8\pi\rho R^2}{3}\label{eq:model1}\\
        &\dot{\rho}+3(P+\rho)\frac{\dot{R }}{R }\;=\;0\label{eq:model2}
    \end{align}
The standard Friedmann model arise from the case \(p = 0\).\\
In the case \(p = 0\), we have from the equation \eqref{eq:model2},
\begin{align*}
    &\dot{\rho}+\frac{3\rho \dot{R }}{R }=0\\
    \Rightarrow\;\;&\log\;\rho+3\log\;{R }=\log\;c \qquad[\text{After integration}]\\
    \Rightarrow\;\;&\log\;\left(\rho{R }^3\right)=\log\;c\\
    \Rightarrow\;\;&\rho{R }^3=c\\
\end{align*}
If the present age of the universe is \(t_0\), then
\(\rho_0=\rho(t_0),\quad, R_0=R(t_0)\), so \(\rho_0 {R_0}^3=c\)\\
Hence
\begin{equation}
    \rho{R }^3=\rho_0 {R_0}^3 \label{eq:model3}
\end{equation}
Using \eqref{eq:model3} in \eqref{eq:model1} we get,
\begin{align}
    \dot{R}^2+K&=\frac{8\pi\rho R^3}{3R}\notag\\
    &=\frac{8\pi\rho_0 {R_0}^3}{3R}\notag\\
    &=\frac{A^2 }{R }\qquad\text{where   }\quad A^2=\frac{8\pi\rho_0 {R_0}^3}{3}\label{eq:model4}
\end{align}
Now Hubbles constant \(H(t)\) is defined by
\begin{equation}
    H(t)=\frac{\dot{R }}{R(t )}\qquad\text{ and }\qquad H_0=\frac{\dot{R }}{R_0}\label{eq:model5}
\end{equation}
Equation \eqref{eq:model1} gives,
\begin{align}
    &\frac{\dot{R}^2}{{R_0}^2}+\frac{k}{{R_0}^2}\;=\;\frac{8\pi\rho_0 }{3}\notag\\
    \Rightarrow\;\;& \frac{k}{{R_0}^2}\;=\;\frac{8\pi\rho_0 }{3}-{H_0}^2\notag\\
    \Rightarrow\;\;& \frac{k}{{R_0}^2}\;=\;\frac{8\pi}{3}\left( \rho_0-\frac{3{H_0}^2}{8\pi} \right)\notag\\
    \Rightarrow\;\;& \frac{k}{{R_0}^2}\;=\;\frac{8\pi}{3}\left( \rho_0-\rho_c \right)\label{eq:model6}
\end{align}
where, \(\rho_c\) is the critical density given by
\begin{equation}
    \rho_c=\frac{3{H_0}^2}{8\pi}\label{eq:model7}
\end{equation}
The three fried models arise are,
\begin{enumerate}[label=(\roman*)]
    \item Flat model: When \(k = 0\), then \(\rho_0=\rho_c\) and equation \eqref{eq:model4} becomes
    \begin{align*}
        &\dot{R}^2=\frac{A^2}{R }\\
        \Rightarrow\;\;&\dot{R}=\frac{A}{\sqrt{R} }\\
        \Rightarrow\;\;&\sqrt{R} \D R=A\D t \\
        \Rightarrow\;\;&{R}^{3/2} =\frac{3}{2}A t \\
        \Rightarrow\;\;&{R}(t)=\left(\frac{3}{2}A\right)^{2/3} t^{2/3}
    \end{align*}
    This is also known as the Einstein's de-silter model.
    \item Closed model: When \(k = 1\), then \(\rho_0>\rho_c\) and equation \eqref{eq:model4} becomes
    \begin{align*}
        &\dot{R}^2+1=\frac{A^2}{R }\\
        \Rightarrow\;\;&\dot{R}=\frac{\sqrt{A^2-R}}{\sqrt{R} }\\
        \Rightarrow\;\;&\D t=\frac{\sqrt{R} }{\sqrt{A^2-R}}\D R \\
        \Rightarrow\;\;&t=\int \frac{A \sin \frac{\psi}{2}}{A \cos \frac{\psi}{2}}A^2\sin \frac{\psi}{2}\cos\frac{\psi}{2} \D \psi\quad[\text{let, }R=A^2\sin^2\psi/2]\\
        \Rightarrow\;\;&t=\frac{A^2}{2}\int (1- \cos \psi)\D \psi\\
        \Rightarrow\;\;&t=\frac{A^2}{2}(\psi- \sin \psi)
    \end{align*}
    So,
    \[R=\frac{A^2}{2}(1- \cos \psi)\qquad\text{and}\qquad t=\frac{A^2}{2}(\psi- \sin \psi)\]
    These two equations gives cycloid and shown in figure.
    \item Open model: When \(k=-1\), then \(\rho_0<\rho_c \) and equation \eqref{eq:model4} becomes
    \begin{align*}
        &\dot{R}^2-1=\frac{A^2}{R }\\
        \Rightarrow\;\;&\dot{R}=\frac{\sqrt{A^2+R}}{\sqrt{R} }\\
        \Rightarrow\;\;&\D t=\frac{\sqrt{R} }{\sqrt{A^2+R}}\D R \\
        \Rightarrow\;\;&t=\int \frac{A \sinh \frac{\psi}{2}}{A \cosh \frac{\psi}{2}}A^2\sinh \frac{\psi}{2}\cosh\frac{\psi}{2} \D \psi\quad[\text{let, }R=A^2\sinh^2\psi/2]\\
        \Rightarrow\;\;&t=\frac{A^2}{2}\int (\cosh \psi-1)\D \psi\\
        \Rightarrow\;\;&t=\frac{A^2}{2}(\sinh \psi- \psi)
    \end{align*}
    So,
    \[R=\frac{A^2}{2}(\cosh \psi-1)\qquad\text{and}\qquad t=\frac{A^2}{2}(\sinh \psi-\psi)\]
\end{enumerate}
\end{soln}
\end{document}