\documentclass[../main-sheet.tex]{subfiles}
\usepackage{../style}

\graphicspath{ {../img/} }
\backgroundsetup{contents={}}
\begin{document}
\chapter{Automorphism}
\begin{defn}[Automorphism]
    An automorphism of a group \(G\) is an isomorphism\footnote{
        Homomorphism: Suppose \(G,\,G'\) are multiplicative groups. A mapping \(f:G\to G'\) is called a group homomorphism iff \(f(ab)=f(a)f(b)\) holds for all \(a,b\in G\).\\
        Isomorphism: A bijective group homomorphism is called an isomorphism. 
    } of \(G\) onto itself.
\end{defn}
\begin{thm}
    The set \(Aut(G)\) of all automorphisms of a group \(G\) is a group under the operation of composition of mappings.
\end{thm}
\begin{proof}
    Here \(Aut(G)\) is the set of all automorphisms of a group \(G\) and the operation is the composition of mappings.\\
    Let \(f,g\in Aut(G)\).\\
    Then the composite map \(g \circ f\) is bijective, because \(f\) and \(g\) are bijective.\\
    Using the hypotheses that \(f\) and \(g\) are group homomorphisms, we can conclude that \(g\circ f\) is also a group homomorphism, because
    \begin{align*}
        (g\circ f)(ab)&=g(f(ab))\\
        &=g(f(a)f(b))\\
        &=g(f(a))g(f(b))\\
        &=(g\circ f)(a)(g\circ f)(b)
    \end{align*} 
    So, \(g\circ f \in Aut(G)\).\\
    This is the closure property.\\

    The associative law holds for \(Map(G)\), the set of all mappings of \(G\) into itself; so it holds in \(Aut(G)\), because \(Aut(G)\) is closed under composition of mappings.\\

    Clearly, \(1_G\) is the identity element of \(Aut(G)\).\\

    If \(f\in Aut(G)\), the inverse mapping \(f^{-1}:G\to G\) exists and is likewise bijective.\\
    Let \(f\in Aut(G)\) and \(a,b,x,y\in G\) such that \(f(a)=x\) and \(f(b)-y\).
    Then we have \(a=f^{-1}(x)\) and \(b=f^{-1}(y)\).\\
    Since \(f\) is a group homomorphism, we have \(f(ab)=f(a)f(b)=xy\).\\
    It gives, \(f^{-1}(xy)=ab=f^{-1}(x)f^{-1}(y)\).\\
    This implies that \(f^{-1}\) is also a group homomorphism.\\
    Hence, \(f^-1\in Aut(G)\).\\
    Therefore, \(Aut(G)\) is a group under composition of mappings.
\end{proof}
\section{Inner Automorphisms}
For any fixed \(a\in G\), we define a mapping \(f_a:G\to G\) by setting \(f_a(x)=axa^{-1}\), \(f_a\in Aut(G)\) for every \(a\in G\).
\begin{proof}
    \(f_a\) is injective (by the cancellation law), for
    \[f_a(x)=f_a(y)\;\;\Rightarrow\; axa^{-1}=aya^{-1}\;\;\Rightarrow\;x=y.\]
    \(f_a\) is surjective, because
    \[f_a(a^{-1}xa)=a(a^{-1}xa)a^{-1}=x.\]
    \(f_a\) is group homomorphism, because for all \(x,y\in G\), we have
    \[f_a(xy)=a(xy)a^{-1}=(axa^{-1})(aya^{-1})=f_a(x)f_a(y).\]
\end{proof}
\begin{defn}[Inner Automorphism]
    For any fixed \(a\in G\) the mapping \(f_a:G\to G\) defined by \(f_a(x)=axa^{-1}\) is called the inner automorphism determined by \(a\).
\end{defn}
\begin{thm}
    The  set \(Inn(G)\) of all inner  automorphisms of a group \(G\)  is a
    subgroup of \(Aut(G)\).
\end{thm}
\begin{proof}
    The relation \(f_a \circ f_b = f_{ab}\) is the key.\\
    This is easily proved, for
    \begin{align*}
        (f_a\circ f_b)(x)&=f_a(f_b(x))\\
        &=f_a(bxb^{-1})\\
        &=a(bxb^{-1})a^{-1}\\
        &=(ab)x(ab)^{-1}\\
        &=f_{ab}(x)\qquad \text{holds for all } x\in G
    \end{align*}
    
    So, \(Inn(G)\) Is closed under composition of mappings.\\
    The identity mapping \(l_G\) belongs to \(Inn(G)\), because \(f_e = 1_G\).\\
The inverse of \(f_a\), which is obviously an automorphism, is the inner automorphism
determined by \(a^{-1}\), because
\[f_a \circ  f_{a^{-1}}= f_{aa^{-1}}=f_e=1_G\]
and 
\[f_{a^{-1}} \circ  f_{a}= f_{a^{-1}a}=f_e=1_G\]
So, \(Inn(G)\) is a subgroup of \(Aut(G)\).
It remains to show that \(Inn(G)\) is a normal subgroup of \(Aut(G)\).
For any \(\sigma \in Aut(G)\), we have \(\sigma \circ f_a \circ \sigma^{-1}  = f_{\sigma(a)}\), because
\begin{align*}
    (\sigma \circ f_a \circ \sigma^{-1})(x)&=(\sigma \circ f_a)(\sigma^{-1}(x))\\
    &=\sigma(a\sigma^{-1}(x)a^{-1})\\
    &=\sigma(a)\sigma(\sigma^{-1}(x))\sigma(a^{-1})\\
    &=\sigma(a)x\sigma(a^{-1})\\
    &=\sigma(a)x(\sigma(a))^{-1}\\
    &=f_{\sigma (a)}(x)\;\;\in Inn(G)
\end{align*}
So, \(Inn(G)\) is a normal subgroup of \(Aut(G)\)
\end{proof}

\end{document}