\documentclass[../main-sheet.tex]{subfiles}
\usepackage{../style}

\graphicspath{ {../img/} }
\backgroundsetup{contents={}}
\begin{document}
\chapter{Applications of Sylow Theorems}
\textbf{Some Examples on the Use of Sylows Theorems}\\
% ^ple 12.7 No group G of order pq (where p and q are distinct primes) can be 't
% i
% Ex *
% simple. \ ;
% Solution. Suppose, without loss of generality, that p > q. I
% The number of Sylow p-subgroups is of the form 1 + kp and divides q.
% This is impossible unless k = 0:
% Thus there is just one Sylow p-subgroup of G .
% Consequently, it is normal in G and we are done
\begin{ex}
    A group of order 40 must contain a normal subgroup of order 5.
\end{ex}
\begin{soln}
    Here \(|G| = 40 = 2^3\cdot5\).\\
    \(s_5\) divides 8 and has the form \(1+5k\).
    So, necessarily \(s_5 = 1\) .\\
    Since there is only one subgroup of order 5, then this subgroup is normal in \(G\).
\end{soln}
\begin{ex}
    Show that no group of order 30 is simple.
\end{ex}
\begin{soln}
    To show: \(G\) has at least one non-trivial normal subgroup.\\
Here \(|G| =30 = 2\cdot 3\cdot 5\).\\
\(s_5\), the number of Sylow 5-subgroups, divides 6 and has the form \((1+ 5k)\).
Should \(k = 0\). We are done (i .e., there exists unique subgroup of order 5 is then normal in \(G\)).\\
Should \(k = 1 \), the total number of Sylow 5-subgroups is then 6. Such subgroups can have only the identity element \(e\) in common; so they would account for \(4 \times 6 =24 \) distinct elements of \(G\), not counting \(e \).\\
In this situation, \(s_3\) (the number of Sylow 3-subgroups), being a divisor of 10 and of the form \(1 + 3m\), can only be \(= 1 \)(\(m=3\) is ruled out by counting elements of \(G \), since \(G\) would have then at least \(25 + 20 = 45\) elements). So, the Sylow 3-subgroup is a normal subgroup.\\
So, it is proved that either \(G\) has a unique Sylow 5-subgroup or it has a unique Sylow 3-subgroup.\\
In either case, we get a non-trivial normal subgroup of \(G \).
\end{soln}
\end{document}
% Ex^nple 12.10 Show that no group of order 36 is simple.
% U
% Solution. Let|G \ = 36 = 22.32.
% Let P be a Sylow 3-subgroup of G ; then [G : P ] = 4.
% Consider the generalized Cayley representation G -> 54 afforded by P.
% K = K e r^V is a non-trivial normal subgroup of G , because 36 does not divide 4! = 24.
% So, G is not simple.
% ^ample 12.11 Show that no group of order p2q2 is simple, where p > q are primes.
% Solution. Let G be a group of order p2q 2, where p > q are primes.
% sp, the number of Sylow p -subgroups of G , has the form 1
% So, Sp = 1, q or q 2.
% Here, s p = q is ruled out, because 1 4- k p > p > q for A' > 0.
% So, s p = 1 or q2.
% If Sp = 1, the unique Sylow p-subgroup is then a non-trivial normal subgroup of C .
% So, suppose Sp = q 2. Also, we have sP = 14- k p. It gives,
% 1 4- k p = q 2
% => k p = q 2 — 1
% => P I 92 -1 ' -- -
% => p divides q — 1 or q 4- 1.
% But p > q; so p cannot divide q — 1.
% So, p divides q 4- 1.
% Since p and q are primes and p > q, we have p > q H- 1.
% So, p divides q + 1 implies p = q 4- 1.
% Since q 4- 1, q are consecutive integers and p, q are primes, the only possibilities for p
% and q are p = 3, q = 2.
% i
% . k p and divides q 2.
% j »
% f
% %
% \r
% m m
% f•:
% 1 i
% l
% _
% Then \G \ = 36.
% We know that no group of order 36 is simple. This completes the proof.
\begin{ex}
No group of order 108 is simple.
\end{ex}
\begin{soln}
Here \(|G| = 108 = 2^2\cdot 3^3\). \\
Then \(s_3\), the number of Sylow 3-subgroups divides 4 and has the form \(1+ 3k \).
\\Should \(k = 0\), we are done (the unique Sylow 3-subgroup of order 27 is normal in \(G\)).\\
Should \(k = 1\), we have \(s_3 = 4\).\\
Let \(H , K\) be two distinct Sylow 3-subgroups.\\
Then \(|HK|=\frac{|H|\,|K|}{|H\cap K|}\) is \(\leq 108\).\\
This implies, \(|H\cap K|\geq \frac{|H|\,|K|}{108}=\frac{27\cdot 27}{108}=\frac{27}{4}\). So, \(| H \cap K | \geq 7\).\\
% Since |H n /C| divides \H \ = 2 7 (because H n K is a subgroup of tf), therefore n K \-
% 9 (because H n K is a proper subgroup of H ).
% Now we use the fact that "a subgroup of order p
% necessarily normal (wherer > 1)".
% Here, \ H n K \ = 3Z and | = |K|= 33.
% So, (W n tf) < H and (// n <3 K .
% Let N be the ncrmauzer of H f! K in G.
% i1i1 li
% ;•
% 1 r-1 o f . a group of order pr is
% 11 1
% « i
% /! 1 11
% • i \
% .iI Then observe that H c A', K c /V, and so H l< a N . i
% 1!
% I H U X I _ 27.27
% I So,|/V| > = 9 = 81.
% J!
% Now,|A'| divides |G[ = 103 and is > 81.
% So,|/V| = |G|.
% Hence N = G [since N c G ] .
% 1
% 1IJ1I1
% I
% 4I Thus,(W n /f) o N = G 1
% I
% => ( W fl K ) < G.
% Thus we gel z non f*1 via. normal subgroup of G and her.ce T is not simple