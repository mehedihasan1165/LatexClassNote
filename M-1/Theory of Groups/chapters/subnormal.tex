\documentclass[../main-sheet.tex]{subfiles}
\usepackage{../style}

\graphicspath{ {../img/} }
\backgroundsetup{contents={}}
\begin{document}
\chapter{Subnormal Series of Group}
\begin{defn}
    A subnormal (or, subinvariant) series of a group \(G\) is a finite sequence of subgroups
    \begin{equation}
        G = G_0 \vartriangleright  G_1 \vartriangleright  G_2 \vartriangleright  \dots \vartriangleright  G_r=\{e\} \label{eq:subnormal1}
    \end{equation}
    or,
    \[
        \{e\}=G_r\vartriangleleft G_{r-1} \vartriangleleft \dots \vartriangleleft G_1 \vartriangleleft G_0=G
    \]
    such that each \(G_i \) is a normal subgroup of \(G_{i-1}\), where \(i = 1, 2,\dots, r\).
\end{defn}
\begin{note}
In the above series, \(r\) is called the length of the subnormal series; observe
that the number of terms in the subnormal series is \((r + 1)\).
\end{note}
\begin{rem}
In the definition of a subnormal series, it is not demanded that each \(G_i\) is
a proper subgroup of \(G_{i-1}\).
\end{rem}
\begin{ex}
\(S_4 \vartriangleright  V \vartriangleright  1\) is a subnormal series for the group \(S_4\)
where
\[ V = \{(1), (1 2)(3 4), (13)(2 4), (1 4)(2 3)\} \text{ and } 1 = \{(1)\}.\]
\end{ex}
\begin{ex}
\(S_4 \vartriangleright  V \vartriangleright  C \vartriangleright  1\) is another subnormal series for the group \(S_4\), where 
\[C = \{(1), (1 2 )(3 4)\}.\]
\end{ex}
\begin{defn}
    A normal (or, invariant) series of a group \(G\) is a subnormal series such
    that each of its terms is a normal subgroup of \(G\).
\end{defn}
\begin{ex}
    The group \(S_3\) has the normal subgroup \(N = \{(1), (1 2 3), (1 3 2)\}\).\\
    So, \(S3 \supseteq N \supseteq \{(1)\}\) is a normal series.
\end{ex}
\begin{ex}
    The group \(S_4\) has the normal subgroup
    \[V_4 = \{(1), (1 2) (3 4), (1 3) (2 4), (1 4)(2 3)\}.\]
    So, \(S_4 \vartriangleright   V_4 \vartriangleright   \{(1)\}\) is a normal series.\\
    Besides, we observe that \(W = \{(1), (1 2) (3 4)\}\) is a normal subgroup of \(V_4\) (because \(V_4\)
    is abelian), but \(W\) is not a normal subgroup of \(S_4\).\\
    So, \(S_4 \vartriangleright   V_4\vartriangleright  W \vartriangleright   \{(1)\}\) is a subnormal series, but not a normal series.
\end{ex}
\begin{note}
Some authors use the term `normal series' for our `subnormal series'.
\end{note}
\begin{defn}
    Given two subnormal series of \(G \), one is a refinement of the other if
    each term of the latter one series occurs as a term of the former series.
\end{defn}
\begin{ex}
The subnormal series \(S_4 \vartriangleright V \vartriangleright C \vartriangleright  1\)
 is a refinement of the subnormal series \(S_4 \vartriangleright  V \vartriangleright  1\).
\end{ex}
\begin{defn}
    The (normal) subgroups \(G_0\), \(G_1,\,G_2,\,\dots,G_r\) are called the terms of the series and the factor groups \(G_{i-1}/G_i (i=1,2,\dots,r )\) are called the factors of the series.
    The series \eqref{eq:subnormal1} is called  a proper subnormal series if every \(G_i \) is a proper normal subgroup of \(G_{i-1}(i=1,2,\dots,r )\).
\end{defn}
\begin{defn}
    The series
    \[\{e\} =J_0\vartriangleleft J_1 \vartriangleleft J_2 \vartriangleleft \dots \vartriangleleft J_m=G\]
    is said to be a proper refinement of the series
    \[\{e\} =H_0\vartriangleleft H_1 \vartriangleleft H_2 \vartriangleleft \dots \vartriangleleft H_n=G\]
    (of the same group \(G\)) if there is a \(j\in \{0,1,\dots, m\}\) such that \( H_i \neq J_j\) holds for \(i\in \{0,1,\dots, n\}\).
\end{defn}
\begin{defn}
    Two subnormal series of a given group \(G\), say
    \[G= G_0 \vartriangleright G_1\vartriangleright G_2\vartriangleright\dots \vartriangleright G_r = 1\]
    and
    \[G= H_0 \vartriangleright H_1\vartriangleright H_2\vartriangleright\dots \vartriangleright H_s = 1\]
    are called isomorphic (or equivalent) if there is a one-one correspondence between the
    set of non-trivial factor groups \(G_{i-1}/G_i \) and the set of non-trivial factor groups \(H_{j-i}/ H_j\)
    such that the corresponding factor groups are isomorphic.
\end{defn}
\begin{ex}
The following two subnormal series (of the cyclic group \(C_6\) of order 6)
\[C_6 \vartriangleright C_3 \vartriangleright 1 \text{ and } C_6 \vartriangleright C2 \vartriangleright 1\]
are isomorphic, for
\[C_6/C_3 \cong C_2/1 \text{ and } C_3/1 \cong C_6/C_2.\]
\end{ex}
\begin{ex}
Let \(G =\langle a \rangle\) be a cyclic group of order 24 so that \(o(a) = 24\).
Consider the following two normal series
\begin{equation}
    G = \langle a \rangle \vartriangleright \langle a^2 \rangle \vartriangleright \langle a^6 \rangle \vartriangleright \langle a^{12} \rangle \vartriangleright \{e\} \label{eq:c24.1}
\end{equation}
and
\begin{equation}
    G = \langle a \rangle \vartriangleright \langle a^3 \rangle \vartriangleright \langle a^6 \rangle \vartriangleright \langle a^{12} \rangle \vartriangleright \{e\} \label{eq:c24.2}
\end{equation}
The factors of \eqref{eq:c24.1} are
\[
\langle a \rangle/\langle a^3 \rangle, \langle a^3 \rangle/\langle a^6 \rangle, \langle a^6 \rangle/\langle a^{12} \rangle \text{ and } \langle a^{12} \rangle/\{e\},
\]
which are of orders 2, 3, 2, 2 respectively. Since these are of prime orders, they are
simple.
Similarly, the factors of \eqref{eq:c24.2} are
\[\langle a \rangle/\langle a3 \rangle, \langle a^3 \rangle/\langle a^6 \rangle, \langle a^6 \rangle/\langle a^{12} \rangle \text{ and } \langle a^{12} \rangle/\{e\},\]
which are of orders 3, 2, 2, 2 respectively. These are again simple.
So, \eqref{eq:c24.1} and \eqref{eq:c24.2} both are composition series of \(G\).
We see that both of these series are of same length (viz. 4)\\
Since two cyclic groups of same order are isomorphic, we then have
\[\langle a \rangle/\langle a^2 \rangle\cong \langle a^3 \rangle/\langle a^6 \rangle, \langle a^2 \rangle/\langle a^6 \rangle \cong \langle a \rangle/\langle a^3 \rangle,\]
\[\langle a^6 \rangle/\langle a^{12} \rangle \cong \langle a^6 \rangle/\langle a^{12} \rangle \text{ and } \langle a^{12} \rangle/\{e\} \cong \langle a^{12} \rangle/ \{e\}.\]
Thus there is a one-one correspondence between the factors of \eqref{eq:c24.1} and those of \eqref{eq:c24.2}
such that the corresponding factors are isomorphic.
\end{ex}
\begin{thm}[Schreier's Refinement Theorem]
     Any two subnormal series of a given group possess isomorphic
    refinements.
\end{thm}
\begin{proof}
    Suppose \[G = G_0 t> G1 o G2 ... t> G
    % f 7
    % a1
    % «
    % i
    % r-l > Gr = 1
    % and H = tf 0 > > H 2 ... > W
    % are two subnormal series of the group G.
    % We construct a refinement of the first series by inserting between G
    % * = 1* 2, 3 ... , r, the series
    % ^£-1 ~ GiCGi-i n //0) C> G^G,
    % Since each <3 // .
    % 5-1 > Hs = 1
    % i-i and G<, for each
    % implies that each G £(Gj_ x n Hj) o Ct-(C,
    % A similar refinement is carried out for the other series by inserting between and
    % Hj , for each j = 1, 2, 3, , s, t h e s e r i e s
    % G/ (G j_ 2 n H — G[.
    % n Wy.1) and Gi <J (7
    % 1-1 ••
    % y-i £
    % -1 £-!•
    % il
    % = Hy(G0 n t> Hj { G^ n Hj-i) » HJ { G2 n > ... O Hj { Gr n W/-0 = Hj.
    % Now taking X = Gi, A — G£_t, Y = Hj, B = and putting these in the Zassenhaus
    % Lemma, it follows that
    % Hi
    % J ~1
    % Gifa-i n Hj-J / GtiGi- j. n ty) s n HJ .1 ) / Hj { Gi n '
    % This shows that the corresponding factor groups are isomorphic.
    % Hence the above two refinements are isomorphic.
    % This completes the Droof.
\end{proof}
% * •
% Definition 13.7 A subnormal series G = G0 t> G1 > G2 o > Gr = 1 is called proper if
% every G* is a proper normal subgroup of Gj_ 2 (£ = 1, 2, 3, ... ,r).
% Definition 13.8 A proper subnormal series of G which has no proper refinement is
% called a composition series of G. The factor groups G^ i/ Gi are then called the
% composition factors of G .
% n
\end{document}