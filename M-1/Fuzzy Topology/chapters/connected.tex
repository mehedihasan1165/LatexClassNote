\documentclass[../main-sheet.tex]{subfiles}
\usepackage{../style}

\graphicspath{ {../img/} }
\backgroundsetup{contents={}}
\begin{document}
\chapter{Connected Fuzzy Topological Space}
\begin{defn}[Separated Fuzzy Sets]
    Let \(\ft\) be a \fts. Then \(A,B\in \fx\) are called separated sets if \(\bar{A}\wedge B=\underbar{0}=A\wedge\bar{B}\).
\end{defn}
\begin{lem}
    Let \(\ft\) be a \fts\s and \(A\), \(B\), \(C\in \fx\). If \(B\) and \(C\) are separated sets then \(A\wedge B\) and \(A\wedge C\) are separated.
\end{lem}
\begin{proof}
    Since \(B\) and \(C\) are separated, we have, \(\bar{B}\wedge C=\underbar{0}=B\wedge\bar{C}\).\\
    We have, \(A\wedge B < B\;\;\Rightarrow\, \overline{A\wedge B}< \overline{B}\) and \(A\wedge C <C\).\\
    This implies, \(\overline{(A\wedge B)}\wedge (A\wedge C)\leq \bar{B}\wedge C=\underline{0}\). Similarly, \((A\wedge B)\wedge \overline{(A\wedge C)}\leq B\wedge \bar{C}=\underline{0}\).\\
    Hence, \(A\wedge B\) and \(A\wedge C\) are separated.
\end{proof}
\begin{defn}[Connected Fuzzy Sets]
    Let \(\ft\) be a \fts. A fuzzy set \(A\) on \(X\) is called connected if there do not exist \(C,D\in \fx\setminus\set{\underbar{0}}\) such that \(A=C\vee D\).\\
    Or, A set \(A\) is connected if \(A=B\vee C\) then either \(B=\underbar{0}\) or, \(C=\underbar{0}\).
\end{defn}
\begin{thm}
    Let, \(\ft\) be a \fts\space and \(A\in\fx\). Then the following are equivalent:
    \begin{enumerate}
        \item \(A\) is connected.
        \item \(B\), \(C\in\fx\) are separated, \(A\leq B\vee C\) implies \(A\wedge B=\underbar{0}\) or, \(A\wedge C=\underbar{0}\).
        \item \(B\), \(C\in\fx\) are separated, \(A\leq B\vee C\) implies \(A\leq B\) or, \(A\leq C\).
    \end{enumerate}
\end{thm}
\begin{proof}
    \((1)\Rightarrow(2)\), Since, \(B\) and \(C\) are separated set. By the above lemma, we have \((A\wedge B)\) and \((A\wedge C)\) are separated. Since, \(A\) is connected and \(A\leq B\vee C\) implies
    \begin{align*}
        A&=A\wedge (B\vee C)\\
        &=(A\wedge B)\vee(A\wedge C)
    \end{align*}
    then by definition of connectedness, either \(A\wedge B=\underbar{0}\) or, \(A\wedge C=\underbar{0}\). Hence, \((2)\) holds.\\

    \((2)\Rightarrow(3)\), Suppose, \(A\wedge B=\underbar{0}\), then,
    \begin{align*}
        A&=(A\wedge B)\vee (A\wedge C)\\
        &=\underbar{0} \vee (A\wedge C)\\
        &=(A\wedge C).
    \end{align*}
    So, \(A\leq C\). Similarly, if \(A\wedge C=\underbar{0}\), then we can prove that \(A\leq B\). Thus, \((3)\) holds.\\

    Finally, \((3)\Rightarrow(1)\), Suppose, \((3)\) holds, we need to show that, \(A\) is connected. Let \(B,C\in\fx\) are two separated fuzzy sets such that \(A\leq B\vee C\). We need to prove that, either, \(B=\underbar{0}\) or, \(C=\underbar{0}\).\\
    By \((3)\), we have either \(A\leq B\) or, \(A\leq C\). Now if \(A\leq B\) then \(C\wedge A\leq C\wedge B\leq C\wedge\bar{B}\). But since, \(B\), \(C \) are separated sets so, \(C\wedge \bar{B }=\underbar{0}\). \(\therefore\,C\wedge A=\underbar{0}\).\\
    Again, \(C\wedge A=C\wedge(B\vee C)=C\). So \(C=\underbar{0}\). Now if \(A\leq C \), we can similarly prove that \(B=\underbar{0}\).\\
    Thus, \(A \) is connected.
\end{proof}
\newpage
\begin{thm}
    Let \(\ft\) be a \fts, \(A\in\fx\) is connected such that \(A\leq B\leq\bar{A}\). Show that \(B \) is connected.
\end{thm}
\begin{proof}
    Suppose, \(C \) and \(D \) are two separated fuzzy sets such that, \(B=C\vee D \). To show that, \(B \) is connected we need only to show either, \(C=\underbar{0}\) or, \(D=\underbar{0}\).\\
    By the lemma, we have \((A\wedge C )\) and \((A\wedge D )\) are separated sets. Let \(F=A\wedge C \), \(G=A\wedge D \). Now
    \begin{align*}
        F\vee G&=(A\wedge C)\vee (A\wedge D)\\
        &=A\wedge (C\vee D)\\
        &=A\wedge B\\
        &=A
    \end{align*}
    Since, \(A \) is connected, we have either \(F=\underbar{0}\) or, \(G=\underbar{0}\).\\
    Suppose, \(F=\underbar{0}\). Then, \(A=F\vee G =G =A\wedge D    \). This implies, \(A\leq D\). Thus, \(\bar{A }\leq \bar{ D }\). i.e., \(B\leq\bar{A }\leq \bar{D }\).\\
    Now, \(C\wedge B\leq C\wedge\bar{A }\leq C\wedge \bar{D }=\underbar{0}\). i.e., \begin{align*}
        &C\wedge B\leq \underbar{0}\\
        \Rightarrow\,&C\wedge (C\vee D) \leq \underbar{0}\\
        \Rightarrow\,&C= \underbar{0}\\
    \end{align*}
    Similarly, if \(G=\underbar{0}\), then we can show that \(D=\underbar{0}\). Hence, \(B \) is connected.
\end{proof}
\begin{defn}[Connected Fuzzy Topological Space]
    If the fuzzy set \(\underbar{1}\) is connected i.e., there does not exist separated sets \(C, D\in\fx\setminus\{\underbar{0}\}\) such that \(\underbar{1}=C \vee D \), then the \fts \(\ft\) is called a connected \fts.
\end{defn}
\begin{thm}[Characterization Theorem]
    Let \(\ft\) be a \fts. Then the followings are equivalent
    \begin{enumerate}
        \item \(\ft\) is connected.
        \item \(A,B \in \delta\), \(A\vee B=\underbar{1}\), \(A\wedge B=\underbar{0}\), implies \(\underbar{0}\in\set{A,B}\).
        \item \(A,B \in \delta'\), \(A\vee B=\underbar{1}\), \(A\wedge B=\underbar{0}\), implies \(\underbar{0}\in\set{A,B}\).
    \end{enumerate}
\end{thm}
\begin{proof}
    \((1)\Rightarrow(2)\), Suppose, \((2)\) is false. Then there are \(A, B\in\delta\setminus\{\underbar{0}\}\) such that 
    \begin{align*}
        &A\vee B=\underbar{1},\qquad A\wedge B=\underbar{0}\\
        \Rightarrow\,&A^c\wedge B^c=\underbar{0}, \text{ and }\,\, A^c\vee B^c=\underbar{1}\quad\text{[By De Morgan's Law]}\\
        \Rightarrow\,&\bar{A}^c\wedge B^c=\underbar{0}, \text{ and }\,\, A^c\wedge \bar{B}^c=\underbar{0}\quad\text{[Since, \(A^c \), \(B^c \) are closed.]}
    \end{align*}
    \(\therefore\) We have by definition, \(A^c \) and \(B^c \) are two separated sets. Therefore, we have \(A^c \vee B^c =\underbar{1}\) and \(A^c \), \(B^c \) are two separated sets. Hence, \(\ft\) is disconnected. Hence, \((2)\) is true.\\

    \((2)\Rightarrow(3)\), Let \(A,B\in \delta'\) such that \(A\vee B=\underbar{1}\) and \(A\wedge B=\underbar{0}\). Then by De Morgan's Laws, \(A^c \wedge B^c=\underbar{0}\) and \(A^c\vee B^c=\underbar{1}\). By \((2)\), \(\underbar{0}\in \{A^c,B^c \}\). Hence, \(\underbar{0}\in \{A,B \}\).\\

    \((3)\Rightarrow(1)\), If \(\ft\) is not connected, then there exists non-zero separated sets \(A,B\in \delta'\setminus\{\underbar{0}\}\) such that \(A\vee B=\underbar{1}\), which contradicts \((3)\).\\
    Hence, \(\ft\) is connected.
\end{proof}
\end{document}