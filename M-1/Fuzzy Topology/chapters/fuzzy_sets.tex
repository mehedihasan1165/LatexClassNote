\documentclass[../main-sheet.tex]{subfiles}
\usepackage{../style}

\graphicspath{ {../img/} }
\backgroundsetup{contents={}}
\begin{document}
\chapter{Fuzzy Sets}
\begin{defn}[Characteristic function]
    Let \(X\) be a universal set and \(A\subseteq X\). Then the function\footnote{Some authors use \(\mu\) as characteristic function.}
    \[\chi_A(x)=\begin{cases}
        1;\quad x\in A\\
        0;\quad x\notin A
    \end{cases}\]
    is characteristic function of \(A\) in \(X\).
\end{defn}


\begin{defn}[Fuzzy Set]
    A fuzzy set\footnote{Sometimes fuzzy set is denoted by \(\underset{\sim}{A}\).} \(A\subseteq X\) is a mapping \(A:X\to[0,1]\), where, \(A(x)=y\in [0,1]\) is called the membership function or, grade of membership of \(x\) in \(A\). The collection of all fuzzy sets of \(X\) is denoted by \(\fx\).
\end{defn}
\begin{defn}[Fuzzy subset]
    A fuzzy set \(A\) is called a fuzzy subset of another fuzzy set \(B\) if \(A(x)\leq B(x)\) \(\forall x\in X\). We denote it by \(A\leq B\).
\end{defn}
\begin{defn}[Empty fuzzy set]
    A fuzzy set \(A \) is called empty fuzzy set if \(\forall x\in X\) \(A(x)=0\). The empty fuzzy set is denoted by \(\underbar{0}\). Thus, \(\underbar{0}(x)=0\; \forall x\in  X\).
\end{defn}
\begin{defn}[Total fuzzy set] The total fuzzy set \(\underbar{1}\) is defined by  \(\underbar{1}(x)=1\; \forall x\in  X\).
\end{defn}
\begin{defn}[Equality of two fuzzy sets]
    Two fuzzy sets \(A\) and \(B\) of \(X\) is said to be equal iff \(A\leq B\) and \(B\leq A\).
\end{defn}
\begin{ex}[Empty and Total fuzzy set]
    Suppose, \(A:X\to[0,1]\) where \(X=[20,80]\). Then,
    \[
        \underbar{0}(x)=\begin{cases}
        0\quad \text{if } 15<x<90\\
        1\quad \text{otherwise }
    \end{cases}\qquad\text{and}\qquad
        \underbar{1}(x)=\begin{cases}
        1\quad \text{if } 20\leq x<90\\
        0\quad \text{otherwise }
    \end{cases}
    \]
\end{ex}
\begin{ex}[Fuzzy subset]
    Suppose, \(A:X\to[0,1]\) where, \(X=[0,100]\) defined by 
    \[
        A(x)=\begin{cases}
            0;\quad \text{if }0\leq x<40\\
            \frac{x}{75};\quad \text{if }40\leq x<75\\
            1;\quad \text{if }75\leq x\leq 100
        \end{cases}
    \]
    and \(B:X=[0,100]\to[0,1]\) defined by 
    \[
        B(x)=\begin{cases}
            0;\quad \text{if }0\leq x<40\\
            \frac{x}{95};\quad \text{if }40\leq x<95\\
            1;\quad \text{if }95\leq x\leq 100
        \end{cases}
    \]
    Then, \(B(x)\) is a subset of \(A(x)\). Since, \(B(x)\leq A(x)\;\forall x\in X\).
\end{ex}
\section{Fuzzy Set Operations}
\begin{defn}[Union of Fuzzy Sets]
    Let \(A,\; B\in \fx\). Then the union of \(A\) and \(B\) is denoted and defined by, \((A\vee B) (x)=\max\set{A(x),\,B(x)},\;\forall x\in X\).
\end{defn}
\begin{defn}[Intersection of Fuzzy Sets]
    Let \(A,\; B\in \fx\). Then the intersection of \(A\) and \(B\) is denoted and defined by, \((A\wedge B) (x)=\min\set{A(x),\,B(x)},\;\forall x\in X\).
\end{defn}
\begin{defn}[Complement of Fuzzy Set]
    Let \(A\) be a fuzzy set of \(X\). Then, the complement of \(A\) is denoted by \(A^c\) and defined by \(A^c(x)=1-A(x),\;\forall x\in X\).
\end{defn}
\begin{ex}
    Given,
    \[
        A_1=\begin{cases}
            \begin{aligned}
                1; \qquad&\text{if }40\leq x<50\\
                1-\frac{x-50}{10}; \qquad&\text{if }50\leq x<60\\
                0; \qquad&\text{if }60\leq x\leq 100
            \end{aligned}
    \end{cases}
    \qquad\text{and}\qquad
    A_2=\begin{cases}
        \begin{aligned}
            0; \qquad&\text{if }40\leq x<50\\
            \frac{x-50}{10}; \qquad&\text{if }50\leq x<60\\
            1-\frac{x-60}{10}; \qquad&\text{if }60\leq x<70\\
            0; \qquad&\text{if }70\leq x\leq 100
        \end{aligned}
    \end{cases}
    \]
    \begin{enumerate}
        \item Find the complement of \(A_1\) and \(A_2\).
        \item Find \((A_1\wedge A_2)(x)\) and \((A_1\vee A_2)(x)\)
    \end{enumerate}
    \underline{Solution:}
    \begin{enumerate}
        \item Complement of \(A_1\),
        \[
            A_1^c=\begin{cases}
                \begin{aligned}
                    0; \qquad&\text{if }40\leq x<50\\
                    \frac{x-50}{10}; \qquad&\text{if }50\leq x<60\\
                    1; \qquad&\text{if }60\leq x\leq 100
                \end{aligned}
            \end{cases}
            \]
        Complement of \(A_2\),
        \[
            A_2^c=\begin{cases}
                \begin{aligned}
                    1; \qquad&\text{if }40\leq x<50\\
                    \frac{60-x}{10}; \qquad&\text{if }50\leq x<60\\
                    \frac{x-60}{10}; \qquad&\text{if }60\leq x<70\\
                    1; \qquad&\text{if }70\leq x\leq 100
                \end{aligned}
            \end{cases}
            \]
        \item 
        \[
            (A_1\wedge A_2)(x)=\begin{cases}
                \begin{aligned}
                    0; \qquad&\text{if }40\leq x<50\\
                    \frac{x-50}{10}; \qquad&\text{if }50\leq x\leq 55\\
                    1-\frac{x-50}{10}; \qquad&\text{if }55\leq x\leq 60\\
                    0; \qquad&\text{if }60\leq x\leq 100
                \end{aligned}
            \end{cases}
            \]
        \[
            (A_1\vee A_2)(x)=\begin{cases}
                \begin{aligned}
                    1; \qquad&\text{if }40\leq x\leq 50\\
                    1-\frac{x-50}{10}; \qquad&\text{if }50\leq x\leq 55\\
                    \frac{x-50}{10}; \qquad&\text{if }55\leq x< 60\\
                    1-\frac{x-60}{10}; \qquad&\text{if }60\leq x< 70\\
                    0; \qquad&\text{if }70\leq x< 100
                \end{aligned}
            \end{cases}
            \]
\end{enumerate}
\end{ex}
\begin{defn}[Level Set]
    Let \(A :X\to [0,1]\) be a fuzzy set. The \(\alpha\) level set of \(A \) is denoted and defined by, \(A_\alpha\) or \(\alpha_A=\{x\in X|A(x)\geq \alpha\}\) where, \(0<\alpha\leq 1\).
\end{defn}
\begin{defn}[Core level of a fuzzy set]
    When \(\alpha=1\), then \(A_1=\{x\in X|A(x)=1\}\) is called the core level of \(A \).
\end{defn}
\begin{defn}[Support of a fuzzy set]
    Support of a fuzzy set \(A\) is denoted and defined by,\(S_A=\{x\in X|A(x)>0\}\).
\end{defn}
\begin{ex}
    Given,
    \[
        A=\begin{cases}
            \begin{aligned}
                0; \qquad&\text{if } x\leq20 \text{ or, }x\geq 60\\
                \frac{x-20}{15}; \qquad&\text{if }20< x<35\\
                \frac{60-x}{15}; \qquad&\text{if }45< x<60\\
                0; \qquad&\text{if }35\leq x\leq 45
            \end{aligned}
    \end{cases}
    \qquad\text{and}\qquad
    B=\begin{cases}
        \begin{aligned}
            0; \qquad&\text{if }x\leq45\\
            \frac{x-45}{15}; \qquad&\text{if }45< x<60\\
            1; \qquad&\text{if }x\geq 60
        \end{aligned}
    \end{cases}
    \]
    \begin{enumerate}
        \item \begin{enumerate}
            \item Core level of \(A \)?
            \item Support of \(A \)?
            \item Half level of \(A \)?
            \item \(\frac{3}{4}\) level of \(A \)?
        \end{enumerate}
        \item \begin{enumerate}
            \item Core level of \(B \)?
            \item Support of \(B \)?
            \item Half level of \(B \)?
        \end{enumerate}
    \end{enumerate}
\end{ex}
\begin{soln}
    \begin{enumerate}
        \item \begin{enumerate}
            \item Core level of \(A \) is \(A_1=\{x\in X|35\leq x\leq45\}\).
            \item Support level of \(A \) is \(S_A=\{x\in X|20<x<60\}\).
            \item Half level of \(A \) is \(A_{\frac{1}{2}}=\{x\in X|27.5\leq x\leq52.5\}\).
            \item \(\frac{3}{4}\) level of \(A \) is \(A_{\frac{3}{4}}=\{x\in X|31.25\leq x\leq48.75\}\).
        \end{enumerate}
        \item \begin{enumerate}
            \item Core level of \(B \) is \(B_1=\{x\in X|x\geq 60\}\).
            \item Support level of \(B \) is \(S_B=\{x\in X|x>45\}\).
            \item Half level of \(B \) is \(B_{\frac{1}{2}}=\{x\in X|x\geq52.5\}\).
        \end{enumerate}
    \end{enumerate}
\end{soln}
\begin{ex}
    \(A:X\to [0,1]\) defined by 
    \[
        A(x)=\begin{cases}
            \begin{aligned}
                1; \qquad&\text{if } x\leq20\\
                \frac{35-x}{20}; \qquad&\text{if }20\leq x <35\\
                0; \qquad&\text{if }x\geq35
            \end{aligned}
    \end{cases}
    \]
    Then find \(\frac{1}{2}\) level of \(A \).
\end{ex}
\begin{soln}
    \[A_\frac{1}{2}=\{x\in X|x\leq 25\}\]
\end{soln}
\begin{prob}
    Consider, the two fuzzy sets \(A,B:X=[0,100]\to[0,1]\) defined by
    \[
        A(x)=\begin{cases}
            \begin{aligned}
                0; \qquad&\text{if } 0\leq x< 40\\
                \frac{x}{75}; \qquad&\text{if }40\leq x<75\\
                1; \qquad&\text{if }75\leq x<100
            \end{aligned}
    \end{cases}
    \qquad\text{and}\qquad
    B(x)=\begin{cases}
        \begin{aligned}
            0; \qquad&\text{if }0\leq x< 40\\
            \frac{x}{95}; \qquad&\text{if }40\leq x<95\\
            1; \qquad&\text{if }95\leq x\leq 100
        \end{aligned}
    \end{cases}
    \]
    Then find \((A\wedge B)(x )\) and \((A\vee B )(x )\).
\end{prob}
\begin{soln}
    \[
        (A\wedge B)(x)=\begin{cases}
            \begin{aligned}
                0; \qquad&\text{if } 0\leq x< 40\\
                \frac{x}{95}; \qquad&\text{if }40\leq x<95\\
                1; \qquad&\text{if }95\leq x\leq 100
            \end{aligned}
    \end{cases}
    \qquad\text{and}\qquad
    (A\vee B)(x)=\begin{cases}
        \begin{aligned}
            0; \qquad&\text{if }0\leq x\leq 40\\
            \frac{x}{75}; \qquad&\text{if }40\leq x<75\\
            1; \qquad&\text{if }75\leq x\leq 100
        \end{aligned}
    \end{cases}
    \]
\end{soln}
\# Suppose, \(X=\R \) and the fuzzy set of real numbers much greater than \(5\) in \(X \), that could be defined by,
\[
    A(x)=\begin{cases}
        \begin{aligned}
            0; \qquad&\text{if } x\leq 5\\
            \frac{x-5}{50}; \qquad&\text{if }5< x\leq 55\\
            1; \qquad&\text{if }x\geq 55
        \end{aligned}
\end{cases}
\]
\begin{ex}
    Consider, the two fuzzy sets \(A \) and \(B \) of \(\fx \), where \(X=[0,100]\)
    \[
        A(x)=\begin{cases}
            \begin{aligned}
                1; \qquad&\text{if } 40\leq x\leq 50\\
                1-\frac{x-50}{10}; \qquad&\text{if } 50\leq x\leq 60\\
                0; \qquad&\text{if }60\leq x\leq100
            \end{aligned}
    \end{cases}
    \qquad\text{and}\qquad
    B(x)=\begin{cases}
        \begin{aligned}
            0; \qquad&\text{if }40\leq x\leq 50\\
            \frac{x-50}{10} ; \qquad&\text{if }50 \leq x\leq 60\\
            1-\frac{x-60}{10}; \qquad&\text{if }60\leq x\leq 70\\
            0; \qquad&\text{if }70\leq x\leq 100
        \end{aligned}
    \end{cases}
    \]
    Draw \((A\vee B )(x),\,(A\wedge B)(x),\,A',\,B'\).
\end{ex}
\begin{soln}
    Here,
    \[
            (A\vee B)(x)=\begin{cases}
            \begin{aligned}
                1; \qquad&\text{if } 40\leq x\leq 50\\
                1-\frac{x-50}{10}; \qquad&\text{if } 50\leq x\leq 55\\
                \frac{x-50}{10}; \qquad&\text{if } 55\leq x\leq 60\\
                1-\frac{x-60}{10}; \qquad&\text{if } 60\leq x\leq 70\\
                0; \qquad&\text{if } 70\leq x\leq 100
            \end{aligned}
    \end{cases}
    \qquad\text{and}\qquad
    (A\wedge B)(x)=\begin{cases}
        \begin{aligned}
            0; \qquad&\text{if } 40\leq x\leq 50\\
            \frac{x-50}{10} ; \qquad&\text{if }50 \leq x\leq 55\\
            1-\frac{x-50}{10} ; \qquad&\text{if }55 \leq x\leq 60\\
            0; \qquad&\text{if }60\leq x\leq 100
        \end{aligned}
    \end{cases}
    \]
    \[
            A^c(x)=\begin{cases}
            \begin{aligned}
                0; \qquad&\text{if } 40\leq x\leq 50\\
                \frac{x-50}{10}; \qquad&\text{if } 50\leq x\leq 60\\
                1; \qquad&\text{if }60\leq x\leq100
            \end{aligned}
    \end{cases}
    \qquad\text{and}\qquad
    B^c(x)=\begin{cases}
        \begin{aligned}
            1; \qquad&\text{if }40\leq x\leq 50\\
            1-\frac{x-50}{10} ; \qquad&\text{if }50 \leq x\leq 60\\
            \frac{x-60}{10}; \qquad&\text{if }60\leq x\leq 70\\
            1; \qquad&\text{if }70\leq x\leq 100
        \end{aligned}
    \end{cases}
    \]
    \begin{figure}[H]
        \centering
        \import{../img/}{set_operation.tikz}
    \end{figure}
\end{soln}
\begin{defn}[Fuzzy Relation]
    Let \(X \) and \(Y \) be two non-empty classical(Fuzzy) sets. Then a fuzzy relation \(R \) on \(X\times Y \) is a mapping, \(R:X\times Y\to[0,1]\) where, the number \(R(x,y)\in[0,1]\) is called the degree of relationship between \(x \) and \(y \).
\end{defn}
\begin{ex}
    Let \(X=\{a,b,c \},\, Y=\{c,d \}\). Then \(X\times Y=\{(a,c),(a,d),(b,c),(b,d),(c,c),(c,d )\}\) where \(R(a,c)=R(a,d)=0\), \(R(b,c)=R(b,d)=R(c,c)=1\) and \(R(c,d)=0.8\). For the fuzzy relation:
    \begin{enumerate}
        \item Core of \(R \)?
        \item Support of \(R \)?
        \item 0.7 of \(R \)?
    \end{enumerate}
\end{ex}
\begin{soln}
    \begin{enumerate}
        \item Core of \(R=\{(b,c),(b,d),(c,c )\}\) Since, \(R(x,y)=1\) for \(x\in X \) and \(y\in Y \).
        \item Support of \(R=\{(b,c),(b,d),(c,c ),(c,d)\}\) Since, \(R(x,y)>0\) for \(x\in X \) and \(y\in Y \).
        \item 0.7 of \(R=\{(b,c),(b,d),(c,c ),(c,d)\}\) Since, \(R(x,y)>0.7\) for \(x\in X \) and \(y\in Y \).
    \end{enumerate}
\end{soln}
\begin{defn}[Max-min and Min-max Composition]
    Let \(R \) be a fuzzy relation on \(X\times Y \) i.e., \(R\in\mathcal{F }(X\times Y )\) and \(S \) be a fuzzy relation on \(Y\times Z \) i.e., \(S\in \mathcal{F }(Y\times Z )\). Then \(R\circ S \in \mathcal{F }(X\times Z)\) defined by \((R\circ S)(x,z)=\bigvee_{y\in Y }R(x,y)\wedge S(y,z )\) is called the Max-Min composition of \(R \) and \(S \) on \(X\times Z \). And \((R\circ S)(x,z)=\bigwedge_{y\in Y }R(x,y)\vee S(y,z )\) is called the Min-Max composition of \(R \) and \(S \) on \(X\times Z \)
\end{defn}
\end{document}