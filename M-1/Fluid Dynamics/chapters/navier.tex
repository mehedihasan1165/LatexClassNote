\documentclass[../main-sheet.tex]{subfiles}
\usepackage{../style}

\graphicspath{ {../img/} }
\backgroundsetup{contents={}}
\begin{document}
\chapter{Navier-Stokes Equation}
\begin{prob}
    Explaining each term clearly, derive the Navier-Stokes equation for viscous compressible fluid and then deduce the equation in the form 
    \[\rho \ddt{q}=\rho \,F-\nabla\,P+\mu\nabla^2\,q\]
    for viscous incompressible fluid. Hence, derive the Lamb's form of these equations.
\end{prob}
\begin{soln}
    Let us consider the motion of a compressible viscous fluid with velocity vector,
    \[\vec{q}=u\vec{i}+v\vec{j}+w\vec{k}\]
    The equation of the motion of the flow fluid becomes,
    \[\rho\ddt{\vec{q}}=\rho\,\vec{F}+\vec{P}\]
    where, \(\vec{F}=F_x\vec{i}+F_y\vec{j}+F_z\vec{k}\) is the \emph{body force} and \(\vec{P}=P_x\vec{i}+P_y\vec{j}+P_z\vec{k}\) is the \emph{surface force}.\\
    
    Let us consider a small parallelopiped of volume \(\D v=\D x\D y\D z\) inside a fluid. The resuting surface force acting on the volume,
    \begin{align*}
        \vec{P}&=\left( \pardx{\vec{P}_x}+\pardy{\vec{P}_y}+\pardz{\vec{P}_z} \right)\dx\dy\dz\\
        &=\left( \pardx{\vec{P}_x}+\pardy{\vec{P}_y}+\pardz{\vec{P}_z} \right)\D v
    \end{align*}
    Now, the resultant surface force per unit volume \((\D v=1)\) is,
    \[
        \vec{P}=\pardx{\vec{P}_x}+\pardy{\vec{P}_y}+\pardz{\vec{P}_z}
    \]
    Now, the quantites \(\vec{P}_x\), \(\vec{P}_y\), \(\vec{P}_z\) are vectors which can be resolved as,
    \begin{align*}
        \vec{P}_x&=\sigma_{xx}\,\vec{i}+\sigma_{xy}\,\vec{j}+\sigma_{xz}\,\vec{k}\\
        \vec{P}_y&=\sigma_{yx}\,\vec{i}+\sigma_{yy}\,\vec{j}+\sigma_{yz}\,\vec{k}\\
        \vec{P}_z&=\sigma_{zx}\,\vec{i}+\sigma_{zy}\,\vec{j}+\sigma_{zz}\,\vec{k}
    \end{align*}
    where \(\sigma_{xx}\), \(\sigma_{yy}\), \(\sigma_{zz}\) are normal stresses and \(\sigma_{xy}\), \(\sigma_{yx}\), \(\sigma_{xz}\), \(\sigma_{zx}\), \(\sigma_{zy}\), \(\sigma_{yz}\) are shearing stresses.
    \[
        \therefore \,\vec{P}=
        \left( \pardx{\sigma_{xx}}+\pardy{\sigma_{yx}}+\pardz{\sigma_{zx}} \right)\vec{i}+
        \left( \pardx{\sigma_{xy}}+\pardy{\sigma_{yy}}+\pardz{\sigma_{zy}} \right)\vec{j}+
        \left( \pardx{\sigma_{xz}}+\pardy{\sigma_{yz}}+\pardz{\sigma_{zz}} \right)\vec{k}
    \]
    We have the equation of motion,
    \[\rho\ddt{\vec{q}}=\rho\vec{F}+\vec{P}\]
\begin{equation}
    \begin{rcases}
        \displaystyle\rho\ddt{u}=\rho F_x+\pardx{\sigma_{xx}}+\pardy{\sigma_{yx}}+\pardz{\sigma_{zx}}\\
        \displaystyle\rho\ddt{v}=\rho F_y+\pardx{\sigma_{xy}}+\pardy{\sigma_{yy}}+\pardz{\sigma_{zy}}\\
        \displaystyle\rho\ddt{w}=\rho F_z+\pardx{\sigma_{xz}}+\pardy{\sigma_{yz}}+\pardz{\sigma_{zz}}\quad
    \end{rcases}\label{eq:navier1}
\end{equation}
where, \(\ddt{}=\frac{\partial}{\partial\,t}+u\pardx{}+v\pardy{}+w\pardz{}\) is the total derivative with respect to the fluid particle following the motion.\\


We know that the stress and rate of strain relation,
\begin{align*}
    \sigma_{xx}&=2\mu\pardx{u}-\frac{2\mu}{3}(\vec{\nabla}\cdot\vec{q})-P\\
    \sigma_{yy}&=2\mu\pardy{v}-\frac{2\mu}{3}(\vec{\nabla}\cdot\vec{q})-P\\
    \sigma_{zz}&=2\mu\pardz{w}-\frac{2\mu}{3}(\vec{\nabla}\cdot\vec{q})-P\\
    \sigma_{xy}=\sigma_{yx}&=\mu\left( \pardy{u}+\pardx{v} \right)\\
    \sigma_{yz}=\sigma_{zy}&=\mu\left( \pardz{v}+\pardy{w} \right)\\
    \sigma_{zx}=\sigma_{xz}&=\mu\left( \pardz{u}+\pardx{w} \right)
\end{align*}
using these we get from \eqref{eq:navier1}
\begin{equation}
    \rho\ddt{u}=\rho F_x+\pardx{}\left[ 2\mu\pardx{u}-\frac{2\mu}{3}(\vec{\nabla}\cdot\vec{q})-P \right]+\pardy{}\left[ \mu\left( \pardy{u}+\pardx{v} \right) \right]+\pardz{}\left[ \mu\left( \pardz{u}+\pardx{w} \right) \right]
    \label{eq:navier2}
\end{equation}
\begin{equation}
    \rho\ddt{v}=\rho F_y+\pardx{}\left[\mu\left( \pardy{u}+\pardx{v} \right)\right]+\pardy{}\left[2\mu\pardy{v}-\frac{2\mu}{3}(\vec{\nabla}\cdot\vec{q})-P\right]+\pardz{}\left[\mu\left( \pardz{v}+\pardy{w} \right)\right]
    \label{eq:navier3}
\end{equation}
\begin{equation}
    \rho\ddt{w}=\rho F_z+\pardx{}\left[\mu\left( \pardz{u}+\pardx{w} \right)\right]+\pardy{}\left[\mu\left( \pardz{v}+\pardy{w} \right)\right]+\pardz{}\left[2\mu\pardz{w}-\frac{2\mu}{3}(\vec{\nabla}\cdot\vec{q})-P\right]
    \label{eq:navier4}
\end{equation}
Now equation \eqref{eq:navier2} can be written,
\begin{align*}
    \rho\ddt{u}&=\rho F_x-\pardx{P}+2\mu\parxn{u}{2}-\frac{2\mu}{3}\pardx{}(\vec{\nabla}\cdot \vec{q})+\mu\pardy{}\left( \pardy{u}+\pardx{v} \right)+\mu\pardz{}\left( \pardz{u}+\pardx{w}  \right)\\
    &=\rho F_x-\pardx{P}+\mu\left(\parxn{u}{2}+\paryn{u}{2}+\parzn{u}{2}\right) +\mu\pardx{}\left( \pardx{u}+\pardy{v}+\pardz{w} \right)-\frac{2\mu}{3}\pardx{}(\vec{\nabla}\cdot \vec{q})\\
    &=\rho F_x-\pardx{P}+\mu\left(\parxn{u}{2}+\paryn{u}{2}+\parzn{u}{2}\right) +\mu\pardx{}\left( \vec{\nabla}\cdot \vec{q}\right)-\frac{2\mu}{3}\pardx{}(\vec{\nabla}\cdot \vec{q})\\
    &=\rho F_x-\pardx{P}
    +\mu\frac{1}{3}\pardx{}(\vec{\nabla}\cdot \vec{q})+
    \mu\left(\parxn{u}{2}+\paryn{u}{2}+\parzn{u}{2}\right)
\end{align*}
Similarly,
\begin{align*}
    \rho\ddt{v}&=\rho F_y-\pardy{P}+\mu\frac{1}{3}\pardy{}(\vec{\nabla}\cdot \vec{q})+\mu\left(\parxn{v}{2}+\paryn{v}{2}+\parzn{v}{2}\right)\\
    \rho\ddt{w}&=\rho F_z-\pardz{P}+\mu\frac{1}{3}\pardz{}(\vec{\nabla}\cdot \vec{q})+\mu\left(\parxn{w}{2}+\paryn{w}{2}+\parzn{w}{2}\right)
\end{align*}
\newpage
In general,
\begin{equation}
    \rho\ddt{\vec{q}}=\rho \,\vec{F}-\vec{\nabla}{P}+\frac{1}{3}\mu\vec{\nabla}(\vec{\nabla}\cdot \vec{q})+\mu \nabla^2\vec{q}
    \label{eq:navier5}
\end{equation}
where \(\nabla^2=\parxn{}{2}+\paryn{}{2}+\parzn{}{2}\)\\

This is Navier-Stokes equation of motion for viscous compressible fluid.\\

For an incompressible flow, \(\rho=\) constant and the continuity equation is, \(\vec{\nabla}\cdot\vec{q}=0\). Then from \eqref{eq:navier5} we get
\begin{align*}
    \rho\ddt{\vec{q}}&=\rho \,\vec{F}-\vec{\nabla}{P}+\mu \nabla^2\vec{q}\\
    \Rightarrow\;\ddt{\vec{q}}&=\,\vec{F}-\frac{1}{\rho}\vec{\nabla}{P}+\nu \nabla^2\vec{q}\quad[\because\,\frac{\mu}{\rho}=\nu]
\end{align*}
This is Navier-Stokes equation for incompressible flow.\\

\emph{Navier-Stokes equation to Lamb's form:} Navier-Stokes equation,
\begin{align}
    & \ddt{\vec{q}}=\vec{F}-\frac{1}{\rho}\,\vec{\nabla}P+\nu\,\nabla^2\vec{q}\notag\\
    \Rightarrow\;\;& \left( \frac{\partial}{\partial\,t}+u\pardx{}+v\pardy{}+w\pardz{} \right)\vec{q}=\vec{F}-\frac{1}{\rho}\,\vec{\nabla}P+\nu\,\nabla^2\vec{q}\notag\\
    \Rightarrow\;\;& \left( \frac{\partial}{\partial\,t}+\vec{q}\cdot\vec{\nabla}\right)\vec{q}=\vec{F}-\frac{1}{\rho}\,\vec{\nabla}P+\nu\,\nabla^2\vec{q}\label{eq:navVec1}
\end{align}
We know,
\begin{align*}
    & (\vec{q}\cdot\vec{\nabla})\cdot\vec{q}=\frac{1}{2}\vec{\nabla}q^2-\vec{q}\times (\vec{\nabla}\times\vec{q})\\
    \Rightarrow\;\;& (\vec{q}\cdot\vec{\nabla})\cdot\vec{q}=\vec{\nabla}\left(\frac{q^2}{2}\right)-\vec{q}\times (\vec{\nabla}\times\vec{q})
\end{align*}
From \eqref{eq:navVec1}
\[
    \frac{\partial\,\vec{q}}{\partial\,t}+\vec{\nabla}\left(\frac{q^2}{2}\right)-\vec{q}\times (\vec{\nabla}\times\vec{q})=\vec{F}-\frac{1}{\rho}\,\vec{\nabla}P+\nu\,\nabla^2\vec{q}
\]
Which is the Navier-Stokes equation in Lamb's form. 
\end{soln}
\begin{prob}
    Prove that,
    \[\sigma_{ij}=-P\delta_{ij}+2\mu(e_{ij}-\frac{1}{3}\Delta \delta_{ij}).\]
    Then derive the Navier-Stokes equation.
\end{prob}
\begin{soln}
    In a fluid at rest there are only normal components of stress on a surface and the stress does not depend on the orientation of the surface. That means, the stress tensor is isotropic or spherically symmetric.

    An isotropic tensor is defined as one whose components do not change under a rotation of the co-ordinate system. The only second order isotropic tensor is the Kronecker Delta
    \[\delta=\begin{bmatrix}
        1&0&0\\
        0&1&0\\
        0&0&1\\
    \end{bmatrix}\]
    So, any isotropic second tensor will be proportional to \(\delta\). Therefore, the stress in a static fluid, being isotropic must be of the form
    \begin{equation}
        \sigma_{ij}=-P\delta_{ij}\label{eq:navTensor1}
    \end{equation}
    where \(\sigma_{ij}\) is the \(i\)th component of the force per unit area exerted, across a plane surface element normal to the perpendicular direction at position \(\vec{x}\) in the fluid at time \(t\) and the tensor of which it is the general components is the tensor.\\
    From \eqref{eq:navTensor1} we may say,
    \begin{align*}
        \sigma_{ii}&=-P (\delta_{11}+\delta_{22}+\delta_{33})\\
        &=-P (1+1+1)\\
        &=-3P
    \end{align*}
    \begin{equation}
        \therefore \; P=-\frac{1}{3}\sigma_{ii} \label{eq:navTensor2}
    \end{equation}
    Where \(P\) is the hydrostatic pressure.\\
    From this we can define the pressure at a point in a moving fluid to be given by \(-\frac{1}{3}\sigma_{ii}\), where \(\sigma_{ij}   \) is the stress tensor.

    Next we get the stress tensor equal to an isotropic part given by \eqref{eq:navTensor1} plus a non-isotropic part denoted by \(d_{ij}\) known as the deviatoric stress tensor, as follows
    \begin{equation}
        \sigma_{ij}=-P\delta_{ij}+d_{ij} \label{eq:navTensor3}
    \end{equation}
    It can be shown that \(d_{ij}\) must have the following form
    \begin{equation}
        d_{ij}=A_{ijkl}\frac{\partial u_k}{\partial x_l}\label{eq:navTensor4}
    \end{equation}
    where the coefficient \(A_{ijkl}\) depends on the local state of the fluid but not directly on the velocity distribution and is symmetric in \(i\) and \(j\).\\
    We have,
    \[\frac{\partial u_k}{\partial x_l}=e_{kl}+\epsilon_{kl}\]
    \begin{equation}
        \begin{rcases}
            e_{kl}=\frac{1}{2}\left( \frac{\partial u_k}{\partial x_l}+\frac{\partial u_l}{\partial x_k} \right)\\
            \epsilon_{kl}=\frac{1}{2}\left( \frac{\partial u_k}{\partial x_l}-\frac{\partial u_l}{\partial x_k} \right)\quad
        \end{rcases}\label{eq:navTensor5}
    \end{equation}
    So that \(e_{kl}\) is symmetric and \(\epsilon_{kl}\) is antisymmetric in \(k\) and \(l\).\\
    The tensor \(\epsilon_{kl}\) has only three independent components and so can be written in the terms of 3 vectors \((\omega_1,\,\omega_2,\,\omega_3)\) as follows:
    \begin{equation}
        \epsilon_{kl}=-\frac{1}{2}\epsilon_{klm}\omega_3\quad(\omega\text{ is the vorticity})\label{eq:navTensor6}
    \end{equation}
    where \(\epsilon_{klm}\) is the completely antisymmetric 3 tensor.\\
    Thus, \eqref{eq:navTensor4} becomes,
    \begin{equation}
        d_{ij}=A_{ijkl}e_{kl}-\frac{1}{2}A_{ijkl}\epsilon_{klm}\omega_3\label{eq:navTensor7}
    \end{equation}
    For a fluid that is isotropic, \(A_{ijkl}\) must be built up from isotropic two tensors of which there is only one's \(\delta_{ij}\).

    Since it is observed that the basic isotropic tensor is the Kronecker delta tensor, and that all isotropic tensors of even order can be written as the sum of products of delta tensors then,
    \begin{equation}
        A_{ijkl}=\mu\delta_{ik}\delta_{jl}+\mu'\delta_{il}\delta_{jk}+\mu''\delta_{ij}\delta_{kl}\label{eq:navTensor8}
    \end{equation}
    where \(\mu\), \(\mu'\), \(\mu''\) are scalar coefficient and Since \(A_{ijkl}\) is symmetrical in \(i\) and \(j\) we require
    \[\mu'=\mu.\]
    It will be observed that \(A_{ijkl}\) is now symmetrical in the indices \(k\) and \(l\) also, and that as a consequence the term containing \(\omega\) drops out of \eqref{eq:navTensor8} giving,
    \[d_{ij}=2\mu e_{ij}+\mu''\Delta\delta_{ij} \]
    where \(\Delta\) denotes the rate of expansion,
    \[\Delta=\frac{\partial u_k}{\partial x_k}=e_{kk}=\vec{\nabla}\cdot\vec{u}  \]
    Recall that \(d_{ij}\) makes no contribution to the normal stress,
    \begin{align*}
        & d_{ii}=2\mu e_{ii}+\left( \mu''\delta_{ii} \right)\Delta=0\\
        \Rightarrow\;& (2\mu)\Delta+\left( \mu''\delta_{ii} \right)\Delta=0\\
        \Rightarrow\;& \left( 2\mu+\mu''\delta_{ii} \right)\Delta=0\\
        \Rightarrow\;& \left( 2\mu+3\mu''\right)\Delta=0\qquad[\delta_{ii}=\delta_{11}+\delta_{22}+\delta_{33}=3 ]\\
        \intertext{Since, \(\Delta\neq 0\)}
        \Rightarrow\;& 2\mu+3\mu''=0\\
        \Rightarrow\;& \mu''=-\frac{2}{3}\mu
    \end{align*}
    Thus
    \begin{equation}
        d_{ij}=2\mu \left( e_{ij}-\frac{1}{3}\Delta\delta_{ij} \right)\label{eq:navTensor10}
    \end{equation}
    Now from \eqref{eq:navTensor3} and \eqref{eq:navTensor10},
    \[\sigma_{ij}=-P\delta_{ij}+2\mu(e_{ij}-\frac{1}{3}\Delta \delta_{ij})\]
    \qed
\newline

    Let \(\vec{u}\) be the fluid velocity at time \(t\) at the position vector \(\vec{x}\), so that \(\vec{u}\) is a function of \(t\) and \(\vec{x}\).\\
    The components of \(\vec{u}\) are \(u_i=(u_1,u_2,u_3)\), so that each component is a function of \(t\) and \(x\):
    \[u(t,x_1,x_2,x_3,\dots)\quad\text{etc.}\]
    Consider a small volume \(v\) in which the velocity components do not vary significantly. The total momentum in this volume is given by
    \begin{equation}
        \int_V \rho \D v\cdot\vec{u}
        \label{eq:navTensor11}
    \end{equation}
    It can be shown that the rate of change of this quantity 
    \[\int \rho \frac{D}{D t}\vec{u}(t,\vec{x})\D v=\int \rho \D v \{\frac{\partial u}{\partial t}+(\vec{u}\cdot \vec{\nabla})\}\]
    where \(\vec{u}\cdot\vec{\nabla}=(u_1\frac{\partial}{\partial x_1}+u_2\frac{\partial}{\partial x_2}+u_3\frac{\partial}{\partial x_3})\). Which is simply the sum of the products of mass and acceleration for all the elements of the material volume \(V\), can be rewritten as,
    \[\int_V \rho \D v\cdot\frac{D u_i}{D t}=\frac{\partial u_i}{\partial t}+(\vec{u}\cdot\vec{\nabla})u_i\]
    A portion of fluid is acted on, in general by both volume and surface forces.

    We denote the vector resultant of the volume forces per unit mass of fluid, by \(\vec{F}\), so that the total volume force on the selected portion of fluid is
    \[\int F_i \rho \D v\]
    The \(i-\)th component of the surface on contact force exerted across a surface element of area \(\D s\) and normal \(\vec{n} \) may be represented as \(\sigma_{ij}n_j\,\D s\), where \(\sigma_{ij}\) is the stress tensor and the total surface force exerted on the selected portion of fluid by
    \begin{align}
        & \int \sigma_{ij}n_j\,\D s=\int \frac{\partial \sigma_{ij}}{\partial x_j}\D v\notag\\
        & (\text{Total force = Body force + surface force})\notag\\
       \Rightarrow\; & \rho \frac{D u_i}{D t}=\rho F_i+\frac{\partial \sigma_{ij}}{\partial x_j}\label{eq:navTensor12}
    \end{align}
    This is the equation of the motion for a fluid where the stress tensor \(\sigma_{ij}\) can be written as follows
    \[\sij =-P\dij+2\mu (e_{ij}-\frac{1}{3}\Delta \dij)\]
    substituting this into \eqref{eq:navTensor12}, the equation of motion we get,
    \[\rho \frac{D u_i}{D t}=\rho F_i-\frac{\partial P}{\partial x_i}+\frac{\partial}{\partial x_j}{2\mu(e_{ij}-\frac{1}{3}\Delta \delta_{ij})}\]
    \[e_{ij}=\frac{1}{2}\left( \frac{\partial u_i}{\partial x_j}+\frac{\partial u_j}{\partial x_i} \right)\quad \text{and  } \quad \Delta =e_{ij}\]
    \begin{align*}
        \therefore\,\frac{\partial \left( e_{ij}-\frac{1}{3}\Delta \dij \right)}{\partial x_j}&=\frac{1}{2}\left( \frac{\partial^2 u_i}{\partial x_j\,\partial x_j}+\frac{\partial^2 u_j}{\partial x_j\,\partial x_i} \right)-\left( \frac{1}{3}\cdot\frac{\partial \nabla}{\partial x_j}\dij \right)\\
        &=\frac{1}{2}\left( \frac{\partial^2 u_i}{\partial x_j\,\partial x_j}+\frac{\partial^2 u_j}{\partial x_j\,\partial x_i} \right)-\left( \frac{1}{3}\cdot\frac{\partial }{\partial x_j}\left( \frac{\partial u_j}{\partial x_j} \right)\dij \right)\\
        &=\frac{1}{2}\left( \frac{\partial^2 u_i}{\partial x_j\,\partial x_j} \right)+\frac{1}{2}\frac{\partial^2 u_j}{\partial x_j\,\partial x_i}-\frac{1}{3}\frac{\partial^2 u_i}{\partial x_j\,\partial x_i}\\
        &=\frac{1}{2}\left( \frac{\partial^2 u_i}{\partial x_j\,\partial x_j} \right)+\frac{1}{6}\frac{\partial}{\partial x_j}(\frac{\partial u_i}{\partial x_i})\\
        &=\frac{1}{2}\left( \frac{\partial^2 u_i}{\partial x_j\,\partial x_j} \right)+\frac{1}{6}\frac{\partial\,\nabla}{\partial x_j}\\
        &=\frac{1}{2}\left( \frac{\partial^2 u_i}{\partial x_j\,\partial x_j} \right)+\frac{1}{6}\frac{\partial}{\partial x_j}(\vec{\nabla}\cdot\vec{u})
    \end{align*}
    For incompressible fluid, \(\vec{\nabla}\cdot\vec{u}=0\),
    \begin{align*}
        \frac{\partial }{\partial x_j}\left( e_{ij}-\frac{1}{3}\Delta \dij \right)&=\frac{1}{2}\frac{\partial^2 u_i}{\partial x_j\,\partial x_j}\\
        &=\frac{1}{2} \frac{\partial^2 u_i}{\partial x_j^2}\\
        &=\frac{1}{2}\nabla^2u
    \end{align*}
    \[\therefore \, \rho \frac{D u_i}{D t}=\rho F_i-\frac{\partial P}{\partial x_i}+2\mu\cdot \frac{1}{2}\nabla^2u_i\]
    \begin{equation}
        \begin{aligned}
            \therefore \, \rho \frac{D u_i}{D t}&=\rho F_i-\frac{\partial P}{\partial x_i}+\mu\nabla^2u_i\\
            \frac{D u_i}{D x_i}&=0
        \end{aligned}
        \label{eq:navTensor13}
    \end{equation}
    \(\therefore\) \eqref{eq:navTensor13} is the Navier-Stokes equation in tensor form.\\
    We may rewrite \eqref{eq:navTensor13} in Lamb vector form,
    \[\frac{D}{D t}=\frac{\partial }{\partial t}+\vec{u}\cdot \vec{\nabla}\]
    \begin{align*}
        &\frac{\partial \vec{u}}{\partial t}+(\vec{u}\cdot \vec{\nabla})\cdot\vec{u}=\vec{F}-\frac{1}{\rho}\left( \frac{\partial P}{\partial x}\vec{i}+\frac{\partial P}{\partial y}\vec{j}+\frac{\partial P}{\partial z}\vec{k} \right)+\nu\nabla^2 \vec{u}\\
        \Rightarrow\;\;&\frac{\partial \vec{u}}{\partial t}+(\vec{u}\cdot \vec{\nabla})\cdot\vec{u}=\vec{F}-\frac{1}{\rho}\vec{\nabla}P+\nu\nabla^2 \vec{u}\\
        &\qquad\vec{\nabla}\cdot\vec{u}=0
    \end{align*}
\end{soln}
\begin{prob}
    What is boundary layer and also define boundary layer thickness, displacement thickness, momentum thickness and energy thickness.
\end{prob}
\begin{soln}
    \underline{Boundary Layer:} The transition from zero velocity at the boundary to the full magnitude at far away from the boundary takes place in a very thin layer is called boundary layer.\\

    \underline{The boundary layer thickness:} \((\delta)\)\\
    The boundary layer thickness is that distance from the wall, where the velocity differs by one percent from the external velocity. This is denoted by \(\delta\), from Blasius profile we obtain,
    \[\eta=y\sqrt{\frac{U_\infty}{\nu_x}} \approx 5.0\]
    Hence the boundary layer thickness on a flat plate becomes,
    \[\delta=\eta\sqrt{\frac{\nu_x }{U_\infty}}=5.0\sqrt{\frac{\nu_x }{U_\infty}}\]

    \underline{The displacement thickness \((\delta_1)\):}\\
    The displacement thickness \(\delta_1\) is that distance by which the potential field of flow is displaced outwards as a consequence of the decrease in velocity in the boundary layer.\\
    The decrease in volume flow due to the influence of friction is
    \[\int_0^\infty (U_\infty-u )\D y\]
    Thus we get, 
    \begin{align*}
        & U_\infty \delta_1=\int_0^\infty (U_\infty-u )\D y\\
        \Rightarrow\,& \delta_1=\int_0^\infty 1-\frac{u}{U_\infty } \D y
    \end{align*}

    \underline{The momentum thickness \((\delta_2)\) :}\\
    The loss of momentum in the boundary layer as compared with potential flow is given by
    \[\rho\int_{0}^{\infty}u(U_\infty-u )\D y\]
    So that a new thickness can be defined by,
    \begin{align*}
        &\rho U_\infty^2 \delta_2 =\rho\int_{0}^{\infty}u(U_\infty-u )\D y\\
        \Rightarrow\,&\delta_2 =\int_{0}^{\infty}\frac{u}{U_\infty}(1-\frac{u}{U_\infty})\D y
    \end{align*}
    Where \(\delta_2\) denotes the momentum thickness. The numerical evaluation for the plate at zero incidence gives,
    \[\delta_2=0.664\sqrt{\frac{\nu_x }{U_\infty}}\]
    
    \underline{The energy thickness \((\delta_3)\) :}\\
    The energy thickness \(\delta_3\) is defined as
    \begin{align*}
        &U_\infty^3 \delta_3 =\int_{0}^{\infty}u(U_\infty^2-u^2 )\D y\\
        \Rightarrow\,&\delta_3 =\int_{0}^{\infty}\frac{u}{U_\infty}(1-\frac{u^2}{U_\infty^2})\D y
    \end{align*}
    Here, \(\begin{aligned}[t]
        \delta&=\text{ boundary layer thickness}\\
        \delta_1&=\text{ displacement layer thickness}=\frac{1}{2}\delta\\
        \delta_2&=\text{ momentum layer thickness}=\frac{1}{6}\delta\\
        \delta_3&=\text{ energy layer thickness}=\frac{1}{4}\delta
    \end{aligned}\)
\end{soln}
\begin{prob}
    Derivation of Boundary Layer equation for two-dimensional flow along a flat place.\\
    Or, Deduce the Prandtl's boundary layer momentum equations for the steady two-dimensional flow stating all the assumption clearly.
\end{prob}
\begin{soln}
    In two-dimensional flow, let us consider a plate coinciding with \(x-\)direction, \(y-\)axis being perpendicular to it.

    For two-dimensional unsteady incompressible viscous flow, the Navier-Stokes equations are (without body force):
    \begin{align}
        \frac{\partial u}{\partial t }+u\frac{\partial u}{\partial x }+v\frac{\partial u}{\partial y }&=-\frac{1}{\rho}\frac{\partial P}{\partial x }+\nu \nabla^2 u \label{eq:prand1}\\
        \frac{\partial v}{\partial t }+u\frac{\partial v}{\partial x }+v\frac{\partial v}{\partial y }&=-\frac{1}{\rho}\frac{\partial P}{\partial y }+\nu \nabla^2 v \label{eq:prand2}\\
        \intertext{and the equation of continuity,}
        \frac{\partial u}{\partial x }+\frac{\partial v}{\partial y }&=0 \label{eq:prand3}
    \end{align}
    Now for non dimensionalize the above equations, let \(V \) be the characteristic velocity and \(L \) be the characteristic length of the body. The pressure is made dimensionless with \(\rho V^2\) and \(\frac{L }{V }\) refers characteristic time.
    
    Reynolds number \(R=\frac{\rho V L }{\mu}=\frac{VL }{\nu}\), which is assumed very large.
    
    Let the dimensionless quantities by,
    \[
        x'=\frac{x }{L },\,\,y'=\frac{y }{L },\,\, t'=\frac{t V }{L },\,\, p'=\frac{P }{\rho V^2},\,\, u'=\frac{u }{V },\,\,v'=\frac{v }{V }
    \]
    The equation \eqref{eq:prand1} becomes
    \begin{align*}
        & \frac{\partial (Vu')}{\partial \left( \frac{L t' }{V } \right)}+Vu'\frac{\partial \left( Vu' \right)}{\partial (Lx') }+Vv'\frac{\partial (Vu')}{\partial (Ly') }=-\frac{1}{\rho}\frac{\partial (\rho V^2 P')}{\partial (Lx') }+\nu\left[ \frac{\partial^2 (Vu')}{\partial \left( L x'\right)^2}+\frac{\partial^2 (Vu')}{\partial \left( L y'\right)^2}\right]\\
        \Rightarrow\;\;& \frac{V^2 }{L}\frac{\partial u'}{\partial  t'}+\frac{V^2 }{L}u'\frac{\partial u'}{\partial x' }+\frac{V^2 }{L}v'\frac{\partial u'}{\partial y' }=-\frac{V^2 }{L} \frac{\partial  P'}{\partial x' }+\nu\frac{V}{L^2}\left[ \frac{\partial^2 u'}{\partial  x'^2}+\frac{\partial^2 u'}{\partial y'^2}\right]\\
        \Rightarrow\;\;& \frac{\partial u'}{\partial  t'}+u'\frac{\partial u'}{\partial x' }+v'\frac{\partial u'}{\partial y' }=-\frac{\partial  P'}{\partial x' }+\frac{\nu}{L V }\left[ \frac{\partial^2 u'}{\partial  x'^2}+\frac{\partial^2 u'}{\partial y'^2}\right]
    \end{align*}
    Dropping the suffixes, we get
    \begin{equation}
        \frac{\partial u}{\partial  t}+u\frac{\partial u}{\partial x}+v\frac{\partial u}{\partial y}=-\frac{\partial  P}{\partial x}+\frac{1}{R}\left( \frac{\partial^2 u}{\partial  x^2}+\frac{\partial^2 u}{\partial y^2}\right) 
        \label{eq:prand4}
    \end{equation}
    Similarly from equation \eqref{eq:prand2}, we get
    \begin{equation}
        \frac{\partial v}{\partial  t}+u\frac{\partial v}{\partial x}+v\frac{\partial v}{\partial y}=-\frac{\partial  P}{\partial y}+\frac{1}{R}\left( \frac{\partial^2 v}{\partial  x^2}+\frac{\partial^2 v}{\partial y^2}\right) 
        \label{eq:prand5}
    \end{equation}
    and the equation of continuity is
    \begin{equation}
        \frac{\partial u}{\partial x }+\frac{\partial v}{\partial y }=0 \label{eq:prand6}
    \end{equation} 
    Let us consider the non dimensional boundary layer thickness is \(\delta\),\\
    where \(\displaystyle \delta=\frac{\text{Original boundary layer thickness }}{L }<<<1\)\\
    Let us assume that,
    \[\begin{aligned}[t]
        u&\sim 1\\
        v&\sim \delta
    \end{aligned}\,\,\begin{aligned}[t]
        x&\sim 1\\
        y&\sim \delta
    \end{aligned}\,\, p\sim 1,\,\, t\sim1\]
    \[\pardx{u}\sim1,\,\,\pardy{v}\sim 1,\,\,\frac{\partial u }{\partial t}\sim 1,\,\,\pardy{u}\sim\frac{1}{\delta},\,\,\pardx{v}\sim \delta,\,\,\frac{\partial v }{\partial t}\sim \delta\]
    \[\frac{\partial^2 u }{\partial x^2}\sim1,\,\,\frac{\partial^2 u }{\partial y^2}\sim \frac{1 }{\delta^2 },\,\,\frac{\partial^2 v }{\partial x^2}\sim \delta ,\,\,\frac{\partial^2 v }{\partial y^2}\sim\frac{1}{\delta}\]

    If these are introduced in equations \eqref{eq:prand4} and \eqref{eq:prand5}, it follows from equation \eqref{eq:prand3} that the viscous forces in boundary layer can become of the same order of magnitude as the inertia forces only if the Reynolds number is of the order \(\frac{1}{\delta^2}\) i.e., \(\frac{1}{R }\sim \delta^2\).\\
    The Prandtl's equation for unsteady flow is:\\
    Direction \(x \):
    % \begin{align*}
    %     \underset{\lower.8cm\hbox{1}}{\frac{\partial u}{\partial t }}+\underset{\lower.8cm\hbox{1}}{u}\,\underset{\lower.8cm\hbox{1}}{\frac{\partial u}{\partial x }}+\underset{\lower.8cm\hbox{\(\delta\)}}{v}\,\underset{\lower.8cm\hbox{\(\frac{1}{\delta}\)}}{\frac{\partial u}{\partial y }}&=-\frac{\partial P}{\partial x }+
    %     \underset{\lower.8cm\hbox{\(\delta^2\)}}{\frac{1}{R}}\left( \frac{\partial^2 u }{\partial x^2}+\frac{\partial^2 u }{\partial y^2}\right) \\
    %     \intertext{Direction \(y\):}
    %     \frac{\partial v}{\partial t }+u\frac{\partial v}{\partial x }+v\frac{\partial v}{\partial y }&=-\frac{\partial P}{\partial y }+
    %     \frac{1}{R}\left( \frac{\partial^2 u }{\partial x^2}+\frac{\partial^2 u }{\partial y^2}\right) \\
    %     \intertext{and the equation of continuity,}
    %     \frac{\partial u}{\partial x }+\frac{\partial v}{\partial y }&=0
    % \end{align*}
    \begin{align*}
        \frac{\partial u}{\partial t}+u\,\frac{\partial u}{\partial x }+v\,\frac{\partial u}{\partial y }&=-\frac{\partial P}{\partial x }+\frac{1}{R}\left( \frac{\partial^2 u }{\partial x^2}+\frac{\partial^2 u }{\partial y^2}\right) \\
        1\phantom{++}1\,\,\,\,1\phantom{++}\delta\,\,\,\,\frac{1}{\delta}&\phantom{=-\frac{\partial P}{\partial x }+\,\,\,}\delta^2\,\,\left( 1\phantom{+++}\frac{1}{\delta^2}\right) \\
        \intertext{Direction \(y\):}
        \frac{\partial v}{\partial t }+u\frac{\partial v}{\partial x }+v\frac{\partial v}{\partial y }&=-\frac{\partial P}{\partial y }+
        \frac{1}{R}\left( \frac{\partial^2 u }{\partial x^2}+\frac{\partial^2 u }{\partial y^2}\right) \\
        \delta\phantom{++}1\,\,\,\,\delta\phantom{++}\delta\,\,\,\,1&\phantom{=-\frac{\partial P}{\partial x }+\,\,\,}\delta^2\,\,\left( \delta\phantom{+++}\frac{1}{\delta}\right) \\
        \intertext{and the equation of continuity,}
        \frac{\partial u}{\partial x }+\frac{\partial v}{\partial y }&=0\\
        1\quad\quad 1\,&
    \end{align*}
    The boundary conditions are:\\
    \begin{align*}
        u&=v=0 &\text{ for }y&=0&\\
        u&=U(x,t) &\text{ for }y&=\infty&
    \end{align*}
    Now in the case of steady flow, pressure depends only upon \(x \). Hence \(\frac{\partial P }{\partial y }=0\). Again since the boundary layer thickness is very small, the transverse velocity component \(v \) is very small at the edge of the boundary layer. Simplified Navier-Stokes equation is known as Prandtl's boundary layer equation transform to the dimensionless form.\\
    Again we have,
    \begin{align*}
        {\frac{\partial u}{\partial t }}+{u}\,{\frac{\partial u}{\partial x }}+{v}\,\frac{\partial u}{\partial y }&=-\frac{1}{\rho}\frac{\partial P}{\partial x }+
        \nu  \frac{\partial^2 u }{\partial y^2}\quad[\text{ neglecting }\frac{\partial^2 u }{\partial x^2}]\\
        \frac{\partial u}{\partial x }+\frac{\partial v}{\partial y }&=0
    \end{align*}
    which is called the boundary layer equation in plane flow.\\
    In the case of steady flow, the boundary layer equations are:
    \begin{align*}
        {u}\,{\frac{\partial u}{\partial x }}+{v}\,\frac{\partial u}{\partial y }&=-\frac{1}{\rho}\frac{\partial P}{\partial x }+
        \nu  \frac{\partial^2 u }{\partial y^2}\\
        \frac{\partial u}{\partial x }+\frac{\partial v}{\partial y }&=0
    \end{align*}
    with the boundary conditions
    \begin{align*}
        u&=v=0 &\text{ for }y&=0&\\
        u&=U(x) &\text{ for }y&=\infty&
    \end{align*}
\end{soln}
\end{document}