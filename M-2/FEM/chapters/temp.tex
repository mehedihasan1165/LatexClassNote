Shape Function:
The basic idea of the FEM is a piecewise approximation, that is, the solution of a complicated problem is obtained by dividing the region of interest into small regions (finite element) and approximating the solution over each such region by a simple function. Thus a necessary and important step is to choosing a simple function for the solution in each element. The functions used to represent the behavior of the solution within an element are called interpolating functions or approximating functions, those are known as shape function for each element.
The general nth order 1D shape function 𝑁𝑖(𝜉) at node 𝑖 for 𝑖 = 1,2,3, … , (𝑛 + 1) is defined by
𝑁 (𝜉) =	(𝜉 − 𝜉1)(𝜉 − 𝜉2) … (𝜉 − 𝜉𝑖−1)(𝜉 − 𝜉𝑖+1) … (𝜉 − 𝜉𝑛)
 
𝑖	(𝜉𝑖 − 𝜉1)(𝜉𝑖 − 𝜉2) … (𝜉𝑖 − 𝜉𝑖−1)(𝜉𝑖 − 𝜉𝑖+1) … (𝜉𝑖 − 𝜉𝑛)
With the property,	𝑁 (𝜉 ) = {1	;  𝑖𝑓	𝑖 = j
 
𝑖	j
 
0 ; 𝑖𝑓 𝑖 ≠ j
 
Global Space & Local Space:


A




B	C
Global Space	Local Space




A







Global Space	Local Space
 
1.	If the domain in global space is [a, b] which is arbitrary then the domain in local space will be fixed and [-1, 1]
2.	Each shape function has unit value at one node and zero at the other nodes.
3.	Each shape function will be the polynomial of same degree as the original trial solution.


Properties of shape function:

1.	𝑁 (𝜉 ) = {1	;  𝑖𝑓	𝑖 = j
 
𝑖	j
 
0 ; 𝑖𝑓 𝑖 ≠ j
 
2.	∑ 𝑁𝑖(𝜉) = 1
3.	∑ 𝑑𝑁𝑖 = 0
𝑑£


Derivation of Shape function:
Shape function for one dimensional Linear element:


(1)	[e]	(2)

𝜉1 = −1	𝜉2 = 1


Consider the shape function for one dimensional linear element in terms of local variables,
𝑁𝑖(𝜉) = 𝑎 + 𝑏𝜉	… … … … … … . (1)

 
With
 

𝑁 (𝜉 ) = {1	𝑖𝑓	𝑖 = j
 

Now we can write (1) as,
 
𝑖	j
 
0	𝑖𝑓 𝑖 ≠ j
 
𝑁𝑖(𝜉j) = 𝑎 + 𝑏𝜉j	(2)
For 𝑖 = 1, the shape function at node (1) is as,

 


When, j = 1,
 
𝑁1(𝜉j) = 𝑎 + 𝑏𝜉j


𝑁1(𝜉1) = 𝑎 + 𝑏𝜉1
 
⟹ 1 = 𝑎 + 𝑏. (−1)

 


When, j = 2,
 
⟹ 𝑎 − 𝑏 = 1

𝑁1(𝜉2) = 𝑎 + 𝑏𝜉2
⟹ 0 = 𝑎 + 𝑏. 1
 
⟹ 𝑎 + 𝑏 = 0
Solving we get,	𝑎 = 1 ,  𝑏 = − 1
 
2

Hence the shape function for node (1) is,

𝑁1(𝜉) =

For 𝑖 = 2, the shape function at node (2) is as,
 
2



1
  (1 − 𝜉)
2
 

 


When, j = 1,





When, j = 2,
 
𝑁2(𝜉j) = 𝑎 + 𝑏𝜉j


𝑁2(𝜉1) = 𝑎 + 𝑏𝜉1
⟹ 0 = 𝑎 + 𝑏. (−1)
⟹ 𝑎 − 𝑏 = 0

𝑁2(𝜉2) = 𝑎 + 𝑏𝜉2
⟹ 1 = 𝑎 + 𝑏. 1
 
⟹ 𝑎 + 𝑏 = 1
Solving we get,	𝑎 = 𝑏 = 1
2
Hence the shape function for node (2) is,

 
𝑁2(𝜉) =
 
1
  (1 + 𝜉)
2
 
Thus the shape function for one dimensional linear element is

 
𝑁1(𝜉) =
 
1
  (1 − 𝜉)
2
 



Now,
 
𝑁2(𝜉) =
 
1
  (1 + 𝜉)
2

1	1
 
𝑁1(𝜉) + 𝑁2(𝜉)  =
 
  (1 − 𝜉) +   (1 + 𝜉)
2	2
= 1
 

 
And,	𝑑𝑁1
𝑑£
 
= − 1 ,
2
 
𝑑𝑁2 = 1,
𝑑£	2
 

 
𝑑𝑁1
∴
𝑑𝜉
 
𝑑𝑁2
+
𝑑𝜉
 
1	1
= −   +   = 0
2	2
 


Shape function for one dimensional Quadratic element:



(1)
	(2)
	(3)

𝜉1 = −1	[1]	𝜉2 = 0	[2]	𝜉3 = 1


Consider the Shape function for one dimensional Quadratic element in terms of local variables,
𝑁𝑖(𝜉) = 𝑎 + 𝑏𝜉 + 𝑐𝜉2	… … … … … … . (1)

 
With
 

𝑁 (𝜉 ) = {1	𝑖𝑓	𝑖 = j
 

Now we can write (1) as,
 
𝑖	j
 
0	𝑖𝑓 𝑖 ≠ j
 
𝑁𝑖(𝜉j) = 𝑎 + 𝑏𝜉j + 𝑐𝜉2	(2)
For 𝑖 = 1, the shape function at node (1) is as,
𝑁1(𝜉j) = 𝑎 + 𝑏𝜉j + 𝑐𝜉2

 
When, j = 1,
 

𝑁1(𝜉1) = 𝑎 + 𝑏𝜉1 + 𝑐𝜉2
⟹ 1 = 𝑎 + 𝑏. (−1) + 𝑐. (−1)2
 

When, j = 2,





When, j = 3,
 
⟹ 𝑎 − 𝑏 + 𝑐 = 1

𝑁1(𝜉2) = 𝑎 + 𝑏𝜉2 + 𝑐𝜉2
⟹ 0 = 𝑎 + 𝑏. 0 + 𝑐. 0
⟹ 𝑎 = 0

𝑁1(𝜉3) = 𝑎 + 𝑏𝜉3 + 𝑐𝜉2
⟹ 0 = 𝑎 + 𝑏. 1 + 𝑐. 1
 

 


Solving we get,
 


𝑎 = 0 , 𝑏 = −
 
⟹ 𝑎 + 𝑏 + 𝑐 = 0
1 ,  𝑐 = 1
2	2
 
Hence the shape function for node (1) is,

 
𝑁1(𝜉) =

For 𝑖 = 2, the shape function at node (2) is as,
 
1
  𝜉(𝜉 − 1)
2
 

 


When, j = 1,



When, j = 2,



When, j = 3,
 
𝑁2(𝜉j) = 𝑎 + 𝑏𝜉j + 𝑐𝜉2


𝑁2(𝜉1) = 𝑎 + 𝑏𝜉1 + 𝑐𝜉2
⟹ 𝑎 − 𝑏 + 𝑐 = 0

𝑁2(𝜉2) = 𝑎 + 𝑏𝜉2 + 𝑐𝜉2
⟹ 𝑎 = 1

𝑁2(𝜉3) = 𝑎 + 𝑏𝜉3 + 𝑐𝜉2
 
⟹ 𝑎 + 𝑏 + 𝑐 = 0
Solving we get,	𝑎 = 1 , 𝑏 = 0 ,  𝑐 = −1
Hence the shape function for node (2) is,
𝑁2(𝜉) = (1 − 𝜉2)
 
For 𝑖 = 3, the shape function at node (3) is as,
𝑁3(𝜉j) = 𝑎 + 𝑏𝜉j + 𝑐𝜉2

 
When, j = 1,



When, j = 2,



When, j = 3,
 

𝑁3(𝜉1) = 𝑎 + 𝑏𝜉1 + 𝑐𝜉2
⟹ 𝑎 − 𝑏 + 𝑐 = 0

𝑁3(𝜉2) = 𝑎 + 𝑏𝜉2 + 𝑐𝜉2
⟹ 𝑎 = 0

𝑁3(𝜉3) = 𝑎 + 𝑏𝜉3 + 𝑐𝜉2
 
⟹ 𝑎 + 𝑏 + 𝑐 = 1
Solving we get,	𝑎 = 0 ,  𝑏 = 1 ,  𝑐 = 1
2	2

Hence the shape function for node (3) is,

 
𝑁3(𝜉) =
 
1
  𝜉(𝜉 + 1)
2
 
Thus the shape function for one dimensional Quadratic element is

 
𝑁1(𝜉) =
 
1
  𝜉(𝜉 − 1)
2
 
𝑁2(𝜉) = (1 − 𝜉2)
1
 
𝑁3(𝜉) =
 
  𝜉(𝜉 + 1)
2
 


Shape function for one dimensional cubic element:
From Lagrange interpolating polynomial, we have the general nth order one dimensional shape function 𝑁𝑖(𝜉) at node 𝑖 for 𝑖 = 1,2,3, … , (𝑛 + 1) is defined by
𝑁 (𝜉) =	(𝜉 − 𝜉1)(𝜉 − 𝜉2) … (𝜉 − 𝜉𝑖−1)(𝜉 − 𝜉𝑖+1) … (𝜉 − 𝜉𝑛)
 
𝑖	(𝜉𝑖 − 𝜉1)(𝜉𝑖 − 𝜉2) … (𝜉𝑖 − 𝜉𝑖−1)(𝜉𝑖 − 𝜉𝑖+1) … (𝜉𝑖 − 𝜉𝑛)
 
Now using this formula we derive the shape function for cubic element.


(1)	(2)	(3)	(4)
 
 
𝜉1
 
= −1	[1]	𝜉2
 
= − 1
3
 
[2]	𝜉  =  
3
 
[3]	𝜉4 = 1
 


For 𝑖 = 1 , the shape function for node (1) is,
𝑁 (𝜉) =	(𝜉 − 𝜉2)(𝜉 − 𝜉3)(𝜉 − 𝜉4)
 
1	(𝜉1 − 𝜉2)(𝜉1 − 𝜉3) (𝜉1 − 𝜉4)
1	1
=	(𝜉 + 3) (𝜉 − 3) (𝜉 − 1)
1	1
(−1 + 3) (−1 − 3) (−1 − 1)

 
(𝜉2
=
 
1
− 9) (𝜉 − 1)
− 16
9
 
= − 1 (9𝜉2 − 1)(𝜉 − 1)
16
= − 1 (9𝜉3 − 9𝜉2 − 𝜉 + 1)
16
For 𝑖 = 2 , the shape function for node (2) is,
𝑁 (𝜉) =	(𝜉 − 𝜉1)(𝜉 − 𝜉3)(𝜉 − 𝜉4)
 
2	(𝜉2 − 𝜉1)(𝜉2 − 𝜉3) (𝜉2 − 𝜉4)

 
(𝜉 + 1)
=
 
1
(𝜉 − 3) (𝜉 − 1)
 
1	1	1	1
 
(− 3 + 1) (− 3 −
 
) (−	− 1)
3	3
 
(𝜉2 − 1)(3𝜉 − 1)
= 	3	
16
27
9(𝜉2 − 1)(3𝜉 − 1)
=
16
 
= 9 (3𝜉3 − 𝜉2 − 3𝜉 + 1)
16
For 𝑖 = 3 , the shape function for node (3) is,
𝑁 (𝜉) =	(𝜉 − 𝜉1)(𝜉 − 𝜉2)(𝜉 − 𝜉4)
 
3	(𝜉3 − 𝜉1)(𝜉3 − 𝜉2) (𝜉3 − 𝜉4)

 
(𝜉 + 1)
=
 
1
(𝜉 + 3) (𝜉 − 1)
 
1	1	1	1
(3 + 1) (3 + 3) (3 − 1)
(𝜉2 − 1)(3𝜉 + 1)
= 	3	
− 16
27
9(𝜉2 − 1)(3𝜉 + 1)
= −
16
= − 9 (3𝜉3 + 𝜉2 − 3𝜉 − 1)
16
For 𝑖 = 4 , the shape function for node (4) is,
𝑁 (𝜉) =	(𝜉 − 𝜉1)(𝜉 − 𝜉2)(𝜉 − 𝜉3)
4	(𝜉4 − 𝜉1)(𝜉4 − 𝜉2) (𝜉4 − 𝜉3)
1	1
= (𝜉 + 3) (𝜉 − 3) (𝜉 + 1)
1	1
(1 + 3) (1 − 3) (1 + 1)

 
(𝜉2
=
 
1
− 9) (𝜉 + 1)
16
9
 
= 1 (9𝜉2 − 1)(𝜉 + 1)
16
= 1 (9𝜉3 + 9𝜉2 − 𝜉 − 1)
16
Thus the shape function for one dimensional cubic element is

 
1
𝑁 (𝜉) = −
 
(9𝜉3 − 9𝜉2 − 𝜉 + 1)
 
 
1	16
 
9
𝑁 (𝜉) =
 
(3𝜉3 − 𝜉2 − 3𝜉 + 1)
 
 
2	16
 
9
𝑁 (𝜉) = −
 

(3𝜉3 + 𝜉2 − 3𝜉 − 1)
 
 
3	16
 
1
𝑁 (𝜉) =
 

(9𝜉3 + 9𝜉2 − 𝜉 − 1)
 

Prove that:
3
𝑖=1

3
𝑖=1

3
𝑖=1

(4) ∑3
 



𝑁𝑖(𝜉) = 1

𝜉𝑖𝑁𝑖(𝜉) = 𝜉

𝜉2𝑁𝑖(𝜉) = 𝜉2

𝑑𝑁𝑖 = 0
 
 
4	16
 
𝑖=1 𝑑£

Solution: the shape function for one dimensional Quadratic element is

 
𝑁1(𝜉) =
 
1
  𝜉(𝜉 − 1)
2
 
𝑁2(𝜉) = (1 − 𝜉2)
1
 
𝑁3(𝜉) =
 
  𝜉(𝜉 + 1)
2
 
3
𝑖=1
 
𝑁𝑖(𝜉) = 𝑁1(𝜉) + 𝑁2(𝜉) + 𝑁3(𝜉)
= 1 𝜉(𝜉 − 1) + (1 − 𝜉2) +
2
 


1
  𝜉(𝜉 + 1)
2
 
1
=   𝜉2
2
 
1
−   𝜉 + 1 − 𝜉2
2
= 1
 
1
+   𝜉2
2
 
1
+   𝜉 2
 

 
3
𝑖=1
 
𝜉𝑖𝑁𝑖(𝜉) = 𝜉1 𝑁1(𝜉) + 𝜉2𝑁2(𝜉) + 𝜉3𝑁3(𝜉)
 

 
1
= (−1).  
2
 
𝜉(
 
𝜉 − 1)
 
+ 0.
 
(1 − 𝜉2)
 
1
+ 1.  
2
 
𝜉(
 
𝜉 + 1)
 
1
= −   𝜉2
2
 
1
+   𝜉 + 2
 
1 𝜉2
2
 
1
+   𝜉 = 𝜉 2
 
(3)	∑3	𝜉2𝑁𝑖(𝜉) = 𝜉2 𝑁1(𝜉) + 𝜉2𝑁2(𝜉) + 𝜉2𝑁3(𝜉)
 
𝑖=1  𝑖
 
1	2
2 1  (	)
 
3
(	2)
 
2 1  (	)
 
= (−1)
 
.   𝜉
2
 
𝜉 − 1
 
+ 0.
 
1 − 𝜉
 
+ (1)
 
.   𝜉
2
 
𝜉 + 1
 
1
=   𝜉2
2
 
1
−   𝜉 + 2
 
1 𝜉2
2
 
1
+   𝜉 2
 
= 𝜉2
(4)

 
𝑁1(𝜉) =
 
1
  𝜉(𝜉 − 1)
2
 
⟹ 𝑑𝑁1 = 1 (2𝜉 − 1)
𝑑𝜉	2
1
= 𝜉 −  
2
𝑁2(𝜉) = (1 − 𝜉2)
𝑑𝑁2
 
⟹
𝑑𝜉
 
= −2𝜉
 

 
𝑁3(𝜉) =
 
1
  𝜉(𝜉 + 1)
2
 
⟹ 𝑑𝑁3 = 1 (2𝜉 + 1)
𝑑𝜉	2
1
= 𝜉 +  
2
 
3
∴ ∑
𝑖=1
 
𝑑𝑁𝑖

 
𝑑𝜉
 
𝑑𝑁1
=
𝑑𝜉
 
𝑑𝑁2
+
𝑑𝜉
 
𝑑𝑁3
+
𝑑𝜉
 

 
= 𝜉 −
 
1 − 2𝜉 + 𝜉 + 1
2	2
 
= 0
 
Problem:
If L is the length of the linear element then evaluate [𝐾] , {𝐹} 𝑎𝑛𝑑 {𝐵} where,

𝐾	= ∫	𝑑𝑁𝑖 𝑑𝑁j 𝑑𝑥
 
𝑖j
 
[𝐿] 𝑑𝑥
 
𝑑𝑥
 


 



Solution:
 
𝐹𝑖 = ∫ 𝑁𝑖 (𝑥)𝑑𝑥 = 𝐵𝑖
[𝐿]
 
The linear shape function are:

 
𝑁1(𝜉) =

𝑁2(𝜉) =
 
1
  (1 − 𝜉)
2
1
  (1 + 𝜉)
2
 

 
Then,	𝑑𝑁1
𝑑£

Now the transformation
 
= − 1 ,
2
 
𝑑𝑁2 = 1
𝑑£	2
 
𝑥 = 𝑥1𝑁1(𝜉) + 𝑥2𝑁2(𝜉)
1	1
= 𝑥1. 2 (1 − 𝜉) + 𝑥2. 2 (1 + 𝜉)
= (𝑥1+𝑥3) + (𝑥3−𝑥1)𝜉
 
2

⟹ 𝑑𝑥 = 𝑥2−𝑥1 = 𝐿
 
2

, where, 𝐿 = (𝑥
 

− 𝑥
 

) is the length of the element.
 
𝑑£	2	2	2	1

 
Now,	𝐾
 
= ∫	𝑑𝑁𝑖 𝑑𝑁j 𝑑𝑥
 
𝑖j
 
[𝐿] 𝑑𝑥  𝑑𝑥
 

 
= ∫
[𝐿]
 
𝑑𝑁𝑖

 
𝑑𝜉
 
𝑑𝜉
.
𝑑𝑥
 
𝑑𝑁j
.
𝑑𝜉
 
𝑑𝜉
.
𝑑𝑥
 
𝐿
.   . 𝑑𝜉 2
 

 
= ∫
[𝐿]
 
𝑑𝑁𝑖

 
𝑑𝜉
 
2 𝑑𝑁j
.   .
𝐿  𝑑𝜉
 
2 𝐿
.   .  
𝐿 2
 
. 𝑑𝜉
 
2
=   ∫
𝐿
[𝐿]
 
𝑑𝑁𝑖

 
𝑑𝜉
 
𝑑𝑁j
.
𝑑𝜉
 
. 𝑑𝜉
 


 
Now,	𝐾
 
= 2 ∫1
 
𝑑𝑁1 𝑑𝑁1 𝑑𝜉
 
11	𝐿
 
−1 𝑑£
 
 
𝑑£
 

 
1
2
=   ∫ (−
𝐿
−1
 
1
 ) . (− 2
 
1
)𝑑𝜉 2
 

 
2 1
=   (
𝐿  4

2
 
1
) ∫ 𝑑𝜉
−1
1
 
=   .   . 2
𝐿 4
1
=  
𝐿
 
𝐾	= 2 ∫1
 

𝑑𝑁1 𝑑𝑁2 𝑑𝜉
 
12	𝐿
 
−1 𝑑£
 
 
𝑑£
 

 
1
2	1	1
=   ∫ (−  ) .  
𝐿	2	2
−1
 

𝑑𝜉
 
1
2	1
 
=   (−
𝐿
 
) ∫ 𝑑𝜉 4
 

2
=   . (−
𝐿
1
 
−1
1
).2
4
 


𝐾	= 2 ∫1
 
= − 𝐿 = 𝐾21
𝑑𝑁2 𝑑𝑁2 𝑑𝜉
 
22	𝐿
 
−1 𝑑£
 
 
𝑑£
 

 
1
2	1
=   ∫   .
𝐿	2
−1
2 1
 
1
  𝑑𝜉 2

1
 
=   .  
𝐿 4
 
. 2 =  
𝐿
 
Therefore,	[𝐾]
 
= 1 [
 
1	−1
]
 
𝐿 −1	1


Now,	𝐹𝑖 = ∫[𝐿] 𝑁𝑖(𝑥)𝑑𝑥

1
𝐿
= 2 ∫ 𝑁𝑖(𝜉)𝑑𝜉
−1
1
𝐿
∴ 𝐹1 = 2 ∫ 𝑁1(𝜉)𝑑𝜉
−1
1
𝐿	1
⟹ 𝐹1 = 2 ∫ 2 (1 − 𝜉)𝑑𝜉
−1
 
𝐿	2
⟹ 𝐹1 =   (
 

− 0)
 
2 0 + 1
𝐿	𝐿
=   . 2 =  
4	2
1
𝐿
∴ 𝐹2 = 2 ∫ 𝑁2(𝜉)𝑑𝜉
−1
1
𝐿	1
⟹ 𝐹2 = 2 ∫ 2 (1 + 𝜉)𝑑𝜉
−1
 
𝐿	2
⟹ 𝐹2 =   (
 

+ 0)
 
2 0 + 1
𝐿	𝐿
 


Therefore,	{𝐹} =  𝐿 {1	1}𝑇
2
 
=   . 2 =  
4	2
 
𝐿
= {2} = {𝐵}
𝐿
2
 
Important lemma for integration:
1
∫ 𝜉𝑚𝑑𝜉 = {
−1
 



2

 
𝑚 + 1
 


0 ;	𝑖𝑓 𝑚 𝑖𝑠 𝑜𝑑𝑑
;	𝑖𝑓 𝑚 𝑖𝑠 0 𝑜𝑟 𝑒𝑣𝑒𝑛
 




Problem:
If L is the length of the quadratic element then evaluate [𝐾] , {𝐹} 𝑎𝑛𝑑 {𝐵} where,

𝐾	= ∫	𝑑𝑁𝑖 𝑑𝑁j 𝑑𝑥
 
𝑖j
 
[𝐿] 𝑑𝑥
 
𝑑𝑥
 

𝐹𝑖 = ∫ 𝑁𝑖 (𝑥)𝑑𝑥
[𝐿]


 



Solution:
 
𝐵𝑖j = ∫ 𝑁𝑖 (𝑥). 𝑁j(𝑥)𝑑𝑥
[𝐿]
 
The shape function for one dimensional Quadratic element is

 
𝑁1(𝜉) =
 
1
  𝜉(𝜉 − 1)
2
 
𝑁2(𝜉) = (1 − 𝜉2)
1
 
𝑁3(𝜉) =
 
  𝜉(𝜉 + 1)
2
 
∴ 𝑑𝑁1 = 1 (2𝜉 − 1)
 
𝑑𝜉
𝑑𝑁2
∴
𝑑𝜉
 
2

= −2𝜉
 
∴ 𝑑𝑁3 = 1 (2𝜉 + 1)
𝑑𝜉	2
 
Now consider the transformation,

3
𝑥 = ∑ 𝑥𝑖𝑁𝑖(𝜉)
𝑖=1
⟹ 𝑥 = 𝑥1𝑁1(𝜉) + 𝑥2𝑁2(𝜉) + 𝑥3𝑁3(𝜉)
𝑥1 + 𝑥3
⟹ 𝑥 = 𝑥 𝑁 (𝜉) +	. 𝑁 (𝜉) + 𝑥 𝑁 (𝜉)
1  1	2	2	3  3

1	1
⟹ 𝑥 = 𝑥1𝑁1 + 2 𝑥1𝑁2 + 2 𝑥3𝑁2 + 𝑥3𝑁3

 
1
⟹ 𝑥 = 𝑥  
 
1	2	1	2	1
 
1 2 𝜉(𝜉 − 1) + 2 𝑥1(1 − 𝜉
 
) + 2 𝑥3(1 − 𝜉
 
) + 𝑥3 2 𝜉(𝜉 + 1)
 

 
1	2
⟹ 𝑥 = 2 𝑥1(𝜉
 
− 𝜉 + 1 − 𝜉

1
 
2	1	2
) + 2 𝑥3(1 − 𝜉
1
 
+ 𝜉2
 
+ 𝜉)
 
⟹ 𝑥 = 2 𝑥1(1 − 𝜉) + 2 𝑥3(1 + 𝜉)
 

⟹ 𝑥 =
 
𝑥1 + 𝑥3 2
 
𝑥3 − 𝑥1
+ (
2
 

) 𝜉
 
∴ 𝑑𝑥 = (𝑥3−𝑥1) = 𝐿
 
; L is the length of the element
 
𝑑£	2	2


 
Now,	𝐾
 
= ∫	𝑑𝑁𝑖 𝑑𝑁j 𝑑𝑥
 
𝑖j
 
[𝐿] 𝑑𝑥  𝑑𝑥
 

 
= ∫
[𝐿]
 
𝑑𝑁𝑖

 
𝑑𝜉
 
𝑑𝜉
.
𝑑𝑥
 
𝑑𝑁j
.
𝑑𝜉
 
𝑑𝜉
.
𝑑𝑥
 
𝐿
.   . 𝑑𝜉 2
 

 
= ∫
[𝐿]
 
𝑑𝑁𝑖

 
𝑑𝜉
 
2 𝑑𝑁j
.   .
𝐿  𝑑𝜉
 
2 𝐿
.   .  
𝐿 2
 
. 𝑑𝜉
 

 
1
2	𝑑𝑁𝑖
=   ∫
𝐿	𝑑𝜉
−1
 
𝑑𝑁j
.
𝑑𝜉
 

. 𝑑𝜉
 

 
𝐾	= 2 ∫1
 
𝑑𝑁1 𝑑𝑁1 𝑑𝜉
 
11	𝐿
 
−1 𝑑£
 
 
𝑑£
 
1
2	1
=   ∫  
𝐿	2
−1
 
1
(2𝜉 − 1).  
2
 

(2𝜉 − 1)𝑑𝜉
 
1
= 2 1	2
 
  ( ) ∫(4𝜉
𝐿  4
 
− 4𝜉 + 1)𝑑𝜉
 
−1
1	2
=	[4.
 

2
− 0 +	]
 
2𝐿	2 + 1
 
0 + 1
 

 
1	8
=	( 
2𝐿  3
 
+ 2) =
 
7

 
3𝐿
 


 
𝐾	= 2 ∫1
 
𝑑𝑁1 𝑑𝑁2 𝑑𝜉
 
12	𝐿
 
−1 𝑑£
 
 
𝑑£
 

 
1
2	1
=   ∫  
𝐿	2
−1
 

(2𝜉 − 1). (−2𝜉)𝑑𝜉
 
1
= 2 ∫(−1)(2𝜉2 − 𝜉)𝑑𝜉
𝐿
 
−1
2
= −   [2.
𝐿
8
 

2

 
2 + 1
 

− 0]
 
= − 3𝐿 = 𝐾21


 
𝐾	= 2 ∫1
 
𝑑𝑁1 𝑑𝑁3 𝑑𝜉
 
13	𝐿
 
−1 𝑑£
 
 
𝑑£
 

 
1
2	1
=   ∫  
𝐿	2
−1
 
1
(2𝜉 − 1).  
2
 

(2𝜉 + 1)𝑑𝜉
 
1
= 2 . 1 ∫(4𝜉2 − 1)𝑑𝜉
𝐿 4
−1
1	2	2
 
=
2𝐿
 
[4.
 

 
2 + 1
 
−	]
0 + 1
 
1  8
=	[ 
2𝐿 3
 
− 2]
 

 
1  2
=	.   =
2𝐿 3
 
1

 
3𝐿
 
= 𝐾31
 


 
𝐾	= 2 ∫1
 
𝑑𝑁2 𝑑𝑁2 𝑑𝜉
 
22	𝐿
 
−1 𝑑£
 
 
𝑑£
 

1
2
=   ∫(−2𝜉). (−2𝜉)𝑑𝜉
𝐿
−1
1
= 2 . 4 ∫ 𝜉2𝑑𝜉
𝐿
 
−1
8	2
=   [	] =
𝐿 2 + 1
 

16

 
3𝐿
 


 
𝐾	= 2 ∫1
 
𝑑𝑁2 𝑑𝑁3 𝑑𝜉
 
23	𝐿
 
−1 𝑑£
 
 
𝑑£
 

 
1
2	1
=   ∫  
𝐿	2
−1
 

(2𝜉 + 1). (−2𝜉)𝑑𝜉
 
1
= 2 ∫(−1)(2𝜉2 + 𝜉)𝑑𝜉
𝐿
 
−1
2
= −   [2.
𝐿
8
 

2

 
2 + 1
 

+ 0]
 
= − 3𝐿 = 𝐾32


 
𝐾	= 2 ∫1
 
𝑑𝑁3 𝑑𝑁3 𝑑𝜉
 
33	𝐿
 
−1 𝑑£
 
 
𝑑£
 

 
1
2	1
=   ∫  
𝐿	2
−1
 
1
(2𝜉 + 1).  
2
 

(2𝜉 + 1)𝑑𝜉
 
1
= 2 1	2
 
  ( ) ∫(4𝜉
𝐿  4
 
+ 4𝜉 + 1)𝑑𝜉
 
−1
1	2
=	[4.
 

2
+ 0 +	]
 
2𝐿	2 + 1
 
0 + 1
 

 
1	8
=	( 
2𝐿  3
 
+ 2) =
 
7

 
3𝐿
 


 

Therefore,	[𝐾] = 1 3𝐿
 
7	−8	1
[−8	16	−8]
1	−8	7
 


Again,

𝐹𝑖 = ∫ 𝑁𝑖 (𝑥)𝑑𝑥
[𝐿]

1
𝑑𝑥
⟹ 𝐹𝑖 = ∫ 𝑁𝑖 (𝜉) 𝑑𝜉 𝑑𝜉
−1
1
𝐿
⟹ 𝐹𝑖 = 2 ∫ 𝑁𝑖 (𝜉)𝑑𝜉
−1


 
Now,	𝐹 = 𝐿 ∫1 𝑁
 
(𝜉)𝑑𝜉
 
1	2 −1  1

 
1
𝐿	1
=   ∫  
2	2
−1
𝐿	2
=   [
 

𝜉(𝜉 − 1) 𝑑𝜉


− 0]
 
4 2 + 1
𝐿
=  
6
 
1
𝐿
𝐹2 = 2 ∫ 𝑁2 (𝜉)𝑑𝜉
−1
1
= 𝐿 ∫(1 − 𝜉2) 𝑑𝜉 2
−1
𝐿	2	2
=   [	−	]
4 0 + 1	2 + 1
𝐿	2
=   (2 −  )
2	3
 
𝐿 4
=   .  
2 3
 
2𝐿
=
3
 


 
,	𝐹 = 𝐿 ∫1 𝑁
 
(𝜉)𝑑𝜉
 
3	2 −1  3

 
1
𝐿	1
=   ∫  
2	2
−1
𝐿	2
=   [
 

𝜉(𝜉 + 1) 𝑑𝜉


+ 0]
 
4 2 + 1
𝐿
=  
6
Thus,

 

{𝐹} =
 
⎛ 𝐿⁄6 ⎞
2𝐿⁄
 

𝐿	1
=   . {4}
 
⎨ 𝐿  3⎫	6	1
⎝ ⁄6 ⎠



𝐵𝑖j = ∫ 𝑁𝑖 (𝑥). 𝑁j(𝑥)𝑑𝑥
[𝐿]

1
𝑑𝑥
= ∫ 𝑁𝑖 (𝜉). 𝑁j(𝜉) 𝑑𝜉 𝑑𝜉
−1
 
1
𝐿
= 2 ∫ 𝑁𝑖 (𝜉). 𝑁j(𝜉)𝑑𝜉
−1
1
𝐿
∴ 𝐵11 = 2 ∫ 𝑁1 (𝜉). 𝑁1(𝜉)𝑑𝜉
−1
 
1
𝐿	1
=   ∫  
2	2
−1
 

1
𝜉(𝜉 − 1).  
2
 

𝜉(𝜉 − 1) 𝑑𝜉
 

 
𝐿 1
=   .  
2 4
 
1
∫(𝜉4
−1
 

− 2𝜉3
 

+ 𝜉
 

2) 𝑑𝜉
 

 
𝐿	2
=   [
 
2
− 0 +	]
 
8 4 + 1
𝐿 2	2
=   [  +  ] =
8 5	3
 
2 + 1
𝐿   16		2   .	=
8 15	15
 
1
𝐿
∴ 𝐵12 = 2 ∫ 𝑁1 (𝜉). 𝑁2(𝜉)𝑑𝜉
−1
 
1
𝐿	1
=   ∫  
2	2
−1
 

𝜉(
 

𝜉 − 1). (1 − 𝜉2
 

) 𝑑𝜉
 
1
= 𝐿 ∫(−𝜉4 + 𝜉3 + 𝜉2 − 𝜉) 𝑑𝜉 4
−1

 
𝐿	2
=   [
4 2 + 1
 
− 0 −
 
2

 
4 + 1
 
+ 0]
 
𝐿 2	2	𝐿
=   [  −  ] =
4 3	5	15
1
𝐿
∴ 𝐵13 = 2 ∫ 𝑁1 (𝜉). 𝑁3(𝜉)𝑑𝜉
−1
 
1
𝐿	1
=   ∫  
2	2
−1
 

1
𝜉(𝜉 − 1).  
2
 

𝜉(𝜉 + 1) 𝑑𝜉
 
1
= 𝐿 ∫(𝜉4 − 𝜉2) 𝑑𝜉 8
−1

 
𝐿	2
=   [
8 4 + 1
 
2
−	]
2 + 1
 

 



Thus,
 
𝐿 2
=   [ 
8 5
 
2	𝐿
−  ] = −
3	30
 

 

{𝐵} =
 
⎛ 2𝐿⁄15 ⎞	2
𝐿⁄	=	𝐿 . {	1	}
 
⎨	15 ⎫
−𝐿
 
 
15	− 1⁄
 
⎝	⁄30⎠	2
 
# Shape function for the linear unit triangular element:


Y
(-1,1) (3)




(1)
(-1, -1)	(1, -1)


1 𝑖𝑓 𝑖 = j
We know that, 𝑁 (𝜉 , 𝜉 ) = {
0 𝑖𝑓 𝑖 ≠ j
For linear let the shape function be ,
𝑁𝑖(𝜉, 𝜂) = 𝑎 + 𝑏𝜉 + 𝑐𝜂 ; 𝑖 = 1,2,3
At node 1:
For i=1;	𝑁1(𝜉j, 𝜂j )=a+ b𝜉j + 𝜂j
When, j=1;	𝑁1(𝜉1, 𝜂1 )=a+ b𝜉1 + 𝜂1
⟹ 1 = 𝑎 − 𝑏 − 𝑐
∴ 𝑎 − 𝑏 − 𝑐 = 1
When, j=2;	𝑁1(𝜉2, 𝜂2 )=a+ b𝜉2 + 𝜂2
⟹ 0 = 𝑎 + 𝑏 − 𝑐
∴ 𝑎 + 𝑏 − 𝑐 = 0
When, j=3;	𝑁1(𝜉3, 𝜂3 )=a+ b𝜉3 + 𝜂3
⟹ 0 = 𝑎 − 𝑏 + 𝑐
∴ 𝑎 − 𝑏 + 𝑐 = 0

 
Solving we get, a=0, b=−
 
1	1
 , c=−  
2	2
 
∴ 𝑁 (𝜉, 𝜂)=− 1 (𝜉 + 𝜂).
1	2




At node 2:
For i=2;	𝑁2(𝜉j, 𝜂j )=a+ b𝜉j + 𝜂j
When, j=1;	𝑁2(𝜉1, 𝜂1 )=a+ b𝜉1 + 𝜂1
⟹ 0 = 𝑎 − 𝑏 − 𝑐
∴ 𝑎 − 𝑏 − 𝑐 = 0
When, j=2;	𝑁2(𝜉2, 𝜂2 )=a+ b𝜉2 + 𝜂2
⟹ 1 = 𝑎 + 𝑏 − 𝑐
∴ 𝑎 + 𝑏 − 𝑐 = 1
When, j=3;	𝑁2(𝜉3, 𝜂3 )=a+ b𝜉3 + 𝜂3
⟹ 0 = 𝑎 − 𝑏 + 𝑐
∴ 𝑎 − 𝑏 + 𝑐 = 0

1	1
Solving we get, a= , b= , c=0
2	2
∴ 𝑁 (𝜉, 𝜂)=1 (1 + 𝜉).
2	2


At node 3:
For i=1;	𝑁3(𝜉j, 𝜂j )=a+ b𝜉j + 𝜂j
When, j=1;	𝑁3(𝜉1, 𝜂1 )=a+ b𝜉1 + 𝜂1
⟹ 0 = 𝑎 − 𝑏 − 𝑐
∴ 𝑎 − 𝑏 − 𝑐 = 0
When, j=2;	𝑁3(𝜉2, 𝜂2 )=a+ b𝜉2 + 𝜂2
⟹ 0 = 𝑎 + 𝑏 − 𝑐
 
∴ 𝑎 + 𝑏 − 𝑐 = 0
When, j=3;	𝑁3(𝜉3, 𝜂3 )=a+ b𝜉3 + 𝜂3
⟹ 1 = 𝑎 − 𝑏 + 𝑐
∴ 𝑎 − 𝑏 + 𝑐 = 1

1	1
Solving we get, a= , b=0, c= 
2	2
∴ 𝑁 (𝜉, 𝜂)=1 (1 + 𝜂).
3	2
Hence, the shape function for linear unit tri-angular elements are:
∴ 𝑁 (𝜉, 𝜂)=− 1 (𝜉 + 𝜂)
1	2
∴ 𝑁 (𝜉, 𝜂)=1 (1 + 𝜉)
2	2
∴ 𝑁 (𝜉, 𝜂)=1 (1 + 𝜂)
3	2


η (0,1) (3)



(1)	(2)	ξ
(0,0)	(1,0)


# For this triangle the shape function are:
𝑁1(𝜉, 𝜂) = 1 − 𝜉 + 𝜂
𝑁2(𝜉, 𝜂) = 𝜉
𝑁3(𝜉, 𝜂) = 𝜂
Solution:
 
Shape function for node 1:
Node line passes through the (2) & (3),
∴ £−0 = 𝜂−1
 
0−1

⟹ £
−1
 
1−0

= 𝜂−1
1
 
⟹ 𝜉 = −(𝜂 − 1)
⟹ 𝜉 + 𝜂 − 1 = 0
⟹ 1 − 𝜉 − 𝜂 = 0
Now the shape function is defined as
𝑁1(𝜉, 𝜂) = 𝑐1(1 − 𝜉 − 𝜂)
At node 1, ξ=0, η=0
∴ 𝑁1(𝜉1, 𝜂1) = 𝑐1(1 − 𝜉 − 𝜂)
1== 𝑐1(1 − 0 − 0)
∴ 𝑐1 = 1
∴ 𝑁1(𝜉1, 𝜂1) = (1 − 𝜉 − 𝜂)
Shape function for node 2:
Node line passes through the (1) & (3),
∴ ξ=0
𝑁2(𝜉2, 𝜂2) = 𝑐2𝜉
⟹ 1 = 𝑐2
∴ 𝑐2 = 1
∴ 𝑁2(𝜉, 𝜂) = 𝜉
Shape function for node 3:
Node line passes through the (1) & (2),
∴ η=0
 
𝑁3(𝜉3, 𝜂3) = 𝑐3𝜂
⟹ 1 = 𝑐3
∴ 𝑐3 = 1
∴ 𝑁3(𝜉3, 𝜂3) = 𝜂




# Derivation of bi-linear shape function for the quadrilateral element in the local parametric shape(using geometry concept).
Solution:
y	(4)
(-1,1)




(1)
X	(-1,-1)	(1,-1)


For node 1:
Node line passes through the (2) & (3), ξ=1
⟹ 1 − 𝜉 = 0
And line passes through the (3) & (4), η=1
⟹ 1 − 𝜂 = 0
Now the shape function is defined by,
𝑁1(𝜉, 𝜂) = 𝑐(1 − 𝜉)(1 − 𝜂)
 
At node 1; ξ=-1, η=-1

 
Thus,
 

𝑁1(𝜉1, 𝜂1) = 𝑐(1 + 1)(1 + 1)
 

 
⟹ 1 = 4𝑐
⟹c=1
4
 




1
𝑁1(𝜉, 𝜂) = 4 (1 − 𝜉)(1 − 𝜂)
 


For node 2:
Node line passes through the (1) & (4), ξ=-1
⟹ 1 + 𝜉 = 0
And line passes through the (3) & (4), η=1
⟹ 1 − 𝜂 = 0
Now the shape function is defined by,
𝑁2(𝜉, 𝜂) = 𝑐(1 + 𝜉)(1 − 𝜂)
At node 2; ξ=1, η=-1

 
Thus,
 

𝑁2(𝜉2, 𝜂2) = 𝑐(1 + 1)(1 + 1)
 

 
⟹ 1 = 4𝑐
⟹c=1
4
 




1
𝑁2(𝜉, 𝜂) = 4 (1 + 𝜉)(1 − 𝜂)
 


For node 3:
 
Node line passes through the (1) & (2), η=-1
⟹ 1 + 𝜂 = 0
And line passes through the (1) & (4), ξ=-1
⟹ 1 + 𝜉 = 0
Now the shape function is defined by,
𝑁3(𝜉, 𝜂) = 𝑐(1 + 𝜉)(1 + 𝜂)
At node 3; ξ=1, η=1

 
Thus,
 

𝑁3(𝜉3, 𝜂3) = 𝑐(1 + 1)(1 + 1)
 

 
⟹ 1 = 4𝑐
⟹c=1
4
 




1
𝑁3(𝜉, 𝜂) = 4 (1 + 𝜉)(1 + 𝜂)
 


For node 4:
Node line passes through the (1) & (2), η=-1
⟹ 1 + 𝜂 = 0
And line passes through the (2) & (3), ξ=1
⟹ 1 − 𝜉 = 0
Now the shape function is defined by,
𝑁4(𝜉, 𝜂) = 𝑐(1 − 𝜉)(1 + 𝜂)
 
At node 4; ξ=-1, η=1

 
Thus,
 

𝑁4(𝜉4, 𝜂4) = 𝑐(1 + 1)(1 + 1)
 

 
⟹ 1 = 4𝑐
⟹c=1
4




Hence we get,
 




1
𝑁4(𝜉, 𝜂) = 4 (1 − 𝜉)(1 + 𝜂)
 

1
𝑁1(𝜉, 𝜂) = 4 (1 − 𝜉)(1 − 𝜂)
1
𝑁2(𝜉, 𝜂) = 4 (1 + 𝜉)(1 − 𝜂)
𝑁3(𝜉, 𝜂) = 𝑐(1 + 𝜉)(1 + 𝜂)
𝑁 (𝜉, 𝜂) = 1 (1 − 𝜉)(1 + 𝜂)
4	4

 
Generally,
 

1
𝑁𝑖(𝜉, 𝜂) = 4 (1 + 𝜉𝜉𝑖)(1 + 𝜂𝜂𝑖)
 
#Derivation of quadratic shape function for triangular element in (𝝃, 𝜼) plane:
Solution:
η
(3) (0,1)




(0,1/2)  (6)	(5)	(1/2,1/2)



(1)	(4)	(2)	
(0,0)	(1/2,0)	(1,0)	ξ


Shape function for node 1:
The straight line passes through the nodes (2), (3),&(5) is,

 
𝜉 − 0

 
0 − 1
 
𝜂 − 1
=
1 − 0
 
⟹ 𝜉 = −𝜂 + 1
⟹ 1 − 𝜉 − 𝜂 = 0


The straight line passes through the nodes (4),&(6) is,

 
𝜉 − 0

 
0 − 1
2
 
𝜂 − 1
=	2
1 − 0
2
 
1
⟹ 𝜉 = −𝜂 +  
2
⟹ 1 − 2𝜉 − 2𝜂 = 0
Now shape function at node 1 is defined by,
 
𝑁1(𝜉, 𝜂) = 𝑐(1 − 2𝜉 − 2𝜂)(1 − 𝜉 − 𝜂)
At node 1, 𝜉1 = 0, 𝜂1 = 0
∴ 𝑁1(𝜉1, 𝜂1) = 𝑐. 1.1
⟹ 1 = 𝑐
⟹ 𝑐 = 1
∴ 𝑁1(𝜉, 𝜂) = (1 − 2𝜉 − 2𝜂)(1 − 𝜉 − 𝜂)
Shape function for node 2:
The straight line passes through the nodes (1), (3),&(6) is,

 
𝜉 − 0

 
0 − 0
 
𝜂 − 0
=
0 − 1
 
⟹ 𝜉 = 0
The straight line passes through the nodes (4),&(5) is,

1
ξ=
2
⟹ 2𝜉 − 1 = 0
Now shape function at node 2 is defined by,
𝑁2(𝜉, 𝜂) = 𝑐𝜉(2𝜉 − 1)
At node 2, 𝜉2 = 1, 𝜂2 = 0
∴ 𝑁2(𝜉2, 𝜂2) = 𝑐. 1.1
⟹ 1 = 𝑐
⟹ 𝑐 = 1
∴ 𝑁2(𝜉, 𝜂) = 𝜉(2𝜉 − 1)
Shape function for node 3:
Since the point (2)&(3) are symmetric then,
∴ 𝑁3(𝜉, 𝜂) = 𝜂(2𝜂 − 1)
Shape function for node 4:
 
The straight line passes through the nodes (1), (3),&(6) is,
⟹ 𝜉 = 0
The straight line passes through the nodes (2),(3)&(5) is,
1-ξ-η=0
Now shape function at node 4 is defined by,
𝑁4(𝜉, 𝜂) = 𝑐𝜉( 1 − ( − η)

 
1
At node 4, 𝜉4 = 2 , 𝜂4 = 0
∴ 𝑁 (𝜉
 


, 𝜂 )	1	1
 
4	4	4
 
= 𝑐.   . (1 −   − 0)
2	2
𝑐
 
⟹ 1 =  
4
⟹ 𝑐 = 4
𝑁4(𝜉, 𝜂) = 4𝜉( 1 − ( − η)
Shape function for node 5:
The straight line passes through the nodes (1), (3),&(6) is,
⟹ 𝜉 = 0
The straight line passes through the nodes (1),(4)&(2) is,
η=0 Now shape function at node 5 is defined by,
𝑁5(𝜉, 𝜂) = 𝑐𝜉𝜂

1	1
At node 5, 𝜉5 = 2 , 𝜂5 = 2
∴ 𝑁 (𝜉 , 𝜂 )	1 1
 
5	5	5
 
= 𝑐.   .  
2 2
 

𝑐
⟹ 1 =  
4
 
⟹ 𝑐 = 4
∴ 𝑁5(𝜉, 𝜂) = 4𝜉𝜂
 
Shape function for node 6:
Since the point (4)&(6) are symmetric then,
∴ 𝑁6(𝜉, 𝜂) = 4𝜂(1 − 𝜉 − 𝜂)
#Derivation of quadratic shape function for 2 squares elements in (ξ, η) plane:
Solution:
η
(-1,1)	(0,1)	(1,1)
(4)	(7)	(3)


(-1,0) (8)	(0,0)	(6) (1,0)	ξ
(1)	(5)	(2)
(-1,-1)	(0,-1)	(1,-1)


Shape function for node 1:
The straight line passes through the nodes (8),&(5) is,

 
𝜉 + 1

 
−1 − 0
 
𝜂 − 0
=
0 + 1
 
⟹ 1 + 𝜉 + 𝜂 = 0
The straight line passes through the nodes (2),(3)&(6) is,
⟹ 𝜉 = 1
⟹ 1 − 𝜉 = 0
The straight line passes through the nodes (4), (3) & (7) is,
⟹ 𝜂 = 1
⟹ 1 − 𝜂 = 0
 
Now shape function at node 1 is defined by,
𝑁1(𝜉, 𝜂) = 𝑐(1 − 𝜉)(1 − 𝜂)(1 − 𝜉 − 𝜂)
At node 1, 𝜉1 = −1, 𝜂1 = −1
∴ 𝑁1(𝜉1, 𝜂1) = 𝑐. 2.2. (−1)
⟹ 1 = −4𝑐
1
⟹ 𝑐 = −  
4
1
∴ 𝑁1(𝜉, 𝜂) = − 4 (1 − 𝜉)(1 − 𝜂)(1 − 𝜉 − 𝜂)
Shape function for node 2:
The straight line passes through the nodes (6),&(5) is,

 
𝜉 − 0
=
0 − 1
 
𝜂 + 1

 
−1 − 0
 
⟹ 1 − 𝜉 + 𝜂 = 0


The straight line passes through the nodes (1),(8)&(4) is,
⟹ 𝜉 = −1
⟹ 1 + 𝜉 = 0
The straight line passes through the nodes (4),(3)&(7) is,
⟹ 𝜂 = 1
⟹ 1 − 𝜂 = 0
Now shape function at node 2 is defined by,
𝑁2(𝜉, 𝜂) = 𝑐(1 + 𝜉)(1 − 𝜂)(1 − 𝜉 + 𝜂)
At node 2, 𝜉2 = 1, 𝜂2 = −1
∴ 𝑁2(𝜉2, 𝜂2) = 𝑐. 2.2. (−1)
⟹ 1 = −4𝑐
 
1
⟹ 𝑐 = −  
4
1
∴ 𝑁2(𝜉, 𝜂) = − 4 (1 + 𝜉)(1 − 𝜂)(1 − 𝜉 + 𝜂)
Shape function for node 3:
Since the point (1)&(3) are symmetric then,
1
∴ 𝑁3(𝜉, 𝜂) = − 4 (1 + 𝜉)(1 + 𝜂)(1 − 𝜉 − 𝜂)
Shape function for node 4:
Since the point (2)&(4) are symmetric then,
1
∴ 𝑁4(𝜉, 𝜂) = − 4 (1 − 𝜉)(1 + 𝜂)(1 + 𝜉 − 𝜂)
Shape function for node 5:
The straight line passes through the nodes(2), (6),&(3) is,
⟹ 1 − 𝜉 = 0
The straight line passes through the nodes (1),(8)&(4) is,
⟹ 𝜉 = −1
⟹ 1 + 𝜉 = 0
The straight line passes through the nodes (4),(3)&(7) is,
⟹ 𝜂 = 1
⟹ 1 − 𝜂 = 0


Now shape function at node 5 is defined by,
𝑁5(𝜉, 𝜂) = 𝑐(1 − 𝜉2)(1 − 𝜂)

At node 5	⟹ 𝑐 = 1 2
∴ 𝑁 (𝜉, 𝜂) = 1 (1 − 𝜉2)(1 − 𝜂)
5	2
 
Shape function for node 6:
The straight line passes through the nodes(1), (5),&(2) is,
⟹ 1 + 𝜂 = 0
The straight line passes through the nodes (1),(8)&(4) is,
⟹ 𝜉 = −1
⟹ 1 + 𝜉 = 0
The straight line passes through the nodes (4),(3)&(7) is,
⟹ 𝜂 = 1
⟹ 1 − 𝜂 = 0


Now shape function at node 6 is defined by,
𝑁6(𝜉, 𝜂) = 𝑐(1 + 𝜉)(1 − 𝜂2)

At node 6	⟹ 𝑐 = 1 2
∴ 𝑁 (𝜉, 𝜂) = 1 (1 + 𝜉)(1 − 𝜂2)
6	2

Shape function for node 7:
Since the point (6)&(7) are symmetric then,
∴ 𝑁 (𝜉, 𝜂) = 1 (1 − 𝜉2)(1 + 𝜂)
7	2
Shape function for node 8:
Since the point (5) & (8) are symmetric then,
∴ 𝑁 (𝜉, 𝜂) = 1 (1 − 𝜉)(1 − 𝜂2)
 
8	2
 
Question -1: Construct a piecewise linear approximation and quadratic approximation to the
function sin(x) using the two element (0,1) and (1,2) and estimate the value of	2 sin(𝑥)𝑑𝑥 .
0
Solution : Linear approximation –



x=0
	x=1
	x=3

1	[1]	2	[2]	3


The linear shape functions are
1
𝑁1(𝜉) = 2 (1 − 𝜉)
1
𝑁2(𝜉) = 2 (1 + 𝜉)
For element [1] : Consider the transformation
𝑥 = 𝑥1𝑁1(𝜉) + 𝑥2𝑁2(𝜉)
1
 
= 0 + 1 *

1
 
  (1 + 𝜉)
2
 
=   (1 + 𝜉) 2
1
 
𝑑𝑥 =
 
  𝑑𝜉 2
 
Now the linear approximation of sin(𝑥) within (0,1) is
sin(𝑥) = 𝑢1𝑁1(𝜉) + 𝑢2𝑁2(𝜉)
1	1
= sin(𝑥1) 2 (1 − 𝜉) + sin(𝑥2) 2 (1 + 𝜉)
 
1
= sin(0)  
2
 
1
(1 − 𝜉) + sin(1)  
2
 

(1 + 𝜉)
 
= 0.4207(1 + 𝜉)
For element [2] : Consider the transformation
𝑥 = 𝑥1𝑁1(𝜉) + 𝑥2𝑁2(𝜉)
 
= 1 *
 
1
  (1 − 𝜉)2 * 2
 
1
  (1 + 𝜉)
2
 
1	1
=   −   𝜉 + 1 + 𝜉
2	2
3	1
  +   𝜉
2	2
1
 
𝑑𝑥 =
 
  𝑑𝜉 2
 
Now the linear approximation of sin(𝑥) within (1,2) is
sin(𝑥) = 𝑢1𝑁1(𝜉) + 𝑢2𝑁2(𝜉)
1	1
= sin(𝑥1) 2 (1 − 𝜉) + sin(𝑥2) 2 (1 + 𝜉)
 
1
= sin(1)  
2
 
1
(1 − 𝜉) + sin(2)  
2
 

(1 + 𝜉)
 

 


Therefore
 
= 0.8754 + 0.0339𝜉
 
∫2 sin(𝑥) 𝑑𝑥 = ∫1 sin(𝑥) 𝑑𝑥 + ∫2 sin(𝑥) 𝑑𝑥
0	0	1
1	1	1	1
 
= ∫  0.4207(1 + 𝜉)  
−1	2
 
𝑑𝜉 + ∫ (0.8754 + 0.0339𝜉)
−1
 
  𝑑𝜉 2
 

 
= 0.21035 [
 
2

 
0 + 1
 
+ 0] +
 
1
  [0.8754 *
2
 
2

 
0 + 1
 
+ 0]
 

 


Quadratic approximation :
The quadratic shape functions are
 
= 1.2961
 

 
𝑁1
 
(𝜉)
 
1
=   𝜉(𝜉 − 1)
2
 
𝑁2(𝜉) = 1 − 𝜉2
1
𝑁3(𝜉) = 2 𝜉(𝜉 + 1)
For element [1] : Consider the transformation
 
𝑥 = 𝑥1𝑁1(𝜉) + 𝑥2𝑁2(𝜉) + 𝑥3𝑁3(𝜉)

 
= 0 + 1 (1 − 𝜉2) +
2
1	1
 
1
  𝜉(𝜉 + 1)
2
 
=   +   𝜉
2	2
 
𝑑𝑥 =
 
1 𝑑𝜉
2
 
Now the quadratic approximation of sin(𝑥) within (0,1) is
sin(𝑥) = 𝑢1𝑁1(𝜉) + 𝑢2𝑁2(𝜉) + 𝑢3𝑁3(𝜉)

 
= sin(𝑥
 
1
)   𝜉(1 − 𝜉) + sin(𝑥
 
) (1 − 𝜉2) + sin(𝑥
 
1
)   𝜉(1 + 𝜉)
 
1  2	2	3  2

 
1	1
= sin(0)    (	)	 
 
1
2	( )   𝜉(1 + 𝜉)
 



Now
 
2	2	2
= 0 + 0.4794 + 0.4207𝜉 − 0.0587𝜉2


1	1	1
 
∫ sin(𝑥) 𝑑𝑥 = ∫ (0.4794 + 0.4207𝜉 − 0.0587𝜉2)
 
  𝑑𝜉
 
0	−1
1
  [0.4794 *
2
 

2

 
0 + 1
 


+ 0 − 0.0587 *
 
2
2
]
2 + 1
 
0.4598


For element [2] : Consider the transformation
𝑥 = 𝑥1𝑁1(𝜉) + 𝑥2𝑁2(𝜉) + 𝑥3𝑁3(𝜉)

 
= 1 * 1 𝜉(1 − 𝜉) + 3 * (1 − 𝜉2) + 2 * 1 𝜉(
 
1 + 𝜉)
 
2	2	2
1	3
=   𝜉 +  
2	2
1
 
𝑑𝑥 =
 
  𝑑𝜉 2
 
Now the quadratic approximation of sin(𝑥) within (1,2) is



x=0
	x=0.5
x=1 (1)
(2)
	(3)

(1)	[1]	(2)	(3)  x=1	x=1.5	[2]	x=2


sin(𝑥) = 𝑢1𝑁1(𝜉) + 𝑢2𝑁2(𝜉) + 𝑢3𝑁3(𝜉)

 
= sin(𝑥
 
1
)   𝜉(1 − 𝜉) + sin(𝑥
 
) (1 − 𝜉2) + sin(𝑥
 
1
)   𝜉(1 + 𝜉)
 
1  2	2	3  2

 
1	3
= sin(1)    (	)	 
 
1
2	( )   𝜉(1 + 𝜉)
 
2	2	2


 


Now ,
 
= 0.9975 + 0.0339𝜉 − 0.1221𝜉2

2	1	1
 
∫ sin(𝑥) 𝑑𝑥 = ∫ (0.9975 + 0.0339𝜉 − 0.1221𝜉2)
 
  𝑑𝜉
 
1	−1	2


 
1
  [0.9975 *
2
 
2

 
0 + 1
 
+ 0 − 0.1221 *
 
2
]
2 + 1
 
0.9568


Therefore
∫2 sin(𝑥) 𝑑𝑥 = ∫1 sin(𝑥) 𝑑𝑥 + ∫2 sin(𝑥) 𝑑𝑥
0	0	1
= 0.4598 + 0.9568

 


The analytic result is
 
= 1.4166
 

2
∫ sin(𝑥) 𝑑𝑥 = [−cos(𝑥)]2
0
 
= cos(0) − cos(2) 1.4162
S0 the quadratic approximation is very good approximation and which gives better result then linear approximation .
Gaussian quadrature integration over element :
In this method the formulae is given by ,
1	3
𝐼 = ∫ 𝑓(𝜉)𝑑𝜉 = ∑ 𝑤𝑖𝑓(𝜉𝑖)
−1	𝑖=1


Where 𝑓(𝜉𝑖) is the functional value of 𝑓(𝜉) at Gauss point 𝜉𝑖 and 𝑤𝑖 is the corresponding weight
.
An 𝑛-th order formulae contains 𝑛unknown , i.e 𝑛 values of 𝜉𝑖 and n values of 𝑤𝑖and consequently gives the exact result when applied to a polynomial of degree (2𝑛 − 1) .
Values of 𝜉𝑖 and 𝑤𝑖 for 𝑛 = 2,3,4,…	are given in the following table



Number of points, n	Values of 𝜉𝑖	Weighted factors, 𝑤𝑖
2	±0.577350	1.0
3	±0.774597
0.0	0.555556
0.888889
4	±0.861136
±0.338891	0.347855
0.652145
5	±0.906180
0
±0.538469	0.236927
0.568889
0.478629

Example - :

(a)	Determine	1 𝑠𝑖𝑛3(𝑥)𝑑𝑥 exactly by analytic technique .
0
(b)	Use Gaussian quadrature to determine an approximation to	1 𝑠𝑖𝑛3(𝑥)𝑑𝑥
0
(c)	Assuming that 0 ≤ 𝑥 ≤ 1 is single element express 𝑠𝑖𝑛3(𝑥) in terms of quadratic shape
 
functions and calculate	1 𝑠𝑖𝑛3(𝑥)𝑑𝑥
0
 
numerically .
 
Solution :  (a)
 


1	1
∫ 𝑠𝑖𝑛3(𝑥)𝑑𝑥 = ∫   (3 sin(𝑥) − sin(3𝑥))𝑑𝑥
 
0	0 4
1	1
=  	( )	 	1
 
[−3 cos 𝑥
4
1
 
+ 3 cos(3𝑥)]0
 
  * 0.7158
4

 


(b) The quadratic shape functions are
 
= 0.1789
 

 







Consider the transformation
 
1
𝑁1(𝜉) = 2 𝜉(𝜉 − 1)
𝑁2(𝜉) = 1 − 𝜉2
1
𝑁3(𝜉) = 2 𝜉(𝜉 + 1)
 
𝑥 = 𝑥1𝑁1(𝜉) + 𝑥2𝑁2(𝜉) + 𝑥3𝑁3(𝜉)

 
= 0 + 1 (1 − 𝜉2) +
2
1	1
 
1
  𝜉(𝜉 + 1)
2
 
=   +   𝜉
2	2
 


Now ,
∫1 𝑠𝑖𝑛3(𝑥)𝑑𝑥 = ∫1
 






1+£  1
𝑠𝑖𝑛 (	)   𝑑𝜉
 
𝑑𝑥 =
 
1 𝑑𝜉
2
 
0	−1	2	2
Using the three point Gaussian quadrature formula We can write
1	3
∫ 𝑓(𝜉)𝑑𝜉 = ∑ 𝑤𝑖𝑓(𝜉𝑖)
−1	𝑖=1
= 𝑤1𝑓(𝜉1) + 𝑤2𝑓(𝜉2) + 𝑤3𝑓(𝜉3)
 
Now
1	1	1 + 𝜉  1
 
∫ 𝑠𝑖𝑛3(𝑥)𝑑𝑥 = ∫  𝑠𝑖𝑛3 (
 
)   𝑑𝜉
 
0

= 1	3  1 + 𝜉1
 
−1	2	2
3  1 + 𝜉2	3  1 + 𝜉3
 
2 [𝑤1𝑠𝑖𝑛 (
 
2	) + 𝑤2𝑠𝑖𝑛 (	2	) + 𝑤3𝑠𝑖𝑛 (	2	)]
 

 


𝜉 = −0.774597 ,	𝑓(𝜉 )
 
…………………………………… (1)
1 − 0.774597
3	) = 0.001422
 
1	1  = 𝑠𝑖𝑛 (	2


 
𝜉2
 
= 0 , 𝑓(𝜉2)
 
= 𝑠𝑖𝑛3
 
1
( ) = 0.110195 2
 

 
𝜉3
 
= 0.774597 ,	𝑓(𝜉3
 
) = 𝑠𝑖𝑛3 (1 + 0.774597) = 0.46615
2
 
Using approximate weighting functions we get from (1) ;

 
1
∫ 𝑠𝑖𝑛3(𝑥)𝑑𝑥 =
0
 
1
  [(0.555556 * 0.001422) + (0.888889 * 0.110195) + (0.555556
2
 
* 0.46615)]
= 0.178858
(c)
The quadratic approximation of 𝑠𝑖𝑛3(𝑥) is

 
1+£
𝑠𝑖𝑛 (	) = 𝑢
 
𝑁 (𝜉) + 𝑢
 
𝑁 (𝜉) + 𝑢
 
𝑁 (𝜉)
 
2	1  1
 
2  2	3  3
 

 




Therefore
 
= 𝑠𝑖𝑛3(0) 1 𝜉(1 − 𝜉) + 𝑠𝑖𝑛3(0.5)(1 − 𝜉2) + 𝑠𝑖𝑛3(1) 1 𝜉(1 + 𝜉)
2	2
= 0.110195 + 0.29791𝜉 + 0.18772𝜉2


1	1	1
 
∫ 𝑠𝑖𝑛3(𝑥)𝑑𝑥 = ∫ [0.110195 + 0.29791𝜉 + 0.18772𝜉2]   𝑑𝜉
0	−1	2
= 1 [0.110195 *  2  + 0 − 0.18772 *  2 ]  = 0.17277
2	0+1	2+1
 
Question -: Evaluate	𝜋⁄2 𝑠𝑖𝑛2(𝑥)𝑑𝑥
0

(a)	By an exact analytic method .
(b)	Using four point Gaussian quadratic .
(c)	By dividing the interval 0 ≤ 𝑥 ≤ 𝜋⁄2 into two linear element .
(d)	By treating the interval 0 ≤ 𝑥 ≤ 𝜋⁄2 as one quadratic element .
Solution -:
(a)	∫𝜋⁄2 𝑠𝑖𝑛2(𝑥)𝑑𝑥 = 1 ∫𝜋⁄2(1 − cos(2𝑥))𝑑𝑥
0	2 0
1	1	𝜋⁄2
= 2 [𝑥 + 2 sin(2𝑥)]0
= 0.785398
(b)	We know that the transformation

𝑥 = 𝑥1+𝑥2 + 𝑥2−𝑥1 𝜉
 
2	2
0 + 𝜋⁄
=
2
𝜋⁄
=
2
 

𝜋⁄ − 0
+	𝜉
2
𝜋⁄
+	𝜉
2
 
= 𝜋⁄4 + 𝜋⁄4 𝜉
𝜋
 
𝑑𝑥 =
 
  𝑑𝜉 4
 
According to the four point Gaussian quadrature

 
𝜋⁄2
 
𝜋	1	𝜋
 
∫	𝑠𝑖𝑛2(𝑥)𝑑𝑥 =
0	4

𝜋
 
∫ 𝑠𝑖𝑛2 (
−1	4
1
 
(1 + 𝜉)) 𝑑𝜉
 


We have
 
=   ∫ 𝑓(𝜉)𝑑𝜉
4 −1
 
𝜉1 = 0.861136 ,	𝑓(𝜉1) = 0.988152
𝜉2 = −0.861136 ,	𝑓(𝜉2) = 0.01185
 
𝜉3 = 0.339981 ,	𝑓(𝜉3) = 0.75451
𝜉4 = −0.339981 ,	𝑓(𝜉4) = 0.24549
Using the appropriate weighting factor we get

 
𝜋⁄2
∫	𝑠𝑖𝑛2(𝑥)𝑑𝑥 =
0
 
𝜋	1
  ∫ 𝑓(𝜉)𝑑𝜉
4 −1
 
4
𝜋
= 4 ∑ 𝑤𝑖𝑓(𝜉𝑖)
𝑖=1
𝜋
= 4 [𝑤1𝑓(𝜉1) + 𝑤2𝑓(𝜉2) + 𝑤3𝑓(𝜉3) + 𝑤4𝑓(𝜉4)]
𝜋
=   [0.347855(0.99152 + 0.01185) + 0.652145(0.75451 + 0.24549)]
4
𝜋
 




(c)	For element [1] We know that
𝑥 = 𝑥1+𝑥2 + 𝑥2−𝑥1 𝜉
 
=   * 1
4
𝜋
=  
4
 
2	2
0 + 𝜋⁄
=
2
𝜋
=
8
 

𝜋⁄ − 0
+	𝜉
2
𝜋
+   𝜉 8
 



The linear transformation of 𝑠𝑖𝑛2(𝑥) in
 



(0,
 




𝜋) 4
 

𝑑𝑥 =

is
 
𝜋
  𝑑𝜉 8
 

 
𝑠𝑖𝑛2
 
𝜋

(  (1 + 𝜉)) = 𝑠𝑖𝑛2 8
 
1
(0)  
2
1
 
(1 − 𝜉) + 𝑠𝑖𝑛2
 
𝜋  1
( )  
4  2
 
(1 + 𝜉)
 


Therefore
 
=   (1 + 𝜉) 4
 
𝜋⁄4
 
𝜋	1	𝜋
 
∫	𝑠𝑖𝑛2(𝑥)𝑑𝑥 =
0
 
  ∫ 𝑠𝑖𝑛2 (
8 −1	8
 
(1 + 𝜉)) 𝑑𝜉
 


 
𝜋	1 1
=   ∫  
 
(1 + 𝜉)𝑑𝜉
 
8 −1 4


𝜋	2
=	[	]
32 0 + 1
𝜋
=
16


For element [2] We know that
𝑥 = 𝑥2+𝑥3 + 𝑥3−𝑥2 𝜉
2	2
 
𝜋⁄4 + 𝜋⁄2 2
 
𝜋⁄ − 𝜋⁄
𝜉
2
 
3𝜋
=
8
 
𝜋
+   𝜉 8
𝜋
 

2  𝜋
 

2  𝜋
 
𝑑𝑥 =

1
  (
 
  𝑑𝜉 8

)
 

2  𝜋  1
 
𝑠𝑖𝑛
 
(  (3 + 𝜉)) = 𝑠𝑖𝑛 8
 
( )	1 − 𝜉
4  2
 
+ 𝑠𝑖𝑛
 
( )   (1 + 𝜉)
2  2
 
1
=   (3 + 𝜉) 4


 
Therefore
 


𝜋⁄2
 


𝜋	1	𝜋
 
∫	𝑠𝑖𝑛2(𝑥)𝑑𝑥 =
𝜋⁄4
 
  ∫ 𝑠𝑖𝑛2 (
8 −1	8
 
(3 + 𝜉)) 𝑑𝜉
 
𝜋	1 1
=   ∫  
 
(3 + 𝜉)𝑑𝜉
 
8 −1 4


3𝜋	2
=	[	]
 





𝜋⁄2
 
32

=

𝜋⁄4
 
0 + 1
3𝜋

 
16
 





𝜋⁄2
 
∫	𝑠𝑖𝑛2(𝑥)𝑑𝑥 = ∫	𝑠𝑖𝑛2(𝑥)𝑑𝑥 + ∫	𝑠𝑖𝑛2(𝑥)𝑑𝑥
 
0	0

𝜋	3𝜋
=	+
16	16
𝜋
=  
4
 
𝜋⁄4
 
(d)	The quadratic shape functions are
(1)	(2)	(3)
𝜉1 = −1	𝜉2 = 0	𝜉3 = 1

 
x=0	x= 𝜋
4
 
x = 𝜋
2
 


1
𝑁1(𝜉) = 2 𝜉(𝜉 − 1)
𝑁2(𝜉) = 1 − 𝜉2
1
𝑁3(𝜉) = 2 𝜉(𝜉 + 1)


 
Now the transformations are
 


𝑥 =
 

0 + 𝜋⁄2
 
2
 

𝜋⁄ − 0
+	𝜉
2
 
𝜋⁄
=
2
 
𝜋⁄
+	𝜉
2
 
= 𝜋⁄4 + 𝜋⁄4 𝜉
𝜋
 

2  𝜋
 

1
2( )    (
 
𝑑𝑥 =
4

)	2
 
𝑑𝜉

𝜋
 

2	2  𝜋  1  (	)
 
𝑠𝑖𝑛
 
(  (1 + 𝜉)) = 𝑠𝑖𝑛 4
 
0	𝜉
2
 
1 − 𝜉
 
+ 𝑠𝑖𝑛
 
( ) (1 − 𝜉 4
 
) + 𝑠𝑖𝑛
 
( )   𝜉
2  2
 
1 + 𝜉
 


 
= 0 + 1 (1 − 𝜉2) +
2
1	1
 
1
  𝜉( 2
 
1 + 𝜉)
 
=   +   𝜉
2	2


 
Thus
 


𝜋⁄2
∫	𝑠𝑖𝑛2(𝑥)𝑑𝑥 =
 


𝜋	1 1   ∫  
 


(1 + 𝜉)𝑑𝜉
 
0	4 −1 2


 
𝜋 1
=   [
4 2
 
2
*
0 + 1
𝜋
=  
4
 
+ 0]
 

Poisson