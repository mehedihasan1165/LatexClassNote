\documentclass[12pt,class=book,crop=false]{standalone}
\usepackage{./style}
\graphicspath{ {./img/} }
\begin{document}
\begin{prob}
    Show that $ J_{-n}(x)=(-1)^nJ_n(x) $
\end{prob}
\begin{soln}
    We have,
    \begin{equation}
        J_{-n}(x)=\sum_{r=0}^\infty \frac{(-1)^r\left( \frac{x}{2} \right)^{-n+2r}}{r!\,(-n+r)!} \label{eq:prob2.2.1}
    \end{equation}
    Let,\\
    \indent $ \begin{aligned}[t]
        &r-n=s\\
        \Rightarrow & r=n+s
    \end{aligned} $\\
    From \eqref{eq:prob2.2.1},
    \begin{align*}
        J_{-n}(x)&=\sum\frac{(-1)^{n+s}\left( \frac{x}{2} \right)^{-n+2(n+s)}}{(n+s)!\,(-n+n+s)!}\\
        &=(-1)^n\sum\frac{(-1)^{s}\left( \frac{x}{2} \right)^{n+2s}}{s!\,(n+s)!}\\
        &=(-1)^nJ_n(x)
    \end{align*}
\end{soln}
Thus, we obtain
\begin{soln}
    
    \begin{align*}
        a2	&=-\frac{\alpha(\alpha+1)}{1\cdot 2}a_0\\
        a_4&=-\frac{(\alpha-2)(\alpha + 3)}{3\cdot4}=(-1)^2\frac{\alpha(\alpha-2)(\alpha + 1)(\alpha + 3)}{4!}a_0\\
        &\vdots\\
        a_{2n}&=(-1)^n\frac{\alpha(\alpha - 2)\dots (\alpha - 2n + 2) \cdot (\alpha + 1)(\alpha + 3) \dots (\alpha + 2n - 1)}{(2n)!}a_0
    \end{align*}
    Similarly, we can compute $ a_3, a_5, a_7,\dots, $ in terms of $ a_1 $ and obtain
    hello there how are you asda ada asa 
\end{soln}
% \centering
% \rule{5cm}{0.5pt} $  \chi $ \rule{5cm}{0.5pt}
% % \line(1, 0){10pt}
\end{document}
% \documentclass{article}
% \usepackage{amsthm}
 
% \newtheorem{mytheo}{Theorem}
 
% \newenvironment{theo}{\begin{mytheo}}{\par\noindent\hrulefill\end{mytheo}}
 
% \begin{document}
 
% \begin{theo}
% text text text text text text text text text text text text text text text text text text text text text text text text text text text text text text text text text text text text text text text text text
% \end{theo}
 
% \end{document}