\documentclass[12pt,class=book,crop=false]{standalone}
\usepackage{../style}
\graphicspath{ {../img/} }
\begin{document}
\chapter{Solution of System of Linear Equations}
\section{Solutions of system of linear equations}
Consider the following system of linear equations
\begin{equation}
    \left.\begin{aligned}
        a_{11}x_1+a_{12}x_2+\dots+a_{1n}x_n & =b_1 \\
        a_{21}x_1+a_{22}x_2+\dots+a_{2n}x_n & =b_2 \\
        a_{31}x_1+a_{32}x_2+\dots+a_{3n}x_n & =b_3
    \end{aligned}\,\right\}\label{eq:sys}
\end{equation}
Let the coefficient matrix
\[
    A=\begin{pmatrix}
        a_{11} & a_{12} & \dots  & a_{1n} \\
        \vdots & \vdots & \ddots & \vdots \\
        a_{n1} & a_{n2} & \dots  & a_{nn}
    \end{pmatrix},
    \quad B=\begin{pmatrix}
        b_1    \\
        \vdots \\
        b_n
    \end{pmatrix},
    \quad X=\begin{pmatrix}
        x_1    \\
        \vdots \\
        x_n
    \end{pmatrix}
\]
then the system (\ref{eq:sys}) reduces to \( AX=B \). To find the solution of \( AX=B \,\Rightarrow X=A^{-1}B \), there are two method solving the system (\ref{eq:sys}) namely (i) Direct method and (ii) Iteration method.\\
Direct or Gauss-Elimination method is used when the coefficient matrix \( A \) is a dense matrix and the iteration method is used when the coefficient matrix \( A \) is large ans sparse.
\section{Gauss-Elimination Method}
In this method, we use \( k \)th equation to eliminate \( x_k \) from equations \( (K+1),(K+2),\dots,n \) during the \( k \)th step \( (K=1,2,\dots, n) \).

By this process the system is reduced to an upper triangular form. This is possible only if at the beginning of the \( k \)th step the coefficient \( a_{kk} \neq 0\). Otherwise, straight forward reduction does not work. Then rearrangement is necessary.

Thus in an \( n\times m \) matrix, the value of the element and its position is hence important.
\begin{ex}
    Solve the system of equations:
    \begin{align}
        x_1+x_2+x_3   & = 1 \label{eq:ex1.1} \\
        x_1+x_2+2x_3  & = 2 \label{eq:ex1.2} \\
        x_1+2x_2+2x_3 & = 1 \label{eq:ex1.3}
    \end{align}
    \begin{align}
        x_1+x_2+x_3                                              & = 1 \label{eq:ex1.1d} \\
        \text{\eqref{eq:ex1.2}-\eqref{eq:ex1.1}}\qquad\qquad x_3 & = 1 \label{eq:ex1.2d} \\
        \text{\eqref{eq:ex1.3}-\eqref{eq:ex1.1}}\qquad x_2+x_3   & = 0 \label{eq:ex1.3d}
    \end{align}
    Since \( a_{22}^{'} =0 \), rearrangement is necessary \eqref{eq:ex1.2d}\( \longleftrightarrow \)\eqref{eq:ex1.3d}
    \begin{align}
        x_1+x_2+x_3 & = 1 \label{eq:ex1.1dd} \\
        x_2+x_3     & = 0 \label{eq:ex1.2dd} \\
        x_3         & = 1 \label{eq:ex1.3dd}
    \end{align}
    Back substitution starts
    \begin{align*}
        \eqref{eq:ex1.3dd} \Rightarrow & x_3=1           \\
        \eqref{eq:ex1.2dd} \Rightarrow & x_2=-x_3=-1     \\
        \eqref{eq:ex1.1dd} \Rightarrow & x_1=1-x_3-x_2=1
    \end{align*}
    So the solutions is \( x=\begin{pmatrix}
            x_1 \\
            x_2 \\
            x_3
        \end{pmatrix} =\begin{pmatrix*}[r]
            1  \\
            -1 \\
            1
        \end{pmatrix*} \)
\end{ex}
\begin{ex}
    Solve the system
    \[x_2,x_3=1,\,x_1+x_3=1,\,x_1+x_2=1\]
\end{ex}
\begin{ex}
    Solve the system of equations:
    \begin{align}
        x_1+x_2+x_3        & = 1 \label{eq:ex3.1} \\
        x_1+1.0001x_2+2x_3 & = 2 \label{eq:ex3.2} \\
        x_1+2x_2+2x_3      & = 1 \label{eq:ex3.3}
    \end{align}
    \begin{align}
        x_1+x_2+x_3                                                   & = 1 \label{eq:ex3.1d} \\
        \text{\eqref{eq:ex3.2}-\eqref{eq:ex3.1}}\quad 0.0001x_2+x_3   & = 1 \label{eq:ex3.2d} \\
        \text{\eqref{eq:ex3.3}-\eqref{eq:ex3.1}}\qquad \qquad x_2+x_3 & = 0 \label{eq:ex3.3d}
    \end{align}
    \begin{align}
        x_1+x_2+x_3                                                                      & = 1 \label{eq:ex3.1dd}      \\
        0.0001x_2+x_3                                                                    & = 0 \label{eq:ex3.2dd}      \\
        \text{\( 10\,000\times \) \eqref{eq:ex3.2d}-\eqref{eq:ex3.3d}} \qquad \qquad 9999x_3 & = 100\,000 \label{eq:ex3.3dd}
    \end{align}
    By back substitution,
    \begin{align*}
        \eqref{eq:ex3.3dd} \Rightarrow & x_3=\frac{10\,000}{9\,999}=1.000\,100\,1\approx 1\text{ (using \( 3 \) significant digit)} \\
        \eqref{eq:ex3.2dd} \Rightarrow & x_2=\frac{1-1}{0.0001}=0                                                         \\
        \eqref{eq:ex3.1dd} \Rightarrow & x_1=1-0-1=0
    \end{align*}
    So the solutions is \( x=\begin{pmatrix}
            x_1 \\
            x_2 \\
            x_3
        \end{pmatrix} =\begin{pmatrix}
            0 \\
            0 \\
            1
        \end{pmatrix} \)\\
    \underline{Alternatively}
    \begin{align}
        x_1+x_2+x_3        & = 1 \label{eq:ex4.1} \\
        x_1+1.0001x_2+2x_3 & = 2 \label{eq:ex4.2} \\
        x_1+2x_2+2x_3      & = 1 \label{eq:ex4.3}
    \end{align}
    \begin{align}
        x_1+x_2+x_3                                                               & = 1 \label{eq:ex4.1d} \\
        \text{\eqref{eq:ex4.2}-\eqref{eq:ex4.1}}\qquad \qquad 0.000\,1x_2+x_3       & = 1 \label{eq:ex4.2d} \\
        \text{\eqref{eq:ex4.3}-\eqref{eq:ex4.1}}\qquad\qquad\,\,\, \qquad x_2+x_3 & = 0 \label{eq:ex4.3d}
    \end{align}
    Interchanging \eqref{eq:ex4.2d} and \eqref{eq:ex4.3d} we get
    \begin{align}
        x_1+x_2+x_3   & = 1 \label{eq:ex4.1dd} \\
        x_2+x_3       & = 0 \label{eq:ex4.2dd} \\
        0.000\,1x_2+x_3 & = 1 \label{eq:ex4.3dd}
    \end{align}
    \begin{align}
        x_1+x_2+x_3& = 1 \label{eq:ex4.1ddd} \\
        x_2+x_3& = 0 \label{eq:ex4.2ddd} \\
        \text{\eqref{eq:ex4.3dd}-0.000\,1\eqref{eq:ex4.2dd}}\qquad \qquad  0.999\,9x_3 & = 1 \label{eq:ex4.3ddd}
    \end{align}
    By back substitution,
    \begin{align*}
        \eqref{eq:ex4.3ddd} \Rightarrow & x_3=\frac{1}{0.999\,9}=1.000\,100\,1\approx 1 \\
        \eqref{eq:ex4.2ddd} \Rightarrow & x_2=-x_3=-1                             \\
        \eqref{eq:ex4.1ddd} \Rightarrow & x_1=1-(-1)-1=1
    \end{align*}
    So the solutions is \( x=\begin{pmatrix}
            x_1 \\
            x_2 \\
            x_3
        \end{pmatrix} =\begin{pmatrix*}[r]
            1  \\
            -1 \\
            1
        \end{pmatrix*} \) (correct result)
\end{ex}
\begin{rem}
    In this problem, we see that there are different solution in different way. It is caused by using different pivot equation which is discusses below.
\end{rem}
\section{Pivot Strategy}
If \( a_{kk}^{(k)}=0 \) in the \( k \)th step for some \( k=1,2,\dots,n-1 \) then from the \( k \)th row to the \( n \)th row is searched for the first non-zero entry.

If \( a_{pk}^{(k)}\neq 0 \) for some \( p \), \( k+1\leq p\leq n \), then perform \( E_k \longleftrightarrow E_p \) (interchange) where \( E_k \) is the \( k \)th equation in the system.

If \( a_{pk}^{(k)}= 0 \), \( p=k,k+1,\dots,n \) then the system does not have unique solution and the procedure fails.

So obtaining a zero for the pivot element, necessarily takes a row interchange. But in practice it is often desirable to perform row interchanges involving pivot elements even when they are non-zero.
\begin{prob}
    Solve
    \[
        \left.\begin{aligned}
             & E_1: \\
             & E_2:
        \end{aligned}\right.
        \left.\begin{aligned}
            0.003x_1+29.140x_2 & =59.17 \\
            5.291x_1-6.130x_2  & =46.78
        \end{aligned}\right.
    \]
    Use \( 4 \) digits arithmetic to find its solution. Exact solution is \( x_1=10.00,\,x_2=1.00 \)
\end{prob}
\begin{soln}
    Use \( a_{11}=0.003\,0  \) as pivot element,\\
    Find the multiplier \( m=\frac{a_{21}}{a_{11}}=\frac{5.291\,0}{0.003}=1\,763.666\,6\approx1764 \)\\
    Eliminate \( x_1 \) from \( E_2 \) by \( (E_2-mE_1)\rightarrow E_2^{'} \)
    \[
        \left.\begin{aligned}
             & E_1^{'}: \\
             & E_2^{'}:
        \end{aligned}\right.
        \left.\begin{aligned}
            0.003x_1+59.140x_2 & =59.17     \\
            -104\,329x_2         & =-104\,329.1
        \end{aligned}\right.
    \]
    By back substitution,
    \[
        x_2=1.000\,000\,959\approx1.001
    \]
    \[
        \therefore E_1=\frac{59.19-59.140\times1.001}{0.003\,0}=-9.7133\approx-10.0
    \]
    The large error in the solution of \( x_1 \) resulted from the small error in solving \( x_2 \). The errors magnified \( 20000 \) times in the solution of \( x_1 \). This happened due to division by pivot element \( 0.003\,0 \).
\end{soln}
This sort of difficulties arise in choosing the pivot element \( a_{kk}^{(k)} \) (in the \( k \)th step) when they are relatively small, compared to the other entries,
\( a_{ij}^{(k)}\) for \( k\leq i\leq n \), \( k\leq j\leq n \).

So we choose a new pivot element \( a_{pq}^{(k)} \) (in the \( k \)th step).\\

Pivot Strategies in general are accomplished by selecting a new element for the pivot \( a_{pq}^{(k)} \) and interchanging the \( k \)the row and \( p \)th rows, followed by interchanging \( k \)th and \( p \)th column if necessary.

The simplest way is to choose the element in the same column, that is below the diagonal and has the largest absolute value; that is, we choose \( p \) such that,
\[
    a_{pk}^{(k)}=\begin{aligned}[t]
        \text{max} & \abs{a_{ik}^{(k)}} \\
        k\leq      & i\leq n
    \end{aligned}
\]
Then perform \( E_k\longleftrightarrow E_p \). In this case no interchange of column is necessary.
\begin{soln}
    \[
        \left.\begin{aligned}
             & E_1: \\
             & E_2:
        \end{aligned}\right.
        \left.\begin{aligned}
            0.003x_1+59.140x_2 & =59.17 \\
            5.291x_1-6.130x_2  & =46.78
        \end{aligned}\right.
    \]
    By the pivot Strategy: Find \(( k=1 )\)
    \begin{align*}
        \text{max} \left\{ \abs{a_{11}^{(1)}}, \abs{a_{21}^{(1)}} \right\} & =\text{max} \left\{ \abs{1.003},\abs{5.291}\right\} \\
                                                                       & =5.291\\
                                                                       & =a_{21}^{(1)}
    \end{align*}
    The operation \( E_2 \longleftrightarrow E_1 \) is performed to give the system
    \[
        \left.\begin{aligned}
            & E_1 \\
            & E_2:
        \end{aligned}\right.
        \left.\begin{aligned}
            5.291x_1-6.13x_2&=46.78\\
            0.003x_1+59.140x_2 & =59.17
        \end{aligned}\right.
    \]
    The multiplier for this system is 
    \[m=\frac{a_{21}^{(1)}}{a_{11}^{(1)}}=\frac{0.003}{5.291}=0.000\,567
    \]
    and the operation \( (E_2-mE_1)\rightarrow E_2^{'} \) reduces the system to 
    \[
        \left.\begin{aligned}
             & E_1^{'}: \\
             & E_2^{'}:
        \end{aligned}\right.
        \left.\begin{aligned}
            5.29 x_1-6.130x_2 & =46.78     \\
            59.14x_2         & =59.14
        \end{aligned}\right.
    \]
    Then by back substitution \( x_2=1.000 \), \( x_1=10.00 \).\\
    Which is the correct solution.
\end{soln}
This technique is known as maximal-column pivoting or partial pivoting.
\begin{prob}
    Solve the following system using \( 4 \) digits arithmetic, once with the \( 1 \)st equation as pivot equation and then \( 2 \)nd equation as pivot equation and then compare the results with exact solution \( x_1=1.00,\, x_2=0.2500  \).
    \[
        \left.\begin{aligned}
            & E_1: \\
            & E_2:
        \end{aligned}\right.
        \left.\begin{aligned}
            0.1410\times 10^{-2}x_1+0.400\,4\times 10^{-1}x_2& =0.114\,2\times 10^{-1}     \\
            0.200\,0\times 10^{0}x_1+0.491\,2\times 10^{1}x_2& =0.142\,8\times 10^{1}
        \end{aligned} \right.
    \]
\end{prob}
\section{Gauss-Seidal Iterative Method}
An iterative technique here starts with an initial approximation \( X^{(0)} \) of the solution \( X \) and generates a sequence of vectors \( \left\{X^{(k)},k=0,1,2,\dots\right\} \) that converges to \( X \).

In this case first we transform the system \( AX=B \) into an equivalent system of the form \( X=TX+C \) for some \( n\times n \) matrix \( T \) and a vector \( C \) and we suppose
    \begin{equation}
        \left.\begin{aligned}
            &E_1:&x_1&= -\frac{a_{12}}{a_{11}}x_2-\frac{a_{13}}{a_{11}}x_3-\dots-\frac{a_{1n}}{a_{11}}x_n+\frac{b_1}{a_{11}}\\
            &E_2:&x_2&= -\frac{a_{21}}{a_{22}}x_1-\frac{a_{23}}{a_{22}}x_3-\dots-\frac{a_{2n}}{a_{22}}x_n+\frac{b_2}{a_{22}}\\
            &E_2:&x_3&= -\frac{a_{31}}{a_{33}}x_1-\frac{a_{32}}{a_{34}}x_4-\dots-\frac{a_{3n}}{a_{33}}x_n+\frac{b_3}{a_{33}}\\
            &\dots &\dots &\quad \dots\quad \dots\quad \dots\\
            &E_n:&x_n&= -\frac{a_{na}}{a_{nn}}x_1-\frac{a_{n2}}{a_{nn}}x_2-\dots-\frac{a_{n\cdot n-1}}{a_{nn}}x_{n-1}+\frac{b_n}{a_{nn}}
        \end{aligned}\quad\right\} \label{eq:star}
    \end{equation}
    In this case, it is reasonable to compute \( x_i^{(k)} \) using these most recently calculated values i.e.,
    \begin{enumerate}
        \item First substitute \( x_1^{(0)},x_2^{(0)},\dots,x_n^{(0)} \) in the first equation \( E_1 \) in the right-hand side of \eqref{eq:star} and we get the new value \( x_1^{(1)} \).
        \item Then substitute \( x_1^{(1)},x_2^{(0)},x_3^{(0)}\dots,x_n^{(0)} \) in the second equation \( E_2 \) in the right-hand side of \eqref{eq:star} and we get the new value \( x_2^{(1)} \).
        \item Then substitute \( x_1^{(1)},x_2^{(1)},x_3^{(0)}\dots,x_n^{(0)} \) in the third equation \( E_3 \) in the right-hand side of \eqref{eq:star} and we get the new value \( x_3^{(1)} \).
        \item Continuing in this way we put \( x_1^{(1)},x_2^{(1)},\dots,x_n^{(0)} \) in the last equation of \eqref{eq:star}
    \end{enumerate}
    Then the first iteration is complete.\\
    Repeat the entire process till \( (x_1,x_2,\dots,x_n) \) is obtained.\\
    This method is known as Gauss-Seidal iterative method.
    \begin{prob}
        Solve the system
        \begin{equation}
            \left.\begin{aligned}
                10x_1-x_2+2x_3&=6\\
                -x_1+11x_2-x_3+3x_4&=25\\
                2x_1-x_2+10x_3-x_4&=-11\\
                3x_2-x_3+8x_4&=15\\
            \end{aligned}\quad\right\}\label{eq:star1}
        \end{equation}
        using Gauss-Seidal method.
    \end{prob}
    \begin{soln}
        The given system can be written as
        \[
            \begin{aligned}
                &E_1:&x_1&= \frac{1}{10}x_2-\frac{2}{10}x_3+\frac{6}{10}\\
                &E_2:&x_2&= \frac{1}{11}x_1+\frac{1}{11}x_3-\frac{3}{11}x_4+\frac{25}{11}\\
                &E_3:&x_3&= -\frac{2}{10}x_1+\frac{1}{10}x_2+\frac{1}{10}x_4-\frac{11}{10}\\
                &E_4:&x_4&= -\frac{3}{8}x_2+\frac{1}{8}x_3+\frac{15}{8}
            \end{aligned}
        \]
        which is of the form \( X=TX+C \)\\
        Where,
        \[
            T=\begin{pmatrix}
                0&\frac{1}{10}&\frac{-2}{10}&0\\[.5cm]
                \frac{1}{11}&0&\frac{1}{11}&\frac{-3}{11}\\[.5cm]
                \frac{-2}{10}&\frac{1}{10}&0&\frac{1}{10}\\[.5cm]
                0&\frac{-3}{8}&\frac{1}{8}&0
            \end{pmatrix}
        \]
        and 
        \[
            C=\begin{pmatrix*}[r]
                \frac{6}{10}\\[.5cm]
                \frac{25}{11}\\[.5cm]
                -\frac{11}{10}\\[.5cm]
                \frac{15}{8}
            \end{pmatrix*}
        \]
        Suppose the initial approximation is \( X^{(0)}=\begin{pmatrix}
            0\\
            0\\
            0\\
            0
        \end{pmatrix} \)\\
        from \( X^{(1)}=TX^{(0)}+C\Rightarrow X^{(1)}=
        \begin{pmatrix*}[r]
            \frac{6}{10}\\[0.5cm]
            \frac{25}{11}\\[0.5cm]
            -\frac{11}{10}\\[0.5cm]
            \frac{15}{8}
        \end{pmatrix*} \).\\
        Similarly, we calculate the other approximation which are shown in the following iteration table
        \begin{center}
        \begin{tabular}{cc*{6}{S[table-format=2.4]}}
            \toprule
            k\( \rightarrow \)&\( 0 \)&\multicolumn{1}{c}{\( 1 \)}&\multicolumn{1}{c}{\( 2 \)}&\multicolumn{1}{c}{\( 3 \)}&\multicolumn{1}{c}{\( 4 \)}&\multicolumn{1}{c}{\( 5 \)}&\multicolumn{1}{c}{\( 6 \)}\\\midrule
            \( x_1^{(k)} \)\rule[-1em]{0pt}{1em}&\( 0 \)& 0.6000 & 1.0302 & 1.0066 & 1.0004 & 1.0001 & 1.0000 \\
            \( x_2^{(k)} \)&\rule[-1em]{0pt}{1em}\( 0 \)& 2.3273 & 2.0370 & 2.0036 & 2.0003 & 2.0000 & 2.0000 \\
            \( x_3^{(k)} \)\rule[-1em]{0pt}{1em}&\( 0 \)& -0.9873 & -1.0145 &-1.0025 & -1.0003 & -1.0000 & -1.0000 \\
            \( x_4^{(k)} \)\rule[-1em]{0pt}{1em}&\( 0 \)& 0.8789 & 0.4843 & 0.9984 & 0.9998 & 1.0000 & 1.0000 \\\bottomrule
        \end{tabular}
    \end{center}
        Therefore, the required solution is 
        \[
            X=\begin{pmatrix}
                x_1\\
                x_2\\
                x_3\\
                x_4
            \end{pmatrix}=\begin{pmatrix*}[r]
                1\\
                2\\
                -1\\
                1
            \end{pmatrix*}
        \]
    \end{soln}
    \begin{prob}
        Solve the following system by Gauss-Seidal method
        \begin{align*}
            21x_1+6x_2-x_3&=85\\
            6x_1+15x_2+2x_3&=72\\
            x_1+x_2+54x_3&=110
        \end{align*}
    \end{prob}
\end{document}