\documentclass[12pt,class=book,crop=false]{standalone}
\usepackage{../style}
\graphicspath{ {../img/} }
\begin{document}
\chapter{Numerical Differentiation}
\section{Introduction}
Recall the definition of the derivative of a function
\[
    f'(x)=\lim_{\Delta x\to 0}  \frac{f(x+\Delta x)-f(x)}{\Delta x}
\]
For finite $ \Delta x $,
\[
    f' (x)\approx \frac{f(x+\Delta x)-f(x)}{\Delta x}
\]
To find the derivative at $ x=x_i $, we choose another point $ x_{i+1}=x_i+h $ ahead of $ x_i $. This gives two point forward difference formulae
\begin{equation}
    \label{eq:intro1}
    f'(x_i)\approx \frac{f(x_i+h)-f(x_i)}{h}
\end{equation}
If $ \Delta x $ is chosen as a negative number, say $ \Delta x=-h  (h> 0) $, we have
\begin{equation}
    \label{eq:intro2}
    f'(x_i)\approx \frac{f(x_i-h)-f(x_i)}{-h}\approx \frac{f(x_i)-f(x_i-h)}{h}
\end{equation}
This is backward difference formula for first derivative.
Adding Eq.\eqref{eq:intro1} and Eq.\eqref{eq:intro2}, we have
\begin{equation}
    \label{eq:intro3}
    f' (x_i)\approx \frac{f(x_i+h)-f(x_i-h)}{2h}
\end{equation}
which is a-point central difference formula for first derivative.
\section{Derivative Formula from Taylor  Series}
For clear idea about the different formulas and their order of errors we may use the Taylor series expansion of $ f(x) $.\\
From Taylor series expansion for $ h > 0 $, we have
\begin{align}
    f(x_0+h) & =f(x_0)+hf'(x_0)+\frac{h^2}{2!} f''(x_0)+\frac{h^3}{3!}+\dots\label{eq:tay1}  \\
    f(x_0-h) & =f(x_0)-hf'(x_0)+\frac{h^2}{2!} f''(x_0)-\frac{h^3}{3!}+\dots \label{eq:tay2}
\end{align}
From the expansion of $ f(x_0+h) $, we have
\[
    f'(x_0)=\frac{f(x_0+h)-f(x_0)}{h}-\frac{h}{2!}f''(x_0)-\frac{h^2}{3!}-\dots
\]
which leads to the two-point forward difference formula for $ f'(x_0) $ as
\[
    f'(x_0)=\frac{f(x_0+h)-f(x_0)}{h}+E
\]
where the error series is
\[
    E=-\left[ \frac{h}{2!} f''(x_0)+\frac{h^2}{3!}+\dots  \right]
\]
From the expansion of $ f(x_0-h) $, we have 2-point backward difference formula
\[
    f'(x_0)=\frac{f(x_0)-f(x_0-h)}{h}+E
\]
with error term
\[
    E=\frac{h}{2!}f''(x_0)-\frac{h^2}{3!}+\dots
\]
In the two point formula the error series is of the form
\[
    E=a_1 h+a_2h^2+a_3h^3+\dots
\]
where $ a $'s does not depend on $ h $.\\

By subtraction, we obtain
\[
    f(x_0+h)-f(x_0-h)=2hf(x_0)+\frac{2}{3!} h^3 f'''(x_0)+\frac{2}{5!}h^5f^{(v)}(x_0)+\dots
\]
This leads to the 3-point central formula for approximating $ f' (x_0) $
\[
    f'(x_0)=\frac{f(x_0+h)-f(x_0-h)}{2h}+E
\]
with
\[
    E=-\left[ \frac{1}{3!}h^2 f'''(x_0)+\frac{1}{5!}h^4f^{(v)}(x_0)+\dots\right]
\]
Adding the Taylor series for $ f(x_0+h) $ and $ f(x_0-h) $, we get
\[
    f(x_0+h)+f(x_0-h)=2f(x_0)+h^2 f''(x_0)\frac{2}{4!}h^4f^{(4)}(x_0)
\]
When this is rearranged, we get 3-point central difference formula for $ f''(x_0) $
\[
    f''(x_0)=\frac{f(x_0+h)-2f(x_0)+f(x_0-h)}{h^2}+E
\]
where the error series is
\[
    E=-2\left[ \frac{1}{4!} h^2 f^{(4)} (x_0)+\frac{1}{6!} h^4 f^{(6)}(x_0)\dots\right]
\]
In the three point central difference formula the error series is of the form
\[
    E=a_2 h^2+a_4h^4+a_6h^6+\dots
\]
\section{Formulas for Computing Derivatives}
\textbf{First Derivatives}
\begin{align*}
    f' (x_0) & \approx \frac{f_1-f_0}{h},                   & O(h)\qquad   & 2-\text{points forward difference}  \\
    f' (x_0) & \approx \frac{f_0-f_(-1)}{h},                & O(h)\qquad   & 2-\text{points backward difference} \\
    f' (x_0) & \approx \frac{f_1-f_(-1)}{2h},               & O(h^2)\qquad & 3-\text{points central difference}  \\
    f' (x_0) & \approx \frac{1}{2h} [-3f_0+4f_1-f_2 ],      & O(h^2)\qquad & 3-\text{points forward difference}  \\
    f'(x_0)  & \approx \frac{1}{2h} [3f_0-4f_{-1}+f_{-2} ], & O(h^2)\qquad & 3-\text{points backward difference}
\end{align*}
\textbf{Second Derivatives}
\begin{align*}
    f''(x_0) & \approx \frac{1}{h^2}[f_{-1}-2f_0+f_1 ],    & O(h^2)\qquad & 3-\text{point central difference}  \\
    f''(x_0) & \approx\frac{1}{h^2} [f_0-2f_1+f_2 ],       & O(h)\qquad   & 3-\text{point forward difference}  \\
    f''(x_0) & \approx \frac{1}{h^2} [f_0-2f_{-1}+f_{-2}], & O(h)\qquad   & 3-\text{point backward difference}
\end{align*}
\section{Richardson Extrapolation}
If the two approximations of order $ O(h^n) $ for $ M $ are $ M(h) $ and $ M(rh) $, then the Richardson's extrapolated estimate $ M_R $ of $ M $ can be written as
\begin{align}
    M_R & =M(h)+Ah^n\label{eq:rich1}     \\
    M_R & =M(rh)+A(rh)^n\label{eq:rich2}
\end{align}
where we have assumed the constant multiplicative factor is same.\\
Subtracting \eqref{eq:rich1} from \eqref{eq:rich2},
\begin{align*}
    0                  & =M(rh)-M(h)+Ah^n (r^n-1)  \\
    \Rightarrow\, Ah^n & =\frac{M(h)-M(rh)}{r^n-1}
\end{align*}
Substituting in \eqref{eq:rich1}, we have
\[
    M_R=M(h)+\frac{M(h)-M(rh)}{r^n-1}
\]
which is the \emph{Richardson extrapolation} formula.


Lower order formula and Richardson extrapolation can be used to deduce the higher order formula. For convenience, we have used the notation $ f'(x_0,h) $ to indicate clearly the approximation of $ f'(x_0) $ with step size $ h $ and $f(x_0+rh)=f_r$.
Thus, the 3-point central difference formula for first derivative will be written as
\[
    f'(x_0,h)=\frac{f_1-f_{-1}}{2h}
\]
\begin{prob}
    Derive three-point forward difference formulas for the first and second derivative using two point derivative formula\\
    The values of distance at various times are given below
    \begin{table}[H]
        \centering
        \begin{tabular}{cS[table-format=3.2]}
            \toprule
            Time $ (t) $    &
            \multicolumn{1}{c}{Distance$ (s) $} \\\midrule
             6    &
             7.38 \\
              7     &
              12.07 \\
               8     &
               18.37 \\
                9     &
                26.42 \\
                 10    &
                 36.40 \\\bottomrule
        \end{tabular}
    \end{table}
    Estimate the velocity $ v=\frac{\D s}{\D t} $ and acceleration $ a=\frac{\D^2 s}{\D t^2}$ at time $ t=6 $ using three point formula and extrapolation.
\end{prob}
\begin{soln}
    Three point forward derivate formula:\\
    From 2-point forward difference formula, we have
    \begin{align*}
        f'(x_0,h)   & =\frac{f(x_0+h)-f(x_0)}{h}=\frac{f_1-f_0}{h}    \\
        f' (x_0,2h) & =\frac{f(x_0+2h)-f(x_0)}{2h}=\frac{f_2-f_0}{2h}
    \end{align*}
    Using Richardson extrapolation
    \begin{align*}
        f_R' (x_0) & =f' (x_0,h)+\frac{f' (x_0,h)-f' (x_0,2h)}{2^1-1}                       \\
                   & =\frac{f_1-f_0}{h}+\left[ \frac{f_1-f_0}{h}-\frac{f_2-f_0}{2h} \right] \\
                   & =\frac{1}{2h} (4f_1-4f_0-f_2+f_0 )                                     \\
                   & =\frac{1}{2h} (-3f_0+4f_1-f_2 )
    \end{align*}
    which is the three point forward difference formula for first derivative and its order of error is two $ O(h^2) $.\\


    Froward derivative formula for second derivatve:


    Differentiating two-point first derivative formula we have
    \begin{align*}
        f''(x_0,h) & =\frac{f'(x_0+h)-f' (x_0)}{h}                                                      \\
                   & =\frac{1}{h} \left[ \frac{f(x_0+2h)-f(x_0+h)}{h}-\frac{f(x_0+h)-f(x_0)}{h} \right] \\
                   & =\frac{1}{h^2} [f(x_0+2h)-2f(x_0+h)+f(x_0)]                                        \\
                   & =\frac{1}{h^2} (f_0-2f_1+f_2)
    \end{align*}
    Here error term is not eliminated by extrapolation and hence the order of the error is $ O(h) $.\\


    Velocity at $ t=6 $
    \begin{align*}
        v(6,1) & =\frac{1}{2(1)}[-3s(6)+4s(7)-s(8)]    \\
               & =\frac{1}{2}[-3(7.38)+4(12.07)-18.37] \\
               & =3.885                                \\
        v(6,2) & =\frac{1}{2(2)}[-3s(6)+4s(8)-s(10)]   \\
               & =\frac{1}{2}[-3(7.38)+4(18.37)-36.40] \\
               & =3.735
    \end{align*}
    Extrapolated value is
    \[
        v_R (6)=v(6,1)+\frac{v(6,1)-v(6,2)}{2^2-1}=3.885+\frac{3.885-3.735}{3}=3.935
    \]

    Acceleration at $ t=6 $
    \begin{align*}
        a(6,1) & =\frac{1}{(1)^2} [s(6)-2s(7)+s(8)] \\
               & =[7.38-2(12.07)+18.37]             \\
               & =1.61                              \\
        a(6,2) & =\frac{1}{(2)^2}[s(6)-2s(8)+s(10)] \\
               & =\frac{1}{4}[7.38-2(18.37)+36.40]  \\
               & =1.76
    \end{align*}
    Extrapolated value is
    \[
        a_R (6)=a(6,1)+\frac{a(6,1)-a(6,2)}{2^1-1}=1.61+\frac{1.61-1.76}{1}=1.46
    \]
\end{soln}
\section{Derivatives from Interpolating Polynomials}
We can fit a polynomial through the data points and then by differentiating we may find the derivatives at a point. This is useful when the data values are not evenly distributed or derivatives are required at points other than tabulated points.\\
The method is discussed with an example.
\begin{prob}
    The distance $ D=D(t) $ traveled by an object is given in the table below:
    \begin{table}
        \centering
        \begin{tabular}{cS[table-format=3.3]}
            \toprule
            $ t $    &
            \multicolumn{1}{c}{$ D(t) $} \\\midrule
             8.0    &
             17.453 \\
              9.0    & 
              21.460 \\
              10.0   &
              25.752 \\
               11.0   &
               30.301 \\
                12.0   &
                     35.084 \\\bottomrule
        \end{tabular}
    \end{table}
    \begin{enumerate}[label=(\alph*)]
        \item Estimate the velocity and acceleration at $ t = 10.4 $ using three suitable points.
        \item Estimate the velocity and acceleration at $ t = 10.4 $ using four suitable points.
    \end{enumerate}
    In each case compare your results with exact results obtained from
    \[
        D(t)=-70+7t+70 e^{\frac{-t}{10}}
    \]
\end{prob}
\begin{soln}
    To construct  polynomial through the given points, the divided difference table is as follows:
    \begin{table}
        \centering
        \begin{tabular}{c*{3}{S[table-format=3.4]}S[table-format=2.5]}
            \toprule
            $ t $ & \multicolumn{1}{c}{$ D(t) $} & \multicolumn{1}{c}{$ D^1(t) $} &\multicolumn{1}{c}{ $ D^2(t) $} &\multicolumn{1}{c}{ $ D^3(t) $} \\\midrule
            9     & 21.460   &            &            &            \\
            10    & 25.752   & 4.292      &            &            \\
            11    & 30.301   & 4.549      & 0.1285     &            \\
            12    & 35.084   & 4.783      & 0.117      & -0.00383   \\\bottomrule
        \end{tabular}
    \end{table}
    \begin{enumerate}[label=(\alph*)]
        \item We need to consider 3 points that are closest to $ t=10.4 $ and we choose the points as $ t=9 $,  $ t=10 $ and $ t=11 $. Then
              \begin{align*}
                  D(t) & =21.46+4.292(t-9)+0.1285(t-9)(t-10),\qquad 9\leq t\leq 11 \\
                  v(t) & =\frac{\D D}{\D t}                                        \\
                       & =4.292+0.1285(2t-9-10)                                    \\
                  a(t) & =\frac{\D^2 D}{\D t^2}                                    \\
                       & =0.1285(2)                                                \\
                       & =0.257
              \end{align*}
              Thus,
              \[
                  v(10.4)=4.5233\qquad \text{and}\qquad a(10.4)=0.257
              \]
        \item We need to consider 4 points that are closest to $ t=10.4 $ and we choose the points as $ t=9 $,  $ t=10 $, $ t=11 $ and $ t=12 $. Then
              \begin{align*}
                  D(t) & =21.46+4.292(t-9)+0.1285(t-9)(t-10)-0.003\,83(t-9)(t-10)(t-12),\qquad 9\leq t\leq 12 \\
                  v(t) & =\frac{\D D}{\D t}                                                                 \\
                       & =4.292+0.1285(2t-9-10)-0.003\,83\left[ 3t^2-2t(9+10+12)+(90+108+120) \right]         \\
                  a(t) & =\frac{\D^2 D}{\D t^2}                                                             \\
                       & =0.1285(2)-0.003\,83[6t-2(31)]
              \end{align*}
              Thus
              \[
                  v(10.4)=4.5322 \qquad\text{and}\qquad a(10.4)=0.2554
              \]
    \end{enumerate}
    \textbf{Error estimation:}\\
    From the exact expression for distance we have
    \[
        \frac{\D D}{\D t}=7-7e^{\frac{-t}{10}}\qquad \text{and}\qquad \frac{\D^2 D}{\D t^2}=\frac{7}{10} e^{\frac{-t}{10}}
    \]
    which give
    \[
        v(10.4)=4.5258 \qquad \text{and}\qquad a(10.4)=0.2474
    \]
    Error with 3-point polynomial:
    \begin{align*}
        \text{Absolute error in velocity}     & = \abs{\frac{4.5233-4.5258}{4.5258}}\times 100=0.055\% \\
        \text{Absolute error in acceleration} & = \abs{\frac{0.257-0.2474}{0.2474}}\times 100=3.88\%
    \end{align*}
    Error with 4-point polynomial:
    \begin{align*}
        \text{Absolute error in velocity}     & = \abs{\frac{4.5322-4.5258}{4.5258}}\times 100=0.14\% \\
        \text{Absolute error in acceleration} & = \abs{\frac{0.2554-0.2474}{0.2474}}\times 100=3.23\%
    \end{align*}
\end{soln}
\section{Exercise}
\begin{enumerate}
    \item The values of  $ f(x) $ are given in the following table:
          \begin{table}[H]
              \centering
              \begin{tabular}{cS[table-format=2.3]}
                  \toprule
                  $ x $    &
                  \multicolumn{1}{c}{$ f(x) $} \\\midrule
                   1.2   &
                   4.448 \\
                    1.3   &
                    3.567 \\
                     1.4   &
                     2.624 \\
                      1.5   &
                      1.625 \\
                       1.6   &
                            0.576 \\\bottomrule
              \end{tabular}
          \end{table}
          \begin{enumerate}
              \item Using two-point  formulae estimate the values of $ f' (1.2) $, $ f' (1.4) $, $ f' (1.6) $.
              \item  Using three-point formulae estimate the values of $ f' (1.4) $ and $ f''(1.4 ) $.
          \end{enumerate}
    \item The table below shows the values of $ f(x) $ at different values of  $ x $:
          \begin{table}[H]
              \centering
              \begin{tabular}{cS[table-format=2.4]}
                  \toprule
                  $ x $    &
                  \multicolumn{1}{c}{$ f(x) $ }\\\midrule
                   1.4    &
                   1.3796\\
                    1.5    &
                    1.4962 \\
                     1.6    &
                     1.5993 \\
                     1.7    &
                     1.6858 \\
                       1.8    &
                            1.7629 \\\bottomrule
              \end{tabular}
          \end{table}
          \begin{enumerate}
              \item Derive three-point forward and backward difference formulae for the first and second derivatives using two-point first derivative formula.\\
                    Use three-point formulae to estimate the values of $ f' (1.4) $, $ f' (1.8) $, $ f''(1.4 ) $ and $ f''(1.8 ) $.
              \item Derive five-point central difference formula for $ f' (x_0) $ and $ f''(x_0) $ using three-point central difference formula with Richardson extrapolation.\\
                    Use five-points formulae to estimate the values of $ f' (1.6) $ and $ f''(1.6) $.
          \end{enumerate}
          [The table is constructed for $ f(x)=x \sin x $]
    \item Estimate $ f' (1.2) $ and $ f''(1.2) $ using Richardson extrapolation for the following data:
          \begin{table}
              \centering
              \begin{tabular}{cS[table-format=1.4]}
                  \toprule
                  $ x $    &
                  \multicolumn{1}{c}{$ f(x) $} \\\midrule 
                   0.8    &
                   0.9548 \\
                    1.0    &
                    1.6487 \\
                     1.2    &
                     2.6239 \\
                      1.4    & 
                      3.9470 \\
                      1.6    &
                          5.6974 \\\bottomrule
              \end{tabular}
          \end{table}
          Compare your result with the exact value  $ f' (1.2) = 5.6850 $  and $ f''(1.0) $ correct to 4 decimal places.


              [The table is constructed for $ f(x)=x^2 e^{\frac{x}{2}} $]
    \item Use the following table of values of  $ f(x) $ to estimate $ f' (1.0) $ and $ f''(1.0) $ by using three point central difference formulae with Richardson extrapolation.
    \begin{table}
        \begin{minipage}[c]{0.48\linewidth}
            \centering
            \begin{enumerate}[label=(a)]
                \item \hfill
                      \begin{tabular}{cS[table-format=1.3]}
                          \toprule
                          $ x $ & \multicolumn{1}{c}{$ f(x) $} \\\midrule
                          0.7   & 1.297                        \\
                          0.8   & 1.597                        \\
                          1.0   & 2.287                        \\
                          1.2   & 3.094                        \\
                          1.3   & 3.536                        \\\bottomrule
                      \end{tabular}
            \end{enumerate}
        \end{minipage}\hfill
        \begin{minipage}[c]{0.48\linewidth}
            \centering
            \begin{enumerate}[label=(b)]
                \item \hfill
                      \begin{tabular}{cS[table-format=1.3]}
                          \toprule
                          $ x $ & \multicolumn{1}{c}{$ f(x) $} \\\midrule
                          0.7   &
                          1.297                                \\
                          0.9   &
                          1.927                                \\
                          1.0   &
                          2.287                                \\
                          1.1   &
                          2.677                                \\
                          1.3   &
                          3.536                                \\\bottomrule
                      \end{tabular}
            \end{enumerate}
        \end{minipage}
    \end{table}
          [The table is constructed for $ f(x)=e^x  \sin x $]
    \item The voltage $ E=E(t) $ in an electric circuit obeys the differential equations
          \[
              E(t)=L \frac{\D I}{\D t}+RI,
          \]
          where $ R $ is the resistance and $ L $ is the inductance. Use $ L =0.05 $ and $ R = 2 $ and the $ I(t) $ in the table
          \begin{table}[H]
              \centering
              \begin{tabular}{cS[table-format=2.4]}
                \toprule
                $ t $    & \multicolumn{1}{c}{$ I(t) $} \\\midrule
                 1.0    &8.2277 \\
                  1.1    &7.2428 \\
                   1.2    &5.9908 \\
                    1.3    &4.5260 \\
                     1.4    &2.9122 \\\bottomrule
            \end{tabular}
          \end{table}
          \begin{enumerate}
              \item Find $ I' (1.2) $ by numerical differentiation as accurately as possible and use it to compute $ E(1.2) $.
              \item Compare your result with the exact solution $ I(t)=10 e^{\frac{-t}{10}}\sin 2 t $.
          \end{enumerate}
    \item Using Taylor series expansion derive three-point forward derivative formulas for the  first and second derivatives.\\
          In each case estimate the order of the errors.
    \item The table below gives the values of the distance traveled by a car at various time intervals during its journey
          \begin{table}[H]
              \centering
              \begin{tabular}{cS[table-format=3.2]}
                \toprule
                Time, $ t $ (min) &\multicolumn{1}{c}{Distance traveled $ s(t) $ (km)} \\\midrule
                 4   &7.5 \\
                  5    &11.0 \\
                   6    &15.0 \\
                    7    &19.5 \\
                     8    &24.0 \\\bottomrule
            \end{tabular}
          \end{table}
          Estimate the velocity, $ v=\frac{\D s}{\D t} $ and acceleration $ v=\frac{\D^2 s}{\D t^2 } $, at time $ t=4 $, $ t=6 $ and $ t=8 $ using three-point formulae with Richardson extrapolation.\label{en:7}
    \item The following data shows the distance of a particle from a fixed point at different time.
          \begin{table}[H]
              \centering
              \begin{tabular}{cS[table-format=3.0]}
                  \toprule
                  Time, $ t $ (sec)&
                  \multicolumn{1}{c}{Distance $ s(t) $  $ (m) $} \\\midrule
                   4  &
                   16 \\
                    5  &
                    20 \\
                     8   &
                     128 \\
                      10  &
                          340 \\\bottomrule
              \end{tabular}
          \end{table}
          \begin{enumerate}
              \item Find the initial velocity and acceleration of the particle.
              \item Find the velocity and acceleration of the particle at time $ t=6 $ and $ t=8 $.
              \item Find the times when the particle is at rest.
          \end{enumerate}
    \item Find the velocity and acceleration at time $ t = 4.5 $ and $ 6.5 $ using 3 and 4 suitable points for the data of \ref{en:7}
    \item Estimate the first and second derivatives at $ x=3 $ and at $ x=6 $ for the function represented by the following tabular data.
          \begin{table}[H]
              \centering
              \begin{tabular}{cS[table-format=2.3]}
                  \toprule
                  $ x $    &
                  \multicolumn{1}{c}{$ f(x) $} \\\midrule 
                   2     &
                   2.704 \\
                    3     &
                    4.841 \\
                    5     &
                    9.625 \\
                    8      &
                    17.407 \\\bottomrule
              \end{tabular}
          \end{table}
\end{enumerate}
\end{document}