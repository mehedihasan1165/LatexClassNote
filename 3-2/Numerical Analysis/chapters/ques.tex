\documentclass[12pt,class=book,crop=false]{standalone}
\usepackage{../style}
\graphicspath{ {../img/} }
\begin{document}
\chapter*{Questions from Previous Years}
\section*{2012-2013 (2015)}
\begin{enumerate}
    \item Marks : $ 4+4+6=14 $
          \begin{enumerate}
              \item Derive Newton-Raphson formula using Taylor's series expansion to solve a nonlinear equation.
              \item Show that Newton-Raphson method converges quadratically.
              \item Find a root, correct to three decimal places, of the equation $ \cos x-xe^x=0 $ using Regula-Falsi method, and then find its percentage error.
          \end{enumerate}
    \item Marks : $ (6+1)+7=14 $
          \begin{enumerate}
              \item Obtain Newton's formula for backward interpolation. Discuss its drawback if any compare to Lagrange interpolation formula.
              \item Find the Lagrange interpolation polynomial to fit the following data:
                    \begin{table}[H]
                        \centering
                        \begin{tabular}{|c|c|c|c|c|}
                            \hline
                            $ x $    & $ 0 $ & $ 1 $      & $ 2 $      & $ 3 $       \\\hline
                            $ f(x) $ & $ 0 $ & $ 1.7183 $ & $ 6.3891 $ & $ 19.0855 $ \\\hline
                        \end{tabular}
                    \end{table}
          \end{enumerate}
    \item Marks : $ 7+7=14 $
          \begin{enumerate}
              \item Describe the Gauss-Seidal iterative method to solve a system of linear equations numerically. Write down the conditon that for any choice of the first approximation of the Gauss-Seidal method converges.
              \item Solve the tri-diagonal system:
                    \begin{table}[H]
                        \centering
                        \begin{tabular}{rcrcrcll}
                            $ 0.5 x_1 $  & $ + $ & $ 0.25x_2 $ &       &            &       &            & $ =0.35 $  \\
                            $ 0.35x_1 $ & $ + $ & $ 0.8x_2 $  & $ + $ & $ 0.4x_3 $ &       &            & $ =0.77 $  \\
                                        &       & $ 0.25x_2 $ & $ + $ & $ x_3 $    & $ + $ & $ 0.5x_4 $ & $ =-0.5 $  \\
                                        &       &             &       & $ x_3 $    & $ - $ & $ 2.0x_4 $ & $ =-2.25 $
                        \end{tabular}
                    \end{table}
                    % \begin{align*}
                    %     0.5x_1+0.25x_2&=0.35\\
                    %     0.35x_1+0.8x_2+0.4x_3&=0.77\\
                    %     0.25x_2+x_3+0.5x_4&=-0.5\\
                    %     x_3-2.0x_4&=-2.25
                    % \end{align*}
          \end{enumerate}
    \item Marks : $ 2+4+8=14 $
          \begin{enumerate}
              \item What are spline functions and spline interpolation? Discuss briefly.
              \item Derive Newton's general interpolation formula with divided difference.
              \item Fit cubic splines to the following data, and utilize the results to estimate the value at $ x=5 $.
                    \begin{table}[H]
                        \centering
                        \begin{tabular}{|c|c|c|c|c|}
                            \hline
                            $ x $    & $ 2.0 $ & $ 4.5 $ & $ 7.0 $ & $ 9.0 $ \\\hline
                            $ f(x) $ & $ 2.5 $ & $ 1.0 $ & $ 2.5 $ & $ 0.5 $ \\\hline
                        \end{tabular}
                    \end{table}
          \end{enumerate}
    \item Marks : $ 8+6=14 $
          \begin{enumerate}
              \item Derive general quadrature formula for equidistant ordinates to integrate a function numerically and hence deduce Weddle's rule.
              \item Compute the value of the integral $ \displaystyle \int_0^3 \left( e^{x^2}-1 \right)\dx,\,n=6 $ by Trapezoidal rule and Simpson's $ {}^1/{}_3 $ rule.
          \end{enumerate}
    \item Marks : $ 7+7=14 $
          \begin{enumerate}
              \item Derive a formula for numerical evaluation of $ \ddx{y}  $ and $ \ddxn{2}{y} $ from a given data set of $ (x,y) $.
              \item The deflection $ y $, measured at various distance $ x $ from one end of a cantilever, is given by
                    \begin{table}[H]
                        \centering
                        \begin{tabular}{|c|c|c|c|c|c|c|}
                            \hline
                            $ x: $ & $ 0.0 $    & $ 0.2 $    & $ 0.4 $    & $ 0.6 $    & $ 0.8 $    & $ 1.0 $    \\\hline
                            $ y: $ & $ 0.0000 $ & $ 0.0347 $ & $ 0.1173 $ & $ 0.2160 $ & $ 0.2987  $& $ 0.3333 $ \\\hline
                        \end{tabular}
                    \end{table}
                    Find $ f'(0.8) $, where $ y=f(x) $.
          \end{enumerate}
    \item Marks : $ 7+7=14 $
          \begin{enumerate}
              \item Explain the Euler's method to solve the ordinary differential equation $ \ddx{y}=f(x,y),\,y(x_0)=y_0 $.
              \item Use the 4th order Runge-Kutta method to solve $ 10\ddx{y}=x^2+y^2,\,y(0)=1 $, for the interval $ 0\leq x\leq 0.4 $ with $ h=0.2 $.
          \end{enumerate}
    \item Marks : $ 7+7=14 $
          \begin{enumerate}
              \item Obtain finite difference formula for $ \frac{\partial u}{\parx} $ and $ \frac{\partial^2 u }{\partial x^2} $. Derive explicit finite difference scheme to solve the BVP:
                    \[
                        u''(x)+p(x)u'(x)+q(x)u(x)+r(x)=0
                    \]
                    with boundary conditions $ u(x_0)=a $ and $ u(x_n)=b $, $ x_0\leq x\leq x_n $.
              \item Solve the boundary value problem $ y''-y=0 $ with boundary conditions $ y(0)0=0 $ and $ y(2)=3.627 $ by the finite difference method with $ h=0.5 $.
          \end{enumerate}
\end{enumerate}
\newpage
\section*{2015-2016 (2018)}
\begin{enumerate}
    \item Marks: $ (2+2)+6+4=14 $
    \begin{enumerate}
        \item Define algebraic and transcendental equations. Discuss how the intermediate value theorem exhibits the bracketing interval.
        \item Describe the method of iteration for finding a root of the equation $ f(x) = 0 $. Also establish the condition of convergence of this method.
        \item Find a real root, correct to three decimal places, of the equation $ \sin^2 x = x^2 - 1 $ lying in the interval $ [1.3, 1.5] $ by using iteration method.
    \end{enumerate}
    \item Marks: $ 7+7=14 $
    \begin{enumerate}
        \item Write down the iteration formula of Newton-Raphson method to find a root of the equation $ f(x) = 0 $. Then show that Newton-Raphson method converges quadratically.
        \item Solve the equation $ e^x -x-2 = 0 $ by Newton-Raphson method.
    \end{enumerate}
    \item Marks: $ 5+4+5=14 $
    \begin{enumerate}
        \item Define interpolation and extrapolation. Derive the Newton's formula for the backward interpolation.
        \item Derive the first fourth backward difference formula and construct its backward difference table.
        \item The population of a town in the decennial census was as given below.
        \begin{table}[H]
            \centering
            \begin{tabular}{|c|c|c|c|c|}
                \hline
                Year: $ x $& $ 1981 $ &$ 1991 $ &$  2001 $ &$ 2011 $\\\hline
                Population: $ y $ (in thousands)& $ 46 $& $ 66 $ &$ 81 $&$ 93 $\\\hline
            \end{tabular}
        \end{table}
        Estimate the population for the year $ 1985 $.
    \end{enumerate}
    \item Marks: $ 7+7=14 $
    \begin{enumerate}
        \item Derive the Lagrange's interpolation formula for unequal intervals.
        \item Given the following data:
        \begin{table}[H]
            \centering
            \begin{tabular}{|c|c|c|c|c|c|}
                \hline
                $ x $& $ 10.1 $& $ 22.2 $&$  32.0 $&$  41.6 $&$  50.5 $\\\hline               
                $ f(x) $ & $ 0.17537 $& $ 0.37784 $& $ 0.52992 $&$ 0.66393 $& $ 0.63608 $\\\hline
            \end{tabular}
        \end{table}        
        Estimate the value of $ f(27.5) $ by using divided difference formula.
    \end{enumerate}
    \item Marks: $ 7+7=14 $
    \begin{enumerate}
        \item Derive general quadrature formula for equidistance ordinates to integrate $ f(x) $ numerically and hence deduce Simpson's rule.
        \item Evaluate the value of the integral $\displaystyle \int_{0.2}^{1.4} \left( \cos x- \log_e x+ e^x \right) \dx $ by 
        \begin{enumerate}[noitemsep]
            \item Simpson's $ {}^{1}/{}_3 $ rule and
            \item Weddle's rule.
        \end{enumerate}
    \end{enumerate}
    \item Marks: $ 8+6=14 $
    \begin{enumerate}
        \item Derive cubic spline interpolating method.
        \item Fit a natural cubic spline to the following data:
        \begin{table}[H]
            \centering
            \begin{tabular}{|c|c|c|c|}
                \hline
                $ x $& $ 1 $&$  2 $& $ 3 $\\\hline               
                $ y $ & $ -8 $& $ -1 $& $ 18 $\\\hline
            \end{tabular}
        \end{table}
        and compute $ y(1.5) $ and $ y'(1) $.
    \end{enumerate}
    \item Marks: $ 7+7=14 $
    \begin{enumerate}
        \item Use the decomposition method to solve the system of equations
        \begin{align*}
            x_1 + x_2 + x_3 &= 1\\
            4x_1 +3x_2 - x_3 &= 6\\
            3x_1 + 5x_2 + 3x_3 &= 4.
        \end{align*}
        \item Use Gauss-Seidel iteration method to solve the following system of equations:
        \begin{align*}
            4x +y + 2z &= 4\\
            3x + 5y +z &= 7\\
            x +y+ 3z &= 3
        \end{align*}
        up to four decimal places.
    \end{enumerate}
    \item Marks: $ 7+7=14 $
    \begin{enumerate}
        \item Derive Euler's method to solve the IVP:
        \[
            \ddx{y} = f(x, y),\quad y(x_0) = y_0
        \]
        Also explain its modification.
        \item Solve the initial value problem:
        \[
            \frac{\D u}{\D t} =-2tu^2,\quad u(0) = 1
        \]
        with $ h = 0.2 $ on the interval $ [0,0.4] $ by using the fourth-order classical Runge-Kutta method.
    \end{enumerate}
\end{enumerate}
\newpage
\section*{2016-2017 (2019)}
\begin{enumerate}
    \item Marks: $ 5+4+5 $
    \begin{enumerate}
        \item What is meant by error in numerical analysis? Explain the following error:\\
        Truncation error, round-off error and inherent error.
        \item Define bracketing interval. Derive the formula for the chord method to find a real root of a transcendental equation.
        \item Find a root of the equation $ x \log_{10} x = 4.77 $ by Newton-Raphson method, correct to three decimal places.
    \end{enumerate}
    \item Marks: $ 7+7 $
    \begin{enumerate}
        \item Derive the close form formulae to solve the tridiagonal system of linear equations:
        \[
            a_r x_{r-1} + b_rx_r+ c_rx_{r +1} = dr,\quad a_1 = c_n = 0 \quad \text{for } r = 1,2, 3,\dots, n
        \]
        \item Apply above formulae to solve the following system of linear equations:
        \begin{align*}
            0.5x_1 +0.25x_2 &= 0.35\\
            0.35x_1 +0.8x_2 +0.4x_3 &= 0 . 77\\
            0. 25x_2 + x_3 + 0.5x_4 &= -0.50\\
            x_3 -2. 0x_4& = -2.25
        \end{align*}
    \end{enumerate}
    \item Marks: $ 4+5+5 $
    \begin{enumerate}
        \item Discuss Gauss-Seidel and Gauss-Jacobi iteration methods to find solutions of the system of linear equations $ Ax = b $ and write their advantages and disadvantages.
        \item Use Gauss-Seidel iteration method to find the solutions, correct up to 3 decimal places of the following system of linear equations:
        \begin{align*}
            4x_1 + x_2 + 2x_3 &= 4\\
            x_1 + x_2 + 3x_3 &= 3\\
            3x_1+ 5x_2 + x_3 &=7
        \end{align*}
        \item Solve the following system of Linear equations:
        \begin{align*}
            5x - 2y + z &= 4\\
            7x + y - 5z &= 8\\
            3x + 7y+4z &= 10
        \end{align*}
        by Crout's reduction method.
    \end{enumerate}
    \item Marks: $ 6+8 $
    \begin{enumerate}
        \item Define interpolation and extrapolation. Derive the expression for the error in polynomial interpolation.
        \item The following table gives the population of a town during the last six censuses. Estimate using any suitable interpolation formula, the increase in the population during the period from $ 1976 $ to $ 1978 $.
        \begin{table}[H]
            \centering
            \begin{tabular}{|c|c|c|c|c|c|c|}
                \hline
                Year& $ 1941 $& $ 1951 $& $ 1961 $& $ 1971 $& $ 1981 $& $ 1991 $\\\hline               
                Population (in thousand)& $ 12 $&$ 15 $&$ 20 $&$ 27 $&$ 39 $&$ 52 $\\\hline
            \end{tabular}
        \end{table}
    \end{enumerate}
    \item Marks: $ 7+7 $
    \begin{enumerate}
        \item Discuss briefly the properties of cubic spline and defines its knots. Fits a cubic spline curve that passes through $ (0, 0.0) $, $ (1, 0.5) $. $ (2,2.0) $ and $ (3,1.5) $ with the natural-end boundary conditions.
        \[
            S''(0) = 0,\quad S''(3) = 0
        \]
        \item Find interpolating polynomial for the following data using Lagrange's formula
        \begin{table}[H]
            \centering
            \begin{tabular}{|c|c|c|c|}
                \hline
                $ x $& $ 1 $& $ 2 $& $ -4 $\\\hline               
                $ y=f(x) $ & $ 3 $& $ -5 $& $ 4 $\\\hline
            \end{tabular}
        \end{table}
        Hence, find $ f(2.25) $.
    \end{enumerate}
    \item Marks: $ 7+7 $
    \begin{enumerate}
        \item  Derive general quaderature formula to evaluate the integral $ \displaystyle I=\int_a^b y(x)\dx $, hence deduce the trapezoidal, Simpsons $ {}^1/{}_3 $ and Simpson's $ {}^3/{}_8 $ formula to find $ I $.
        \item Compute the value of the definite integral $ \displaystyle \int_0^{\frac{\pi}{2}}e^{\sin x}\dx $ by Simpson's $ {}^3/{}_8 $ rule and Weddle's rule. After finding the true value of the integral, compare the errors in both cases and comment which method is better. 
    \end{enumerate}
    \item Marks: $ 7+7 $
    \begin{enumerate}
        \item Define initial value problem. Derive Euler's method to solve the IVP.
        \[
            \ddx{y}= (fx, y),\quad y(x_0) =y_0
        \]
        Also explain its modification.
        \item Using modified Euler's method, obtain the solution of the differential equation $ \ddx{y}=t+\sqrt{y} $ with the initial conditon $ y (0) = 1 $, for the range $ 0\leq t\leq 0.6 $ in steps of $ 0.2 $.
    \end{enumerate}
    \item Marks: $ 6+8 $
    \begin{enumerate}
        \item Use fourth order Runge-Kutta method to solve numerically the initial value problem
        \[
            \frac{\dy}{\D t}=y^2-100e^{-100(t-1)^2},\quad y(0.8)=4.9491
        \]
        and find $ y(0.85) $ taking $ h = 0.01 $.
        \item Find $ y(0.8) $ using Milne's predictor-corrector Method, if $ y(x) $ is the solution of the differential equation $ \ddx{y}=-xy^2 $, $ y(0)=2 $ assuming $ y(0.2) = 1.92308$, $Y(0.4) = 1.72414 $ and $ y(0.6) =1.47059 $.
    \end{enumerate}
\end{enumerate}
\end{document}