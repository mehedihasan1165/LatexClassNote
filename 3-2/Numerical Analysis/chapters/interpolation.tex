\documentclass[12pt,class=book,crop=false]{standalone}
\usepackage{../style}
\graphicspath{ {../img/} }
\begin{document}
\chapter{Interpolation}
\section{Interpolation}
This is the technique of obtaining the most likely estimate of a certain quantity under certain assumption.

Suppose the values if a function \(  f(x) \) are given for a discrete set of values of \(  x \) (independent variable). Interpolation is defined as the method of estimating the values of \(  f(x) \) for any intermediate value of the arguments \(  x_i (i=0,1,2,\dots ,n) \).

Let us suppose we are given the census figures for the population of Bangladesh for four years \(  1931, 1941, 1951 \) and \(  1961 \) and we want to estimate the figures for any intermediate year, e.g. for \(  1955 \) or \(  1958 \) etc. This can be done by applying the technique of interpolation.
\section{Extrapolation}
If we have to estimate the value of \(  f(x) \) for any value outside the given range then the technique is known as extrapolation.
\section{Fundamental Assumptions}
\begin{enumerate}
    \item There are no sudden jumps or falls in the values of the data; i.e. data can be represented by a smooth continuous curve.

          Data can be represented by a polynomial of certain degree, which can be determined by,
          \begin{thm}
              One and only one polynomial (curve) of degree less than or equal to \(  n \) passes through a given set of \(  (n+1) \) distinct points.
          \end{thm}
    \item The data can be expressed as a polynomial function with fair degree of accuracy.
    \item The rise or fall of data is uniform.
    \item The method is not exact.
    \item The method becomes complicated when the numbers of observations (data) is large.
    \item This method gives closer approximation than the graphical method.
\end{enumerate}
\begin{thm}[Existance and Uniqueness]
    If \(  x_0,x_1,x_2,\dots,x_n \) are \(  (n+1) \) distinct points and \(  f(x) \) is a function whose values are known at these points; then there exists a unique polynomial \(  P_n(x) \) of \( \text{degree}  \leq n \) which interpolates \(  f(x) \) such that \(  P_n(x_i)=f(x_i),\; i=0,1,2,\dots ,n \) and
    \[P_n(x)=\sum_{k=0}^n f(x_k)l_k(x)\]
\end{thm}
\begin{proof}[Proof (Existance)]
    Let \(  x_0,x_1,\dots,x_n \) be \(  (n+1) \) distinct points on the real axis and let \(  f(x) \) be a real values function defined on some interval \(  [a,b] \) containing these points. We need to construct a polynomial \(  P_n(x) \) of degree \(  \leq n \) which interpolates \(  f(x) \) at the \(  (n+1) \) points and satisfies \(  P_n(x_i)=f(x_i),\; i=0,1,2,\dots ,n \).

    For this, we use the form, called Lagrange's form:
    \[P_n(x)=\sum_{k=0}^na_kl_k(x)\] where \[l_k(x)=\prod_{\substack{i=0\\ i\neq k}}^n\frac{x-x_i}{x_k-x_i},\,k=0,1,2,\dots,n\]are called the Lagrange polynomials for the points \(  x_0,x_1,\dots,x_n \).

    This function \(  l_k(x) \) is the product of \(  n \) linear factors, hence gives a polynomial of exact degree \(  n \).  Hence, the Lagrange form given by\[P_n(x)=\sum_{k=0}^na_kl_k(x)\]describes a polynomial of degree \(  \leq n \).

    Also, \(  l_k(x) \) vanishes at \(  x_i \) for all \(  i\neq k \) and takes the value \(  1 \) at \(  x_k \), ie,
    \[l_k(x_i)=\begin{cases}
            1, \,\,\, i=k \\
            0, \,\,\, i\neq k
        \end{cases}\quad i=0,1,2,\dots,n\]
    \begin{align*}
        \therefore P_n(x_i) & = \sum_{k=0}^n a_kl_k(x_i)                                                           \\
                            & = a_0l_0(x_i)+ a_1l_1(x_i)+\dots+ a_il_i(x_i)+ a_{i+1}l_{i+1}(x_i)+\dots+ a_nl_n(x_i) \\
                            & = 0+0+\dots+a_i\times 1+0+\dots                                                      \\
        P_n(x_i)            & =a_i\quad (i=0,1,2,\dots ,n)
    \end{align*}
    The co-efficient \(  a_0,a_1,\dots,a_n \) are simply the values of the polynomials \(  P_n(x) \) at the points \(  x_0,x_1,\dots,x_n \).

    Consequently, for any function \(  f(x) \),\[P_n(x)=\sum_{k=0}^nf(x_k)l_k(x)\]is a polynomial of degree \(  \leq n \) which interpolates \(  f(x) \) at \(  x_0,x_1,\dots,x_n \).
\end{proof}
\newpage
\begin{proof}[Proof (Uniqueness)]
    Let \(  P_n(x) \) and \(  q_n(x) \) are two Lagrange's interpolating polynomials of degree \(  \leq n \) which interpolates \(  f(x) \) at \(  x_0,x_1,\dots,x_n \), then we have,
    \begin{equation}
        P_n(x_i)=f(x_i) \label{eq:uniq1}
    \end{equation}
    \begin{equation}
        q_n(x_i)=f(x_i)\label{eq:uniq2}
    \end{equation}
    Consider the polynomial \(  p_n(x) \) given by
    \[\phi_n(x)=P_n(x)-q_n(x)\]
    Then \(  \phi_n(x) \) vanishes at the points \(  x_0,x_1,\dots,x_n \).\\
    Hence we have \(  \phi_n(x)\equiv 0 \)\\
    \(  \Rightarrow P_n(x)=q_n(x) \)\\
    Thus Lagrange interpolating polynomial is unique.
\end{proof}
\begin{prob}
    Find the linear form of Lagrange's interpolation polynomial.
\end{prob}
\begin{soln}
    We know, the Lagrange's polynomial is
    \[P_n(x)=\sum_{k=0}^nf(x_k)l_k(x)\]
    where,
    \[l_k(x)=\prod_{\substack{i=0\\ i\neq k}}^n\frac{x-x_i}{x_k-x_i}\]
    Put \(  n=1 \), then we have only two distinct points namely \(  x_0,x_1 \).\\
    Now\[l_0(x)=\prod_{\substack{i=0\\ i\neq 0}}^1\frac{x-x_i}{x_0-x_i}=\frac{x-x_1}{x_0-x_1}\]
    and,\[l_1(x)=\prod_{\substack{i=0\\ i\neq 1}}^1\frac{x-x_i}{x_1-x_i}=\frac{x-x_0}{x_1-x_0}\]
    Now,
    \begin{align*}
        P_1(x) & = \sum_{k=0}^1f(x_k)l_k(x)                                                    \\
               & = f(x_0)l_0(x)+f(x_1)l_1(x)                                                   \\
               & = f(x_0)\frac{x-x_1}{x_0-x_1}+f(x_1)\frac{x-x_0}{x_1-x_0}                     \\
               & = f(x_0)\frac{x_1-x}{x_1-x_0}+f(x_1)\frac{x-x_0}{x_1-x_0}                     \\
               & =f(x_0)\left(\frac{x_1-x_0+x_0-x}{x_1-x_0}\right)+f(x_1)\frac{x-x_0}{x_1-x_0} \\
               & = f(x_0)\left(1+\frac{x_0-x}{x_1-x_0}\right)+f(x_1)\frac{x-x_0}{x_1-x_0}      \\
               & = f(x_0)+ f(x_0) \frac{x-x_0}{x_1-x_0}+f(x_1)\frac{x-x_0}{x_1-x_0}            \\
               & = f(x_0)+\frac{f(x_1)-f(x_0)}{x_1-x_0}\left(x-x_0\right)
    \end{align*}
    ie, \(  P_1(x)= f(x_0)+\frac{f(x_1)-f(x_0)}{x_1-x_0}\left(x-x_0\right) \)\\

    This is the linear form of Lagrange's interpolation polynomial.
\end{soln}
\begin{prob}
    Suppose \(  f(x)=\frac{1}{x}, x_0=2, x_1=2.5, x_2=4 \). Find the second degree interpolating polynomial and determine the coefficients \(  l_0,l_1,l_2 \) and hence find an approximate value of \(  f(3) \).
\end{prob}
\begin{soln}
    We know, the Lagrange's polynomial is
    \begin{equation}
        P_n(x)=\sum_{k=0}^nf(x_k)l_k(x)\label{eq:ex1.1}
    \end{equation}
    where
    \begin{equation}
        l_k(x)=\prod_{\substack{i=0\\ i\neq k}}^n\frac{x-x_i}{x_k-x_i},\; k=0,1,\dots,n\label{eq:ex1.2}
    \end{equation}
    In this case we are given \(  x_0.x_1,x_2 \)\\
    So that \(  n=2 \)\\
    Thus from \ref{eq:ex1.2} we have
    \[l_k(x)=\prod_{\substack{i=0\\ i\neq k}}^2\frac{x-x_i}{x_k-x_i},\; k=0,1,2\]
    \begin{align*}
        l_0(x) & = \prod_{\substack{i=0                            \\ i\neq 0}}^2\frac{x-x_i}{x_0-x_i}\\
               & =\frac{x-x_1}{x_0-x_1}\cdot \frac{x-x_2}{x_0-x_2} \\
               & = \frac{x-2.5}{2-2.5}\cdot \frac{x-4}{2-4}        \\
               & = \frac{x-2.5}{-0.5}\cdot \frac{x-4}{-2}          \\
               & = x^2-6.5x+10
    \end{align*}
    \begin{align*}
        l_1(x) & = \prod_{\substack{i=0                            \\ i\neq 1}}^2\frac{x-x_i}{x_1-x_i}\\
               & =\frac{x-x_0}{x_1-x_0}\cdot \frac{x-x_2}{x_1-x_2} \\
               & = \frac{x-2}{2.5-2}\cdot \frac{x-4}{2.5-4}        \\
               & = \frac{x-2}{0.5}\cdot \frac{x-4}{-1.5}           \\
               & = \frac{x^2-6x+8}{-0.75}                          \\
               & = -\frac{1}{3}(4x^2-24x+32)
    \end{align*}
    \begin{align*}
        l_2(x) & = \prod_{\substack{i=0                            \\ i\neq 2}}^2\frac{x-x_i}{x_2-x_i}\\
               & =\frac{x-x_0}{x_2-x_0}\cdot \frac{x-x_1}{x_2-x_1} \\
               & = \frac{x-2}{4-2}\cdot \frac{x-2.5}{4-2.5}        \\
               & = \frac{x-2}{2}\cdot \frac{x-2.5}{1.5}            \\
               & = \frac{1}{3}(x^2-4.5x+5)
    \end{align*}
    Now we are given that \(  f(x)=\frac{1}{x} \)\\
    So writing the following table we get\\
    \begin{center}
        \begin{tabular}{cr@{ = }Sr@{ = }S}
            \toprule
            \(i\) & \multicolumn{2}{c}{\(  x_i \)} & \multicolumn{2}{c}{\(  f(x)=\frac{1}{x} \)}                       \\\midrule
            \(0\) & \(  x_0\)                      & 2                                           & \(  f(x_0)\) & 0.5  \\
            \(1\) & \(  x_1\)                      & 2.5                                         & \(  f(x_1)\) & 0.4  \\
            \(2\) & \(  x_2\)                      & 4                                           & \(  f(x_2)\) & 0.25 \\\bottomrule
        \end{tabular}
    \end{center}
    Now, the Lagrange's polynomial is
    \[P_n(x)=\sum_{k=0}^nf(x_k)l_k(x)\]
    Here
    \begin{align*}
        P_2(x) & =P_n(x)=\sum_{k=0}^2f(x_k)l_k(x)                                                 \\
               & =f(x_0)l_0(x)+f(x_1)l_1(x)+f(x_2)l_2(x)                                          \\
               & = (0.5)(x^2-6.5x+10)-(0.4)\frac{1}{3}(4x^2-24x+32)+(0.25)\frac{1}{3}(x^2-4.5x+5) \\
               & =0.05x^2-0.425x+1.15
    \end{align*}
    \(  \therefore P_2(x)=0.05x^2-0.425x+1.15\approx f(x) \)\\
    Which is the interpolating polynomial of 2nd degree.\\
    \(  \therefore f(3)=(0.05)(3)^2-(0.425)(3)+1.15=0.325\approx 0.33 \)\\
\end{soln}
\begin{prob}
    Use the following table for the function \(  f(x)=\log_{10}(\tan x) \), to find the interpolating polynomial to estimate the value of \(  f(1.09) \)
    \begin{center}
        \begin{tabular}{cS[table-format=2.4]}
            \toprule
            \(  x \)    & \multicolumn{1}{c}{\(  f(x) \)} \\\midrule
            \(  1.00 \)   &  0.1924  \\
            \(  1.05 \)  &   0.2414 \\
            \(  1.10 \)   &  0.2933  \\
            \(  1.15 \)   &  0.3492  \\\bottomrule
        \end{tabular}
    \end{center}
\end{prob}
\newpage
\begin{soln}
    We know, the Lagrange's interpolation polynomial is
    \[
        P_n(x)=\sum_{k=0}^nf(x_k)l_k(x)
    \]
    where
    \[
        l_k(x)=\prod_{\substack{i=0\\ i\neq k}}^n\frac{x-x_i}{x_k-x_i}, k=0,1,\dots,n
    \]
    Now,\begin{align*}
        l_0(x) & = \prod_{\substack{i=0                                                      \\ i\neq 0}}^3\frac{x-x_i}{x_0-x_i}\\
               & = \frac{x-x_1}{x_0-x_1}\cdot\frac{x-x_2}{x_0-x_2}\cdot\frac{x-x_3}{x_0-x_3} \\
               & = \frac{x-1.05}{1-1.05}\cdot\frac{x-1.10}{1-1.10}\cdot\frac{x-1.15}{1-1.15} \\
               & = \frac{(x^2-2.15x+1.155)(x-1.15)}{(-0.05)(-0.10)(-0.15)}                   \\
               & = -\frac{x^3-3.30x^2+3.6275x-1.328\,25}{0.000\,75}                              \\
               & = -\frac{1}{0.000\,75}(x^3-3.3x^2+3.6275x-1.328\,25)
    \end{align*}
    \begin{align*}
        l_1(x) & = \frac{x-x_0}{x_1-x_0}\cdot\frac{x-x_2}{x_1-x_2}\cdot\frac{x-x_3}{x_1-x_3}  \\
               & = \frac{x-1}{1.05-1}\cdot\frac{x-1.1}{1.05-1.1}\cdot\frac{x-1.15}{1.05-1.15} \\
               & = \frac{(x^2-2.1x+1.1)(x-1.15)}{(0.05)(-0.05)(-0.1)}                         \\
               & = \frac{1}{0.000\,25}(x^3-3.25x^2+3.515x-1.265)
    \end{align*}
    \begin{align*}
        l_2(x) & = \frac{x-x_0}{x_2-x_0}\cdot\frac{x-x_1}{x_2-x_1}\cdot\frac{x-x_3}{x_2-x_3} \\
               & = \frac{x-1}{1.1-1}\cdot\frac{x-1.05}{1.1-1.05}\cdot\frac{x-1.15}{1.1-1.15} \\
               & = \frac{(x^2-2.05x+1.05)(x-1.15)}{(0.01)(0.05)(-0.05)}                      \\
               & = -\frac{1}{0.000\,25}(x^3-3.20x^2+3.4075x-1.2075)
    \end{align*}
    \begin{align*}
        l_3(x) & = \frac{x-x_0}{x_3-x_0}\cdot\frac{x-x_1}{x_3-x_1}\cdot\frac{x-x_2}{x_3-x_2}  \\
               & = \frac{x-1}{1.15-1}\cdot\frac{x-1.05}{1.15-1.05}\cdot\frac{x-1.1}{1.15-1.1} \\
               & = \frac{(x^2-2.05x+1.05)(x-1.1)}{(0.15)(0.1)(0.05)}                          \\
               & = \frac{1}{0.000\,75}(x^3-3.15x^2+3.305x-1.155)
    \end{align*}
    So the required polynomial is
    \begin{align*}
        P_3(x)             & = f(x_0)l_0(x)+f(x_1)l_1(x)+f(x_2)l_2(x)+f(x_3)l_3(x)                            \\
                           & = 0.1924\left\{-\frac{1}{0.000\,75}\left(x^3-3.3x^2+3.6275x-1.328\,25\right)\right\} \\
                           & + 0.2414\left\{\frac{1}{0.000\,25}\left(x^3-3.25x^2+3.515x-1.265\right)\right\}    \\
                           & + 0.2933\left\{-\frac{1}{0.000\,25}\left(x^3-3.20x^2+3.4075x-1.2075\right)\right\} \\
                           & + 0.3492\left\{\frac{1}{0.000\,75}\left(x^3-3.15x^2+3.305x-1.155\right)\right\}    \\
        \Rightarrow P_3(x) & = 1.466\,666\,667x^3-4.04x^2+4.638\,333\,33x-1.8726\approx f(x)
    \end{align*}
    \(  \therefore f(1.09)=0.282\,635\,2 \)\\
    Hence the required polynomial is \(  P_3(x) = 1.466\,66x^3-4.04x^2+4.638\,33x-1.8726 \) and \(  f(1.09)=0.2826 \)
\end{soln}
\begin{prob}
    Use the following table, find an approximate value of \(  f(1.25) \) when \(  f(x)=e^{x^2}-1 \)
    \begin{center}
        \begin{tabular}{cS[table-format=3.5]}
            \toprule
            \(  x \)   & \multicolumn{1}{c}{\(  f(x) \)}    \\\midrule
            \(  1.0 \) &  1.00000  \\
            \(  1.1 \) &  1.23368  \\
            \(  1.2 \) &  1.55271  \\
            \(  1.3 \) &  1.49372  \\
            \(  1.4 \) &  2.61170  \\\bottomrule
        \end{tabular}
    \end{center}
\end{prob}
\begin{soln}
    We know, the Lagrange's interpolation polynomial is
    \[
        P_n(x)=\sum_{k=0}^nf(x_k)l_k(x)
    \]
    where
    \[
        l_k(x)=\prod_{\substack{i=0\\ i\neq k}}^n\frac{x-x_i}{x_k-x_i}, k=0,1,\dots,n
    \]
    In this case we are given, \(  x_0,x_1,x_2,\text{ and }x_4 \) so that \(  n=4 \) and we get
    \[
        P_4(x)=\sum_{k=0}^4f(x_k)l_k(x)
    \]
    where
    \[
        l_k(x)=\prod_{\substack{i=0\\ i\neq k}}^4\frac{x-x_i}{x_k-x_i}, k=0,1,2,3,4
    \]
    \begin{align*}
        \therefore l_0(x)     & = \frac{x-x_1}{x_0-x_1}\cdot\frac{x-x_2}{x_0-x_2}\cdot\frac{x-x_3}{x_0-x_3}\cdot\frac{x-x_4}{x_0-x_4}                 \\
        \Rightarrow l_0(1.25) & = \frac{1.25-1.1}{1.00-1.1}\cdot\frac{1.25-1.2}{1.00-1.2}\cdot\frac{1.25-1.3}{1.00-1.3}\cdot\frac{1.25-1.4}{1.00-1.4} \\
                              & = 0.023\,437\,5
    \end{align*}
    \begin{align*}
        \therefore l_1(x)     & = \frac{x-x_0}{x_1-x_0}\cdot\frac{x-x_2}{x_1-x_2}\cdot\frac{x-x_3}{x_1-x_3}\cdot\frac{x-x_4}{x_1-x_4}             \\
        \Rightarrow l_1(1.25) & = \frac{1.25-1.0}{1.1-1.0}\cdot\frac{1.25-1.2}{1.1-1.2}\cdot\frac{1.25-1.3}{1.1-1.3}\cdot\frac{1.25-1.4}{1.1-1.4} \\
                              & = \frac{0.25}{0.1}\cdot\frac{0.05}{-0.1}\cdot\frac{-0.05}{-0.2}\cdot\frac{-0.15}{-0.3}                             \\
                              & = -\frac{0.000\,093\,75}{0.000\,6}                                                                                      \\
                              & = -0.156\,25
    \end{align*}
    \begin{align*}
        \therefore l_2(x)     & = \frac{x-x_0}{x_2-x_0}\cdot\frac{x-x_1}{x_2-x_1}\cdot\frac{x-x_3}{x_2-x_3}\cdot\frac{x-x_4}{x_2-x_4}             \\
        \Rightarrow l_2(1.25) & = \frac{1.25-1.0}{1.2-1.0}\cdot\frac{1.25-1.1}{1.2-1.1}\cdot\frac{1.25-1.3}{1.2-1.3}\cdot\frac{1.25-1.4}{1.2-1.4} \\
                              & = \frac{0.25}{0.2}\cdot\frac{0.15}{0.1}\cdot\frac{-0.05}{-0.1}\cdot\frac{-0.15}{-0.2}                             \\
                              & = \frac{0.000\,281\,25}{0.0004}                                                                                       \\
                              & = 0.703\,125
    \end{align*}
    \begin{align*}
        \therefore l_3(x)     & = \frac{x-x_0}{x_3-x_0}\cdot\frac{x-x_1}{x_3-x_1}\cdot\frac{x-x_2}{x_3-x_2}\cdot\frac{x-x_4}{x_3-x_4}             \\
        \Rightarrow l_3(1.25) & = \frac{1.25-1.0}{1.3-1.0}\cdot\frac{1.25-1.1}{1.3-1.1}\cdot\frac{1.25-1.2}{1.3-1.2}\cdot\frac{1.25-1.4}{1.3-1.4} \\
                              & = \frac{0.25}{0.3}\cdot\frac{0.15}{0.2}\cdot\frac{0.05}{0.1}\cdot\frac{-0.15}{-0.1}                               \\
                              & = \frac{0.000\,281\,25}{0.0006}                                                                                       \\
                              & = 0.468\,75
    \end{align*}
    \begin{align*}
        \therefore l_4(x)     & = \frac{x-x_0}{x_4-x_0}\cdot\frac{x-x_1}{x_4-x_1}\cdot\frac{x-x_2}{x_4-x_2}\cdot\frac{x-x_3}{x_4-x_3}             \\
        \Rightarrow l_4(1.25) & = \frac{1.25-1.0}{1.4-1.0}\cdot\frac{1.25-1.1}{1.4-1.1}\cdot\frac{1.25-1.2}{1.4-1.2}\cdot\frac{1.25-1.3}{1.4-1.3} \\
                              & = \frac{0.25}{0.4}\cdot\frac{0.15}{0.3}\cdot\frac{0.05}{0.2}\cdot\frac{-0.05}{0.1}                                \\
                              & = -\frac{0.000\,093\,75}{0.0024}                                                                                      \\
                              & = -0.039\,062\,5
    \end{align*}
    So writing Lagrange's interpolating polynomial, we get
    \begin{align*}
        P_4                & = f(x_0)l_0(x)+f(x_1)l_1(x)+f(x_2)l_2(x)+f(x_3)l_3(x)+f(x_4)l_4(x) \equiv f(x)      \\
        \Rightarrow f(x)   & = (1.00) l_0(x)+(1.233\,68) l_1(x)+(1.552\,75) l_2(x)+(1.993\,72) l_3(x)+(2.611\,70) l_4(x) \\
                           &                                                                                     \\
        \therefore f(1.25) & = (1.00 )(0.023\,437\,5 )+(1.233\,68)(-0.156\,25)+(1.552\,75) (0.468\,75)                       \\
                           & +(1.993\,72)(0.468\,75)+(2.611\,70)(-0.039\,062\,5)                                           \\
                           & =1.754\,96
    \end{align*}
\end{soln}
\begin{prob}
    If \(  f(x)=e^x \), using the following table
    \begin{center}
        \begin{tabular}{cS[table-format=3.5]}
            \toprule
            \(  x \)   & \multicolumn{1}{c}{\(  f(x) \)} \\\midrule
            \(  0.0 \) & 1.0                             \\
            \(  0.5 \) & 1.64872                         \\
            \(  1.0 \) & 2.71858                         \\
            \(  2.0 \) & 7.38906                         \\\bottomrule
        \end{tabular}
    \end{center}
    \begin{enumerate}
        \item Find the approximate value \(  f(0.25) \) using linear interpolation with \(  x_0=0,\, x_1=0.5 \)
        \item Find the approximate value \(  f(0.75) \) using linear interpolation with \(  x_0=0.5,\, x_1=1.0 \)
        \item Approximate value of \( f(0.25), \,f(0.75) \) using second degree interpolating polynomial with \( x_0=0,x_1=1.0,x_2=2.0 \)
    \end{enumerate}
    Which approximation is better and why?
\end{prob}
\begin{soln}
    We know, the Lagrange's interpolation polynomial is
    \[
        P_n(x)=\sum_{k=0}^nf(x_k)l_k(x)
    \]
    where
    \[
        l_k(x)=\prod_{\substack{i=0\\ i\neq k}}^n\frac{x-x_i}{x_k-x_i}, k=0,1,\dots,n
    \]
    \emph{Part i}\\
    Here, we are given \(  x_0=0,\,x_1=0.5 \)\\So that \(  n=1 \)
    We know, the Lagrange's interpolation polynomial is
    \[
        P_1(x)=\sum_{k=0}^1f(x_k)l_k(x)
    \]
    where
    \[
        l_k(x)=\prod_{\substack{i=0\\ i\neq k}}^1\frac{x-x_i}{x_k-x_i}, k=0,1
    \]
    \begin{align*}
        \therefore l_0(x)     & = \frac{x-x_1}{x_0-x_1}   \\
        \Rightarrow l_0(0.25) & = \frac{0.25-0.25}{0-0.5} \\
                              & = \frac{-0.25}{-0.5}      \\
                              & =0.5
    \end{align*}
    \begin{align*}
        \therefore l_1(x)     & = \frac{x-x_0}{x_1-x_0} \\
        \Rightarrow l_1(0.25) & = \frac{0.25-0}{0.5-0}  \\
                              & =0.5
    \end{align*}
    \begin{align*}
        P_1(x)              & \approx f(x) = f(x_0)l_0(x)+f(x_1)l_1(x) \\
        \Rightarrow f(0.25) & = (1.0)(0.5)+(1.648\,72)(0.5)              \\
                            & =1.324\,36
    \end{align*}
    \emph{Part ii}\\
    Here, we are given \(  x_0=0.5,\,x_1=1.0 \)\\So that \(  n=1 \)
    We know, the Lagrange's interpolation polynomial is
    \[
        P_1(x)=\sum_{k=0}^1f(x_k)l_k(x)
    \]
    where
    \[
        l_k(x)=\prod_{\substack{i=0\\ i\neq k}}^1\frac{x-x_i}{x_k-x_i}, k=0,1
    \]
    \begin{align*}
        \therefore l_0(x) & = \frac{x-x_1}{x_0-x_1} \\
                          & = \frac{x-1.0}{0.5-1.0} \\
                          & = -\frac{1}{5(x-1)}     \\
                          & =-2(x-1)
    \end{align*}
    \begin{align*}
        \therefore l_1(x) & = \frac{x-x_0}{x_1-x_0} \\
                          & = \frac{x-0.5}{1-0.5}   \\
                          & =2(x-.5)
    \end{align*}
    So the linear interpolating polynomial is
    \begin{align*}
        P_1(x) & = f(x_0)l_0(x)+f(x_1)l_1(x)                                       \\
               & = (1.648\,72)\left\{-2(x-1)\right\}+(2.718\,28)\left\{-2(x-1)\right\} \\
               & =2.113\,912x+0.579\,16                                                \\
               & \approx f(x)
    \end{align*}
    \begin{align*}
        \therefore f(0.75) & = (2.139\,12)(0.75)+0.579\,16 \\
                           & =2.183\,50
    \end{align*}
    \emph{Part iii}\\
    Here, we are given \(  x_0=0,\,x_1=1.0 \text{ \& }x_2=2.0 \)\\So that \(  n=2 \)
    We know, the Lagrange's interpolation polynomial is
    \[
        P_2(x)=\sum_{k=0}^2f(x_k)l_k(x)
    \]
    where
    \[
        l_k(x)=\prod_{\substack{i=0\\ i\neq k}}^2\frac{x-x_i}{x_k-x_i}, k=0,1,2
    \]
    \begin{align*}
        \therefore l_0(x) & = \frac{x-x_1}{x_0-x_1}\cdot\frac{x-x_2}{x_0-x_2} \\
                          & = \frac{x-1.0}{0-1.0}\cdot\frac{x-2.0}{1.0-2.0}   \\
                          & =\frac{1}{2}(x^2-3.0x+2.0)
    \end{align*}
    \begin{align*}
        \therefore l_1(x) & = \frac{x-x_0}{x_1-x_0}\cdot\frac{x-x_2}{x_1-x_2} \\
                          & = \frac{x-0.0}{1.0-0.0}\cdot\frac{x-2.0}{1.0-2.0} \\
                          & = -(x^2-2.0x)                                     \\
                          & = 2.0x-x^2
    \end{align*}
    \begin{align*}
        \therefore l_2(x) & = \frac{x-x_0}{x_2-x_0}\cdot\frac{x-x_1}{x_2-x_1} \\
                          & = \frac{x-0.0}{2.0-0.0}\cdot\frac{x-1.0}{2.0-1.0} \\
                          & = \frac{1}{2}(x^2-1.0x)
    \end{align*}
    So the second degree interpolating polynomial is
    \begin{align*}
        P_2(x) & = f(x_0)l_0(x)+f(x_1)l_1(x)+f(x_2)l_2(x)                                                                          \\
               & = (1.0)\left\{\frac{1}{2}(x^2-3.0x+2.0)\right\}+(2.718\,28)(2.0x-x^2)+(7.389\,06)\left\{\frac{1}{2}(x^2-1.0x)\right\} \\
               & =1.476\,25x^2+0.242\,03x+1                                                                                            \\
               & \approx f(x)
    \end{align*}
    \begin{align*}
        \therefore f(0.25) & = (1.476\,25)(0.25)^2+(0.242\,03)(0.25)+1 \\
                           & =1.152\,77
    \end{align*}
    \begin{align*}
        \therefore f(0.75) & = (1.476\,25)(0.75)^2+(0.242\,03)(0.75)+1 \\
                           & =2.011\,91
    \end{align*}
    \emph{Comment:}\\
    We see that,\\
    \indent from part i, \(  f(0.25)=1.324\,36 \)\\
    \indent from part iii, \(  f(0.25)=1.152\,77 \)\\
    But exactly \(  f(0.25)=1.284\,02 \)\\
    So the approximate value of \(  f(0.25)=1.324\,36 \) is better because in part i the interval is closer than in part iii.\\
    Similarly, \(  f(0.75)=2.183\,50 \) is better.
\end{soln}
\begin{prob}
    Consider the curve \(  y=x^3 \). Five points on this curve are \(  (0,0), (1,1)(,2,8),(3,27),(4,68) \). Compute \(  \sqrt[3]{20} \) by inverse Lagrange's interpolating polynomial. Find the result by
    \begin{enumerate}
        \item cubic polynomial (using first 4 points)
        \item quadratic polynomial (using all the points)
        \item linear polynomial (using \(  y_0=8,\;y_1=27 \))
    \end{enumerate}
\end{prob}
\begin{soln}
    \emph{Part i}\\
    We have the (inverse) Lagrange's polynomial is
    \[
        P_n(y)=\sum_{k=0}^nf(y_k)l_k(y)
    \]
    where
    \[
        l_k(y)=\prod_{\substack{i=0\\ i\neq k}}^n\frac{y-y_i}{y_k-y_i}, k=0,1,2,\dots,n
    \]
    In this case, we are given\\
    \indent \(  y_0=1,y_1=1,y_2=8,y_3=27 \) so that \(  n=3 \)\\
    Thus
    \[
        P_3(y)=\sum_{k=0}^3f(y_k)l_k(y)
    \]
    where
    \[
        l_k(y)=\prod_{\substack{i=0\\ i\neq k}}^3\frac{y-y_i}{y_k-y_i}, k=0,1,2,3
    \]
    \begin{align*}
        \therefore l_0(y)   & = \frac{y-y_1}{y_0-y_1}\cdot\frac{y-y_2}{y_0-y_2}\cdot\frac{y-y_3}{y_0-y_3} \\
        \Rightarrow l_0(20) & = \frac{20-1}{0-1}\cdot\frac{20-8}{0-8}\cdot\frac{20-27}{0-27}              \\
                            & = \frac{19\cdot12\cdot(-7)}{(-1)(-8)(-27)}                                  \\
                            & =\frac{1596}{216}                                                           \\
                            & =7.388\,88
    \end{align*}
    \begin{align*}
        \therefore l_1(y)   & = \frac{y-y_0}{y_1-y_0}\cdot\frac{y-y_2}{y_1-y_2}\cdot\frac{y-y_3}{y_1-y_3} \\
        \Rightarrow l_1(20) & = \frac{20-0}{1-0}\cdot\frac{20-8}{1-8}\cdot\frac{20-27}{1-27}              \\
                            & = \frac{20\cdot 12\cdot(-7)}{1(-7)(-26)}                                    \\
                            & =-\frac{1680}{182}                                                          \\
                            & =-9.230\,76
    \end{align*}
    \begin{align*}
        \therefore l_2(y)   & = \frac{y-y_0}{y_2-y_0}\cdot\frac{y-y_1}{y_2-y_1}\cdot\frac{y-y_3}{y_2-y_3} \\
        \Rightarrow l_2(20) & = \frac{20-0}{8-0}\cdot\frac{20-1}{8-1}\cdot\frac{20-27}{8-27}              \\
                            & = \frac{20\cdot19\cdot(-7)}{8\cdot7(-19)}                                   \\
                            & =\frac{2660}{1064}                                                          \\
                            & =2.5
    \end{align*}
    \begin{align*}
        \therefore l_3(y)   & = \frac{y-y_0}{y_3-y_0}\cdot\frac{y-y_1}{y_3-y_1}\cdot\frac{y-y_2}{y_3-y_2} \\
        \Rightarrow l_3(20) & = \frac{20-0}{27-0}\cdot\frac{20-1}{27-1}\cdot\frac{20-8}{27-8}             \\
                            & = \frac{20\cdot19\cdot12}{27\cdot26\cdot19}                                 \\
                            & =0.341\,88
    \end{align*}
    So,
    \begin{align*}
        P_3(y)            & = f(y_0)l_0(y)+f(y_1)l_1(y)+f(y_2)l_2(y)+f(y_3)l_3(y) \approx f(y) \\
        \Rightarrow f(20) & = 0\cdot7.388\,88+1\cdot(-923\,076)+2\cdot(2.5)+\cdot(0.341\,88)         \\
                          & =-3.205\,12
    \end{align*}
    i.e.\ \(  \sqrt[3]{20}=-3.2 \)\\
    \begin{note}
        In this case we first have to find polynomial.
    \end{note}
    \emph{Part iii}\\
    Here, we are given,\\
    \indent \(  y_0=8,y_1=27\) so that \(  n=1 \)\\
    Thus
    \[
        P_1(y)=\sum_{k=0}^1f(y_k)l_k(y)
    \]
    where
    \[
        l_k(y)=\prod_{\substack{i=0\\ i\neq k}}^1\frac{y-y_i}{y_k-y_i}, k=0,1
    \]
    \[\therefore l_0(y) = \frac{y-y_1}{y_0-y_1}=\frac{y-28}{8-27}=-\frac{1}{19}(y-27)\]
    \[\therefore l_1(y) = \frac{y-y_0}{y_1-y_0}=\frac{y-8}{27-8}=\frac{1}{19}(y-8)\]
    So the (inverse) linear interpolating polynomial is
    \begin{align*}
        P_1(y)           & = f(y_0)l_0(y)+f(y_1)l_1(y)                                            \\
                         & =2\left\{-\frac{1}{19}(y-27)\right\}+3\left\{\frac{1}{19}(y-8)\right\} \\
                         & =0.052\,63y+1.578\,94\approx f(y)                                          \\
                         &                                                                        \\
        \therefore f(20) & = 0.052\,63(20)+1.578\,94                                                  \\
                         & =2.631\,54
    \end{align*}
    i.e.\ \(  \sqrt[3]{20}=2.63 \)
\end{soln}
\section{Power form of the polynomial}
A polynomial of degree \(  \leq n \) is \(  P(x)=a_0+a_1x+a_2x^2+\dots+a_nx^n \) if \(  a_n \neq 0 \text{; deg}(P(x))=n\)
\begin{rem}
    The power form sometimes leads to loss of significance (in digits).
\end{rem}
\begin{prob}
    Find equation of the straight line passing through \(  (6000,\frac{1}{3}),(6001,-\frac{2}{3}) \)
\end{prob}
\begin{soln}
    We know, the equation of the straight line passing through \(  (x_1,y_1) \) and \(  (x_2,y_2) \) is\[\frac{x-x_1}{x_1-x_2}=\frac{y-y_1}{y_1-y_2}\]
    In this case,
    \begin{align*}
                    & \frac{y-\frac{1}{3}}{\frac{1}{3}+\frac{2}{3}}=\frac{x-6000}{6000-6001} \\
        \Rightarrow & \frac{3y-1}{3} =\frac{x-6000}{-1}                                      \\
        \Rightarrow & 3y-1 = 3(6000-3x)                                                      \\
        \Rightarrow & y = 6000+\frac{1}{3}-x                                                 \\
        \Rightarrow & P(x) \equiv y=6\,000.333\,33-x
    \end{align*}
    Suppose a machine only allow upto 5 significant digit\\
    Therefore, in five significant digits, it becomes,\\
    \indent \(  P(x)\approx 6000.3-x \)\\
    Now,\\
    \indent \(  P(60000)=6000.3-6000=0.3 \)
    \indent \(  P(60001)=6000.3-6001=-0.7 \)
    On the other hand,
    \indent\(  \frac{1}{3}=0.333\,33 \)
    \indent\(  -\frac{2}{3}=-0.666\,67 \)
\end{soln}
\section{Shifted Power Form}
The shifted power form of the polynomial
\begin{equation}
    P(x)=a_0+a_1(x-c)+a_2(x-c)^2+\dots+a_n(x-c)^n  \,\text{  for some } c . \label{eq:shift}
\end{equation}

\begin{note}
    \(  c \) is called the center.
\end{note}
[In the above example \(  c  \) may be \(  6000 \)]\\
Now, \(  P(x)=6000.3-x \) can be written as \\
\indent\(  P(x)=0.333\,33-1(x-6000) \)\\
So by comparing we get,
\indent \(  a_0=0.333\,33 \)\\
\indent \(  a_1=-1 \)\\
\indent \(  c=6000 \)\\
Now,\\
\[
    \left.\begin{aligned}
        P(6000) & =0.333\,33  \\
        P(6001) & =-0.666\,67 \\
    \end{aligned}\,\right\}\text{ exact value}
\]
\begin{rem}
    Shifted form is better than power form.
\end{rem}
\section{A Generalization of Shift Form}
\subsection{Newton's form}
\begin{equation}
    P(x)=a_0+a_1(x-c_1)+a_2(x-c_1)(x-c_2)+a_3(x-c_1)(x-c_2)(x-c_3)+\dots+a_n(x-c_1)(x-c_2)\dots(x-c_n) \label{eq:Newtonform}
\end{equation}
The above form is known as Newton's form.
\begin{rem}
    If \(  c_1=c_2=\dots=c_n=c \) then (\ref{eq:Newtonform}) is same as (\ref{eq:shift})
\end{rem}
\subsection{Evaluation of The Newton's form of interpolating polynomial}
From (\ref{eq:Newtonform}) we can write,
\begin{align*}
    P(x) & = a_0 +(x-c_1)(a_1+a_2(x-c_2)+a_3(x-c_2)(x-c_3)+\dots+a_n(x-c_2)(x-c_3)\dots(x-c_n)) \\
         & = a_0 +(x-c_1)(a_1+(x-c_2)(a_2+a_3(x-c_3)+\dots+a_n(x-c_3)\dots(x-c_n)))
\end{align*}
In this way we get,
\indent \(  P(x) \) in nested form
\[
    P(x)= a_0 +(x-c_1)(a_1+(x-c_2)(a_2+(x-c_3)(a_3+\dots+(x-c_{n-1})(a_{n-1}+(x-c_n)a_n))\dots))_{(n-1)\text{ bracket}}
\]
\begin{ex}
    \begin{align*}
        P(x)             & = 1+2(x-1)+3(x-1)(x-2)+4(x-1)(x-2)(x-3) \\
        \Rightarrow P(x) & = 1+(x-1)[2+(x-2)\{3+(x-3)\cdot4\}]     \\
                         &                                         \\
        \therefore P(4)  & = 1+(4-1)[2+(4-2)\{3+(4-3)\cdot4\}]     \\
                         & = 1+3[2+2\{3+1\cdot4\}]                 \\
                         & = 1+3[2+14]                             \\
                         & = 1+3\dot 16                            \\
                         & = 49
    \end{align*}
\end{ex}
\section{Newton's Interpolating Polynomial}
\subsection{Divided Difference}
Writing the interpolating polynomial in shifted power form (\ref{eq:Newtonform}) using the points \(  x_0,x_1,\dots,x_{n-1} \) as centers, then
\begin{equation}
    P_n(x)=a_0+a_1(x-x_0)+a_2(x-x_0)(x-x_1)+\dots+a_n(x-x_0)(x-x_1)\dots(x-x_{n-1})\label{eq:divdiff}
\end{equation}
\begin{note}
    Put \(  x=x_0 \) then,\\
    \indent\(  P_n(x_0)=a_0\approx f(x_0) \)\\
    Similarly, if we put \(  x=x_1 \), then\\
    \begin{align*}
        P_n(x_1)        & =a_0+a_1(x_1-x_0)\approx f(x_1) \\
                        & = f(x_0)+a_1(x_1-x_0)=f(x_1)    \\
        \Rightarrow a_1 & = \frac{f(x_1)-f(x_0)}{x_1-x_0}
    \end{align*}
\end{note}
\section{Divided Difference Notation for \(  f \)}
\subsection{Zeroth divided difference}
The zeroth divided difference of \(  f \) is \\
\indent \(  a_0=f(x_0)=f[x_0] \)\\


In general, \(  f(x_i)=f[x_i] \)
\subsection{First Divided Difference}
The first divided difference of \(  f \) is
\begin{align*}
    a_1 & =\frac{f(x_1)-f(x_0)}{x_1-x_0} \\
        & =\frac{f[x_1]-f[x_0]}{x_1-x_0} \\
        & = f[x_1,x_0]                   \\
        & = f[x_0,x_1]
\end{align*}
i.e.\ \(  f[x_0,x_1]=\frac{f[x_0]-f[x_1]}{x_0-x_1} \)\\
In general,  \(  f[x_i,x_{i+1}]=\frac{f[x_i]-f[x_{i+1}]}{x_i-x_{i+1}} \)\\
Similarly the second divided difference is,
\[
    f[x_0,x_1,x_2]=\frac{f[x_0,x_1]-f[x_1,x_2]}{x_0-x_2}
\]
In general,
\[
    f[x_i,x_{i+1},x_{i+2}]=\frac{f[x_i,x_{i+1}]-f[x_{i+1},x_{i+2}]}{x_i-x_{i+2}}
\]
Inductively, we can define the \(  k \)-th divided difference as,
\[
    f[x_0,x_1,\dots,x_k]=\frac{f[x_0,x_1,\dots,x_{k-1}]-f[x_1,x_2,\dots,x_k]}{x_0-x_k}
\]
For any integer \(  k \) between \(  0 \) and \(  n \), let \(  q_k(x) \) be the sum of the first \(  (k+1) \) times, i.e.,
\[q_k(x)=a_0+a_1(x-x_0)+a_2(x-x_0)(x-x_1)+\dots+a_k(x-x_0)\dots(x-x_{k-1}) \quad(0\leq k\leq n)\]
Then every one of the remaining terms in (\ref{eq:divdiff}) has the factor \(  (x-x_0)\dots(x-x_k) \) common.\\
Therefore, equation (\ref{eq:divdiff}) can be written as
\[P_n(x)=q_k(x)+(x-x_0)\dots(x-x_k)r(k)\]
for some polynomial \(  r(k) \).\\
The last term here vanishes at \(  x=x_0,x_1,\dots,x_k \)\\
\(  \therefore r(x) \) is of no further interest.\\
Hence, \(  q_k(x) \) interpolates \(  f(x) \) at \(  x_0,x_1,\dots,x_k  \), since \(  P_n(x) \) does.\\
Now \(  q_k(x) \) is of degree \(  \leq k \)\\
\(  \therefore q_k(x)=P_k(x) \) is unique and interpolates \(  f(x) \) at \(  x_0,x_1,\dots,x_k  \). This shows that Newton's form in (\ref{eq:divdiff}) can be built up step by step from the sequence \(  P_0(x),P_1(x),\dots \) with \(  P_k(x) \) obtained from \(  P_{k-1}(x) \) just by adding the next term in (\ref{eq:divdiff}).\\
i.e. \(  P_k(x)=P_{k-1}(x)+a_k(x-x_0)\dots (x_k-x_{k-1}) \)\\
The coefficient \(  a_k \) depends only on the values of \(  f(x) \).\\
\indent \(  a_k=f[x_0,x_1,\dots,x_k] \) the \(  k \)th divided difference.

So we get the Newton's interpolating divided formula as
\begin{align*}
    P_n(x) & =f[x_0]+f[x_0,x_1](x-x_0)+\dots+f[x_0,x_1,\dots,x_n](x-x_0)\dots(x-x_{n-1}) \\
           & =\sum_{i=0}^nf[x_0,x_1,\dots,x_i]\prod_{j=0}^{i-1}(x-x_j)
\end{align*}
\section{Newton's Divided Difference Formula}
\[P_n(x)=\sum_{i=0}^nf[x_0,x_1,\dots,x_i]\prod_{j=0}^{i-1}(x-x_j)\]
For \(  n=1 \)
\[
    \left.\begin{aligned}
        P_1(x)             & =f[x_0]+f[x_0,x_1](x-x_0)                      \\
        \Rightarrow P_1(x) & =f(x_0)+\frac{f(x_0)-f(x_1)}{(x_0-x_1)}(x-x_0)
    \end{aligned}\quad\right\vert\begin{aligned}
        \text{where, } & f[x_0]=f(x_0)                               \\
                       & f[x_0,x_1] =\frac{f(x_0)-f(x_1)}{(x_0-x_1)}
    \end{aligned}
\]
% \newpage
\begin{prob}
    Use a divided difference table and Newton's form of interpolating polynomial for \(  f(x)=\log_{10}(x) \) to estimate \(  \log_{10}(4.5) \)
    \begin{center}
        \begin{tabular}{cr@{ = }S[table-format=2.6]}
            \toprule
            \(  x_i \) & \multicolumn{2}{c}{\(  f(x_i) \)}          \\\midrule
            \(  x_0=1.2 \) & \(  f(x_0)\)&0.079181  \\
            \(  x_1=1.4 \) & \(  f(x_1)\)&0.146128  \\
            \(  x_2=1.6 \) & \(  f(x_2)\)&0.204120  \\
            \(  x_3=1.8 \) & \(  f(x_3)\)&0.255273  \\\bottomrule
        \end{tabular}
    \end{center}
\end{prob}
\begin{soln}
    First we construct the following divided difference table
    % \begin{table}[h]
    %     % \resizebox{10pt\textwidth}{!}{%
    %     \footnotesize
    %     \begin{tabular}{|c|p{2.5cm}|p{3cm}|c|c|}
    %         \hline
    %                 & {Zeroth \newline divided \newline difference} & {First \newline divided difference}                                   & Second divided difference         & Third divided difference           \\\hline
    %         \(  x_i \) & \(  f(x_i)=f[x_i] \)         & {\( \begin{aligned} &f[x_i,x_{i+1}]\\
    %             =&\frac{f[x_i]-f[x_{i+1}]}{x_i-x_{i+1}}\end{aligned} \)} & {\( \begin{aligned}
    %                           & f[x_i,x_{i+1},x_{i+2}]                                \\
    %                         = & \frac{f[x_i,x_{i+1}]-f[x_{i+1},x_{i+2}]}{x_i-x_{i+2}}
    %                     \end{aligned}\)} & {\(  \begin{aligned}   & f[x_i,x_{i+1},x_{i+2},x_{i+3}]                                        \\
    %                         = & \frac{f[x_i,x_{i+1},x_{i+2}]-f[x_{i+1},x_{i+2},x_{i+3}]}{x_i-x_{i+3}}\end{aligned} \)} \\\hline
    %     {\(  \begin{aligned}
    %         &x_0=1.2\\
    %         & \\
    %         & \\
    %         &x_1=1.4\\
    %         & \\
    %         & \\
    %         &x_2=1.6\\
    %         & \\
    %         & \\
    %         &x_3=1.8
    %     \end{aligned} \)} & \(  {\begin{aligned}
    %         f[x_0]&=0.079181\\
    %         & \\& \\
    %         f[x_1]&=0.146128\\
    %         & \\& \\
    %         f[x_2]&=0.204120\\
    %         & \\& \\
    %         f[x_3]&=0.255273 
    %     \end{aligned}} \) & {\( 
    %         \begin{aligned}
    %             &f[x_0,x_1]\\
    %             =&\frac{f[x_0]-f[x_1]}{x_0-x_1}\\
    %             % =&\frac{0.079181-0.146128}{1.2-1.4}\\
    %             =&0.334735\\
    %             & \\
    %             &f[x_1,x_2]\\
    %             =&\frac{f[x_1]-f[x_2]}{x_1-x_2}\\
    %             % =&\frac{0.146128-0.204120}{1.4-1.6}\\
    %             =&0.28996\\
    %             & \\
    %             &f[x_2,x_3]\\
    %             =&\frac{f[x_2]-f[x_3]}{x_2-x_3}\\
    %             % =&\frac{0.204120-0.255273}{1.6-1.8}\\
    %             =&0.255765\\
    %         \end{aligned}
    %     \)} &{\(  \begin{aligned}
    %         & f[x_0,x_1,x_2] \\
    %         = & \frac{f[x_0,x_1]-f[x_1,x_2]}{x_0-x_2}\\
    %         =& -0.1119375\\
    %         & \\
    %         & f[x_1,x_2,x_3] \\
    %         = & \frac{f[x_1,x_2]-f[x_2,x_3]}{x_1-x_3}\\
    %         =& -0.0854875\\
    %     \end{aligned} \)} &{\(  \begin{aligned}
    %         & f[x_0,x_1,x_2,x_3]\\
    %         = & \frac{f[x_0,x_1,x_2]-f[x_1,x_2,x_3]}{x_0-x_3}\\
    %         =& 0.044083333
    %     \end{aligned} \)}\\\hline
    %     \end{tabular}%}
    % \end{table}
    \begin{table}[h]
        \footnotesize
        \everymath{\displaystyle}
        \begin{tabular}{ccccc}
            \toprule
            & \begin{tabular}[c]{@{}l@{}}Zeroth divided\\ difference\end{tabular}
            & \begin{tabular}[c]{@{}l@{}}First divided\\ difference\end{tabular}
            & Second divided difference  & Third divided difference           \\
            \multirow[b]{2}{*}{$ x_i $}& \multirow[b]{2}{*}{\(f(x_i)=f[x_i]\)} & $ f[x_i,x_{i+1} ]$ \rule[-1em]{0pt}{1em}& \multicolumn{1}{l}{\hspace*{1em}$ f[x_i,x_{i+1},x_{i+2}] $} & \multicolumn{1}{l}{\hspace*{1em}$ f[x_i,x_{i+1},x_{i+2},x_{i+3}] $} \\
                && $ =\frac{f[x_i]-f[x_{i+1}]}{x_i-x_{i+1}} $\rule[-1em]{0pt}{1em} & $ =\frac{f[x_i,x_{i+1}]-f[x_{i+1},x_{i+2}]}{x_i-x_{i+2}} $& $ =\frac{f[x_i,x_{i+1},x_{i+2}]-f[x_{i+1},x_{i+2},x_{i+3}]}{x_i-x_{i+3}} $\\\midrule
            {\(  \begin{aligned}
                     & x_0=1.2 \\
                     &         \\
                     &         \\
                     & x_1=1.4 \\
                     &         \\
                     &         \\
                     & x_2=1.6 \\
                     &         \\
                     &         \\
                     & x_3=1.8
                \end{aligned} \)} & \(  {\begin{aligned}
                        f[x_0] & =0.079\,181 \\
                               &           \\& \\
                        f[x_1] & =0.146\,128 \\
                               &           \\& \\
                        f[x_2] & =0.204\,120 \\
                               &           \\& \\
                        f[x_3] & =0.255\,273
                    \end{aligned}} \)   & {\(
                    \begin{aligned}
                          & f[x_0,x_1]                    \\
                        = & \frac{f[x_0]-f[x_1]}{x_0-x_1} \\
                        % =&\frac{0.079181-0.146128}{1.2-1.4}\\
                        = & 0.334\,735                      \\
                          &                               \\
                          & f[x_1,x_2]                    \\
                        = & \frac{f[x_1]-f[x_2]}{x_1-x_2} \\
                        % =&\frac{0.146128-0.204120}{1.4-1.6}\\
                        = & 0.289\,96                       \\
                          &                               \\
                          & f[x_2,x_3]                    \\
                        = & \frac{f[x_2]-f[x_3]}{x_2-x_3} \\
                        % =&\frac{0.204120-0.255273}{1.6-1.8}\\
                        = & 0.255\,765                      \\
                    \end{aligned}
            \)}                                 & {\(  \begin{aligned}
                          & f[x_0,x_1,x_2]                        \\
                        = & \frac{f[x_0,x_1]-f[x_1,x_2]}{x_0-x_2} \\
                        = & -0.111\,937\,5                            \\
                          &                                       \\
                          & f[x_1,x_2,x_3]                        \\
                        = & \frac{f[x_1,x_2]-f[x_2,x_3]}{x_1-x_3} \\
                        = & -0.085\,487\,5                            \\
                    \end{aligned} \)}   & {\(  \begin{aligned}
                          & f[x_0,x_1,x_2,x_3]                            \\
                        = & \frac{f[x_0,x_1,x_2]-f[x_1,x_2,x_3]}{x_0-x_3} \\
                        = & 0.044\,083\,333
                    \end{aligned} \)}                                                                   \\\bottomrule
        \end{tabular}%}
    \end{table}
    We know, the Newton's interpolating divided difference formula is
    \[
        P_n(x)=\sum_{i=0}^nf[x_0,x_1,\dots,x_i]\prod_{j=0}^{i-1}(x-x_j)
    \]
    In this case, \(  n=3 \)
    \begin{align*}
        \therefore P_3(x) &=\sum_{i=0}^3f[x_0,x_1,\dots,x_i]\prod_{j=0}^{i-1}(x-x_j)\\
        &= f[x_0]+f[x_0,x_1](x-x_0)+f[x_0,x_1,x_2](x-x_0)(x-x_1)+f[x_0,x_1,x_2,x_3](x-x_0)(x-x_1)(x-x_2)\\
        &=0.079\,181+0.334\,735(x-1.2)+0.111\,937\,5(x-1.2)(x-1.4)+0.044\,083\,333(x-1.2)(x-1.4)(x-1.6)\\
        &=0.079\,181+(x-1.2)[0.334\,735-(x-1.4)\{0.111\,937\,5+(x-1.6)(0.044\,083\,333)\}]
    \end{align*}
    Since \(  \log_{10}(14.5)=1+\log_{10}(1.45) \)
    \begin{align*}
        \therefore P_3(1.45)&=0.079\,181+(1.45-1.2)[0.334\,735-(1.45-1.4)\{0.111\,937\,5+(1.45-1.6)(0.044\,083\,333)\}]\\
        &=0.161\,548\,187\\
        & \\
        \therefore P_3(14.5)&=1+0.161\,548\,187\\
        &=1.161\,588
    \end{align*}
\end{soln}
\begin{prob}
    Construct the divided difference table to find the Newton's form of polynomial of deg\(  \leq 3 \) and estimate \(  f(1.25) \). Also evaluate \(  f(1.25) \) in Lagrange's form and compare the results. Exact value of \(  f(1.25)=3.0096 \).
    \begin{center}
        \begin{tabular}{r@{ = }S[table-format=1.3]r@{ = }S[table-format=1.4]}
            \toprule
            \multicolumn{2}{c}{\(  x_i \)} & \multicolumn{2}{c}{\(  f(x_i) \)}\\\midrule
            \(  x_0\)& 1.05 & \(  f(x_0)\)& 1.7433 \\
            \(  x_1 \)& 1.20 & \(  f(x_1)\)&2.5722 \\
            \(  x_2 \)& 1.30 & \(  f(x_2)\)&3.6021 \\
            \(  x_3 \)& 1.43 & \(  f(x_3)\)&8.2381\\\bottomrule
        \end{tabular}
    \end{center}
\end{prob}
\begin{soln}
    First we construct the (Newton) divided difference table as follows,
    % \begin{table}[h]
    %     % \resizebox{10pt\textwidth}{!}{%
    %     \footnotesize
    %     \begin{tabular}{|c|p{2.5cm}|p{3cm}|c|c|}
    %         \hline
    %                 & {Zeroth \newline divided \newline difference} & {First \newline divided difference}                                   & Second divided difference         & Third divided difference           \\\hline
    %         \(  x_i \) & \(  f(x_i)=f[x_i] \)         & {\( \begin{aligned} &f[x_i,x_{i+1}]\\
    %             =&\frac{f[x_i]-f[x_{i+1}]}{x_i-x_{i+1}}\end{aligned} \)} & {\( \begin{aligned}
    %                           & f[x_i,x_{i+1},x_{i+2}]                                \\
    %                         = & \frac{f[x_i,x_{i+1}]-f[x_{i+1},x_{i+2}]}{x_i-x_{i+2}}
    %                     \end{aligned}\)} & {\(  \begin{aligned}   & f[x_i,x_{i+1},x_{i+2},x_{i+3}]                                        \\
    %                         = & \frac{f[x_i,x_{i+1},x_{i+2}]-f[x_{i+1},x_{i+2},x_{i+3}]}{x_i-x_{i+3}}\end{aligned} \)} \\\hline
    %     {\(  \begin{aligned}
    %         &x_0=1.05\\
    %         & \\
    %         & \\
    %         &x_1=1.20\\
    %         & \\
    %         & \\
    %         &x_2=1.30\\
    %         & \\
    %         & \\
    %         &x_3=1.45
    %     \end{aligned} \)} & \(  {\begin{aligned}
    %         f[x_0]&=1.7433\\
    %         & \\& \\
    %         f[x_1]&=2.5722\\
    %         & \\& \\
    %         f[x_2]&=3.6021\\
    %         & \\& \\
    %         f[x_3]&=8.2381 
    %     \end{aligned}} \) & {\( 
    %         \begin{aligned}
    %             &f[x_0,x_1]\\
    %             =&\frac{f[x_0]-f[x_1]}{x_0-x_1}\\
    %             % =&\frac{0.079181-0.146128}{1.2-1.4}\\
    %             =&5.526\\
    %             & \\
    %             &f[x_1,x_2]\\
    %             =&\frac{f[x_1]-f[x_2]}{x_1-x_2}\\
    %             % =&\frac{0.146128-0.204120}{1.4-1.6}\\
    %             =&10.299\\
    %             & \\
    %             &f[x_2,x_3]\\
    %             =&\frac{f[x_2]-f[x_3]}{x_2-x_3}\\
    %             % =&\frac{0.204120-0.255273}{1.6-1.8}\\
    %             =&30.90666667\\
    %         \end{aligned}
    %     \)} &{\(  \begin{aligned}
    %         & f[x_0,x_1,x_2] \\
    %         = & \frac{f[x_0,x_1]-f[x_1,x_2]}{x_0-x_2}\\
    %         =& 19.092\\
    %         & \\
    %         & f[x_1,x_2,x_3] \\
    %         = & \frac{f[x_1,x_2]-f[x_2,x_3]}{x_1-x_3}\\
    %         =& 82.43066668\\
    %     \end{aligned} \)} &{\(  \begin{aligned}
    %         & f[x_0,x_1,x_2,x_3]\\
    %         = & \frac{f[x_0,x_1,x_2]-f[x_1,x_2,x_3]}{x_0-x_3}\\
    %         =& 158.3466667
    %     \end{aligned} \)}\\\hline
    %     \end{tabular}%}
    % \end{table}
    \begin{table}[H]
        \footnotesize
        \everymath{\displaystyle}
        \begin{tabular}{ccccc}
            \toprule
            & \begin{tabular}[c]{@{}l@{}}Zeroth divided\\ difference\end{tabular}
            & \begin{tabular}[c]{@{}l@{}}First divided\\ difference\end{tabular}
            & Second divided difference  & Third divided difference           \\
            \multirow[b]{2}{*}{$ x_i $}& \multirow[b]{2}{*}{\(f(x_i)=f[x_i]\)} & $ f[x_i,x_{i+1} ]$ \rule[-1em]{0pt}{1em}& \multicolumn{1}{l}{\hspace*{1em}$ f[x_i,x_{i+1},x_{i+2}] $} & \multicolumn{1}{l}{\hspace*{1em}$ f[x_i,x_{i+1},x_{i+2},x_{i+3}] $} \\
                && $ =\frac{f[x_i]-f[x_{i+1}]}{x_i-x_{i+1}} $\rule[-1em]{0pt}{1em} & $ =\frac{f[x_i,x_{i+1}]-f[x_{i+1},x_{i+2}]}{x_i-x_{i+2}} $& $ =\frac{f[x_i,x_{i+1},x_{i+2}]-f[x_{i+1},x_{i+2},x_{i+3}]}{x_i-x_{i+3}} $\\\midrule
            {\(  \begin{aligned}
                     & x_0=1.05 \\
                     &         \\
                     &         \\
                     & x_1=1.20 \\
                     &         \\
                     &         \\
                     & x_2=1.30 \\
                     &         \\
                     &         \\
                     & x_3=1.45
                \end{aligned} \)} & \(  {\begin{aligned}
                        f[x_0] & =1.7433 \\
                               &           \\& \\
                        f[x_1] & =2.5722 \\
                               &           \\& \\
                        f[x_2] & =3.6021 \\
                               &           \\& \\
                        f[x_3] & =8.2381
                    \end{aligned}} \)   & {\(
                    \begin{aligned}
                          & f[x_0,x_1]                    \\
                        = & \frac{f[x_0]-f[x_1]}{x_0-x_1} \\
                        % =&\frac{0.079181-0.146128}{1.2-1.4}\\
                        = & 5.526                      \\
                          &                               \\
                          & f[x_1,x_2]                    \\
                        = & \frac{f[x_1]-f[x_2]}{x_1-x_2} \\
                        % =&\frac{0.146128-0.204120}{1.4-1.6}\\
                        = & 10.299                       \\
                          &                               \\
                          & f[x_2,x_3]                    \\
                        = & \frac{f[x_2]-f[x_3]}{x_2-x_3} \\
                        % =&\frac{0.204120-0.255273}{1.6-1.8}\\
                        = & 30.906\,666\,67                      \\
                    \end{aligned}
            \)}                                 & {\(  \begin{aligned}
                          & f[x_0,x_1,x_2]                        \\
                        = & \frac{f[x_0,x_1]-f[x_1,x_2]}{x_0-x_2} \\
                        = & 19.092                            \\
                          &                                       \\
                          & f[x_1,x_2,x_3]                        \\
                        = & \frac{f[x_1,x_2]-f[x_2,x_3]}{x_1-x_3} \\
                        = & 82.430\,666\,68                            \\
                    \end{aligned} \)}   & {\(  \begin{aligned}
                          & f[x_0,x_1,x_2,x_3]                            \\
                        = & \frac{f[x_0,x_1,x_2]-f[x_1,x_2,x_3]}{x_0-x_3} \\
                        = & 158.346\,666\,7
                    \end{aligned} \)}                                                                   \\\bottomrule
        \end{tabular}%}
    \end{table}
    We know, the Newton's interpolating divided difference formula is
    \[
        P_n(x)=\sum_{i=0}^nf[x_0,x_1,\dots,x_i]\prod_{j=0}^{i-1}(x-x_j)
    \]
    In this case, \(  n=3 \)
    \begin{align*}
        \therefore P_3(x) &=\sum_{i=0}^3f[x_0,x_1,\dots,x_i]\prod_{j=0}^{i-1}(x-x_j)\\
        &= f[x_0]+f[x_0,x_1](x-x_0)+f[x_0,x_1,x_2](x-x_0)(x-x_1)+f[x_0,x_1,x_2,x_3](x-x_0)(x-x_1)(x-x_2)\\
        &=1.7433+5.526(x-1.05)+19.092(x-1.05)(x-1.20)+158.346\,666\,7(x-1.05)(x-1.30)(x-1.30)\\
        &=1.7433+5.526(x-1.05)+19.092(x^2-2.25x+1.26)+158.346\,666\,7(x^2-2.25x+1.26)(x-1.30)\\
        &=1.7433+5.526(x-1.05)+19.092(x^2-2.25x+1.26)+158.346\,666\,7(x^3-3.55x^2+4.185x-1638)\\
        &= 158.346\,666\,7x^3-543.038\,666\,8x^2+625.249\,800\,1x-239.374\,920\,1\\
        &\approx f(x)
    \end{align*}
    \(  \therefore f(1.25)=2.960\,24 \)
\end{soln}
\end{document}