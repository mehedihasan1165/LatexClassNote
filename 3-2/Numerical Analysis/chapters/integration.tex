\documentclass[12pt,class=book,crop=false]{standalone}
\usepackage{../style}
\graphicspath{ {../img/} }
\begin{document}
\chapter{Numerical Integration}
The process of computing the value of a definite integral from a set of numerical values of the integrand is called \emph{Numerical Integration}. When applied to the integration of a function of a single variable, the process is known as quadrature.\\

The problem of Numerical Integration is solved by representing the integrand by an interpolation formula and then integrating this formula between the desired limits.
\section{A General Quadrature Formula for Equidistant Ordinates}
Let $ I=\int_a^b y\dx  $ where $ y=f(x) $. Let $ f(x) $ be given for certain equidistant value of $ x $ say $ x_0,x_0+h,x_0+2h,\dots $. Let the range $ (a,b) $ be divided into $ n $ equal parts, each of width $ h $ so that $ b-a=nh $.\\
Let $ x_0=a$, $x_1=x_0+h=a+h$, $x_2=a+2h$, $\dots$, $x_n=a+nh=b $. We have assumed that the $ n+1 $ ordinates $ y_0,\,y_1,\dots,y_n $ are at equal intervals.
\[
    \therefore\, I=\int_a^b y\dx=\int_{x_0}^{x_0+nh} y_x \dx
\]
Let, $ u=\frac{x-x_0}{h}\qquad \therefore \dx=h \D u $\\
when $ \begin{aligned}[t]
        x & =x_0,    & u=0 \\
        x & =x_0+nh, & u=n
    \end{aligned} $\\
Now we have (from Newton's forward formula)
\[
    y_{x_0+uh}=y_0+u\Delta y_0+\frac{u(u-1)}{2!}\Delta^2 y_0+\dots
\]
\begin{align}
    \therefore\,I  & =h\int_0^n \left[ y_0+u\Delta y_0+\frac{u(u-1)}{2!}\Delta^2 y_0+\dots \right]\D u\notag                 \\
    \Rightarrow\,I & =h\left[ ny_0+\frac{n^2}{2}\Delta y_0+\left( \frac{n^3}{3}-\frac{n^2}{2} \right)\frac{\Delta^2 y_0}{2!}
        +\dots \text{ upto $ (n+1)$ terms}\right] \label{eq:quad1}
\end{align}
This is the general quadrature formula.\\
We can deduce a number of formulae from thus by putting $ n=1,2,\dots $.
\section{Kinds of Rule for Determining Numerical Integration}
\subsection{The Trapezoidal Rule}
Putting $ n=1 $ in the formula \eqref{eq:quad1} and neglecting second and higher order differences, we get
\[
    \int_{x_0}^{x_0+h} y\dx=h\left[ y_0+\frac{1}{2}\left( y_1-y_0 \right) \right]=\frac{h[y_0+y_1]}{2}
\]
Similarly,
\begin{align*}
    \int_{x_0+h}^{x_0+2h} y\dx      & =h \,\frac{y_1+y_2}{2}    \\
    \int_{x_0+(n-1)h}^{x_0+nh} y\dx & =h\,\frac{y_{n-1}+y_n}{2}
\end{align*}
Adding these $ n $ integrals we get
\begin{align*}
    \int_{x_0}^{x_0+nh} y\dx & =h\left[ \frac{1}{2}\left\{ (y_0+y_n)+(y_1+y_2)+\dots+(y_{n-1}+y_n) \right\} \right]                                \\
                             & =h\left[ \frac{1}{2}(y_0+y_n)+(y_1+y_2+\dots+y_{n-1}) \right]                                                       \\
                             & =\text{ distance between two consecutive ordinates} \times \left\{ \text{mean of the 1st and last ordinates}\right. \\
                             & \, +\left. \text{ sum of all the intermediate ordinates} \right\}
\end{align*}
This rule is known as the Trapezoidal Rule.
\subsection{Simpson's $ {}^1/{}_3 $ Rule (Simpson's rule)}
Putting $ n=2 $ in the formula \eqref{eq:quad1} and neglecting third and higher order differences, we get
\begin{align*}
    \int_{x_0}^{x_0+2h} y \dx & =h\left[ 2y_0+\frac{4}{2}\Delta y_0+\left( \frac{8}{3}-\frac{4}{2} \right)\frac{\Delta^2 y_0}{2}+\dots \right] \\
                              & =h\left[ 2y_0+2(y_1-y_0)+\frac{1}{3}\left\{ 4(y_1-y_0) \right\}\right]                                         \\
                              & =h\left[ 2y_0+2y_1-2y_0+\frac{1}{3}\left( y_2-y_1-(y_1-y_0) \right)\right]                                     \\
                              & =\frac{h}{3}\left[ 6y_1+y_2-y_1-y_1+y_0\right]                                                                 \\
                              & =\frac{h}{3}\left[ y_0+4y_1+y_2\right]
\end{align*}
Similarly,
\begin{align*}
    \int_{x_0+2h}^{x_0+4h} y \dx     & =\frac{h}{3}\left[ y_2+4y_3+y_4\right]                                       \\
    \int_{x_0+(n-2)h}^{x_0+nh} y \dx & =\frac{h}{3}\left[ y_{n-2}+4y_{n-1}+y_n\right]\qquad  \text{[$ n $ is even]}
\end{align*}
Adding all the integrals we get,
\begin{align*}
    \int_{x_0}^{x_0+nh} y \dx & =\frac{h}{3}\left[ (y_0+4y_1+y_2)+(y_2+4y_3+y_4)+\dots+(y_{n-2}+4y_{n-1}+y_n)\right]      \\
                              & =\frac{h}{3}\left[ (y_0+y_n)+4(y_1+y_3+y_5+\dots+y_{n-1})+2(y_2+y_4+\dots+y_{n-2})\right]
\end{align*}
This formula is known as Simpson's $ {}^1/{}_{3} $ -rule also known as Simpson's rule.
\subsection{Simpson's $ {}^3/{}_8 $ Rule}
Putting $ n=3 $ in the formula \eqref{eq:quad1} and neglecting fourth and higher order differences, we get
\begin{align*}
    \int_{x_0}^{x_0+3h} y \dx & =h\left[ 3y_0+\frac{9}{2}\Delta y_0+\left( \frac{27}{3}-\frac{9}{2} \right)\frac{\Delta^2 y_0}{2!}+\left( \frac{81}{4}-27+9\right)\frac{\Delta^3 y_0}{3!} \right] \\
                              & =h\left[ 3y_0+\frac{9}{2}(y_1-y_0)+\frac{9}{2}\cdot\frac{1}{2}(y_2-2Y_1+y_0)+\frac{81-104+36}{4}\cdot\frac{1}{6}\Delta(y_2-2y_1+y_0)\right]                       \\
                              & =h\left[ 3y_0+\frac{9}{2}(y_1-y_0)+\frac{9}{4}\left( y_2-2y_1+y_0\right)+\frac{9}{4}\cdot\frac{1}{6}(y_3-y_2-2y_2+2y_1+y_1-y_0)\right]                            \\
                              & =\frac{3h}{8}\left[ y_0+3y_1+3y_2+y_3\right]
\end{align*}
Similarly,
\begin{align*}
    \int_{x_0+3h}^{x_0+6h} y \dx     & =\frac{3h}{8}\left[ y_3+3y_4+3y_5+y_6\right]             \\
    \int_{x_0+(n-3)h}^{x_0+nh} y \dx & =\frac{3h}{8}\left[ y_{n-3}+3y_{n-2}+3y_{n-1}+y_n\right]
\end{align*}
Adding all the integrals we get,
\[
    \int_{x_0}^{x_0+nh} y \dx =\frac{3h}{8}\left[ (y_0+y_n)+3(y_1+y_2+y_4+\dots+y_{n-2}+y_{n-1})+2(y_3+y_6+\dots+y_{n-3})\right]
\]
This formula is known as Simpson's $ {}^3/{}_{8} $ rule.
\begin{prob}
    Calculate the value of the integral
    \[
        \int_4^{5.2} \log_e x \dx
    \]
    by
    \begin{enumerate}[label=(\roman*)]
        \item Trapezoidal rule
        \item Simpson's $ {}^1/{}_3 $ rule
        \item Simpson's $ {}^3/{}_8 $ rule
        \item Weddle's rule
    \end{enumerate}
    After finding the true value of the integral, compare the errors in tn the four cases.
\end{prob}
\begin{soln}
    Divide the range of integration $ (4,5.2) $ in 6 equal parts each of width $ \frac{5.2-4}{6}=0.2 $, so that $ h=0.2 $. The value of the function $ f(x)=\ln x $ for each point of sub-division are given below:
    \begin{table}[H]
        \centering
        \begin{tabular}{r@{ = }S[table-format=1.2]r@{ = }S[table-format=1.7]}
            \toprule
            \multicolumn{2}{c}{$ x $}          & \multicolumn{2}{c}{$ y=f(x)=\ln x $  } \\\midrule
            $ x_0$& 4.0 & $y_0$ &1.3862944  \\
            $ x_0+h$&4.2   & $ y_1$& 1.4350845 \\
            $ x_0+2h$&4.4  & $ y_2$& 1.4816045 \\
            $ x_0+3h$&4.6  & $ y_3$& 1.5260563 \\
            $ x_0+4h$&4.8  & $ y_4$& 1.5686159 \\
            $ x_0+5h$&5.0  & $ y_5$& 1.6094379 \\
            $ x_0+6h$&5.2  & $ y_6$& 1.6486586  \\\bottomrule
        \end{tabular}
    \end{table}
    \begin{enumerate}[label=(\roman*)]
        \item We have from Trapezoidal rule
              \[
                  \int_{x_0}^{x_0+nh} y\dx = h\left[ \frac{1}{2}(y_0+y_n)+(y_1+y_2+\dots+y_{n-1}) \right]
              \]
              In this case:
              \begin{align*}
                  \int_{4}^{5.2} \ln x \dx & = 0.2\left[ 1.517\,476\,5+7.620\,799\,2\right] \\
                                           & =1.827\,655\,1
              \end{align*}
        \item By Simpson's $ {}^1/{}_3 $ rule:
              \[
                  \int_{x_0}^{x_0+nh} y \dx =\frac{h}{3}\left[ (y_0+y_n)+4(y_1+y_3+y_5+\dots+y_{n-1})+2(y_2+y_4+\dots+y_{n-2})\right]
              \]
              In this case:
              \begin{align*}
                  \int_{4}^{5.2} \ln x \dx & = \frac{0.2}{3}\left[ 3.034\,953\,0+18.282\,314\,9+6.100\,844\,09\right] \\
                                           & =1.827\,847\,2
              \end{align*}
        \item By Simpson's $ {}^3/{}_8 $ rule:
              \[
                  \int_{x_0}^{x_0+nh} y \dx =\frac{3h}{8}\left[ (y_0+y_n)+3(y_1+y_2+y_4+\dots+y_{n-2}+y_{n-1})+2(y_3+y_6+\dots+y_{n-3})\right]
              \]
              In this case:
              \begin{align*}
                  \int_{4}^{5.2} \ln x \dx & = \frac{3\times 0.2}{8}\left[ 3.034\,953\,0+18.284\,228\,7+3.052\,112\,6\right] \\
                                           & =1.827\,847
              \end{align*}
    \end{enumerate}
    Actual value of
    \begin{align*}
        \int_4^{5.2}\ln x\dx & =(x\ln x-x)\Biggr\vert_4^{5.2} \\
                             & =1.827\,847\,5
    \end{align*}
    Hence the errors are:
    \begin{enumerate}[label=(\roman*)]
        \item 0.000\,192\,4
        \item 0.000\,000\,3
        \item 0.000\,000\,5
    \end{enumerate}
    So Simpson's $ {}^1/{}_3 $ rule is more accurate.
\end{soln}
\vspace{2.5cm}
Note To Self: May be missing a page\\
\vspace{2.5cm}


Similarly,
\begin{align*}
    \int_{x_0+6h}^{x_0+12h}y\dx    & =\frac{3h}{10}[y_6+5y_7+y_8+6y_9+y_{10}+5y_{11}+y_{12}]                  \\
    \int_{x_0+(n-6)h}^{x_0+nh}y\dx & =\frac{3h}{10}[y_{n-6}+5y_{n-5}+y_{n-4}+6y_{n-3}+y_{n-2}+5y_{n-1}+y_{n}]
\end{align*}
Adding these integrals, we get
\[
    \int_{x_0}^{x_0+nh}y\dx=\frac{3h}{10}[y_{0}+5y_{1}+y_{2}+6y_{3}+y_{4}+5y_{5}+2y_{6}+5y_7+y_8+\dots]
\]
This formula is known as Weddle's rule.
\end{document}