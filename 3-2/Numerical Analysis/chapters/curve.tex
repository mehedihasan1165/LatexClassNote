\documentclass[12pt,class=book,crop=false]{standalone}
\usepackage{../style}
\graphicspath{ {../img/} }
\begin{document}
\chapter{Curve Fitting and Spline Interpolation}
\section{Curve Fitting by Least Squares Method}
The method of least squares may be one of the most systematic procedure to fit a curve through given data points.\\
Consider the problem of fitting a set of data points
\[
	(x_r,y_r),\qquad r=1,2,3,\dots,m
\]
to a curve $ y=f(x) $ whose values depend on $ n $ parameters $ c_1,c_2,c_3,\dots,c_n $. The values of the function at a point depends on the values of the parameter involved. In least square method we determine a set of values of the parameter $ c_1,c_2,c_3,\dots,c_n $ such that the sum of the squares of the error
\[
	E(c_1,c_2,\dots,c_n)=\sum_i^m\left[f\left(x_i,c_1,c_2,\dots,c_n\right)-y_i \right]^2
\]
is minimum.\\
The necessary conditions for $ E $ to have a minimum is that
\[
	\frac{\partial E}{\partial c_i}=0,\qquad i=1,2,3,\dots,n
\]
This condition gives a system of $ n $ equations, called \emph{normal equations}, in $ n $ unknowns $ c_1,c_2,c_3,\dots,c_n $.\\
If the parameters appear in the function in non-linear form, the normal equations become non-linear and is difficult to solve. This difficulty may be avoided if $ f(x) $ is transformed to a form which is linear in parameter.
\subsection{Parameters in Nonlinear Form}
\begin{enumerate}[label=(\alph*)]
	\item \textbf{Power function:} Let the curve
	      \[
		      y=ax^b
	      \]
	      be fitted to the given data.\\
	      Taking logarithm of both sides, we get
	      \[
		      \ln y=\ln a+b \ln x
	      \]
	      which can be written in the form
	      \[
		      Y=A+bX
	      \]
	      where,
	      \[
		      Y=\ln y,\quad A=\ln a,\quad X=\ln x.
	      \]
	\item Let the curve
	      \[
		      y=\frac{400}{1+c e^{bx}}
	      \]
	      be fitted to the given data.\\
	      The equation of the curve can be rewritten as
	      \[
		      \frac{400}{y}-1=c e^{bx}
	      \]
	      Taking logarithm of both sides, we get
	      \[
		      \ln\left(\frac{400}{y}-1\right)=\ln c+bx
	      \]
	      which can be written in the form
	      \[
		      Y=A+bx\qquad \text{ where}\qquad Y=\ln\left(\frac{400}{y}-1\right),\quad A=\ln c.
	      \]
\end{enumerate}
\begin{prob}
	The average price, $ P $, of a certain type of second-hand car is believed to be related to its age, $ x $ years, by an equation of the form
	\[
		P=50+ae^{bx}
	\]
	where $ a $ and $ b $ are constants. Data from a recent newspaper give the following average price (in Taka) for used car of this type,
	\begin{table}[H]
		\centering
		\begin{tabular}{cS[table-format=3.1]}
			\toprule
			$ x $ &
			\multicolumn{1}{c}{$ P $ (in thousands)} \\\midrule
			  1  & 
			  774.4  \\
			  2      &
			  603.4  \\
			   3      &
			   439.2  \\
			    4      &
				360.0  \\
			     5      &
					  328.3  \\\bottomrule
		\end{tabular}
	\end{table}
	\begin{enumerate}[label=(\alph*)]
		\item Estimate the values of $ a $ and $ b $ rounded to 3 significant figures.
		\item Estimate the price of a car of this type that is 6 years old and the original new price of that car.
	\end{enumerate}
\end{prob}
\begin{soln}
	\begin{enumerate}[label=(\alph*)]
		\item The curve $ P=50+ae^bx $ is to be fitted to the given data.\\
		      The equation of the curve can be rewritten as $ P-50=ae^bx $\\
		      Taking logarithm of both sides, we get $ \ln ( P-50)=\ln a+bx $\\
		      which can be written in the form
		      \[
			      Y=A+bx\qquad \text{ where}\quad Y=\ln (P-50),\quad A=\ln a.
		      \]
		      \[
			      E(A,b)=\sum_{i=1}^5 (A+bx_i-Y_i )^2
		      \]
		      At minimum,
		      \[
			      \frac{\partial E}{\partial A}=0\qquad\text{ and }\qquad\frac{\partial E}{\partial b}=0
		      \]
		      which give
		      \begin{align*}
			      2\sum[(A+bx_i-Y_i)]1  & =0 \\
			      2\sum (A+bx_i-Y_i)x_i & =0
		      \end{align*}
		      which can be rearranged as
		      \begin{align*}
			      A\sum 1+b\sum x_i          & =\sum Y_i     \\
			      A\sum x_i + b \sum {x_i}^2 & =\sum x_x Y_i
		      \end{align*}
		      The sum can be calculated in a tabular form as below:
		      \begin{table}
			      \centering
			      \begin{tabular}{ccS[table-format=3.2]*{2}{S[table-format=3.3]}c}
				      \toprule
				      $ n $ & $ x $ & \multicolumn{1}{c}{$ P $} & \multicolumn{1}{c}{$ Y $}  & \multicolumn{1}{c}{$ xY $} & $ x^2 $ \\\midrule
				      1     & 1     & 774.4 & 6.585  & 6.585  & 1       \\
				      2     & 2     & 603.4 & 6.316  & 12.632 & 4       \\
				      3     & 3     & 439.2 & 5.964  & 17.892 & 9       \\
				      4     & 4     & 360.0 & 5.737  & 22.948 & 16      \\
				      5     & 5     & 328.3 & 5.629  & 28.145 & 25      \\\midrule
				      Sum   & 15    &       & 30.231 & 88.202 & 55      \\\bottomrule
			      \end{tabular}
		      \end{table}
		      The normal equations are
		      \begin{align*}
			      5A+15b  & =30.231 \\
			      15A+55b & =88.202
		      \end{align*}
		      Dividing 1st eq. by 5 and 2nd by 15, we have
		      \begin{align*}
			      A+3b     & =6.046 \\
			      A+3.667b & =5.880
		      \end{align*}
		      Subtracting
		      \[
			      0.667b=-0.166 \qquad \text{or}\quad b=-0.249
		      \]
		      and
		      \[
			      A=5.88-3.667(-0.249)=6.793
		      \]
		      and hence
		      \[
			      a=e^6.793\approx 892
		      \]
		      The required best fitting curve is
		      \[
			      P=50+892e^{-0.249x}
		      \]
		\item When $ x = 6 $, we have
		      \[
			      P=50+892e^{-0.249\times 6}\approx 250.2
		      \]
		      The price of the 6 years old car is Tk. 250.2 thousand.
		      New price corresponds to $ x = 0 $ and is $ 50+892=942 $ thousand taka.
	\end{enumerate}
\end{soln}
\subsection{Exercise}
\begin{enumerate}
	\item Find the least square line  $ y=ax+b $ to the following data
	      \begin{table}
		      \centering
		      \begin{tabular}{*{2}{S[table-format=2.1]}}
			      \toprule
			      \multicolumn{1}{c}{$ x $ }&
				  \multicolumn{1}{c}{$ y $} \\\midrule
				    -1  &
					1.5    \\
					 0   &
					 3.7 \\
					  2   & 
					  6.2 \\
					  4   &
					  8.5 \\
					   5    &
						    12.8 \\\bottomrule
		      \end{tabular}
	      \end{table}
	\item Students collected the following set of data to find the gravitational constant $ g $. Use the relation  $ d = (gt^2)/2 $,  where  $ d $  is the distance in metres and $ t $ is time in seconds, to find the value of  $ g $.
	      \begin{table}[H]
		      \centering
		      \begin{tabular}{*{2}{S[table-format=1.4]}}
			      \toprule
			      \multicolumn{1}{c}{Time $ (t) $}     &
				  \multicolumn{1}{c}{Distance $ (d) $} \\\midrule 
				   0.2    &
				   0.2142 \\
				    0.4    &
					0.7789 \\
					 0.6    &
					 1.7676 \\
					  0.8    &
					  3.1365 \\
					   5.0    &
						   4.9075 \\\bottomrule
		      \end{tabular}
	      \end{table}
	\item Fit a curve of the form  $ y=ax^2+be^{-x} $    to the following data.
	      \begin{table}[H]
		      \centering
		      \begin{tabular}{cS[table-format=2.2]}
			      \toprule
			      $ x $ &
				  \multicolumn{1}{c}{$ y $ }\\\midrule
				   1    &
				   5.18 \\
				    2    & 
					6.70 \\
					4     & 
					21.31 \\
					5     &
					33.07 \\
					8     &
					84.48 \\\bottomrule
		      \end{tabular}
	      \end{table}
	\item The temperature in a metal strip was measured at various time intervals. Given that the relation between the temperature $ T ({}^\circ C) $ and time $ t $ (min) is of the form $ T=a+be^{t/2} $. Six pairs of observations of the two variables $ T $ and $ t $ gave the following results.
	      \[
		      \sum T=165.5,\quad    \sum e^{t/2}=48.71,\quad \sum e^t=656.6,\quad \sum Te^{t/2}=1425
	      \]
	      Find the temperature after 7 minutes.
	\item Given the following set of values of $ x $ and $ y :$
	      \begin{table}[H]
		      \centering
		      \begin{tabular}{cS[table-format=1.2]}
			      \toprule
			      $ x $ &
				  \multicolumn{1}{c}{$ y $ }\\\midrule 
				   2    &
				   1.14 \\
				    3    &
					1.45 \\
					 6    &
					 1.97 \\
					  7    &
					  2.41 \\
					   10   &
						   2.99 \\\bottomrule
		      \end{tabular}
	      \end{table}
	      \begin{enumerate}
		      \item Fit the power equation  $ y=ax^b $ to the given data.
		      \item Find the best fitting curve of the form $ y = a e^{bx} $.
		      \item Fit the saturation growth rate model $ y=\frac{ax}{b+x} $ to the above data.
	      \end{enumerate}
	      By finding errors determine which one of the above is the best fitting curve.
	\item The following table gives the population of a certain country from 1960 to 2000 at ten yearly interval
	      \begin{table}[H]
		      \centering
		      \begin{tabular}{cc}
			      \toprule
			      Year                      & 
				  Population $ P $ (in Lac) \\\midrule
				  1970 &
				  20.5 \\
				   1980 &
				   26.4 \\
				    1990 &
					33.1 \\
					 2000 &
					 40.4 \\
					  2010 &
					   48.2 \\\bottomrule
		      \end{tabular}
	      \end{table}
	      It is known that if environmental factors remain constant, the population size,  $ P $,  is given by
	      \[
		      P(t)=\frac{200}{1+c e^{at}}
	      \]
	      where $ c $ and $ a $ are constants.\\
	      Estimate, to 3 significant figures, the values of $ c $ and $ a $.\\
	      Hence, predict the population in the year 2015.
	\item A bowl of hot water is kept in a room of constant temperature $ 25^\circ C $. At 5 minutes interval temperature of the water is recorded and listed as given below.
	      \begin{table}[H]
		      \centering
		      \begin{tabular}{cc}
			      \toprule
			      $ t $ (in min)            &
				  $ T $ (in $ {}^\circ C $) \\\midrule
				   5    &
				   75.3 \\
				    10   &
					70.0 \\
					 15   &
					 63.4 \\
					  20   &
					  58.5 \\
					   25   &
					   54.0 \\\hline
		      \end{tabular}
	      \end{table}
	      The law of cooling can be assumed to be of the form	$ T=25+ae^{-kt} $.\\
	      Find, to 2 significant figures, the best values of $ a $ and $ k $.\\
	      Estimate the initial temperature.\\
	      Hence, find the time, to the nearest minute, when the temperature of the water in the bowl will be  $ 51^\circ C $.
	\item The pressure $ p $ and volume $ V $ of a fixed mass of gas are believed to be related by an equation of the form
	      \[
		      pV^\gamma=c
	      \]
	      where $ \gamma $ and $ c $ are constants.\\
	      In six set of experiments on the fixed mass of gas, in each of which $ p $ was controlled and $ V $ measured. The results are as follows:
	      \begin{table}[H]
		      \centering
		      \begin{tabular}{cc}
			      \toprule
			      $ p\,(Nm^{-2}) $ &
				  $ V\,(m^3) $     \\\midrule
				   0.4   & 
				   1.894 \\
				   0.6   &
				   1.426 \\
				    0.8   &
					1.166 \\
					 1.0   &
					 0.998 \\
					  1.2   &
					  0.878 \\
					   1.4   &
					    0.789 \\\bottomrule
		      \end{tabular}
	      \end{table}
	      Estimate to 2 decimal places
	      \begin{enumerate}[label=(\roman*)]
		      \item the value of $ \gamma $,
		      \item the value of $ V $ when $ p = 0.75 Nm^{-2} $.
	      \end{enumerate}
	\item The method of least squares is used to estimate the constants $ a $, $ b $, $ c $ in the formula
	      \[
		      y=a+b \sin x+ce^{-x/5}
	      \]
	      Eight pairs of data points leads to the following normal equations, where the missing numerical values are to be determined.
	      \begin{align*}
		      {\_\_\_\_\_}a  -  3.616 b  + 5.784 c       & =   100.06  \\
		      {\_\_\_\_\_}a +   3.261 b  -  2.737 c      & = - 46.796  \\
		      {\_\_\_\_\_}a  + {\_\_\_\_\_} b +  4.403 c & =    73.060
	      \end{align*}
	      The method of Gaussian elimination leads to the following equations:
	      \begin{align*}
		      a  -  0.452 b + {\_\_\_\_\_} c & = {\_\_\_\_\_} \\
		      0.450 b -  0.034 c             & =  -0.433      \\
		      1.734 c                        & =  4.895
	      \end{align*}
	      Supply the missing values and complete the solution of the set of equations, giving $ a $, $ b $ and $ c $ to two decimal places.\\
	      Using the estimated values of $ a $, $ b $ and $ c $, find the value of $ y $ given by the formula to one decimal place when $ x = 4 $.
	\item Use a suitable substitution to derive a linearized form for the following functions:
	      \begin{enumerate}[label=(\roman*)]
		      \item \[y=\frac{x}{ax+b}\]
		      \item \[y=(ax+b)^{-1}\]
		      \item \[x=a^x b^y+2e^y\]
		      \item \[y=\frac{5+ax}{b+x^3}\]
		      \item \[y =\frac{x^2}{(a+bx)^2}\]
		      \item \[y=\frac{1}{a(2^y)+b(2^{-x})}\]
		      \item \[y=\frac{x^2}{(ax+1)(bx+2)}\]
		      \item \[y=\frac{x}{1+ae^{bx} }\]
		      \item \[y=\frac{1}{\sqrt{(x+a)(x+b)}}\]
	      \end{enumerate}
\end{enumerate}
\section{Spline Interpolation}
Spline interpolation is a form of interpolation where the interpolant is a special type of  piecewise polynomial called a \textbf{spline}. Spline interpolation is preferred over polynomial interpolation because the interpolation error can be made small even when using low degree polynomials for the spline.\\
Divide the interval containing the tabular points as subintervals 	$ x_0<x_1<x_2<\dots<x_n  $ and replace the function $ f(x) $ by some lower degree interpolating polynomial in each of the subinterval. The tabular points $ x_0, x_1, x_3,\dots, x_n   $ at which the function changes its character is termed as \textbf{knots} in the theory of spline.\\


A function $ S(x) $ of the form
\[
	S(x)=\begin{cases}
		f_0 (x)     & x\in [x_0,x_1]    \\
		f_1 (x)     & x\in [x_1,x_2]    \\
		\vdots      &                   \\
		f_{n-1} (x) & x\in[x_{n-1},x_n]
	\end{cases}
\]
is called a spline of degree $ m $ if
\begin{enumerate}[label=(\roman*)]
	\item the domain of $ S(x) $ is the interval $ [x_0,x_n] $
	\item $ S(x),\, S'(x),\, S''(x),\dots, S^{(m-1)} (x) $ are all continuous functions on $ [x_0, x_n] $
	\item $ S(x) $ is a polynomial of degree less than equal to $ m $ on each subintervals $ [x_k, x_{k+1}]$, $k=0,1,2,3,\dots,n-1 $.
\end{enumerate}
\subsection{Linear Spline Interpolation}
The simplest polynomial to use, a polynomial of degree one, produces a polygon path that consists of line segments that pass through the points. The point-slope formuala for the line segment may be used to represent this piecewise linear curve:
\[
	S(x)=f_k (x)=a_k (x-x_k)+b_k,\qquad \text{for } x_k \leq x \leq x_{k+1}(k = 0, 1, 2,\dots, n-1)
\]
Since the line passes through $ (x_k, y_k) $ and $ (x_{k+1}, y_{k+1}) $ we have
\[
	f_k (x_k)=y_k=b_k
\]
and
\[
	f_k (x_{k+1})=y_{k+1}=a_k (x_{k+1}-x_k)+b_k
\]
or
\[
	a_k=\frac{y_{k+1}-y_k}{x_k+1-x_k}=\frac{ \Delta y_k}{h_k}
\]
where
\[
	\Delta y_k=y_{k+1}-y_k \qquad \text{and}\quad h_k=x_{k+1}-x_k
\]
\begin{figure}[H]
	\centering
	\import{../tikz/}{linear-spline.tikz}
\end{figure}
The resulting linear spline curve $ f_k (x) $ in $ [x_k,x_{k+1}] $ can be written as
\[
	f_k (x)=\frac{\Delta y_k}{h_k} (x-x_k)+y_k,\qquad (k=0,1,2,\dots,n-1)
\]
The resulting curve looks like a ``broken line'' as shown in the diagram.
\subsection{Quadratic Spline Interpolation}
For a quadratic spline through $ (x_k,y_k) $ we may take $ f_k (x) $ is of the form
\begin{equation}
	\label{eq:quadspline1}
	f_k(x)=a_k (x-x_k)^2+b_k (x-x_k)+c_k
\end{equation}
The quadratic spline function $ S(x) $ is
\[
	S(x)=f_k (x)
\]
on the interval $ [x_k, x_{k+1}] $ for $ k = 0, 1, 2,\dots, n-1 $.\\
Each quadratic polynomial, $ f_k (x) $, has three unknown constants, hence there are $ 3n $ unknown coefficients $ a_k, b_k, c_k\,(k = 0, 1, 2, \dots, n-1) $.\\
To find $ 3n $ unknowns, one needs to set up $ 3n $ equations and then simultaneously solve them. These $ 3n $ equations are found as follows:
\begin{enumerate}
	\item As the splines pass through $ (x_k,y_k) $, we have
	      \[
		      f_k (x_k)=c_k=y_k\qquad \text{for}\quad k = 0, 1, 2,\dots, n-1
	      \]
	      and
	      \[
		      f_{n-1} (x_n)=y_n
	      \]
	\item Continuity of $ S(x) $ at the interior points gives
	      \begin{equation}
		      \label{eq:quadspline2}
		      f_k (x_{k+1})=f_{k+1} (x_{k+1})
	      \end{equation}
	      Since there are $ (n-1) $ interior points, we have $ (n-1) $ such equations.
	\item Continuity of $ S' (x) $ at the interior points gives
	      \begin{equation}
		      \label{eq:quadspline3}
		      f_k^{'} (x_{k+1})=f_{k+1}^{'} (x_{k+1})
	      \end{equation}
\end{enumerate}
Differentiating eq. \eqref{eq:quadspline1} we have
\[
	f_k^{'} (x)=2a_k (x-x_k)+b_k
\]
From \eqref{eq:quadspline3} we have
\[
	2a_k h_k+b_k=b_{k+1}
\]
Using the notation,
\[
	f_k^{'}(x_k)=b_k=Z_k
\]
we have
\[
	a_k=\frac{Z_{k+1}-Z_k}{2h_k},\qquad k = 0, 1, 2, \dots, n-1
\]
From Eq. \eqref{eq:quadspline2},
\begin{align*}
	 & a_k (h_k^2)+b_k (h_k)+c_k=c_{k+1}                            \\
	 & a_k h_k+b_k=\frac{c_{k+1}-c_k}{h_k} =\frac{y_{k+1}-y_k}{h_k} \\
	 & \frac{Z_{k+1}-Z_k}{2}+Z_k=\frac{y_{k+1}-y_k}{h_k}            \\
	 & Z_{k+1}+Z_k=2\frac{\Delta y_k}{h_k}
\end{align*}
Here also $ (n-1) $ interior points, we have $ (n-1) $ such equations.\\
So far, the total number of equations are
\[
	(2n)+(n-1)=(3n-1)
\]
We still need one more equation. We can assume that the first spline is linear, that is
\[
	a_0=0
\]
This gives us $ 3n $ equations for $ 3n $ unknowns.  These can be solved by a number of techniques used to solve simultaneous linear equations.\\
It should be mentioned that the curvature of the quadratic spline changes abruptly at each knot, and the curve may not be pleasing to the eye.
\subsection{Cubic Spline Interpolation}
For the cubic spline through the points $ (x_k,y_k ) $, $ k=0,1,2,\dots,n $ we may take $ f_k (x) $ is of the form
\begin{equation}
	\label{eq:cubespline1}
	f_k (x)=a_k (x-x_k )^3+b_k (x-x_k )^2+c_k (x-x_k)+d_k \quad \text{in} [x_k, x_{k+1}]
\end{equation}
Thus the cubic spline function $ S(x) $ is of the form
\[
	S(x)=f_k (x) \qquad\text{on the interval}\quad [x_k,x_{k+1}] \quad\text{for } k = 0, 1, 2, 3,\dots,n-1
\]
with the following properties:
\begin{enumerate}[label=(\alph*)]
	\item \label{enum:a}\[
		      f_k (x_k)=y_k,\qquad k=0,1,2,\dots,n-1\quad \text{and}\quad f_{n-1} (x_n)=y_n
	      \]
	\item \label{enum:b}\[
		      f_k (x_{k+1})=f_{k+1} (x_{k+1}),\qquad k=0,1,2,\dots,n-1
	      \]
	\item \label{enum:c}\[
		      f_k^{'} (x_{k+1})=f_{k+1}^{'} (x_{k+1}),\qquad k=0,1,2,\dots,n-1
	      \]
	\item \label{enum:d}\[
		      f_{k}^{''} (x_{k+1})=f_{k+1}^{''} (x_{k+1}),\qquad k=0,1,2,\dots,n-1
	      \]
\end{enumerate}
Each cubic polynomial, $ f_k (x) $, has four unknown constants, hence there are $ 4n $ coefficients to be determined. The data points supply $ (n+1) $ conditions, and properties \ref{enum:b}, \ref{enum:c} and \ref{enum:d} each supply $ (n-1) $ conditions. Hence, $ n+1+3(n-1)=4n-2 $ conditions are specified. Two more conditions are needed which will be discussed later.\\

The conditions \ref{enum:a} then gives
\[
	d_k=y_k,\qquad k=0,1,2,\dots,n-1
\]
From \ref{enum:b} we have
\begin{align}
	y_{k+1} & =a_k (x_{k+1}-x_k )^3+b_k (x_{k+1}-x_k )^2+c_k (x_{k+1}-x_k)+y_k\notag              \\
	        & =a_k {h_k}^3+b_k {h_k}^2+c_k h_k+y_k,\qquad k=0,1,2,\dots,n-1\label{eq:cubespline3}
\end{align}
where $ h_k=(x_{k+1}-x_k) $.\\
Differentiating \eqref{eq:cubespline1} we have
\begin{align}
	f_k^{'} (x) & =3a_k (x-x_k )^2+2b_k (x-x_k)+c_k\label{eq:cubespline4} \\
	f_k^{''}(x) & =6a_k(x-x_k)+2b_k \label{eq:cubespline5}
\end{align}
Development is simplified if we write the equations in terms of the second derivatives-that is, if we use
\[
	M_k=f^{''}_k(x_k) \qquad \text{for } k=0,1,2,\dots,n-1 \quad \text{ and } M_n=f^{''}_k(x_k)
\]
From Eq.\eqref{eq:cubespline5}, we have
\begin{align*}
	M_k     & =6a_k (x_k-x_k)+2b_k=2b_k              \\
	M_{k+1} & =6a_k (x_{k+1}-x_k)+2b_k=6a_k h_k+2b_k
\end{align*}
Hence we can write
\begin{align*}
	b_k & =\frac{M_k}{2}            \\
	a_k & =\frac{M_{k+1}-M_k}{6h_k}
\end{align*}
and from Eq.\eqref{eq:cubespline3}, we have
\begin{align*}
	y_{k+1}=\left(\frac{M_{k+1}-M_k}{6h_k} \right){h_k}^3+\frac{M_k}{2} {h_k}^2+c_k h_k+y_k \\
	c_k & =\frac{y_{k+1}-y_k}{h_k} -\frac{h_k}{6}(M_{k+1}+2M_k)
\end{align*}
In order to get the cubic splines, it is required to determine the second derivatives
\[
	M_0,M_1,M_2,\dots,M_n
\]
at the knots and can be evaluated by the continuity of the second derivatives.\\
From Eq. \eqref{eq:cubespline1},
\[
	f_k^{'} (x)=3a_k (x-x_k )^2+2b_k (x-x_k)+c_k
\]
At the common knot $ (x_{k+1},y_{k+1}) $ the first derivatives $ f_k^{'} (x) $ and $ f_(k+1)^{'} (x) $ should be equal i.e.
\[
	f_{k+1}^{'} (x_{k+1})=f_k^{'} (x_{k+1})
\]
But
\begin{equation}
	\label{eq:cubespline6a}
	f_{k+1}^{'} (x_{k+1})=c_{k+1}=\frac{y_{k+2}-y_{k+1}}{h_{k+1}}-\frac{h_{k+1}}{6}(M_{k+2}+2M_{k+1})
\end{equation}
and
\begin{align}
	f_k^{'} (x_{k+1}) & =3a_k h_k^2+2b_k h_k+c_k\notag                                                                                                                 \\
	                  & =3\left(\frac{M_{k+1}-M_k}{6h_k}\right){h_{k-1}}^2+2\left(\frac{M_k}{2}\right) h_k+\frac{y_{k+1}-y_k}{h_k} -\frac{h_k}{6}(M_{k+1}+2M_k )\notag \\
	                  & =\frac{y_{k+1}-y_k}{h_k} +\frac{h_k}{6}(M_{k+1}+2M_k)\label{eq:cubespline6b}
\end{align}
Eq. \eqref{eq:cubespline6a} with Eq.\eqref{eq:cubespline6b},
\begin{equation}
	\label{eq:cubespline7}
	h_k M_k+2(h_k+h_{k+1})M_{k+1}+h_{k+1} M_{k+2}=6\left[\frac{\Delta y_{k+1}}{h_{k+1}} -\frac{\Delta y_k}{h_k} \right],\qquad k = 0, 1, 2,\dots, n-2
\end{equation}

\subsection{End Points Constraints}
We need to impose suitable end-conditions to get a unique cubic spline. The standard end points constraints are mentioned below.
\begin{table}[H]
	\centering
	\begin{tabular}{ll}
		\toprule
		\multicolumn{1}{c}{Description of the strategy}&\multicolumn{1}{c}{ Equations involving $ M_0 $ and $ M_n $ }\\\midrule
		
        Natural cubic spline ``a relaxed curve'': & \multicolumn{1}{c}{$ M_0=0,$}\\
		$ S^{'}(x_0) $ and $ S^{''}(x_n )$. & \multicolumn{1}{c}{$M_n=0 $ }\\%\midrule
		&\\
        Clamped cubic spline:& \\
		specify $ S^{'} (x_0) = A $ and $ S^{'} (x_n) = B $. & $\displaystyle 2M_0+M_1 =\frac{6}{h_0}\left[ \frac{\Delta y_0}{h_0}-A \right]$\\
					& $\displaystyle M_n+2M_{n-1} =\frac{6}{h_{n-1}}6\left[ B-\frac{\Delta y_{n-1}}{h_{n-1}} \right] $      \\%\midrule
		& \\
        Extrapolated cubic spline:& \\
		$ M_0 $ as linear extrapolation from & \\
		$ M_1 $ and $ M_2 $ :$ \frac{M_1-M_0}{h_0}=\frac{M_2-M_1}{h_1} $ & $\displaystyle M_0 =M_1-\frac{h_0 (M_2-M_1)}{h_1}$\\
		$ M_n $ as linear extrapolation from & \\
		$ M_{n-1} $ and $ M_{n-2} $: & $\displaystyle M_n  =M_{n-1}-\frac{h_{n-1}(M_{n-1}-M_{n-2})}{h_{n-2}}$\\
		$\displaystyle \frac{M_n-M_{n-1}}{h_{n-1}}=\frac{M_{n-1}-M_{n-2}}{h_{n-2}} $ & \\%\midrule
		& \\
        Parabolically terminated spline & \\
		($ S^{''}(x) $ is constant near the end points)&\multicolumn{1}{c}{ $ M_0=M_1,\quad M_n=M_{n-1} $}\\\bottomrule
	\end{tabular}
\end{table}
\begin{prob}
	Find the linear spline for the following data:
	\begin{table}[H]
		\centering
		\begin{tabular}{cc}
			\toprule
			$ x $	&
			$ y $	\\\midrule
			0	&
			0.0	\\
			1	&
			0.5	\\
			2	&
			2.0	\\
			3&
			1.5 \\\bottomrule
		\end{tabular}
	\end{table}
\end{prob}
\begin{soln}
	Here,  $ h_0=h_1=h_2=1 $.\\
	Linear spline functions are
	\begin{align*}
		f_0 (x)&=\frac{\Delta y_0}{h_0} (x-x_0)+y_0=0.5x			& 0 \leq x \leq 1\\
		f_1 (x)&=\frac{\Delta y_1}{h_1} (x-x_1)+y_1=1.5(x-1)+0.5		& 1 \leq x \leq 2\\
		f_2 (x)&=\frac{\Delta y_2}{h_2} (x-x_2)+y_2=-0.5(x-2)+2		& 2 \leq x \leq 3
	\end{align*}
	Linear spline function is
	\[
		S(x)=\begin{cases}
			0.5x, & 0 \leq x \leq 1\\
			1.5(x-1)+0.5,& 1 \leq x \leq 2\\
			-0.5(x-2)+2, & 2 \leq x \leq 3
		\end{cases}
	\]	
\end{soln}
\begin{prob}
	Find the quadratic spline for the following data:
	\begin{table}[H]
		\centering
		\begin{tabular}{cc}
			\toprule
			$ x $	&
			$ y $	\\\midrule
			0	&
			0.0	\\
			1	&
			0.5	\\
			2	&
			2.0	\\
			3&
			1.5\\\hline
		\end{tabular}
	\end{table}
\end{prob}
\begin{soln}
	Here,  $ h_0=h_1=h_2=1 $.
	\[
		Z_{k+1}+Z_k=2 \frac{\Delta y_k}{h_k}\qquad k = 0, 1, 2   
	\]
	Using the recurrence relation, we obtain the equations
	\begin{align*}
		Z_1+Z_0&=2(0.5)=1\\
		Z_2+Z_1&=2(2-0.5)=3\\
		Z_3+Z_2&=2(1.5-2)=-1\\
	\end{align*}
	Using the end condition $ a_0=0 $, we have
	\[
		\frac{Z_1-Z_0}{2(1)}=0\qquad \text{or}\quad Z_1=Z_0
	\]
	Solving above equations, we have
	\begin{align*}
		Z_0&=Z_1=\frac{1}{2}=0.5\\
		Z_2&=3-0.5=2.5\\
		Z_3&=-1-2.5=-3.5
	\end{align*}
	With these values of Z's the spline coefficients the spline coefficients can be obtained as follows:\\
	With k = 0,
	\[
		a_0=0,\qquad b_0=Z_0=0.5,\qquad c_0=y_0=0
	\]
	and
	\[
		f_0 (x)=0.5x,\qquad 0 \leq x \leq 1
	\]
	With k = 1,
	\[
		a_1=\frac{Z_2-X_1}{2h_1}=\frac{2.5-0.5}{2(1)}=1,\qquad b_1=Z_1=0.5,\qquad c_1=y_1=0.5
	\]
	and
	\[
		f_1 (x)=(x-1)^2+0.5(x-1)+0.5,\qquad 1 \leq x \leq 2
	\]
	With k = 2,
	\[
		a_2=\frac{Z_3-Z_2}{2h_2}=\frac{-3.5-2.5}{2(1)}=3,\qquad b_2=Z_2=2.5,\qquad c_2=y_2=2
	\]
	and
	\[
		f_2 (x)=-3(x-2)^2+2.5(x-2)+2,\qquad 2 \leq x \leq 3
	\]
	The quadratic spline function is
	\[
		S(x)=\begin{cases}
			0.5x, &0 \leq x \leq 1\\
			(x-1)^2+0.5(x-1)+0.5,& 1 \leq x \leq 2\\
			-3(x-2)^2+2.5(x-2)+2, &2 \leq x \leq 3
		\end{cases}
	\]
\end{soln}
\begin{prob}
	Consider the points
	\begin{table}[H]
		\centering
		\begin{tabular}{cc}
			\toprule
			$ x $	&
			$ y $	\\\midrule
			0	&
			0.0	\\
			1	&
			0.5	\\
			2	&
			2.0	\\
			3&
			1.5\\\bottomrule
		\end{tabular}
	\end{table}
	\begin{enumerate}[label=(\alph*)]
		\item Find the natural cubic spline which fits the given data.
		\item Find the clamped cubic spline with conditions $ S^{'} (0)=1 $ and $ S^{'} (3)=-1 $.
		\item Find the extrapolated cubic spline.
	\end{enumerate}	
\end{prob}
\begin{soln}
	The governing recurrence is
	\[
		h_k M_k+2(h_k+h_{k+1})M_{k+1}+h_{k+1} M_{k+2}=6\left[\frac{\Delta y_{k+1}}{h_{k+1}} -\frac{\Delta y_k}{h_k} \right],\qquad k = 0, 1, 2
	\] 
	First, compute the quantities
	\[
		h_0=h_1=h_2=1
	\]
	and
	\[
		\frac{\Delta y_0}{h_0} =\frac{0.5-0}{1}=0.5,\quad\frac{ \Delta y_1}{h_1} =\frac{2-0.5}{1}=1.5,\quad \frac{\Delta y_2}{h_2} =\frac{1.5-2.0}{1}=-1.5
	\]
	\begin{enumerate}[label=(\alph*)]
		\item Using natural cubic spline\\
		Here the end conditions are
		\[
			M_0=M_3=0
		\]
		The equations corresponding to $ k = 0, 1 $ are
		\[
			M_0+4M_1+M_2=6(1.5-0.5)=6
		\]
		or
		\begin{equation}
			\label{eq:soln1}
			4M_1+M_2=6
		\end{equation}
		and
		\begin{equation*}	
			M_1+4M_2+M_3=6(-0.5-1.5)=-12
		\end{equation*}
		or
		\begin{equation}
			\label{eq:soln2}
			M_1+4M_2=-12
		\end{equation}
		Solving \eqref{eq:soln1} and \eqref{eq:soln2},
		\[
			M_1=2.4,\qquad M_2=-3.6
		\]
		With $ k = 0 $,
		\begin{align*}
			a_0&=\frac{M_1-M_0}{6}=\frac{2.4}{6}=0.4\\
			b_0&=\frac{M_0}{2}=0\\
			c_0&=\frac{\Delta y_0}{h_0} -\frac{h_0}{6}(M_1+2M_0)=0.5-\frac{2.4}{6}=0.1\\
			d_0&=y_0=0
		\end{align*}
		and
		\[
			f_0 (x)=0.4x^3+0.1x,\qquad 0 \leq x \leq 1
		\]
		With $ k = 1 $,
		\begin{align*}
			a_1&=\frac{M_2-M_1}{6}=\frac{-6}{6}=-1\\
			b_1&=\frac{M_1}{2}=\frac{2.4}{2}=1.2\\
			c_1&=\frac{\Delta y_1}{h_1} -\frac{h_1}{6}(M_2+2M_1)=1.5-\frac{1.2}{6}=1.3\\	
			d_1&=y_1=0.5
		\end{align*}
		and
		\[
			f_1 (x)=-(x-1)^3+1.2(x-1)^2+1.3(x-1)+0.5\qquad 0 \leq x \leq 1
		\]
		With $ k = 2 $,
		\begin{align*}
			a_2&=\frac{M_3-M_2}{6}=\frac{3.6}{6}=0.6\\
			b_2&=\frac{M_2}{2}=\frac{-3.6}{2}=-1.8\\
			c_2&=\frac{\Delta y_2}{h_2} -\frac{h_2}{6}(M_3+2M_2)=-0.5-\frac{-7.2}{6}=0.7\\
			d_2&=y_2=2\\
		\end{align*}
		and
		\[
			f_2 (x)=0.6(x-2)^3-1.8(x-2)^2+0.7(x-2)+2,\qquad 0 \leq x \leq 1
		\]
		The natural cubic spline function is
		\[
			S(x)=\begin{cases}
				0.4x^3+0.1x, &0 \leq x \leq 1\\
				-(x-1)^3+1.2(x-1)^2+1.3(x-1)+0.5, &1 \leq x \leq 2\\
				0.6(x-2)^3-1.8(x-2)^2+0.7(x-2)+2, &2 \leq x \leq 3	
			\end{cases}
		\]
		\item With clamped spline condition\\
		The first derivative boundary conditions are:
		\[
			S^{'} (0)=0.2 \qquad \text{and} \qquad S^{'} (3)=-1
		\]
		The equations involving M's are
		\begin{align*}
			&\text{At left end:}		&2M_0+M_1=6(0.5-0.2)=1.8\\
			&\text{For } k = 0: &M_0+4M_1+M_2=6(1.5-0.5)=6\\
			&\text{For } k = 1:		&M_1+4M_2+M_3=6(-0.5-1.5)=-12\\
			&\text{At right end:}		&M_2+2M_3=6(-1+0.5)=-3
		\end{align*}
		Solution of the above equations are
		\[
			M_0=-0.36,\qquad M_1=2.52,\qquad M_2=-3.72,\qquad M_3=0.36
		\]
		Corresponding spline coefficients are
		\begin{align*}
			&k=0:		&a_0=0.48,\quad b_0=-0.18,\quad  c_0=0.2,\quad  d_0=0\\
			&k=1:		&a_1=-1.04,\quad  b_1=1.26,\quad  c_1=1.28,\quad  d_1=0.5\\
			&k=2:		&a_2=0.68,\quad  b_2=-1.86,\quad  c_2=0.68,\quad  d_2=2
		\end{align*}
		Thus the clamped cubic spline function is
		\[
			S(x)=\begin{cases}
				0.48x^3-0.18x^2+0.2x, &0 \leq x \leq 1\\
				-1.04(x-1)^3+1.26(x-1)^2+1.28(x-1)+0.5, &1 \leq x \leq 2\\
				0.68(x-2)^3-1.8(x-2)^2+0.68(x-2)+2, &2 \leq x \leq 3
			\end{cases}
		\]
		\item With extrapolated boundary condition\\
		The equations involving M's are
		\begin{align*}
			&\text{At left end:} &M_1-M_0=M_2-M_1\\
			&\text{or}			&M_0-2M_1+M_2=0\\
			&\text{For }k = 0:		&M_0+4M_1+M_2=6(1.5-0.5)=6\\
			&\text{For }k = 1:		&M_1+4M_2+M_3=6(-0.5-1.5)=-1.2\\
			&\text{At right end:}		&M_3-M_2=M_2-M_1\\
			&\text{or}			&M_1-2M_2+M_3=0
		\end{align*}
		Solution of the above equations are
		\[
			M_0=4,\qquad    M_1=1,\qquad    M_2=-2,\qquad    M_3=-5
		\]
		Corresponding spline coefficients are
		\begin{align*}
			&k=0:		&a_0=-0.5,\quad  b_0=2,\quad  c_0=-1,\quad  d_0=0\\
			&k=1:		&a_1=-0.5,\quad  b_1=0.5,\quad  c_1=1.5,\quad  d_1=0.5\\
			&k=2:		&a_2=-0.5,\quad  b_2=-1,\quad  c_2=1,\quad  d_2=2
		\end{align*}
		Thus the cubic spline function with extrapolated boundary condition is
		\[
			S(x)=\begin{cases}
				-0.5x^3+2x^2-x, &0 \leq x \leq 1\\
				-0.5(x-1)^3+0.5(x-1)^2+1.5(x-1)+0.5,& 1 \leq x \leq 2\\
				-0.5(x-2)^3-(x-2)^2+(x-2)+2, &2 \leq x \leq 3
			\end{cases}
		\]
	\end{enumerate}
	\end{soln}
	Comparison of the three types of spline curves are shown below:
	\begin{figure}[H]
		\centering
		\import{../tikz/}{spline.tikz}
	\end{figure}
\section{EXERCISES}
\begin{enumerate}
	\item Determine whether this function is a first degree spline:
	\[
		f(x)=\begin{cases}
			x,& -1 \leq x \leq 1\\
			1-2(x-1),& 1 \leq x \leq 2\\
			-1+3(x-2), &2 \leq x \leq 3
		\end{cases}
	\]
	\item Is $ f(x)=\abs{x} $ a first degree spline? Why or why not?
	\item Are these functions quadratic splines? Explain why or why not.
	\begin{enumerate}
		\item \[
			f(x)=\begin{cases}
				0.1x^2, &0 \leq x \leq 1\\
				9.3x^2-18.4x+9.2,& 1 \leq x \leq 1.3
			\end{cases}
		\]
		\item \[
			f(x)=\begin{cases}
				-x^2, &x \leq 0\\
				x , &x>0
			\end{cases}
		\]
	\end{enumerate}
	\item Find first-degree and quadratic splines for the following data:
	\begin{table}[H]
		\centering
		\begin{tabular}{S[table-format=1.1]c}
			\toprule
			\multicolumn{1}{c}{$ x $}	&
			$ y $	\\\midrule
			-1.0	&
			2	\\
			0.0	&
			1	\\
			0.5	&
			0	\\
			1.0	&
			1	\\
			2.0&
			2\\\bottomrule
		\end{tabular}
	\end{table}
	\item Prove that the derivative of a quadratic spline is a first degree spline.
	\item Show that the indefinite integral of a first-degree spline is a second-degree spline.
	\item Determine whether $ f(x) $ is a cubic spline with knots -1, 0, 1 and 2:
	\[
		f(x)=\begin{cases}
			1+2(x+1)+(x+1)^3, &-1 \leq x \leq 0\\
			4+5x+3x^3, &0 \leq x \leq 1\\
			11+1(x-1)+3(x-1)^2+(x-1)^3, &1 \leq x \leq 2
		\end{cases}
	\]
	\item A natural cubic spline $ S $ on $ [0, 2] $ is defined by
	\[
		S(x)=\begin{cases}
			1+2x-x^3, 0 \leq x \leq 1\\
			2+b(x-1)+c(x-1)^2+d(x-1)^3, &1 \leq x \leq 2
		\end{cases}
	\]
	Find $ b $, $ c $, and $ d $.
	\item A natural cubic spline for a function $ f(x) $ on $ [-1, 2] $ is defined by
	\[
		f(x)=\begin{cases}
			A(x+1)^3+B(x+1)^2-5(x+1)+5, &-1 \leq x \leq 0\\
			x^3+3x^2-2x+1, &0 \leq x \leq 1\\
			a(x+1)^3+b(x+1)^2+c(x+1)+d, &1 \leq x \leq 2
		\end{cases}
	\]
	Find the values of $ A $, $ B $, $ a $, $ b $, $ c $ and $ d $.
	Hence, estimate the values of $ f(-0.5) $ and $ f(1.5) $.
	\item A clamped cubic spline for a function $ f(x) $ is defined on $ [1, 3] $ by
	\[
		f(x)=\begin{cases}
			3(x-1)+2(x-1)^2-(x-1)^3, &0 \leq x \leq 1\\
			a+b(x-2)+c(x-2)^2+d(x-2)^3,& 1 \leq x \leq 2
		\end{cases}
	\]
	Given $ f^{'}(1)=f^{'}(3) $, find $ a $, $ b $, $ c $, and $ d $.
	\item Find the natural cubic splines satisfying the following data points:
	\begin{enumerate}
		\item $ (0,1) $, $ (1,1) $ and $ (2,5) $
		\item $ (-1,1) $, $ (0,2) $ and $ (1,-1) $
	\end{enumerate}
	\item Find the natural cubic spline which fits the following data:
	\begin{table}[H]
		\centering
		\begin{tabular}{cc}
			\toprule
			$ x $	&
			$ y $	\\\midrule
			1	&
			1	\\
			2	&
			5	\\
			3	&
			11	\\
			4&
			8\\\bottomrule
		\end{tabular}
	\end{table}
	and hence find the values of $ y(1.5) $.
	\item Find the natural cubic spline which fits the following data:
	\begin{table}[H]
		\centering
		\begin{tabular}{cS[table-format=1.0]}
			\toprule
			$ x $	&
			\multicolumn{1}{c}{$ f(x) $}	\\\midrule
			1	&
			6	\\
			2	&
			-3	\\
			3	&
			6	\\
			4	&
			2	\\
			5&
			-6\\\bottomrule
		\end{tabular}
	\end{table}
	Find $ f(x) $ at $ x = 1.3 $.
	\item Consider the points 
	\begin{table}
		\centering
		\begin{tabular}{*{2}{S[table-format=2.0]}}
			\toprule
			\multicolumn{1}{c}{$ x $}	&
			\multicolumn{1}{c}{$ y $}	\\
			-1	&
			9	\\
			0	&
			26	\\
			3	&
			56	\\
			4&
			29\\\bottomrule
		\end{tabular}
	\end{table}
	\begin{enumerate}
		\item Find the natural cubic spline which fits this data and hence estimate the value of  $ y(1) $.
		\item Find the clamped cubic spline with conditions $ S^{'} (-1)=1 $ and $ S^{'} (4)=-1 $.
		\item Find the extrapolated cubic spline.
	\end{enumerate}
	\item Consider the points  $ (0,1) $, $ (1,4) $, $ (2,0) $ and $ (3,-2) $. Find
	\begin{enumerate}
		\item the natural cubic spline.
		\item the clamped cubic spline with conditions $ S' (0)=2 $ and $ S' (3)=2 $.
		\item the extrapolated cubic spline.
		\item the parabolically terminated cubic spline.
		\item the curvature adjusted cubic spline with the second derivative boundary conditions $ S^{''}(0)=-1.5 $ and $ S^{''} (3)=3 $.
	\end{enumerate}
\end{enumerate}
\end{document}
