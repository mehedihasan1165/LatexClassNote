\documentclass[12pt]{article}
\usepackage{./style}
\graphicspath{ {./img/} }
\begin{document}
\section{Semigroups and Group}
\subsection{Binary Operations. Semigroups}
\begin{defn}
    A \emph{binary operation} on a nonempty set $ S $ is any mapping of $ S\times S $ into $ S $.
\end{defn}

So, with every ordered pair $ (a,b) $ of elements of $ S $ a binary operation on $ S $ associates an element of $ S $, uniquely determined by $ a $ and $ b $.This element is denoted by a symbol such as $ a + b $, $ a.b $, $ ab $, $ a \,O\, b $, etc. The requirement that the image of every element of $ S \times S $ under the given mapping must belong to $ S $, is referred to as the \emph{closure property}. As a rule, we use the symbol $ ab $ for the image of $ (a, b) $ and any given binary operation; in doing this we follow the multiplicative notation. Occasionally we use the notation $ a + b $ for the image of $ (a, b) $; in doing this we follow the \emph{additive notation}.
\begin{defn}
    An element $ e \in S $ is called a \emph{left identity}, or a \emph{right identity}, or a \emph{twosided identity} for a given binary operation on $ S \text{ iff } ea = a $, or $ ae = a $, or $ ea = ae = a $ holds, respectively, for every $ a \in S $.
\end{defn}
\begin{defn}\label{defn:2.3}
    \hfill
    \begin{enumerate}[label={(\roman*)}]
        \item A binary operation on $ S $ is called \emph{associative}; or \emph{commutative}, iff $ (ab)c = a(bc) $ holds for all $ a, b, c \in S $; or $ ab = ba $ holds for all $ a, b \in S $, respectively.
        \item A \emph{semigroup} is any nonempty set $ S $ equipped with an associative binary operation.
        \item A semigroup is called \emph{abelian} (after N. H. Abel, 1802-1829) iff the binary operation is commutative (in addition to being associative).
    \end{enumerate}
\end{defn}
\begin{ex}
    Addition, as well as multiplication of numbers, is a binary operation on each of the sets $ \N , \Z, \Q, \R. \C$; each of these binary operations is associative and commutative. So each of the sets $\N, \Z, \Q, \R, \C $ is an abelian semigroup under addition, as Well as abelian semigroup under multiplication. Each of these sem groups under multiplication has a (unique) two-sided identity. viz. $ 1 $ The same is true for each of the semigroups $ \Z, \Q, \R, \C $ under addition, for $ 0 $  is the identity element for addition, but not for $ \N $ because $ 0\notin \N $.
\end{ex}
\begin{rem}
    There are some modem authors who include $ 0 $ among the natural numbers. We do not agree with practice because it is historically unjustified.
\end{rem}
\begin{ex} 
    Let $ X $ be any non-empty set and $ P(X) $ be the power set of $ X $.
    Each of the mappings $ (A, B) \to  A \cup B, (A, B) \to A \cap B, (A, B) \to A\Delta B$, is an associative and commutative binary operation on $ P(X) $. So $ P(X) $ is an abelian semigroup under each of these binary operations. Each of these semigroups has a (unique) two-sided identity, viz. $ \varnothing , X, \varnothing $, respectively. Thus, two distinct binary operations on a set may have the same identity element.
\end{ex}
\begin{rem}
    If a given binary operation has a two-sided identity $ e $ then $ e $ is the only two-sided identity for that binary operation.
\end{rem}
\begin{ex}
    Let Map $ (X) $ be the set all mappings of a nonempty set $ X $ into itself. The composition of mappings is a binary operation on Map $ (X) $, which is associative but in general not commutative. By Example 1.4.6 (p. 13) composition of mappings is a binary operation on Bij $ (X) $, the subset of Map $ (X) $ consisting of all bijective mappings $ X $ onto itself. So, Map $ (X) $, as well as Bij $ (X) $, is a semigroup under the composition of mappings. Each of these semigroups has a (unique) twosided identity element, viz. $ l_X $, the identity mapping on $ X $.
\end{ex}
\begin{ex}
    Let $ M_n (D) $ be the set of all $ n \times n $ matírices with entries in $ D $, where $ n > 1 $ and $ D $ is any of the sets $ \Z, \Q, \R, \C $. Addition of matrices, as well as multiplication of matrices, is a binary operation on $ M_n (D) $, because $ D $ is closed under addition and multiplication of numbers. Each of these semigroups under addition is abelian and has an identity element, viz. the zero matrix of order $ n $. Each of the multiplicative semigroups is non-abelian, and has a (unique) twosided identity, viz. the identity matrix of order $ n $.
\end{ex}
\begin{ex}
    Show that the set S of all $ 2 \times 2 $ matrices of the form $ \begin{bmatrix}
        x & y\\
        0 & 0
    \end{bmatrix} $ with entries in $ \Z $, is a nonabelian semigroup under multiplication of matrices. Show also that $ S $ has no right identity, while every matrix of the form $ \begin{bmatrix}
        1 & b\\
        0 & 0
    \end{bmatrix} $, where $ b\in \Z $, is a left identity. The matrix product $  \begin{bmatrix}
        x & y\\
        0 &0
    \end{bmatrix}\begin{bmatrix}
        x' & y'\\
        0 &0
    \end{bmatrix}=\begin{bmatrix}
        xx' & yy'\\
        0 & 0
    \end{bmatrix} $belongs to $ S $, for all $ x, x', y, y' \in \Z $. So $ S $ is closed under multiplication of matrices. The associative law $ (AB)C = A(BC) $ holds in $ M_2(Z) $; so it holds in $ S $, because $ (AB)C $ and $ A(BC) $ belong to $ S $, wherever $ A, B, C $ belong to $ S $.\\
    So $ S $ is a semigroup under multiplication of matrices.
    S is nonabelian, because - for example - we have \\
    $ \begin{bmatrix}
        2 &3\\
        0 &0
    \end{bmatrix}\begin{bmatrix}
        3 &2\\
        0 &1
    \end{bmatrix}=\begin{bmatrix}
        6 &7\\
        0 & 0
    \end{bmatrix} $ while $ \begin{bmatrix}
        3 &2\\
        0 &1
    \end{bmatrix}\begin{bmatrix}
        2 &3\\
        0 &0
    \end{bmatrix}=\begin{bmatrix}
        6 &9\\
        0 & 0
    \end{bmatrix} $
    and $ \begin{bmatrix}
        6 & 9\\
        0 & 0
    \end{bmatrix} \neq \begin{bmatrix}
        6 & 7\\
        0 & 0
    \end{bmatrix} $ Suppose $ \begin{bmatrix}
        a &b \\
        0&0
    \end{bmatrix} $ is a right identity for $ S $.\\
    Since\\
    $ \begin{bmatrix}
        x & y \\
        0 & 0
    \end{bmatrix}\begin{bmatrix}
        a &b\\
        0 &0
    \end{bmatrix}=\begin{bmatrix}
        xa &xb\\
        0 & 0
    \end{bmatrix} $, we should have $ xa=x $ and $ xb=y $.\\
    But there is no fixed $ b \in \Z $ such that $ xb = y $ holds for all $ x, y \in Z $.\\
    Therefore $ S $ has no right identity identity.\\
    For any fixed $ b \in \Z $, the matrix $ \begin{bmatrix}
        1 & b\\
        0& 0
    \end{bmatrix} $ belongs to $ S $ and is a left identity for $ S $, because $ \begin{bmatrix}
        1 & b \\
        0 & 0
    \end{bmatrix}\begin{bmatrix}
        x &y\\
        0 &0
    \end{bmatrix}=\begin{bmatrix}
        x &x\\
        0 & 0
    \end{bmatrix}  $ holds for all $ x, y \in \Z $. So $ S $ has infinitely many left identities.
\end{ex}
\begin{rem}
    If for a given mapping on $ S\times S $, the image does not always belong to $ S $, then that mapping is not a binary operation on $ S $.
\end{rem}
For example, the set $ S $ of all $ 2 \times 2 $ matrices of the form $ \begin{bmatrix}
    x & x\\
    x & 0
\end{bmatrix} $ with entries in $ \Z $, is not closed under multiplication of matrices, because $ \begin{bmatrix}
    x &x\\
    x &0
\end{bmatrix}\begin{bmatrix}
    y&y\\
    y&0
\end{bmatrix} =\begin{bmatrix}
    2xy &xy\\
    xy &xy
\end{bmatrix}$ does not belong to $ S $, unless $ xy=0 $ (that is, unless $ x=0 $ or $ y =0 $). So $ S $ is not a semigroup under multiplication of matrices, even though multiplication of matrices in $ M_2(\Z) $, in particular of those in $ S $, is associative.\\

It is of course possible that a set $ S $ is closed under a given mapping on $ S\times S $, but that binary operation is not associative. An example is subtraction on the set $ \Z $; indeed $ a - (b-c)=(a - b) - c $ holds only for $ c= 0 $. So $ \Z $ is not a semigroup under subtraction.\\

At this stage it is desirable to formulate the definition of semigroup bypassing the notion of binary operation.
\begin{defn}
    A nonempty set $ S $ is called a \emph{semigroup} under a mapping $ (a, b) \to ab $ from $ S\times S $ into $ S $, iff the following properties (referred to \emph{semigroup axioms}) hold:
    \begin{enumerate}
        \item $ ab \in S $ for all $ a, b \in S $(closure property)
        \item $ (ab)c = a(bc) $ for all $ a, b, c\in S $ (associative law)
    \end{enumerate}
\end{defn}

In Definition \ref{defn:2.3} (ii) the closure property was not expressly mentioned, because the notion of binary operation embodies the closure property. Even then it is desirable to list the closure property explicitly, because in a specific example this property has to be checked first.
\begin{defn}
    Suppose $ e $ is a \emph{left}, or \emph{right}, or a \emph{two-sided identity} of a given semigroup $ S $. Given $ a \in S $, an element $ a' \in S $ is called a \emph{left}, or \emph{right}, or a \emph{two-sided inverse} of $ a $ iff $ a'a = e $, or $ aa'=e $ or $ a'a=aa'=e $, holds, respectively. $ a $ is then called \emph{left invertible}, or
\end{defn} 
\end{document}