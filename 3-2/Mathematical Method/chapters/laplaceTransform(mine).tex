\documentclass[../main-sheet.tex]{subfiles}
\usepackage{../style}
\backgroundsetup{contents={}}
\usepackage{enumitem}
% \newcommand{\lap}[1]{\mathcal{L}\left\{ #1\right\}}
\graphicspath{ {../img/} }
\begin{document}
\chapter{The Laplace Transform}
\section{Definition of The Laplace Transform}
Let $ F(t) $ be a function of $ t $ specified for $ t>0 $. The \emph{Laplace transform} of $ F(t) $, denoted by $ \lap{F(t)} $, is defined by
\begin{equation}
    \lap{F(t)}=f(s)=\int_0^\infty e^{-st}F(t)\D t \label{eq:1}
\end{equation}
where we assume at present that the parameter $ s $ is real. Later it will be found useful to consider $ s $ complex.\\

The Laplace transform of $ F(t) $ is said to \emph{exist} if the integer \eqref{eq:1} \emph{converges} for some values of $ s $; otherwise it does not exist.
\section{Laplace Transforms of Some Elementary Functions}
% \begin{table}[H]
%     \centering
%     \begingroup
%     \setlength{\tabcolsep}{10pt}
%     \renewcommand{\arraystretch}{1.9}
%     \begin{tabular}{|c|c|}
%         \hline
%         $ \mathbf{F(t)} $ & $ \mathbf{\lap{F(t)}=f(s)} $\\\hline
%         $ 1 $ & $\displaystyle \frac{1}{s}\qquad s>0 $\\\hline
%         $ t $ & $\displaystyle \frac{1}{s^2}\qquad s>0 $\\\hline
%         $ t^n $ & $\displaystyle \frac{n!}{s^{n+1}}\qquad s>0 $\\
%         $ n=0,1,2,\dots $ & Note. Factorial $n=n!=1\cdot2\dots n $ Also, by definition $ 0!=1 $\\\hline
%         $ e^{at} $ & $\displaystyle \frac{1}{s-a}\qquad s>a $\\\hline
%         $ \sin at $ & $\displaystyle \frac{a}{s^2+a^2}\qquad s>0 $\\\hline
%         $ \cos at $ & $\displaystyle \frac{s}{s^2+a^2}\qquad s>0 $\\\hline
%         $ \sinh at $ & $\displaystyle \frac{a}{s^2-a^2}\qquad s>\abs{a} $\\\hline
%         $ \cosh at $ & $\displaystyle \frac{s}{s^2-a^2}\qquad s>\abs{a} $\\\hline
%     \end{tabular}
%     \endgroup
% \end{table}
\begin{table}[H]
    \centering
    \newcolumntype{A}{>{$\displaystyle \rule{0pt}{1em}\rule[-1.25em]{0pt}{1em}}c<{$}}
    \begin{tabular}{AA}
        \toprule
        \mathbf{F(t)} & \mathbf{\lap{F(t)}=f(s)}          \\\midrule
        1             & \frac{1}{s}\qquad s>0             \\%\hline
        t             & \frac{1}{s^2}\qquad s>0           \\%\hline
        t^n\quad n=0,1,2,\dots             & \frac{n!}{s^{n+1}}\qquad s>0           \\%\hline
        e^{at}        & \frac{1}{s-a}\qquad s>a           \\%\hline
        \sin at       & \frac{a}{s^2+a^2}\qquad s>0       \\%\hline
        \cos at       & \frac{s}{s^2+a^2}\qquad s>0       \\%\hline
        \sinh at      & \frac{a}{s^2-a^2}\qquad s>\abs{a} \\%\hline
        \cosh at      & \frac{s}{s^2-a^2}\qquad s>\abs{a} \\\bottomrule
    \end{tabular}
\end{table}
% \newpage
\section{Some Important Properties of Laplace Transforms}
\begin{enumerate}
    \item First translation or shifting property.
        \begin{thm}
            If $ \lap{F(t)}=f(s) $ then $ \lap{e^{at}F(t)}=f(s-a) $.
        \end{thm}
        \begin{ex}
            Since $ \lap{\cos 2t}=\frac{s}{s^2+4} $, we have
            \[\lap{e^{-t}\cos 2t}=\frac{s+1}{(s+1)^2+4}=\frac{s+1}{s^2+2s+5}\]
        \end{ex}
    \item Second translation or shifting property.
        \begin{thm}
            If $ \lap{F(t)}=f(s) $ and $ G(t)=\begin{cases}
                F(t-a) & t>a\\
                0 & t<a
            \end{cases} $ then $ \lap{G(t)}=e^{-as}f(s) $.
        \end{thm}
    \begin{ex}
        Since $ \lap{t^3}=\frac{3!}{s^4}=\frac{6}{s^4} $, the Laplace transform of the function $ G(t)=\begin{cases}
            (t-2)^3&t>2\\
            0&t<2
        \end{cases} $ is $ \frac{6e^{-2s}}{s^4} $.
    \end{ex}
    \item Change of scale property.
        \begin{thm}
            If $ \lap{F(t)}=f(s) $ then $ \lap{F(at)}=\frac{1}{a}f\left( \frac{s}{a} \right) $
        \end{thm}
        \begin{ex}
            Since $ \lap{\sin t}=\frac{1}{s^2+1} $, we have $ \lap{\sin 3t}=\frac{1}{3}\frac{1}{(s/3)^2+1}=\frac{3}{s^2+9} $.
        \end{ex}
    \item Laplace transform of derivatives.
        \begin{thm}\label{thm:lapder}
            If $ \lap{F(t)}=f(s) $, then $ \lap{F' (t)}=s\,f(s)-F(0) $.\\
            If $ F(t) $ is continuous for $ 0\leqq t \leq N $ and of exponential order for $ t>N $ while $ F'(t) $ is sectionally continuous for $ 0\leqq t\leqq N $.
        \end{thm}
        \begin{ex}
            If $ F(t)=\cos 3t $, then $ \lap{F'(t)}=\frac{s}{s^2+9} $ and we have 
            \[
                \lap{F'(t)}=\lap{-3\sin 3t }=s\left( \frac{s}{s^2+9} \right)-1=\frac{-9}{s^2+9}
            \]
        \end{ex}
        This method is useful in finding Laplace transforms without integration.
        \begin{thm}\label{thm:lapder2}
            If $ \lap{F(t)}=f(s) $, then $ \lap{F''(t)}=s^2\,f(s)-s\,F(0) $.\\
            If $ F(t) $ and $ F'(t) $ is continuous for $ 0\leq t \leq N $ and of exponential order for $ t>N $ while $ F''(t) $ is sectionally continuous for $ 0\leq t\leqq N $.
        \end{thm}
    \item Laplace transform of integrals.
        \begin{thm}
            If $ \lap{F(t)}=f(s) $ then 
            \[ \lap{\int_0^t F(u)\D u}=\frac{f(s)}{s}\]
        \end{thm}
        \begin{ex}
            Since $ \lap{\sin 2t}=\frac{2}{s^2+4} $, we have 
            \[ \lap{\int_0^t \sin 2u \D u}=\frac{2}{s(s^2+4)}\]
            as can be verified directly.
        \end{ex}
        % \newpage
    \item Multiplication by $ t^n $.
        \begin{thm}
            \label{thm:lapmul}
            If $ \lap{F(t)}=f(s) $ then 
            \[ \lap{t^n F(t)}=(-1)^n\frac{\D^n}{\D s^n}f(s)=(-1)^n f^{(n)}(s)\]
        \end{thm}
        \begin{ex}
            Since $ \lap{e^{2t}}=\frac{1}{s-2} $, we have
            \begin{align*}
                \lap{te^{2t}}&=-\frac{\D}{\D s}\left( \frac{1}{s-2} \right)=\frac{1}{(s-2)^2}\\
                \lap{t^2e^{2t}}&=\frac{\D^2}{\D s^2}\left( \frac{1}{s-2} \right)=\frac{2}{(s-2)^2}\\
            \end{align*}
        \end{ex}
\end{enumerate}
\section{Solved Problems}
\subsection{Laplace Transforms of Some Elementary Functions}
\begin{prob}
    Prove that 
    \begin{enumerate}[label={(\alph*)}]
        \item $ \lap{1}=\frac{1}{s},\,s>0 $
        \item $ \lap{t}=\frac{1}{s^2},\,s>0 $
        \item $ \lap{e^{at}}=\frac{1}{s-1},\,s>a $
    \end{enumerate}
\end{prob}
\begin{soln}
    \begin{enumerate}[label={(\alph*)}]
        \item \begin{align*}
            \lap{1}&=\int_0^\infty e^{-st}(1)\D t\\
            &=\lim\limits_{P\to \infty}\,\int_0^P e^{-st}\D t\\
            &=\lim\limits_{P\to \infty}\left.\frac{e^{-st}}{-s}\,\right\vert_0^P\\
            &=\lim\limits_{P\to \infty}\frac{1-e^{-sP}}{s}\\
            &=\frac{1}{s}\quad \text{ if } s>0
        \end{align*}
        \item \begin{align*}
            \lap{t}&=\int_0^\infty e^{-st}(t)\D t\\
            &=\lim\limits_{P\to \infty}\,\int_0^P te^{-st}(t)\D t\\
            &=\lim\limits_{P\to \infty}\,\left.(t)\left( \frac{e^{-st}}{-s} \right)-(1)\left( \frac{e^{-st}}{s^2} \right)\right\vert_0^P\\
            &=\lim\limits_{P\to \infty}\,\left(\frac{1}{s^2}-\frac{e^{-st}}{s^2}-\frac{Pe^{-sP}}{s} \right)\\
            &=\frac{1}{s^2}\quad \text{ if } s>0
        \end{align*}
        \item \begin{align*}
            \lap{e^{at}}&=\int_0^\infty e^{-st}\left( e^{at} \right)\D t\\
            &=\lim\limits_{P\to \infty}\,\int_0^P e^{-(s-a)t}\D t\\
            &=\lim\limits_{P\to \infty}\left.\frac{e^{-(s-a)t}}{-(s-a)}\,\right\vert_0^P\\
            &=\lim\limits_{P\to \infty}\frac{1-e^{-(s-a)P}}{s-a}\\
            &=\frac{1}{s-a}\quad \text{ if } s>a
        \end{align*}
    \end{enumerate}
\end{soln}
\begin{prob}
    Prove that
    \begin{enumerate}[label={(\alph*)}]
        \item $ \lap{\sin at}=\frac{a}{s^2+a^2},\,s>0 $
        \item $ \lap{\cos at}=\frac{s}{s^2+a^2},\,s>0 $
    \end{enumerate}
\end{prob}
\newpage
\begin{soln}
    \hfill
    \begin{enumerate}[label={(\alph*)}]
        \item \begin{align*}
            \lap{\sin at}&=\int_0^\infty e^{-st}\sin at \D t\\
            &=\lim\limits_{P\to \infty}\,\int_0^P e^{-st}\sin at \D t\\
            &=\lim\limits_{P\to \infty}\,\left.\frac{e^{-st}(-s\sin at -a \cos at)}{s^2+a^2}\right\vert_0^P\\
            &=\lim\limits_{P\to \infty}\,\left\{ \frac{a}{s^2+a^2}-\frac{e^{-sP}(a\sin aP+a\cos aP)}{s^2+a^2}\right\}\\
            &=\frac{a}{s^2+a^2}\quad \text{ if } s>0
        \end{align*}
        \item \begin{align*}
            \lap{\cos at}&=\int_0^\infty e^{-st}\cos at \D t\\
            &=\lim\limits_{P\to \infty}\,\int_0^P e^{-st}\cos at \D t\\
            &=\lim\limits_{P\to \infty}\,\left.\frac{e^{-st}(-s\cos at +a \sin at)}{s^2+a^2}\right\vert_0^P\\
            &=\lim\limits_{P\to \infty}\,\left\{ \frac{s}{s^2+a^2}-\frac{e^{-sP}(a\cos aP-a\sin aP)}{s^2+a^2}\right\}\\
            &=\frac{s}{s^2+a^2}\quad \text{ if } s>0
        \end{align*}
    \end{enumerate}
    We have used here the results
    \[\int e^{\alpha t}\sin \beta t \D t=\frac{e^{at}(\alpha \sin \beta t-\beta \cos \beta t)}{\alpha^2+\beta^2}\]
    \[\int e^{\alpha t}\cos \beta t \D t=\frac{e^{at}(\alpha \cos \beta t+\beta \sin \beta t)}{\alpha^2+\beta^2}\]
\end{soln}
\subsection{Translation and Change of Scale Properties}
\begin{prob}
    Prove the first translation or shifting property: If $ \lap{F(t)}=f(s) $, then $ \lap{e^{at}F(t)}=f(s-a) $.
\end{prob}
\begin{soln}
    We have, $ \begin{aligned}
        & \\
        \lap{F(t)}&= \int_0^\infty e^{-st}F(t)\D t\\
        &=f(s)
    \end{aligned} $\\
    Then 
    \begin{align*}
        \lap{e^{at}F(t)}&=\int_0^\infty e^{-st}\left\{ e^{at}F(t)\right\}\D t\\
        &=\int_0^\infty e^{-(s-a)t}F(t)\D t\\
        &=f(s-a)
    \end{align*}
\end{soln}
\begin{prob}
    Find
    \begin{enumerate}[label={(\alph*)}]
        \item $ \lap{t^2 e^{3t}}$
        \item $ \lap{e^{-2t} \sin 4t}$
        \item $ \lap{e^{4t} \cosh 5t}$
        \item $ \lap{e^{-2t} (3\cos 6t - 5\sin 6t)}$
    \end{enumerate}
\end{prob}
\newpage
\begin{soln}
    \hfill
    \begin{enumerate}[label={(\alph*)}]
        \item $ \lap{t^2 }=\frac{2!}{s^3}=\frac{2}{s^3}$. Then $ \lap{t^2 e^{3t}}=\frac{2}{(s-3)^3} $.
        \item $ \lap{\sin 4t}=\frac{4}{s^2+16} $. Then $ \lap{e^{-2t} \sin 4t}=\frac{4}{(s+2)^2+16}=\frac{4}{s^2+4s+20}$.
        \item $ \lap{\cosh 5t}=\frac{s}{s^2-25} $. Then $ \lap{e^{4t} \cosh 5t}=\frac{s-4}{(s-4)^2-25}=\frac{s-4}{s^2-8s-9}$.\\
        
        Another method.
        \begin{align*}
            \lap{e^{4t} \cosh 5t}&=\lap{e^{4t}\left( \frac{e^{5t}+e^{-5t}}{2} \right)}\\
            &=\frac{1}{2}\lap{e^{9t}+e^{-t}}\\
            &=\frac{1}{2}\left\{ \frac{1}{s-9}+\frac{1}{s+1}\right\}\\
            &=\frac{s-4}{s^2-8s-9}
        \end{align*}
        \item \begin{align*}
            \lap{3\cos 6t -5\sin 6t}&=3\lap{\cos 6t}-5\lap{\sin 6t}\\
            &=3\left( \frac{s}{s^2+36} \right)-5\left( \frac{6}{s^2+36} \right)\\
            &=\frac{3s-30}{s^2+36}
        \end{align*}
        Then 
        \begin{align*}
            \lap{e^{-2t} (3\cos 6t - 5\sin 6t)}&=\frac{3(s+2)-30}{(s+2)^2+36}\\
            &=\frac{3s-24}{s^2+4s+40}
        \end{align*}
    \end{enumerate}
\end{soln}
\begin{prob}
    Prove the second translation or shifting property:\\
    If $ \lap{F(t)}=f(s) $ and $ G(t)=\begin{cases}
        F(t-a) & t>a\\
        0 & t<a
    \end{cases} $, then $ \lap{G(t)}=e^{-as}f(s) $.
\end{prob}
\begin{soln}
    \begin{align*}
        \lap{G(t)}&=\int_0^\infty e^{-st}G(t)\D t\\
        &=\int_0^a e^{-st}G(t)\D t+\int_a^\infty e^{-st}G(t)\D t\\
        &=\int_0^a e^{-st}(0)\D t+\int_a^\infty e^{-st}F(t-a)\D t\\
        &=\int_a^\infty e^{-st}F(t-a)\D t\\
        &=\int_a^\infty e^{-s(u+a)}F(u)\D u\\
        &=e^{-as}\int_a^\infty e^{-su}F(u)\D u\\
        &=e^{-as}f(s)
    \end{align*}
    Where we have used the substitution $ t=u+a $.
\end{soln}
\begin{prob}
    Find $ \lap{F(t)}  $ if $ F(t)=\begin{cases}
        \cos\left( t-\frac{2\pi}{3} \right) & t>\frac{2\pi}{3}\\
        0 & t<\frac{2\pi}{3}
    \end{cases} $.
\end{prob}
\newpage
\begin{soln}\hfill
    \begin{enumerate}[label={Method \arabic*.}]
        \item \begin{align*}
            \lap{F(t)}&=\int_0^{\frac{2\pi}{3}} e^{-st}(0)\D t+\int_{\frac{2\pi}{3}}^\infty e^{-st} \cos \left( t-\frac{2\pi}{3} \right)\D t\\
            &=\int_0^\infty e^{-s\left( u+\frac{2\pi}{3} \right)} \cos u \D u\\
            &=e^{-\frac{2\pi}{3}}\int_0^\infty e^{-su} \cos u \D u\\
            &=\frac{se^{-\frac{2\pi}{3}}}{s^2+1}
        \end{align*}
        \item Since $ \lap{\cos t}=\frac{s}{s^2+1} $, it follows from second translation property, with $ a=\frac{2\pi}{3} $, that 
        \[
            \lap{F(t)}=\frac{se^{-\frac{2\pi}{3}}}{s^2+1}
        \]
    \end{enumerate}
\end{soln}
\subsection{Laplace Transform of Derivatives}
% 13*
\begin{prob}
    Prove Theorem \ref{thm:lapder}: If $ \lap{F(t)}=f(s) $ then $ \lap{F'(t)}=s\,f(s)-F(0) $.
\end{prob}
\begin{soln}\label{soln:13}
    Using integration by parts, we have
    \begin{align*}
        \lap{F'(t)}&=\int_0^\infty e^{-st}F'(t)\D t-\lim\limits_{P\to \infty}\int_0^P e^{-st}F'(t)\D t\\
        &=\lim\limits_{P\to \infty}\left\{ \left.e^{-st}F(t)\right\vert_0^P +s\int_0^P e^{-st}F(t)\D t \right\}\\
        &=\lim\limits_{P\to \infty}\left\{ e^{-sP}F(P)-F(0) +s\int_0^P e^{-st}F(t)\D t \right\}\\
        &=s\int_0^\infty e^{-st}F(t)\D t-F(0)\\
        &=s\,f(s)-F(0)
    \end{align*}
    Using the fact that $ F(t) $ is of exponential order $ \gamma $ as $ t \to \infty $, so that $ \lim\limits_{P\to\infty} e^{-sP}F(P)=0$ for $ s>\gamma $.
\end{soln}
% 14*
\begin{prob}
    Prove Theorem \ref{thm:lapder2}: If $ \lap{F(t)}=f(s) $ then $ \lap{F''(t)}=s^2\,f(s)-s\,F(0)-F'(0) $.
\end{prob}
\begin{soln}
    By Problem \ref{soln:13},
    \[
        \lap{G'(t)}=s\lap{G(t)}-G(0)=s\,g(s)-G(0)
    \]
    Let $ G(t)=F'(t) $. Then 
    \begin{align*}
        \lap{F''(t)}&=s\lap{F'(t)}-F'(0)\\
        &=s\left[ s\lap{F(t)}-F(0) \right]-F'(0)\\
        &=s^2 \lap{F(t)}-s\,F(0) -F'(0)\\
        &=s^2 f(s)-s\,F(0) -F'(0)
    \end{align*}
    The generalization to higher derivatives can be proved by using mathematical induction.
\end{soln}
\subsection{Multiplication By Powers of $ t $}
% 19
\begin{prob}
    Prove Theorem \ref{thm:lapmul}: If $ \lap{F(t)}=f(s) $, then 
    \[ \lap{t^n F(t)}=(-1)^n\frac{\D^n}{\D s^n}f(s)=(-1)^n f^{(n)}(s)\qquad \text{where }n=1,2,3,\dots\]
\end{prob}
\begin{soln}
    We have, 
    \[
        f(s)=\int_0^\infty e^{-st}F(t)\D t
    \]
    Then by Leibniz's rule for differentiating under the integral sign,
    \begin{align*}
        \frac{\D f}{\D s}=f'(s)&=\frac{\D}{\D s}\int_0^\infty e^{-st}F(t)\D t=\int_0^{\infty} \frac{\partial}{\partial s}e^{-st}F(t)\D t\\
        &=\int_0^{\infty} -te^{-st}F(t)\D t\\
        &=-\int_0^{\infty} e^{-st}\left\{tF(t)\right\}\D t\\
        &=-\lap{tF(t)}
    \end{align*}
    Thus,
    \begin{equation}
        \lap{tF(t)}=-\frac{\D f}{\D s}=-f'(s)
        \label{eq:mulprob1}
    \end{equation}
    which proves the theorem for $ n=1 $.\\

    To establish the theorem in general, we use mathematical induction. Assume the theorem is true for $ n=k $, i.e., assume
    \begin{equation}
        \int_0^\infty e^{-st}\left\{ t^kF(t) \right\}\D t=(-1)^k f^{(k)}(s)
        \label{eq:mulprob2}
    \end{equation}
    Then
    \[
        \frac{\D}{\D s}\int_0^\infty e^{-st}\left\{ t^kF(t) \right\}\D t=(-1)^k f^{(k+1)}(s)
    \]
    or by Leibniz's rule,
    \[
        -\int_0^\infty e^{-st}\left\{ t^{k+1}F(t) \right\}\D t=(-1)^k f^{(k+1)}(s)
    \]
    i.e.,
    \begin{equation}
        \int_0^\infty e^{-st}\left\{ t^{k+1}F(t) \right\}\D t=(-1)^{k+1} f^{(k+1)}(s) \label{eq:mulprob3}
    \end{equation}
    It follows that if \eqref{eq:mulprob2} is true, i.e., if the theorem holds for $ n=k $, then \eqref{eq:mulprob3} is true, i.e., the theorem holds for $ n=k+1 $. But by \eqref{eq:mulprob1} the theorem is true for $ n=1 $. Hence, it is true for $ n=1+1=2 $ and $ n=2+1=3 $, etc., and thus for all positive integer values of $ n $.
\end{soln}
% 20
\begin{prob}
    Find \begin{enumerate}[label=(\alph*)]
        \item $ \lap{t\sin at} $
        \item $ \lap{t^2 \cos at} $
    \end{enumerate}
\end{prob}
\begin{soln}
    \hfill
    \begin{enumerate}[label=(\alph*)]
        \item Since $ \lap{\sin at}=\frac{a}{s^2+a^2} $, we have by multiplication by the powers of $ t $
        \[
            \lap{t\sin at}=-\frac{\D}{\D s}\left( \frac{a}{s^2+a^2} \right)=\frac{2as}{(s^2+a^2)^2}
        \]
        \emph{Another method}\\
        Since $ \displaystyle \lap{\cos at}=\int_0^\infty e^{-st}\cos at \D t =\frac{s}{s^2+a^2}$\\
        We have by differentiating with respect to the parameter $ a $ [using Leibniz's rule],
        \begin{align*}
            \frac{\D}{\D a}\int_0^\infty e^{-st}\cos at\D t&=\int_0^\infty e^{-st}\left\{-t\sin at\right\}\D t=-\lap{t\sin at}\\
            &=-\frac{\D}{\D a}\left( \frac{s}{s^2+a^2} \right)=-\frac{2as}{(s^2+a^2)^2}
        \end{align*}
        from which 
        \[
            \lap{t\sin at}=\frac{2as}{(s^2+a^2)^2}
        \]
        Note that the result is equivalent to $ \frac{\D}{\D a}\lap{\cos at}=\lap{\frac{\D}{\D a}\cos at} $.
        \item Since $ \lap{\cos at}=\frac{s}{s^2+a^2} $, we have by multiplication by the powers of $ t $
        \[
            \lap{t^2\cos at}=-\frac{\D^2}{\D s^2}\left( \frac{s}{s^2+a^2} \right)=\frac{2s^3-6a^2s}{(s^2+a^2)^3}
        \]
        We can also use the second method of part (a) by writing
        \[
            \lap{t^2\cos at}=\lap{-\frac{\D^2}{\D a^2}\cos at}=-\frac{\D^2}{\D a^2}\lap{\cos at}
        \]
        which gives the same result.
    \end{enumerate}
\end{soln}
% 45
\subsection{Evaluation of Integral}
\begin{prob}
    Evaluate \begin{enumerate}[label=(\alph*)]
        \item $ \displaystyle \int_0^\infty t e^{-2t}\cos t \D t$,
        \item $ \int_0^\infty \frac{e^{-t}-e^{-3t}}{t}\D t $
    \end{enumerate}
\end{prob}
\begin{soln}
    \begin{enumerate}[label=(\alph*)]
        \item By multiplication by the powers of $ t $,
        \begin{align*}
            \lap{t\cos t}&=\int_0^\infty t e^{-st}\cos t \D t\\
            &=-\frac{\D}{\D s}\lap{\cos t}\\
            &=-\frac{\D}{\D s}\left( \frac{s}{s^2+1} \right)\\
            &=\frac{s^2-1}{(s^2+1)^2}
        \end{align*}
        Then letting $ s=2 $, we find
        \[
            \int_0^\infty te^{-2t}\cos t\D t=\frac{3}{25}
        \]
        \item If $ F(t)= e^{-t}-e^{-3t}$, then 
        \[
            f(s)=\lap{F(t)}=\frac{1}{s+1}-\frac{1}{s+3}
        \]
        Thus by division by powers of $ t $,
        \begin{align*}
            & \lap{\frac{e^{-t}-e^{-3t}}{t}}=\int_0^\infty \left\{ \frac{1}{u+1}-\frac{1}{u+3} \right\}\D u\\
            \Rightarrow\,& \int_0^\infty e^{-st}\left(\frac{e^{-t}-e^{-3t}}{t}\right)\D t=\ln\left( \frac{s+3}{s+1} \right)
        \end{align*}
        Taking the limit as $ s\to 0^+ $, we find
        \[
            \int_0^\infty \frac{e^{-t}-e^{-3t}}{t}\D t=\ln 3
        \]
    \end{enumerate}
\end{soln}
\end{document}