\documentclass[12pt,class=book,crop=false]{standalone}
\usepackage{../style}
\graphicspath{ {../img/} }
\newcommand{\lap}[1]{\mathcal{L}\left\{ #1\right\}}
\newcommand{\ilap}[1]{\mathcal{L}^{-1}\left\{ #1\right\}}
\begin{document}
\chapter{Inverse Laplace Transform}
\section{Definition of Inverse Laplace Transform}
If the Laplace transform of a function $ F(t) $ is $ f(s) $, i.e., if $ \lap{F(t)}=f(s) $, then $ F(t) $ is called an inverse Laplace Transform of $ f(s) $, and we write symbolically $ F(t)=\ilap{f(s)} $ where $ \mathcal{L}^{-1} $ is called the inverse Laplace transformation operator.
\section{Some Inverse Laplace Transforms}
Here is a table of some inverse Laplace transforms
\begin{table}[H]
    \centering
    \begingroup
    \setlength{\tabcolsep}{10pt}
    \renewcommand{\arraystretch}{1.9}
    \begin{tabular}{|c|c|}
        \hline
        $ f(s) $                                         & $ \ilap{f(s)}=F(t) $               \\\hline
        $\displaystyle \frac{1}{s} $                     & $\displaystyle 1$                  \\\hline
        $\displaystyle \frac{1}{s^2} $                   & $\displaystyle t$                  \\\hline
        $\displaystyle \frac{1}{s^{n+1}},n=0,1,2,\dots $ & $\displaystyle \frac{t^n}{n!}$     \\\hline
        $\displaystyle \frac{1}{s-a} $                   & $\displaystyle e^{at}$             \\\hline
        $\displaystyle \frac{1}{s^2+a^2} $               & $\displaystyle \frac{\sin at}{a}$  \\\hline
        $\displaystyle \frac{s}{s^2+a^2} $               & $\displaystyle \cos at$            \\\hline
        $\displaystyle \frac{1}{s^2-a^2} $               & $\displaystyle \frac{\sinh at}{a}$ \\\hline
        $\displaystyle \frac{s}{s^2-a^2} $               & $\displaystyle \cosh at$           \\\hline
    \end{tabular}
    \endgroup
\end{table}
\section{Properties}
TODO :: Check class
\section{The Convolution Theorem}
The convolution theorem can be used to solved integral and integral-differential equations.
\begin{thm}
    If $ \ilap{f(s)}=F(t) $ and $ \ilap{g(s)}=G(t) $ then
    \[
        \ilap{f(s)g(s)}=\int_0^t F(u) G(t-u) \D u=F*G.
    \]
\end{thm}

We call $ F*G $ the convolution or faulting of $ F $ and $ G $ and the theorem is called the convolution theorem. [Here, $ * $ (asterisk) denotes convolution in this context, not standard multiplication.]\\

The formulation is especially useful for implementing a numerical convolution on a computer. The standard convolution algorithm has quadratic computational complexity. With the help of convolution theorem and the fast Fourier transform the complexity of the convolution can be reduced from $ O(n^2) $ to $ O(n\log n) $.
\begin{prob}
    Prove the convolution theorem:\\
    If $ \ilap{f(s)}=F(t) $ and $ \ilap{g(s)}=G(t) $ then
    \[
        \ilap{f(s)g(s)}=\int_0^t F(u) G(t-u) \D u=F*G.
    \]
\end{prob}
\begin{proof}
    The required results follows if we can prove that
    \begin{equation}
        \lap{\int_0^t F(u) G(t-u)\D u}=f(s)g(s)\label{eq:convothm1}
    \end{equation}
    Where, $ \begin{aligned}[t]
                 &                             \\
            f(s) & =\lap{F(t)}\qquad\text{and} \\
            g(s) & =\lap{G(t)}
        \end{aligned} $\\
    To show this we note the left side of \eqref{eq:convothm1} is
    \begin{align}
            & \int_{t=0}^\infty e^{-st}\left\{ \int_{u=0}^t F(u)G(t-u)\D u \right\}\D t\notag \\
            & =\, \int_{t=0}^\infty \int_{u=0}^\infty e^{-st} F(u)G(t-u)\D u \D t\notag       \\
            & =\, \lim_{M\to \infty} s_M\notag                                                \\
        \intertext{where,}
        s_M & =\, \int_{t=0}^M \int_{u=0}^t e^{-st} F(u)G(t-u)\D u \D t\label{eq:convothm2}
    \end{align}
    The region in the $ tu $ plane over which the integration \eqref{eq:convothm2} is performed is shown shaded in figure \ref{fig:convothm1}.
    \begin{figure}[H]
        \begin{minipage}[b]{0.5\textwidth}
            \centering
            \import{../tikz/}{convo-thm-1.tikz}
            \caption{}
            \label{fig:convothm1}
        \end{minipage}
        \begin{minipage}[b]{0.5\textwidth}
            \centering
            \import{../tikz/}{convo-thm-2.tikz}
            \caption{}
            \label{fig:convothm2}
        \end{minipage}
    \end{figure}
    Let, $ t-u=v $ or $ t=u+v $, the shaded region $ R_{tu} $ of the $ tu $ plane is transformed into the shaded region $ R_{uv} $ of the $ uv  $ plane shown in figure \ref{fig:convothm2}. Then by a theorem on transformation on multiple integral,\\ We have
    \begin{align}
        s_M & =\iint\limits_{R_{tu}} e^{-st}F(u)G(t-u)\D u\D t\notag                                                            \\
            & =\iint\limits_{R_{uv}} e^{-s(u+v)}F(u)G(v)\abs{\frac{\partial (u,t)}{\partial (u,v)}}\D u\D v\label{eq:convothm3}
    \end{align}
    where the Jacobian of the transformation is
    \[
        J=\frac{\partial (u,t)}{\partial (u,v)}=\begin{vmatrix}
            \frac{\partial u}{\partial u} & \frac{\partial u}{\partial v} \\
            \frac{\partial t}{\partial u} & \frac{\partial t}{\partial v}
        \end{vmatrix}=\begin{vmatrix}
            1 & 0 \\
            1 & 1
        \end{vmatrix}=1
    \]
    Thus, the right side of \eqref{eq:convothm3} is,
    \begin{equation}
        s_M=\int_{v=0}^M\int_{u=0}^M e^{-s(u+v)}F(u)G(v)\D u\D v \label{eq:convothm4}
    \end{equation}
    Let us define a function
    \begin{equation}
        \label{eq:convothm5}
        k(u,v)=\begin{cases}
            e^{-s(u+v)}F(u)G(v) & \text{ if } u+v\leq M \\
            0                   & \text{ if } u+v> M
        \end{cases}
    \end{equation}
    This function is defined over the square of figure \ref{fig:convothm3} but as indicated in \eqref{eq:convothm5}, is zero over
    \begin{figure}[H]
        \centering
        \import{../tikz/}{convo-thm-3.tikz}
        \caption{}
        \label{fig:convothm3}
    \end{figure}
    the unshaded portion of the square. In terms of this new function we can write \eqref{eq:convothm4} as,
    \begin{align*}
        s_M                    & =\int_{v=0}^M\int_{u=0}^M k(u,v)\D u\D v                                                           \\
        \intertext{Then,}
        \lim_{M\to \infty} s_M & =\int_{0}^\infty \int_{0}^\infty k(u,v)\D u\D v                                                    \\
                               & =\int_{0}^\infty \int_{0}^\infty e^{-s(u+v)}F(u)G(v) \D u\D v                                      \\
                               & =\left\{ \int_{0}^\infty e^{-su}F(u)\D u \right\}\left\{ \int_{0}^\infty e^{-sv}G(v) \D v \right\} \\
                               & =f(s) g(s)
    \end{align*}
    Which establishes the theorem.

    We call $ \int_0^t F(u)G(t-u)\D u=F*G $ the convolution integral or convolution of $ F $ and $ G $.
\end{proof}
\begin{prob}
    Evaluate each of the following by the use of the convolution theorem
    \begin{enumerate}[label=(\alph*)]
        \item $ \displaystyle \ilap{\frac{s}{\left( s^2+a^2 \right)^2}} $
        \item $ \displaystyle \ilap{\frac{1}{s^2\left( s+1 \right)^2}} $
    \end{enumerate}
\end{prob}
\begin{soln}
    \begin{enumerate}[label=(\alph*)]
        \item We can write
              \[
                  \frac{s}{\left( s^2+a^2 \right)^2}=\frac{s}{s^2+a^2}\times \frac{1}{s^2+a^2}
              \]
              Now, $ \begin{aligned}[t]
                                        &                       \\
                      \quad \frac{s}{s^2+a^2} & =\cos at \qquad\text{ and } \\
                      \quad \frac{1}{s^2+a^2} & =\frac{\sin at}{a}
                  \end{aligned} $\\
              By the convolution theorem we get,
              \begin{align*}
                  \ilap{\frac{s}{\left( s^2+a^2 \right)^2}} & =\int_0^t \cos au \,\frac{\sin a(t-u)}{a}\D u                                                                                    \\
                                                            & =\frac{1}{a}\int_0^t (\cos^2 au) (\sin at \cos au -\cos at \sin au)\D u                                                          \\
                                                            & =\frac{1}{a}\sin at \int_0^t \cos^2 au\D u-\frac{1}{a}\cos at \int_0^t \sin au \cos au \D u                                     \\
                                                            & =\frac{1}{a}\sin at \int_0^t \frac{1+\cos 2au}{2}\D u-\frac{1}{a}\cos at \int_0^t \frac{\sin 2au}{2} \D u                       \\
                                                            & =\frac{1}{a}\sin at \left(\frac{t}{2}+ \frac{\sin 2at}{4a} \right)-\frac{1}{a}\cos at \left( \frac{1-\cos 2at}{4a} \right)      \\
                                                            & =\frac{1}{a}\sin at \left(\frac{t}{2}+ \frac{\sin at\cos at}{2a} \right)-\frac{1}{a}\cos at \left( \frac{\sin^2 at}{2a} \right) \\
                                                            & =\frac{2\sin at}{2a}
              \end{align*}
        \item We have, $ \begin{aligned}[t]
                                        &                 \\
                      \frac{1}{s^2}     & =t \qquad\text{ and } \\
                      \frac{1}{(s+1)^2} & =te^{-t}
                  \end{aligned} $\\
              By the convolution theorem we get,
              \begin{align*}
                  \ilap{\frac{1}{s^2\left( s+1 \right)^2}} & =\int_0^t ue^{-u}(t-u)\D u                                                                                     \\
                                                           & =\int_0^t \left(ut-u^2\right)e^{-u}\D u                                                                        \\
                                                           & =\left(ut-u^2\right)\left(-e^{-u}\right)-(t-2u)\left( e^{-u} \right)+(-2)\left( -e^{-u} \right)\Biggl\vert_0^t \\
                                                           & =te^{-t}+2e^{-t}+t-2
              \end{align*}
    \end{enumerate}
\end{soln}

\end{document}