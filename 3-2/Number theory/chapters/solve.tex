\documentclass[12pt,class=book,crop=false]{standalone}
\usepackage{../style}
\graphicspath{ {../img/} }
\newlength{\wdth}
\newcommand{\strike}[1]{\settowidth{\wdth}{#1}\rlap{\rule[.5ex]{\wdth}{.4pt}}#1}

\begin{document}
\chapter*{Various Problem and Solution}
% \section{New Section}
\begin{qn}
    Show that $ n (n+1)(2n+1) $ is divisible by $ 6 $.
\end{qn}
\begin{soln}
    One of the two consecutive integers $ n $ and $ n+1 $ is divisible by 2 and one of the other consecutive integer $ 2n,\;2n+1 $ and $ 2n+2 $ is divisible by 2.\\
    Hence, the product $ \begin{aligned}[t]
                & 2n(2n+1)(2n+2)                                  \\
            =\; & 4n(2n+1)(2n+1) \text{ is divisible by 2 and 3.}
        \end{aligned} $ \\
    Since 4 is divisible by 2 and 2 is prime to 3, so $ n (n+1)(2n+1) $ is divisible by $ 6 $.
\end{soln}
\begin{qn}
    Show that $ n^5-n $ is divisible by $ 30 $.
\end{qn}
\begin{soln}
    Let $ n $ be even.\\
    Then $ n^5 $ is even and $ n^5-n $ is also even and which is divisible by 2.\\
    If $ n $ is odd, then $ n^5-n $ is even and hence divisible by $ 2 $.\\
    Now, $ n^5-n =n(n^4-1)$\\
    But, $ n^4-1 =n^{5-1}-1$\\
    By Fermat's theorem $ \begin{aligned}[t]
                          & n^{5-1}-1 \text{ is divisible by } 5  \\
            \text{i.e., } & n^{4}-1 \text{ is divisible by } 5    \\
                          & n(n^{4}-1) \text{ is divisible by } 5 \\
                          & n^{5}-n \text{ is divisible by } 5
        \end{aligned} $\\
    Again $ n $ can be written any one of the form $ 3m,\;3m+1 $ and $ 3m+2 $.\\
    When $ n=3m $, $ 3m\left\{ (3m)^4-1 \right\} $ which is divisible by 3.\\
    When $ n=3m+1 $,\blfootnote{$\begin{aligned}
        (a+b+c)^2&=a^2+b^2+c^2+2ab+2bc+2ca\\
        (a+b)^4&=a^4+b^4+6a^2b^2+4a^3b+4ab^3
    \end{aligned}$}
    \begin{align*}
          & (3m+1)\left\{ (3m+1)^4-1 \right\}                                  \\
        = & (3m+1)\left\{ 81m^4+108m^3+54m^2+12m+1-1 \right\}                  \\
        = & 3m(3m+1)( 27m^3+36m^2+18m+4 ) \quad\text{ which is divisible by }3
    \end{align*}
    When $ n=3m+2 $,
    \begin{align*}
          & (3m+2)\left\{ (3m+2)^4-1 \right\}                                       \\
        = & (3m+2)\left\{ 81m^4+216m^3+216m^2+96m+16-1 \right\}                     \\
        = & 3(3m+2)( 27m^4+72m^3+72m^2+32m+5 ) \quad\text{ which is divisible by }3
    \end{align*}
    Hence,  $ n^5-n $ is divisible by the product of $ 2,3,5 $\\
    i.e.,  $ n^5-n $ is divisible by 30.
    
\end{soln}
\begin{qn}
    If $ n $ is an integer then prove that one of $ n $, $ n+2 $, $ n+4 $ is divisible by 3.
\end{qn}
\begin{soln}
    Given the number $ n $, $ n+2 $, $ n+4 $ when $ n  $ is an integer, then $ n $ must be any one of the form $ 3m,\;3m+2,\;3m+4 $.\\
    If $ n=3m $, the first integer is divisible by 3.\\
    If $ n=3m+2 $, then $ n+4=3m+2+4=3(m+2) $ which is divisible by 3.\\
    If $ n=3m+4 $, then $ n+2=3m+4+2=3(m+2) $ which is divisible by 3.\\
    Hence, if $ n $ is an integer then one of $ n $, $ n+2 $ and $ n+4 $ is divisible by 3.
\end{soln}
\begin{qn}
    Prove that, $ 3^{2n+1}+2^{n+2} $ is divisible by 7.
\end{qn}
\begin{soln}
    Let $ T=3^{2n+1}+2^{n+2} $\\
    For $ n=0 $, $ T=3+4=7 $ which is divisible by 7.\\
    For $ n=1 $, $ T=27+8=35 $ which is divisible by 7.\\
    For $ n=3 $, $ T=259 $ which is divisible by 7.\\
    Suppose, for $ n=3 $, $ T=3^{2r+1}+2^{r+2}=7q $ which is divisible by 7.\\
    Thus, for $ n=r+1 $,
    \begin{align*}
        T= & 3^{2(r+1)+2}+2^{(r+1)+2}                                \\
        =  & 9\cdot3^{2r+1}+2\cdot2^{r+2}                            \\
        =  & 9\left(3^{2r+1}+2^{r+2}\right)-7\cdot2^{r+2}            \\
        =  & 7\cdot 9q-7\cdot 2^{r+2}                                \\
        =  & 7\left(9q-2^{r+2}\right)\text{ which is divisible by }7
    \end{align*}
    Hence, $ 3^{2n+1}+2^{n+2} $ is divisible by 7.
\end{soln}
\begin{qn}
    Show that $ 2^n+1 $ or $ 2^n-1 $ is divisible by 3 according as $ n $ is odd or even.
\end{qn}
\begin{soln}
    We know that the product $ P=\left( 2^n+1 \right)\left( 2^n-1 \right)=2^{2n}-1 $ is divisible by 3 for all $ n $.\\
    For $ n=0 $, $ P=2\cdot0=0 $ is divisible by 3.\\
    For $ n=1 $, $ P=3\cdot1=3 $ is divisible by 3.\\
    For $ n=2 $, $ P=5\cdot(4-1)=15 $ is divisible by 3.\\
    Suppose, for $ n=r $, $ P=\left( 2^r+1 \right)\left( 2^r-1 \right) $ is divisible by 3. i.e., $ \left( 2^r+1 \right)\left( 2^r-1 \right)=3q $ where $ q $ is an integer.\\
    Then for $ n=r+1 $,
    \begin{align*}
        P= & \left( 2^{r+1}+1 \right)\left( 2^{r+1}-1 \right)    \\
        =  & 2^{2(r+1)}-1                                        \\
        =  & 4\left(2^{2r}-1\right)+3                            \\
        =  & 4\cdot3q+3                                          \\
        =  & 3\left( 4q+1 \right)\text{ which is divisible by }3
    \end{align*}
\end{soln}
\begin{qn}
    Compute $ \varphi(210) $, $ \varphi(2187) $, $ \varphi(2000) $, $ \varphi(1026) $, $ \varphi(13912) $, $ \varphi(1981) $, $ \varphi(1350) $.
\end{qn}
\begin{soln}
    \begin{align*}
        \varphi(210) & =\varphi(2\cdot 5\cdot3\cdot7)                   \\
                     & =\varphi(2)\;\varphi( 5)\;\varphi(3)\;\varphi(7) \\
                     & =1\cdot4\cdot2\cdot6                             \\
                     & =48
    \end{align*}
    \begin{align*}
        \varphi(2187) & =\varphi(3\cdot 3\cdot 3\cdot 3\cdot 3\cdot 3\cdot 3) \\
                      & =\varphi\left(3^7\right)                              \\
                      & =3^7\left( 1-\frac{1}{3} \right)                      \\
                      & =1458
    \end{align*}
    \begin{align*}
        \varphi(2000) & =\varphi(2\cdot 1000)                                \\
                      & =\varphi\left(2\cdot5\cdot200\right)                 \\
                      & =\varphi\left(2\cdot5\cdot5\cdot4\cdot2\cdot5\right) \\
                      & =\varphi\left( 2^4\cdot5^3\right)                    \\
                      & =\varphi\left( 2^4\right)\;\varphi\left(5^3\right)   \\
                      & =\left( 2^4-2^3\right)\;\left(5^3-5^2\right)         \\
                      & =8\cdot100                                           \\
                      & =800
    \end{align*}
    \begin{align*}
        \varphi(1026) & =\varphi(2\cdot 3\cdot3\cdot3\cdot19)                         \\
                      & =\varphi\left( 2\right)\;\varphi\left(3^3\right)\;\varphi(19) \\
                      & =1\left( 3^3-3^2\right)\;\left(18\right)                      \\
                      & =324
    \end{align*}
    \begin{align*}
        \varphi(13912) & =\varphi(2\cdot 2\cdot2\cdot37\cdot47)                       \\
                       & =\varphi\left( 8\right)\;\varphi\left(37\right)\;\varphi(47) \\
                       & =\left( 8-4\right)\;\left(36\right)\;(46)                    \\
                       & =6624
    \end{align*}
    \begin{align*}
        \varphi(1981) & =\varphi(7\cdot 283)                             \\
                      & =\varphi\left( 7\right)\;\varphi\left(283\right) \\
                      & =7\cdot282                                       \\
                      & =1974
    \end{align*}
\end{soln}
\begin{qn}
    Show that the sum of the integers less than $ n $ and prime to it is $ \frac{1}{2}n \varphi(n) $ if $ n\geq 2 $.
\end{qn}
\begin{soln}
    Let $ x $ be an integer less than $ n $ and prime to it, then $ n-x $ is also an integer less than $ n $ and prime to it.

    Define the integer by $ 1,p,q,r,\dots,(n-1) $ and their sum by $ S $. Then
    \[S=1+p+q+r+\dots+(n-p)+(n-q)+(n-r)+(n-1)\]
    which is the series consisting of $ \varphi(n) $ terms.\\
    Rearranging, we have
    \begin{align*}
        S=             & (n-1)+(n-r)+(n-q)+(n-p)+\dots+r+q+p+1                            \\
        \therefore 2S= & n+n+n+n+\dots \text{ upto } \varphi(n)\text{ terms} =n\varphi(n) \\
        \therefore S=  & \frac{1}{2}n\varphi(n)                                           \\
    \end{align*}
\end{soln}
\begin{qn}
    Show that the following congruence holds for all integer values of $ n $.
    \begin{enumerate}[label=(\roman*)]
        \item $ 2^{2n}-1\equiv 0 \pmod{3} $
        \item $ 2^{3n}-1\equiv 0 \pmod{7} $
        \item $ 2^{4n}-1\equiv 0 \pmod{15} $
    \end{enumerate}
\end{qn}
\begin{soln}
    We show that $ T=2^{2n}-1 $ is divisible by 3.\\
    For $ n=1 $, $ T=4-1=3 $ which is divisible by 3.\\
    For $ n=2 $, $ T=16-1=15 $ which is divisible by 3.\\
    For $ n=3 $, $ T=64-1=63 $ which is divisible by 3.\\

    Let for $ n=r $, $ t=2^{2r}-1=3q $ which is divisible by 3.\\
    Now for $ n=r+1 $,
    \begin{align*}
        T&= 2^{2(r+1)}-1\\
        &= 4\cdot2^{2r}-1\\
        &= 4\left(2^{2r}-1\right)+3\\
        &= 3\left(4q+1\right) \text{ which is divisible by }3
    \end{align*}
    Hence, $ 2^{2n}-1\equiv 0 \pmod{3} $.
\end{soln}
\begin{qn}
    Prove that if $ a $ is an integer, then $ 6\mid a(a^2+11) $.
\end{qn}
\begin{proof}
    Let $ a $ is an even integer, then for any integer values of $ n $, we can write $ a=2n $.\\
    So,
    \begin{align*}
        &2n\left\{ (2n)^2+11 \right\}\\
        =&2n\left\{ 4n^2-1+12 \right\}\\
        =&2n(2n+1)(2n-1)+24n\\
        =&(2n-1)(2n)(2n+1)+24n
    \end{align*}
    Since, $ (2n-1)(2n)(2n+1) $ is the multiple of three consecutive integers and hence is divisible by $ 3!=6 $ and 24 is evidently divisible by 6.

    Hence, $ a(a^2+11) $ is divisible by 6 for all even values of $ a $.\\

    Again, let $ a $ is odd integer, then for any integer values of $ n $, we write $ a=2n+1 $.\\
    So,
    \begin{align*}
        &(2n+1)\left\{ (2n+1)^2+11 \right\}\\
        =&(2n+1)\left\{ (2n+1)^2-1^2+12 \right\}\\
        =&(2n+1)\left\{(2n+2)(2n+1-1)\right\}+12(2n+1)\\
        =&2n(2n+1)(2n+2)+12(2n+1)
    \end{align*}
    Since, $ 2n(2n+1)(2n+2) $ is the product of three consecutive integers and hence is divisible by $ 3!=6 $, again $ 12(2n+1) $ is divisible by 6.\\

    Thus, $ a(a^2+11) $ is divisible by 6.
\end{proof}
\begin{qn}
    Prove that if $ a $ is an odd integer, then  24 divides $ a(a^2-1) $ i.e., $ 24\mid a(a^2-1) $.
\end{qn}
\begin{soln}
    Let $ a $ is an odd integer, then $ a-1  $ and $ a+1 $ are two even integers, hence one of them is divisible by 2 and the other by 4.

    Again, $ a-1\;a,\;a+1 $ are three consecutive numbers and hence one of them is divisible by 3.

    Thus, the product $ a(a^2-1) $ is divisible by $ 2\cdot3\cdot4 $ i.e., $ a(a^2-1) $ is divisible by 24.
\end{soln}
\begin{qn}
    If $ a $ and $ b $ are odd integers then show that $ 8\mid (a^2-b^2) $.
\end{qn}
\begin{soln}
    Here $ a^2-b^2=(a-b)(a+b) $\\
    Since $ a $ and $ b $ are odd integers here, so $ a-b $ and $ a+b $ are two even numbers. Hence, one of them is divisible by 2 and the other is by 5.\\
    Hence, the product $ (a-b)(a+b) $ is divisible by product of $ 2 $ and 4 i.e., by 8.
\end{soln}
\begin{qn}
    Show that $ n^4+4 $ is composite for all $ n>1 $.
\end{qn}
\begin{soln}
    Suppose, $ \begin{aligned}[t]
        f(n)&=n^4+4\\
        &=(n^2+2)^2-4n^2\\
        &=(n^2+2n+2)(n^2-2n+2)
    \end{aligned} $\\
    If $ n=1 $, $ f(1)=5 $ which is not composite.\\
    But when $ n>1 $ then $ f(n) $ is a product of two factors and hence is a composite number.
\end{soln}
\begin{qn}
    Show that $ n^4+n^2+1 $ is composite for all $ n>1 $.
\end{qn}
\begin{soln}
    Let, $ \begin{aligned}[t]
        f(n)&=n^4+n^2+1\\
        &=(n^2+1)^2-n^2\\
        &=(n^2+n+1)(n^2-n+1)
    \end{aligned} $\\
    If $ n=1 $, $ f(1)=3 $ which is not composite.\\
    Thus, when $ n>1 $ then $ f(n) $ is a product of two factors and hence is a composite number.
\end{soln}
\begin{qn}
    Show that $ n^4+n^2+1 $ is composite for all $ n>1 $.
\end{qn}
\begin{qn}
    If $ (a,7)=1 $, then prove that $ a^3+1$ or $a^3-1 $ is divisible by 7.
\end{qn}
\begin{soln}
    Since $ (a,7)=1 $ and 7 is a prime so by Fermat's theorem,
    \begin{align*}
        &a^{7-1}\equiv 1\pmod{7}\\
        \Rightarrow\;&a^6-1 \equiv 0\pmod{7}\\
        \Rightarrow\;&(a^3)^2-1 \equiv 0\pmod{7}\\
        \Rightarrow\;&(a^3+1)(a^3-1) \equiv 0\pmod{7}\\
    \end{align*}
    Hence, $ a^3+1$ or $a^3-1 $ is divisible by 7.
\end{soln}
\begin{qn}
    If $ (a,p)=1 $, $ (b,p)=1 $ then show that $ a^p\equiv b^p\pmod{p} $ implies that $ a\equiv b\pmod{p} $.
\end{qn}
\begin{soln}
    Since,
    \begin{align}
        &(a,p)=1\qquad \text{ and }\qquad (b,p)=1\notag\\
        \Rightarrow\;&a^{p-1}\equiv 1\pmod{p}\qquad \text{ and }\qquad b^{p-1}\equiv1\pmod{p}\notag\\
        \Rightarrow\;&a^{p}\equiv a\pmod{p}\qquad \text{ and }\qquad {b}^{p}\equiv b\pmod{p}\notag\\
        \therefore\;&a^{p}-b^p\equiv a-b\pmod{p}\label{eq:q13.1}
    \end{align}
    Again,
    \[
        a^p\equiv b^p\pmod{p}\quad \Rightarrow \; a^p-b^p\equiv 0 \pmod{p}
    \]
    Hence, \eqref{eq:q13.1} implies that $ a\equiv b\pmod{p} $.
\end{soln}
\begin{qn}
    If $ (a,p)=1 $, $ (b,p)=1 $ then show that $ a^p\equiv b^p\pmod{p} $ implies that $ a\equiv b\pmod{p^2} $.
\end{qn}
\begin{soln}
    Since,
    \begin{align}
        &(a,p)=1\qquad \text{ and }\qquad (b,p)=1\notag\\
        \Rightarrow\;&(a,p^2)=1\qquad \text{ and }\qquad (b,p^2)=1\notag\\
        \Rightarrow\;&a^{p^2-1}\equiv 1\pmod{p^2}\qquad \text{ and }\qquad b^{p^2-1}\equiv1\pmod{p^2}\notag\\
        \Rightarrow\;&a^{p^2}\equiv a\pmod{p^2}\qquad \text{ and }\qquad {b}^{p^2}\equiv b\pmod{p^2}\notag\\
        \Rightarrow\;&\left(a^{p^2}\right)^p\equiv a^p \pmod{p^2}\qquad \text{ and }\qquad \left({b}^{p^2}\right)^p\equiv b\pmod{p^2}\notag\\
        \therefore\;&a^{p}-b^p\equiv \left(a^{p^2}\right)^p-\left(b^{p^2}\right)^p \pmod{p^2}\notag
    \end{align}
    Again, $ a^p\equiv b^p\pmod{p} $\\
    We have, $ \begin{aligned}[t]
        &\left(a^{p}\right)^{p^2}\equiv \left(b^{p}\right)^{p^2} \pmod{p^2}\\
        \Rightarrow\; &a^{p} \equiv b^{p}\pmod{p^2}\\
        \Rightarrow\; &a \equiv b\pmod{p^2}
    \end{aligned} $
\end{soln}
\begin{qn}
    If $ p $ is a prime of the form $ 4n+1 $, then show that $ 28!+233\equiv 0\pmod{899} $ i.e., $ 28!+233 $ is divisible by $ 899 $.
\end{qn}
\begin{soln}
    Here, $ \begin{aligned}[t]
        899&=29\cdot31\\
        233&=8\cdot29+1\\
        233&=3\cdot7+16
    \end{aligned} $
    \begin{equation}
        \therefore \;233\equiv1\pmod{29}\label{eq:q15.1}
    \end{equation}
    and
    \begin{equation}
        \therefore \;233\equiv16\pmod{31}\label{eq:q15.2}
    \end{equation}
    Now, by using Wilson's theorem we have,
    \begin{align}
        & (29-1)!+1\equiv 0\pmod{29}\notag\\
        \Rightarrow\;& 28!+1\equiv 0\pmod{29}\label{eq:q15.3}
    \end{align}
    Combining \eqref{eq:q15.1} and \eqref{eq:q15.3},
    \begin{equation}
        28!+233\equiv0\pmod{29}\label{eq:q15.4}
    \end{equation} 
    Again by using Wilson's theorem,
    \begin{align}
        & (31-1)!+1\equiv 0\pmod{31}\notag\\
        \Rightarrow\;& 30\cdot 29\cdot 28!+1\equiv 0\pmod{31}\notag\\
        \Rightarrow\;& -1\cdot -2\cdot 28!+1\equiv 0\pmod{31}\notag\\
        \Rightarrow\;& 2\cdot 28!+1+31\equiv 0\pmod{31}\notag\\
        \Rightarrow\;& 28!+16\equiv 0\pmod{31}\label{eq:q15.5}
    \end{align}
    Combining \eqref{eq:q15.2} and \eqref{eq:q15.5},
    \[
        28!+233\equiv 0\pmod{29\cdot31} \quad \Rightarrow\; 28!+233\equiv 0\pmod{899}
    \]
\end{soln}
\begin{qn}
    Prove that $ 18!+1 \equiv 0\pmod{437} $ i.e., $ 18!+1 $ is divisible by 437.
\end{qn}
\begin{soln}
    Here $ 437=19\cdot 23 $\\
    Thus, using Wilson's theorem,
    \begin{align}
        &(19-1)!+1\equiv 0\pmod{19}\notag\\
        \Rightarrow\;&18!+1\m{0}{19}\label{eq:q16.1}\\
        &\notag\\
        &(23-1)!+1\m{0}{23}\notag\\
        \Rightarrow\;&22!+1\m{0}{23}\notag\\
        \Rightarrow\;&22\cdot21\cdot20\cdot19\cdot18!+1\m{0}{23}\notag\\
        \Rightarrow\;&-1\cdot -2\cdot -3\cdot -4\cdot 18!+1\m{0}{23}\notag\\
        \Rightarrow\;&24\cdot 18!+1\m{0}{23}\notag\\
        \Rightarrow\;&(23+1)\cdot 18!+1\m{0}{23}\notag\\
        \Rightarrow\;&23\cdot18! +18!+1\m{0}{23}\label{eq:q16.2}
    \end{align}
    Now, from \eqref{eq:q16.1} and \eqref{eq:q16.2} we have $ 18!+1\m{0}{19\cdot23} $ i.e., $ 18!+1\m{0}{437} $.
\end{soln}
\begin{qn}
    If $ p $ is a prime of the form $ 4n+1 $, then $ (2n)! $ is a solution of the congruence $ x^2\m{-1}{p} $.
\end{qn}
\begin{soln}
    If $ p $ is a prime, then by Wilson's theorem,\\
    We have,
    \begin{equation}
        (p-1)!+1\m{0}{p}\label{eq:q17.1}
    \end{equation}
    Putting $ p=4n+1 $ in \eqref{eq:q17.1}, we get,
    \begin{align}
        & (4n)!+1\m{0}{p}\notag\\
        \Rightarrow\;& 4n\cdot(4n-1)\cdot(4n-2)\dots(2n+1)\cdot(2n)!+1\m{0}{p}\label{eq:q17.2}
    \end{align}
    Now, $ p=4n+1 $
    \begin{align*}
        \therefore\;&4n+1\m{0}{p}\\
        \Rightarrow\;&4n\m{-1}{p}\\
        \Rightarrow\;&4n-1 \m{-2}{p}\\
        \Rightarrow\;&4n-2 \m{-3}{p}\\
        &\dots\quad \dots\quad \dots\\
        \Rightarrow\;&4n-(2n-1) \m{-2n}{p}\\
        \text{i.e., }\;&2n+1 \m{-2n}{p}
    \end{align*}
    Hence multiplying all the congruence, we get,
    \begin{equation}
        4n\cdot(4n-1)\cdot(4n-2)\dots(2n+1)\m{(-1)^{2n}(2n)!}{p}\label{eq:q17.3}
    \end{equation}
    Combining \eqref{eq:q17.2} and \eqref{eq:q17.3}, we get,
    \begin{align*}
        & (-1)^{2n} (2n)!\cdot(2n)!+1\m{0}{p}\\
        \Rightarrow\;& ((2n)!)^2 \m{-1}{p}\\
        \Rightarrow\;& x^2 \m{-1}{p},\\
        &\text{where } x=(2n)!
    \end{align*}
    Thus, $ x=(2n)! $ is a solution of the given congruence $ x^2\m{-1}{p} $.
\end{soln}
\begin{qn}
    Show that, $ a^7-a $ is divisible by 42.
\end{qn}
\begin{soln}
    Let, $ T=a^7-a=a(a^6-1)=a(a^3-1)(a^3+1) $\\
    i.e., $ \begin{aligned}[t]
        T&= a(a-1)(a+1)(a^2+a+1)(a^2-a+1)\\
        &= (a-1)a(a+1)(a^4+a^2+1)
    \end{aligned} $\\
    Since $ (a-1)a(a+1) $ is a product of three consecutive integers, hence is divisible by $ 3!=6 $ and so $ (a-1)a(a+1) $ is divisible by 6.\\
    Now, by Fermat's theorem,
    \begin{align*}
        &a^{7-1} \m{1}{7}\\
        \Rightarrow\; &a(a^{6}-1) \m{0}{7}
    \end{align*}
    Hence the product $ (a-1)a(a+1)(a^4+a^2+1) $ is divisible by the product of $ 6 $ and 7. That is, $ a^7-a $ is divisible by 42.
\end{soln}
\begin{qn}
    Show that, $ a^{36}-1 $ is divisible by 33744. If $ a  $ is prime to 2, 3, 19 and 37.
\end{qn}
\begin{soln}
    Given, $ (a,2)=1 $, $ (a,3)=1 $, $ (a,19)=1 $, and $ (a,37)=1 $.\\
    By Fermat's theorem,
    \begin{align*}
        & a^{2-1}\m{1}{2}\\
        \Rightarrow\;& a^{36}\m{1}{2}\\
        \intertext{Similarly, }
        & a^{36}\m{1}{3}\\
        & a^{36}\m{1}{19}\\
        & a^{36}\m{1}{37}
    \end{align*}
    Since 2, 3, 19 and 37 are relatively prime in pairs.\\
    So, $ \begin{aligned}[t]
        a^{36} &\m{1}{2\cdot3\cdot19\cdot37}\\
        a^{36} &\m{1}{4218}
    \end{aligned} $\\
    That is $ a^{36}-1 $ is divisible by 4218.\\
    Again, $ a^{36}-1=\left( a^{18} \right)^2-1=(a^{18}+1)(a^{18}-1) $\\
    When $ a $ is odd, then $ (a^{18}-1) $ and $ (a^{18}+1) $ are two consecutive even numbers and hence one of them is divisible by 2 and the other is by 4.

    So, their product $ (a^{18}+1)(a^{18}-1) $ is divisible by 8.\\
    Therefore, $ a^{36}-1 $ is divisible by $ 8\times 4218=33744 $.
\end{soln}
\begin{qn}
    Solve these congruences
    \begin{enumerate}[label=(\alph*)]
        \item $ 5x\m{2}{7} $
        \item $ 98x\m{7}{105} $
        \item $ 15x\m{6}{21} $
    \end{enumerate}
\end{qn}
\begin{soln}\hfill
    \begin{enumerate}[label=(\alph*)]
        \item Here $ (5,7)=1 $ so the given congruence $ 5x\m{2}{7} $ has exactly one solution.
        \begin{align*}
            &5x\m{2}{7}\\
            \Rightarrow\;&15x\m{6}{7}\\
            \Rightarrow\;&x\m{6}{7}
        \end{align*}
        Hence, $ x=6 $ is a root of $ 5x\m{2}{7} $.
        \item Here $ (98,105)=7 $ and $ 7\mid7 $. So there are 7 incongruent roots of the congruence $ 98x\m{7}{105} $.
        \begin{align*}
            & 98x\m{7}{105}\\
            \Rightarrow\;& 14x\m{1}{15}\\
            \Rightarrow\;& -x\m{1}{15}\\
            \text{i.e., }\;& x\m{-1+15}{15}\\
            \text{i.e., }\;& x=14 \text{ is a solution.}
        \end{align*}
        Hence, the other incongruent solution are given by,
        \begin{align*}
            x&=14,\;14+\frac{105}{7},\;14+\frac{2\cdot105}{7},\;14+\frac{3\cdot105}{7},\;14+\frac{4\cdot105}{7},\;14+\frac{5\cdot105}{7},\;14+\frac{6\cdot105}{7}\\
            \text{i.e., }x&=14,\;29,\;44,\;59,\;74,\;89,\;104
        \end{align*}
        \item Here $ (98,105)=7 $ and $ 7\nmid 1 $ so $ 98x\m{1}{105} $ has no solution.
    \end{enumerate}
\end{soln}
\begin{qn}
    Solve the following simultaneous congruences
    \begin{enumerate}[label=(\alph*)]
        \item $ x\m{1}{15} $ and $ x\m{11}{21} $
        \item $ x\m{2}{12} $ and $ x\m{5}{13} $
    \end{enumerate}
\end{qn}
\begin{soln}\hfill
    \begin{enumerate}[label=(\alph*)]
        \item \begin{align*}
            & x\m{1}{15}\\
            \Rightarrow\;& \left.\begin{array}{@{}l}
                x \m{1}{2}\\
                x \m{1}{5}
            \end{array}\right\}\text{ Since 15 must be divisible by each of 3 and 5 and $ (3,5)=1 $.}
            \end{align*}
        Again, 
        \begin{align*}
            &x\m{11}{21}\\
            \Rightarrow\;&x\equiv 11 \m{2}{3}\\
            &x\equiv 11 \m{4}{7}
        \end{align*}
        $ \left.\begin{array}{lc}
            \text{But} & x\m{1}{3}\\
            \text{and} & x\m{2}{3}\\
        \end{array}\right\} $ is impossible.\\
        So the given congruences has no solution.
        \item \begin{align*}
            &x\m{2}{12}\\
            \Rightarrow\;&\left\{\begin{array}{l}
                x\m{2}{3}\\
                x\m{2}{4}\\
            \end{array}\right.
        \end{align*}
        Thus, we have to solve
        \begin{align*}
            x&\m{2}{3}\\
            x&\m{2}{4}\\
            x&\m{5}{13}
        \end{align*}
        Here $ \begin{aligned}[t]
            a_1&=2,\quad a_2=2,\quad a_3=5\\
            m_1&=3,\quad m_2=4,\quad m_3=15;\quad m=m_1m_2m_3=156\\
            Q_1&=52,\quad Q_2=39,\quad Q_3=12;\quad \text{where }Q_i=\frac{m}{m_i}
        \end{aligned} $\\
        Consider the congruence $ Q_iy_i\m{1}{m_i} $\\
        \begin{align*}
            &52y_1\m{1}{3}\\
            \Rightarrow\;&y_1\m{1}{3}\\
            &\\
            &39y_2\m{1}{4}\\
            \Rightarrow\;&-y_2\m{1}{4}\\
            &\\
            &12y_3\m{1}{13}\\
            \Rightarrow\;&-y_3\m{1}{13}\\
            \Rightarrow\;&y_3\m{12}{13}
        \end{align*}
        Now, $ \begin{aligned}[t]
            X&= Q_1y_1a_1+Q_2y_2a_2+Q_3y_3a_3\\
            &= 104+234+720\\
            &\m{1058}{156}\\
            &\m{122}{156}
        \end{aligned} $\\
        
        Thus, $ x=122 $ is the least solution and the other solutions are given by $ x=22+156y $.
    \end{enumerate}
\end{soln}
\begin{qn}
    Show that $ 2^{2n+1}-9n^2+3n-2 $ is divisible by 54.
\end{qn}
\begin{soln}
    Let, $ f(n)=2^{2n+1}-9n^2+3n-2 $\\
    Then $ f(1)=8-9+3-2=0 $ is divisible by 54.\\
    Now,
    \begin{align*}
        f(n+1)-f(n)&=2^{2n+3}-9(n+1)^2+3(n+1)-2-2^{2n+1}+9n^2-3n+2\\
        &=2^2\cdot2^{2n+1}-2^{2n+1}-18n-6\\
        &=3\cdot2^{2n+1}-18n-6\\
        &=6(2^2)^n-18n-6\\
        &=6(3+1)^n-18n-6\\
        &=6\left[(1+3\ncr{n}{1}+\ncr{n}{2}3^{n-2}+\ncr{n}{3}3^{n-3}+\dots+3^n)\right]-18n-6\\
        &=6\left(\ncr{n}{2}3^{n-2}+\ncr{n}{3}3^{n-3}+\dots+3^n\right)\\
        &=54\left(\ncr{n}{2}3^{n-4}+\ncr{n}{3}3^{n-5}+\dots+3^{n-2}\right)\\
        &=54k,\text{ where $ k $ is an integer}
    \end{align*}
    Hence, if $ f(n) $ is divisible by 54 then $ f(n+1) $ is divisible by 54. Now, $ f(1) $ is divisible by 54 so $ f(1+1)=f(2) $ is divisible by 54. Thus, it follows that $ f(3) $, $ f(4) $, $ \dots $ etc. are divisible by 54.

    $ \therefore\;2^{2n+1}-9n^2+3n-2 $ is divisible by 54.
\end{soln}
\begin{qn}
    Prove that if $ a $ is an even integer, then $ a(a^2+20) $ is divisible by 48.
\end{qn}
\begin{soln}
    Let $ f(a)=a(a^2+20) $\\
    $ \therefore\; f(2)=2(2^2+20)=48 $ which is divisible by 48.\\
    Now,
    \begin{align*}
        f(2n+2)-f(2n)&=(2n+2)\left\{ (2n+2)^2+20 \right\}-2n\left( 4n^2+20 \right)\\
        &=24n^2+24n+48\\
        &=48\left( \frac{n^2+n+1}{2}\right)\\
        &=48k
    \end{align*}
    Hence if $ f(2n) $ is divisible by 48 then $ f(2n+2) $ is also divisible by 48.

    Now, $ f(2\cdot1) =f(2)=48$ is divisible by 48 so $ f(2\cdot1+2)=f(4) $ is divisible by 48 and hence by succession we get $ f(6),\;f(8),\dots $ etc. are divisible by 48.
\end{soln}
\begin{qn}
    Using Chinese remainder theorem, solve $ 13x\m{17}{42} $.
\end{qn}
\begin{qn}
    Solve $ x\m{5}{6} $ and $ x\m{8}{15} $.
\end{qn}
\begin{qn}
    Find the four roots of the congruence $ x^2\m{-1}{65} $.
\end{qn}
\begin{soln}
    \hfill
    \begin{align*}
        &x^2\m{-1}{65}\\
        \Rightarrow\;&x^2\m{-1+65}{65}\\
        \Rightarrow\;&(x-8)(x+8)\m{0}{65}\\
    \end{align*}
    Now, $ x\m{8}{65} $ and $ x\m{-8}{65} $.\\
    Since $ 65=5\times 13 $ and 5, 13 are relatively prime to each other.\\
    $ \begin{aligned}
        \therefore\;&x\m{8}{5}\\
        &x\m{8}{13}\qquad
    \end{aligned} $ or $ \begin{aligned}
       \qquad &x\m{-8}{5}\\
        &x\m{-8}{13}
    \end{aligned} $\\
    Now, we shall use solve these congruences by Chinese remainder theorem.
    \begin{align*}
        &x\equiv 8\m{3}{5}&m_1&=5,\;m_2=13,\;m=65\\
        &x\m{8}{13}&a_1&=3,\;a_2=8\\
        &&Q_1&=13,\;Q_2=5
    \end{align*}
    Therefore,
    \begin{align*}
        &13y_1\m{1}{5}\\
        \Rightarrow\;&y_1\m{2}{5}\\
        &5y_2\m{1}{13}\\
        &26y_2-y_2\m{5}{13}\\
        \Rightarrow\;&-y_2\m{5}{13}\\
        \text{i.e., }&y_2\m{8}{13}\\
        \therefore\;&x=78+320=398\m{8}{65}
    \end{align*}
    Thus, $ x=8 $ is a solution of the congruence $ x^2\m{-1}{65} $ and $ 65-8=57 $ is ---- root of this congruence.\\

    Again, 
    \begin{align*}
        &x\equiv -8\m{3}{5}&m_1&=5,\;m_2=13,\;m=65\\
        &x\equiv -8\m{5}{13}&a_1&=3,\;a_2=5\\
        &&Q_1&=13,\;Q_2=5
    \end{align*}
    Consider, 
    \begin{align*}
        &13y_1\m{1}{5}\\
        \Rightarrow\;&y_1\m{2}{5}\\
        \text{and }&5y_2\m{1}{13}\\
        &y_2\m{8}{13}\\
    \end{align*}
    $\therefore\;x=78+200=278\m{18}{65}$\\
    Hence, the other solution of $ x^2\m{-1}{65} $ is $ 65-18=47 $.\\
    Hence, the four roots of the congruence $ x^2\m{-1}{65} $ is 8, 18, 47, 57.
\end{soln}
\begin{qn}
    Find the four roots of the congruence $ x^2\m{-2}{33} $.
\end{qn}
\begin{soln}
    \hfill
    \begin{align*}
        &x^2\m{-2}{33}\\
        \Rightarrow\;&x^2\m{-2}{3}\\
        &x^2\m{-2}{11}\qquad\text{ as }(3,11)=1\text{ and }3\times11=33
    \end{align*}
    Now, 
    \begin{align*}
        &x^2\m{-2}{3}\\
        \Rightarrow\;&x^2\m{16}{3}\\
        \Rightarrow\;&x\m{4}{3}\\
        \Rightarrow\;&x\m{1}{3}
    \end{align*}
    Again,
    \begin{align*}
        &x^2\m{-2}{11}\\
        \Rightarrow\;&x^2\m{9}{11}\\
        \Rightarrow\;&x\m{3}{11}
    \end{align*}
    Thus, solving $ x\m{1}{3} $ and $ x\m{3}{11} $ by Chinese remainder theorem, we get 8, 14, 19, 25 are the four incongruent roots of $ x^2\m{-2}{33} $.
\end{soln}
\begin{qn}
    Find the four roots of the congruence $ x^2\m{9}{16} $
\end{qn}
\begin{soln}
    Given, $ \begin{aligned}[t]
        & x^2\m{9}{16}\\
        \Rightarrow\; & x^2\m{\left( \pm 3 \right)^2}{16}
    \end{aligned} $\\

    So, roots of the given congruence are $ \pm 3 $.\\
    The other root is $ 16-3=13 $ of $ x^2\m{9}{16} $\\
    So, $ 3,13 $ are two roots of the congruence $ x^2\m{9}{16} $.\\
    Again, Since,
    \begin{align*}
        & x^2\m{9}{16}\\
        \therefore\;& x^2\m{9}{8}
    \end{align*}
    Now, another roots of the given congruence will be $ \pm 3+8k $ where $ k=0,1 $\\
    For $ k=0 $, $ x=\pm3+0=3,\;-3=3,\;16-3=3,13 $\\
    For $ k=1 $, $ x=\pm3+8=11,5 $\\
    So, $ x^2\m{9}{16} $\\
    Therefore, the four roots of the congruence $ x^2\m{9}{16} $ are $ 3,5,11,13 $.
\end{soln}
\begin{qn}
    If $ p  $ is a prime of the form $ 4n+3 $, show that $ (2n+1)! $ is a root of the congruence $ x^2\m{1}{p} $
\end{qn}
\begin{soln}
    Since $ p $ is a prime of the form $ 4n+3 $,\\
    We have,
    \begin{align*}
        & 4n+3\m{0}{p}\\
        \Rightarrow\; & -3\m{4n}{p}\\
        \intertext{Now,}
        & -1\m{4n+2}{p}\\
        & -2\m{4n+1}{p}\\
        & -3\m{4n}{p}\\
        & -4\m{4n-1}{p}\\
        & \dots\quad \dots\quad \dots\\
        & -(2n+1)\m{4n+\left\{ -(2n+1)+3 \right\}}{p}\\
        \text{i.e., }& -(2n+1)\m{2n+2}{p}
    \end{align*}
    Multiplying both sides we get,
    \begin{align}
        & (-1)^{2n+1} \{1\cdot2\cdot3\cdot \dots (2n+1)\}\m{(4n+2)(4n+1)(4n)(4n-1)\dots (2n+2)}{p}\notag \\
        \Rightarrow\; & -(2n+1)! \m{\frac{(4n+2)(4n+1)(4n)\dots (2n+2)(2n+1)(2n-1)(2n-2)\dots2\cdot1}{(2n+1)\dots3\cdot2\cdot1}}{p}\notag \\
        \Rightarrow\; & -{(2n+1)!}^2 \m{(4n+2)!}{p}\label{eq:q27.1}
    \end{align}
    Again, since $ p $ is a prime of the form $ 4n+3 $, by Wilson's theorem we have,
    \begin{align}
        &(4n+3-1)!+1\m{0}{p}\notag\\
        &(4n+2)!+1\m{0}{p}\label{eq:q27.2}
    \end{align}
    From \eqref{eq:q27.1} and \eqref{eq:q27.2},
    \begin{align*}
        &-{(2n+1)!}^2\m{-1}{p}\\
        \Rightarrow\; &{(2n+1)!}^2\m{1}{p}\\
        \therefore\; &x^2\m{1}{p}\qquad \text{where, } x=(2n+1)!
    \end{align*}
    Thus, $ x=(2n+1)! $ is a root of the congruence $ x^2\m{-1}{p} $.
\end{soln}
\begin{qn}
    If $ p $ is an odd prime and $ h+k=p-1 $ prove that $ h! k! + (-1)^h\m{0}{p} $.
\end{qn}
\begin{soln}
    If $ p $ is a prime of the form $ h+k+1=p $ then we can write,
    \begin{align*}
        h+k+1&\m{0}{p}\\
        h+1&\m{-k}{p}\\
        h+2&\m{-(k-1)}{p}\\
        h+3&\m{-(k-2)}{p}\\
        h+4&\m{-(k-3)}{p}\\
        \dots\quad&\dots\quad \dots\\
        h+k&\m{-1}{p}
    \end{align*}
    Multiplying the above congruences, we get 
    \begin{align}
        & (h+1)(h+2)(h+3)\dots(h+k)\m{(-1)^k k!}{p}\notag\\
        \Rightarrow\;& h!\; (h+1)(h+2)(h+3)\dots(h+k)\m{(-1)^k \,k!\; h!}{p}\notag\\
        \Rightarrow\;& (h+k)! \m{(-1)^k\, k!\; h!}{p}\label{eq:q28.1}
    \end{align}
    Again $ p $ is a prime of the form $ h+k+1 $, so by Wilson's theorem we have,
    \begin{align}
        (h+k+1-1)!&\m{-1}{p}\notag\\
        \text{i.e., }(h+k)!&\m{-1}{p}\label{eq:q28.2}
    \end{align}
    Hence, by \eqref{eq:q28.1} and $ \eqref{eq:q28.2} $, we write,
    \[(-1)^k\, h!\;k!\m{-1}{p}\]
    Since, $ p $ is a prime, $ (-1)=(-1)^{k+h+1} $
    \begin{align*}
        \therefore\;&(-1)^k\, h!\;k!\m{(-1)^{k+h+1}}{p}\\
        \Rightarrow\;&(-1)^h+ h!\;k!\m{0}{p}
    \end{align*}
\end{soln}
\begin{qn}
    Prove/Find the number of divisors and sum of divisors if a composite number.
\end{qn}
\begin{soln}\hfill
    \begin{itemize}
        \item The function $ d(n) $ is the number of divisors of the composite number $ n $ including $ 1 $ and $ n $.
        \item The function $ \sigma(n) $ is the sum of the divisors of the composite number $ n $.
    \end{itemize}

    Consider the factorization of the composite number $ n $ into primes be $ n= {p_1}^{\alpha_1}{p_2}^{\alpha_2} \dots {p_r}^{\alpha_r}$ where $ {p_r}^{\alpha} $s are pairwise relatively prime.\\ %(p_r)^{\alpha}  is not confirm
    Then, the divisors of $ {p_1}^{\alpha_1}  $ are $ 1,p_1,p_1^2,\dots, {p_1}^{\alpha_1} $.\\
    Therefore, $ \di{{p_1}^{\alpha_1}}=\alpha_1+1 $ and hence,
    \begin{align*}
        \di{n}&=\di{{p_1}^{\alpha_1}}\di{{p_2}^{\alpha_2}}\dots\di{{p_r}^{\alpha_r}}\\
        &=(\alpha+1)(\alpha+1)\dots(\alpha+1)\\
        &= \prod_{i=1}^r (\alpha_i+1)
    \end{align*}

    Now, sum of the divisors of $ {p_1}^{\alpha_1} $ is $ 1+p_1+p_1^2+\dots+ {p_1}^{\alpha_1}=\frac{{p_1}^{\alpha_1+1}-1}{p_1-1} $.\\
    i.e., $ \sig{{p_1}^{\alpha_1}}=\frac{{p_1}^{\alpha_1+1}-1}{p_1-1}  $\\
    and therefore,
    \begin{align*}
        \sig{n}&=\sig{{p_1}^{\alpha_1}}\sig{{p_2}^{\alpha_2}}\dots\sig{{p_r}^{\alpha_r}}\\
        &=\frac{{p_1}^{\alpha_1+1}-1}{p_1-1} \cdot \frac{{p_2}^{\alpha_2+1}-1}{p_2-1}\dots\frac{{p_r}^{\alpha_r+1}-1}{p_r-1}\\
        &= \prod_{i=1}^r \left( \frac{{p_i}^{\alpha_i+1}-1}{p_i-1} \right)
    \end{align*}
\end{soln}
\begin{qn}
    Show that
    \[\sum_{d/n}\{f(d)\}^3=\left\{ \sum_{d/n} f(d) \right\}^2\]
\end{qn}
\begin{soln}
    \hfill\newline
    Left-Hand side.\\
    Suppose that $ n=p^k $. Since $ f(d) $ is multiplicative function and $ f(d) $ denotes the numbers of divisors of $ n $.
    \begin{align*}
        \therefore\; \sum_{d/n}\left\{ f(d) \right\}^3&=\left\{ f(1) \right\}^3+\left\{ f(p) \right\}^3+\left\{ f(p^2) \right\}^3+\dots+\left\{ f(p^k) \right\}^3\\
        &=1^3+2^3+3^3+\dots+(k+1)^3\\
        &=\left\{ \frac{(k+1)(k+2)}{2} \right\}^2
    \end{align*}
    Right-Hand side.\\
    \begin{align*}
        \therefore\; \left\{\sum_{d/n} f(d) \right\}^2&=\left\{ f(1) +f(p)+f(p^2) +\dots+f(p^k) \right\}^2\\
        &=\left\{1+2+3+\dots+(k+1)\right\}^2\\
        &=\left\{ \frac{(k+1)(k+2)}{2} \right\}^2
    \end{align*}
    Hence proved.
\end{soln}
\begin{qn}
    If $ a $ is an even number then show that $ 48\mid a(a^2+20) $.
\end{qn}
\begin{soln}
    We have,
    \begin{align*}
        p&=a\left(a^2+20\right)\\
        &=a\left(a^2-4+24\right)\\
        &=a\left((a-2)(a+2)+24\right)\\
        &=(a-2)(a)(a+2)+24a
    \end{align*}
    Now, since $ a $ is an even number so let $ a=2n $ for any integer $ n $.\\
    Then, 
    \begin{align*}
        p&=(2n-2)(2n)(2n+2)+48a\\
        &=8(n-1)(n)(n+1)+48a
    \end{align*}
    Now, since $ (n-1)\;n\;(n+1) $ is the product of three consecutive integers, so it is divisible by $ 3!=6 $. And hence $ 8(n-1)(n)(n+1) $ is divisible by $ 8\times 6=48 $. Again $ 48a $ is ----- divisible by 48. Hence, the term $ p=a(a^2+20) $ is divisible by 48.
\end{soln}
\begin{qn}
    If $ n $ is an odd integer, $ n(n^2+1) $ is divisible by 24.
\end{qn}
\begin{soln}
    Since, $ n $ is odd integer so $ n-1 $ and $ n+1 $ are two consecutive integers and hence one of them is divisible by 2 and the other is divisible by 4.\\

    Again, $ (n-1),n,(n+1) $ are three consecutive integers so one of them is divisible by 3. Thus, the given expression is divisible by 2, 3 and 4 and hence by their product 24.
\end{soln}
\begin{qn}
    Find $ \di{n} $ and $ \sig{n} $ for $ n=21600 $.
\end{qn}
\begin{soln}
    Here, $ n=21600=2^5\cdot3^3\cdot5^2 $\\
    \begin{align*}
        \therefore \di{n}&=\text{ number of divisors of } n\\
        &=\prod_{i=1}^3 (\alpha_i+1)\\
        &=(5+1)(3+1)(2+1)\\
        &=72\\
        \intertext{And,}
        \therefore \sig{n}&=\text{ sum of the divisors }\\
        &=\prod_{i=1}^3 \frac{{p_i}^{\alpha_i+1}-1}{p_i-1}\\
        &=\frac{2^{6}-1}{2-1}\cdot \frac{3^{4}-1}{3-1}\cdot\frac{5^{3}-1}{5-1}\\
        &=78120\\
    \end{align*}
\end{soln}
\begin{qn}
    Find the positive integer solution of the linear Diophantine equation $ 62x+11y=788 $.
\end{qn}
\begin{soln}
    Here, $ a=62 $, $ b=11 $, and $ c=788 $.\\
    Now using Euclid's algorithm,
    \begin{align*}
        62&=11\cdot 5+7\\
        11&=7\cdot 1+4\\
        7&=4\cdot 1+3\\
        4&=3\cdot 1+1\\
        3&=1\cdot 3+0
    \end{align*}
    Now, $ (62,11)=1 $ and $ 1\mid 788 $, so it has a solution.\\
    Now, 
    \begin{align*}
        1&= 4+3\cdot(-1)\\
        &= 4+(-1)\left\{ 7+4\cdot(-1) \right\}\\
        &= 2\cdot4+(-1)\cdot7\\
        &= 2\left\{ 11+(-1)\cdot7 \right\}+(-1)\cdot7\\
        &= 2\cdot 11+(-3)\cdot7 \\
        &= 2\cdot 11+(-3)\left\{ 62+11\cdot(-5) \right\} \\
        &= 11(17)+62(-3)\\
        \Rightarrow\; 62(-2364)+11(13396)&= 788
    \end{align*}
    Hence, $ x_0=-2364 $ and $ y_0=13396 $ is a particular solution of $ 62x+11y=788 $.\\
    Hence, the general solution of the given linear Diophantine equation is given by $ x=x_0+\frac{b}{d}t $, $ y=y_0-\frac{a}{d}t $. Where $ t $ is an integer.\\
    i.e., $ x=-2364+11t $ and $ y=13396-62t $.\\
    Hence, the positive integral solutions are given by 
    \begin{align*}
        &-2364+11t>0 \qquad\text{ and } \qquad 13396-62t>0\\
        \Rightarrow\;&t>214.91 \qquad\text{ and } \qquad t<216.0645 
    \end{align*}
    Now, $ 214.91<t<216.0645 $ and since $ t $ is integer, so we conclude that $ t=215 $ and $ 216 $.\\
    Hence, the positive integral solution is 
    \begin{enumerate}[label=(\roman*)]
        \item $ x=1 $, $ y=66 $  and
        \item $ x=12 $, $ y=4 $
    \end{enumerate}
\end{soln}
\begin{thm}
    If $ p $ is a prime then 
    \[\sum_{i=0}^\alpha \ephi{p^i}=p^\alpha\]
\end{thm}
\begin{proof}
    \hfill
    \begin{align*}
        \sum_{i=0}^\alpha \ephi{p^i}&=\ephi{p^0}+\ephi{p}+\ephi{p^1}+\dots+\ephi{p^\alpha}\\
        &=1+(p-1)+p^2\left( 1-\frac{1}{p} \right)+p^3\left( 1-\frac{1}{p} \right)+\dots+p^\alpha\left( 1-\frac{1}{p} \right)\\
        &=1+(p-1)+p\left( p-1\right)+p^2\left( p-1\right)+\dots+p^{\alpha-1}\left( p-1\right)\\
        &=1+(p-1)\frac{p^{\alpha-1+1}-1}{p-1}\\
        &=p^\alpha
    \end{align*}
\end{proof}
\section*{Properties of Legendre Symbol}
\begin{thm}
    If $ p $ is an odd prime and $ (a,p)=1 $, $ (b,p)=1 $ then
    \begin{enumerate}[label=(\roman*)]
        \item $ \displaystyle \leg{a}{p}\m{a^{\frac{p-1}{2}}}{p} $
        \item $ \displaystyle \leg{a}{p}\leg{b}{p}=\leg{ab}{p}$
        \item $ a\m{b}{p} $ implies $ \displaystyle \leg{a}{p}=\leg{b}{p}$
        \item $ \displaystyle \leg{a^2}{p}=1;\;\leg{a^b}{p}=\leg{b}{p};\;\leg{1}{p}=1,\;\leg{-1}{p}=(-1)^{\frac{p-1}{2}}$
        \item $ \displaystyle \leg{2}{p}=(-1)^{\frac{p^2-1}{8}}$
        \item If $ p $ and $ q $ are distinct odd prime then $ \displaystyle \leg{p}{q}\leg{q}{p}=(-1)^{\leg{p-1}{2}\leg{q-1}{2}}$
    \end{enumerate}
\end{thm}
\begin{qn}
    Find:
    \begin{enumerate*}
        \item $ \displaystyle \leg{231}{997} $
        \item $ \displaystyle \leg{3}{101} $
        \item $ \displaystyle \leg{60}{29} $
        \item $ \displaystyle \leg{20}{7} $
        \item $ \displaystyle \leg{85}{11} $
        \item $ \displaystyle \leg{100}{7} $
    \end{enumerate*}
\end{qn}
\begin{soln}
    \hfill
    \begin{enumerate}
        \item Here 231 and 997 are two distinct odd primes, so
        \begin{align*}
            \leg{231}{997}&=\leg{997}{231}(-1)^{\leg{231-1}{2}\leg{997-1}{2}}\\
            &=\leg{73}{231}(-1)^{(115)(498)}\\
            &=\leg{231}{73}(-1)^{(115)(36)}\\
            &=\leg{12}{73}
        \end{align*}
        Now, by Jacobi symbol, we have
        \[\leg{12}{73}=\leg{1}{73}\leg{2^2}{73}\leg{3}{73}\]
        \begin{align*}
            \therefore\;\leg{1}{73}&=1;\\
            \therefore\;\leg{2^2}{73}&=1;\\
            \therefore\;\leg{3}{73}&=\leg{73}{3}(-1)^{\leg{3-1}{2}\leg{73-1}{2}}\\
            &=\leg{1}{3}\\
            &=1
        \end{align*}
        Hence, $ \displaystyle \leg{231}{997}=\leg{12}{73}=\leg{1}{73}=\leg{2^2}{73}\leg{3}{73}=1\cdot1\cdot1=1 $.
        \item $ \displaystyle \leg{3}{101}=\leg{101}{3}(-1)^{1\cdot 50}=\leg{2}{3}=(-1)^{\leg{3^2-1}{8}}=(-1)^{\frac{8}{9}}=-1 $
        \item $ \displaystyle \leg{60}{29}=\leg{1}{29}\leg{2^2}{29}\leg{3}{29}\leg{5}{29} $\\
        Now,
        \begin{align*}
            \leg{1}{29}&=1;\\
            \leg{2^2}{29}&=1;\\
            \leg{3}{29}&=(-1)^{14\cdot1}\\
            &=\leg{2}{3}\\
            &=(-1)^{\frac{3^2-1}{8}}\\
            &=-1\\
            \leg{5}{29}&=\leg{29}{5}(-1)^{2\cdot14}\\
            &=\leg{4}{5}\\
            &=\leg{2^2}{5}\\
            &=1
        \end{align*}
        $ \displaystyle \therefore\;\leg{60}{29}=1\cdot1\cdot-1\cdot1=-1 $
        \item \[\leg{20}{7}=\leg{1}{7}\leg{2^2}{7}\leg{5}{7} \]
        \[\leg{1}{7}=1,\;\leg{2^2}{7}=1,\;\leg{5}{7}=\leg{7}{5}(-1)^{2\cdot 3 }=\leg{2}{5}=(-1)^{\frac{5^2-1}{8}}=-1\]
        $ \displaystyle \therefore\; \leg{20}{7}=1\cdot1\cdot-1=-1 $.
        \item \begin{align*}
            \leg{85}{11}&=\leg{8}{11} \text{ as } x^2\m{85}{11} \Rightarrow\; x^2\m{8}{11}\\
            &=\leg{2^2\cdot 2}{11}\\
            &=\leg{2}{11}\text{ as }\leg{a^2b}{p}=\leg{b}{p}\\
            &=(-1)^{\frac{11^2-1}{8}}\\
            &=(-1)^{15}\\
            &=-1
        \end{align*}
        \item $ \displaystyle \leg{100}{7}=\leg{2}{7}=(-1)^{\frac{7^2-1}{8}}=(-1)^6=1 $
    \end{enumerate}
\end{soln}
\begin{qn}
    Find the values $ \di{1968} $, $ \di{255} $, $ \di{111} $, $ \di{353650} $, $ \sig{1968} $, $ \sig{255} $, $ \sig{111} $, $ \sig{353650} $, $ \mob{25} $, $ \mob{235} $, $ \mob{300} $.
\end{qn}
\begin{soln}
    Here,
    \begin{align*}
        1968&=3\cdot 4^2\cdot 41=2^4\cdot 3^1 \cdot 41^1\\
        255&=3\cdot 5\cdot 17\\
        111&=3\cdot 37\\
        353650&=2\cdot 5^2\cdot 11\cdot 643\\
        25&=5^2\\
        235&=5\cdot 47\\
        300&=3\cdot 4\cdot 5^2=2^2\cdot 3^1\cdot 5^2
    \end{align*}
    \begin{align*}
        \therefore\; \di{1968}&=\di{2^4\cdot 3^1 \cdot 41^1}=(1+1)(4+1)(1+1)=20\\
        \therefore\; \di{255}&=\di{3\cdot 5 \cdot 13}=(1+1)(1+1)(1+1)=8\\
        \therefore\; \di{111}&=\di{3\cdot 37}=4\\
        \therefore\; \di{353650}&=\di{2\cdot 5^2\cdot 11\cdot 643}=(1+1)(2+1)(1+1)(1+1)=24\\
        \therefore\; \sig{353650}&=\sig{2^4\cdot 3^1 \cdot 41^1}=\prod_{i=1}^3 \frac{p_i^{\alpha_i+1}-1}{p_i-1}\\
        &=\left( \frac{3^{1+1}-1}{3-1} \right)\left( \frac{2^{4+1}-1}{2-1} \right)\left( \frac{41^{1+1}-1}{41-1} \right)\\
        &=\frac{8}{2} \cdot 31 \cdot \frac{1680}{40}=5208\\
        \therefore\; \sig{255}&=\sig{3\cdot 5 \cdot 17}=\left( \frac{3^{2}-1}{3-1} \right)\left( \frac{5^{2}-1}{5-1} \right)\left( \frac{17^{2}-1}{17-1} \right)=432\\
        \therefore\; \sig{111}&=\sig{3\cdot 37}=\left( \frac{3^{2}-1}{3-1} \right)\left( \frac{37^{2}-1}{37-1} \right)=152\\
        \therefore\; \sig{353650}&=\sig{2\cdot 5^2\cdot 11\cdot 643}=\left( \frac{2^{3}-1}{2-1} \right)\left( \frac{5^{3}-1}{5-1} \right)\left( \frac{11^2-1}{11-1} \right)\left( \frac{643^2-1}{643-1} \right)=1676976\\
        \therefore\; \mob{25}&=\mob{5^2}=0 \qquad \text{ as } \mob{n}=0 \text{ if } a^2\mid n \text{ where } a>1\\
        \therefore\; \mob{235}&=\mob{5\cdot 47}=(-1)^2=1\\
        \therefore\; \mob{300}&=\mob{2^2\cdot 3\cdot 5^2}=(-1)^3=-1
    \end{align*}
\end{soln}
\begin{qn}[T-3]
    Find the positive integral solution of the linear Diophantine equation $ 20x+7y=30 $.
\end{qn}
\begin{soln}
    Here, $ a=20 $, $ b=7 $, $ c=30 $\\
    Then applying Euclid's algorithm, we get
    \begin{align*}
        20&=7\cdot 2+6\\
        7&=6\cdot 1+1\\
        6&=1\cdot 6+0\\
    \end{align*}
    Hence, $ (20,7)=1 $ and $ 1\mid 30 $ so a solution of $ 20x+7y=30 $ exists.\\
    Using steps of Euclid's algorithm, 1 can be written as a linear combination of 20 and 7.
    \begin{align*}
        1&=7+(-1)\cdot 6\\
        &=7+(-1)\{20+(-2)\cdot 7\}\\
        &=7(3)+20(-1)\\
        \Rightarrow\;20(-30)+7(90)&=30
    \end{align*}
    Hence, $ x_0=-30 $ and $ y_0=90 $ is a particular solution of $ 20x+7y=30 $, and hence the general solution is given by
    \begin{align*}
        & x=x_0+\frac{b}{d}t;\;\;y=y_0-\frac{a}{d}t\qquad\text{ where } t \text{ is an integer}\\
        \text{i.e., }& x=-30+7t,\;y=90-20t
    \end{align*}
    The positive integral solution is given by the system of inequalities
    \begin{align*}
        &-30+7t>0\\
        &90-20t>0\\
        \Rightarrow\;&t>4.28\text{ and } t<4.5
    \end{align*}
    Hence, $ 4.28<t<4.5 \;\Rightarrow\;t=4 $ as $ t $ is an integer or $ t=5 $
    \begin{enumerate}[label=(\roman*)]
        \item $ x=-30+7\cdot 4=-2 $ and $ y=70-20\cdot 4=10 $
        \item $ x=5 $ and $ y=-10 $
    \end{enumerate}
    Hence, there is no positive integral solution of the given linear Diophantine equation.
\end{soln}
\begin{qn}
    Show that $ 3^{2n}-32n^2+24n-1=M(5/2) $
\end{qn}
\begin{qn}
    Solve the congruence $ 7x\m{15}{40} $
\end{qn}
\begin{qn}[100E]
    Solve
    \begin{align*}
        x&\m{7}{30}\\
        x&\m{25}{42}\\
        x&\m{37}{45}
    \end{align*}
\end{qn}
\begin{soln}
    \underline{Alternative method except Chinese Remainder method:}
    \begin{align}
        x&\m{7}{30}\label{eq:1}\\
        x&\m{25}{42}\label{eq:2}\\
        x&\m{37}{45}\label{eq:3}
    \end{align}
    From \eqref{eq:1},
    \begin{equation}
        x=7+30t\label{eq:A}
    \end{equation}
    where $ t $ is integer and putting this in \eqref{eq:2} we get,
    \begin{align*}
        & 7+30t\m{25}{42}\\
        \Rightarrow\; & 30t\m{25-7}{42}\\
        \Rightarrow\; & 30t\m{18}{42}\qquad \left[(30,42)=6,\;\;\therefore\;\left( \frac{30}{6},\frac{42}{6} \right)=1\right]\\
        \Rightarrow\; & 5t\m{3}{7}\\
        \Rightarrow\; & t\m{2}{7}
    \end{align*}
    Now, $ t=2+7u $, $ u $ is any integer and putting in \eqref{eq:A} we get
    \[x=7+30(2+7u)=67+210u\]
    Putting this value in \eqref{eq:3} we get,
    \begin{align*}
        &210u\m{-30}{45}\\
        \Rightarrow\; &14u\m{-2}{3}\\
        \Rightarrow\; &-u\m{-2}{3}\qquad\text{ as }144\m{-u}{3}\\
        \Rightarrow\; &u\m{2}{3}
    \end{align*}
    Now, $ u=2+3v $, where $ v $ is integer
    \begin{align*}
        \therefore\;&x=67+210(2+3v)=487+630v\\
        \Rightarrow\;&x\m{487}{630}
    \end{align*}
\end{soln}
\begin{qn}[100E]
    Solve $ 371x\m{287}{460} $.
\end{qn}
\begin{soln}
    Given,
    \begin{equation}
        371x\m{287}{460} \label{eq:q1.1}
    \end{equation}
    Here, $ 460=4\cdot5\cdot23 $\\
    $ \therefore $ \eqref{eq:q1.1} can be written as 
    \begin{align*}
        371x\m{287}{4}\\
        371x\m{287}{5}\\
        371x\m{287}{23}
    \end{align*}
    \begin{align}
        \text{i.e., } 3x\m{3}{4}\;\Rightarrow\; & x\m{1}{4}\label{eq:q1.2}\\
        & x\m{2}{5}\label{eq:q1.3}\\
        & 3x\m{11}{23}\label{eq:q1.4}
    \end{align}
    From \eqref{eq:q1.2} $ x=1+4t $, $ t $ is an integer and putting this in \eqref{eq:q1.3}
    \begin{align*}
        4t&\m{1}{5}\\
        t&\m{4}{5}
    \end{align*}
    Now, taking $ t=4+5u $, we have $ x=17+20u $ and putting this value in \eqref{eq:q15.4}
    \begin{align*}
        60u&\m{-40}{23}\\
        3u&\m{-2}{23}\\
        u&\m{7}{23}
    \end{align*}
    Putting $ u=7+23v $ we have, $ x=157+460v $.\\
    $ \therefore\; x\m{157}{460} $ is the required solution of \eqref{eq:q1.1}.
\end{soln}
\begin{qn}
    If $ n $ is an integer, then prove that one of $ n $, $ n+2 $, $ n+4 $ is divisible by 3.
\end{qn}
\begin{soln}\hfill
    
    
    Here, $ n $ must be any one of the form $ 3m $, $ 3m+1 $, $ 3m+2 $.\\
    At $ n=3m $, the first number is divisible by 3.\\
    At $ n=3m+1 $, $ n+2=3(m+1) $ is divisible by 3.\\
    At $ n=3m+2 $, $ n+4=3(m+2) $ is divisible by 3.
\end{soln}
\begin{qn}[C.H.88 E]
    Show that $ a^x+a $ and $ a^x-a $ are always even, whatever $ a $ and $ x $ may be.
\end{qn}
\begin{soln}
    If $ a $ is odd, then $ a^x $ is odd, hence $ a^x+a $ and $ a^x-a $ are both even, for all values of $ x $.\\
    If $ a $ is even, then $ a^x $ is even and hence $ a^x+a $ and $ a^x-a $ are both even, for all values of $ x $.\\

    Hence, the problem is shown in proof.
\end{soln}
\begin{qn}[I]
    Show that the sum of the integers less than $ n $ and prime to $ n $ is $ \frac{1}{2}n\ephi{n} $ if $ n\geq2 $.
\end{qn}
\begin{soln}
    Let $ x $ is any integer less than $ n $ and prime to $ n $, then $ n-x $ is also an integer less than $ n $ and prime to it.\footnote{$ \begin{aligned}[t]
        8&= \overbrace{1, 3, 5, 7}^{x}\\
        \therefore\; 8-x&=8-5=3\\
        \therefore\; (8,3)&=1
    \end{aligned} $}

    Denote the integers by $ 1 $, $ p $, $ q $, $ r, \dots $ and their sum by $ S $; then
    \[
        S=1+p+q+r+\dots+(n-p)+(n-q)+(n-r)+(n-1)
    \]
    Which is the series consisting of $ \ephi{n} $ terms.\\
    Rearranging, we have
    \begin{align*}
        S&=(n-1)+(n-p)+(n-q)+(n-r)+\dots+r+q+p+1\\
        \therefore\;2S&=n+n+n+n+\dots\text{ upto }\ephi{n} \text{ terms }=n\ephi{n}\\
        \therefore\;S&=\frac{1}{2}n\ephi{n}
    \end{align*}
\end{soln}
\begin{thm}[E]
    The product of any $ r $ consecutive (integer) number is divisible by $ r! $.
\end{thm}
\begin{proof}
    Let $ n $ be the first number if the $ r $ consecutive integers.\\
    Then
    \begin{align*}
        &\frac{n(n+1)(n+2)\dots(n+r-1)}{r!}\\
        =&\frac{(n+r-1)(n+r-2)\dots(n+2)(n+1)n(n-1)!}{r!(n-1)!}\\
        =&\frac{(n+r-1)!}{r!(n-1)!}\\
        =&\ncr{n+r-1}{r}
    \end{align*}
    Which is the number of combination of $ (n+r-1) $ things taken $ r $ at a time and to an integer. Hence, the theorem is complete.
\end{proof}
\begin{qn}[T-1]
    Show that $ n^5-n $ is divisible by 30.
\end{qn}
\begin{soln}
    As 5 is a prime, $ n^5-n=x(5)= $ multiple of 5.\footnote{x maybe changed to m/M?}\\
    Again, $ n^5-n=n(n^2+1)(n+1)(n-1)=(n-1)n(n+1)(n^2+1) $\\
    Since, $ (n-1)n(n+1) $ is the product of three consecutive integers so it is divisible by $ 3!=6 $.\\
    Therefore, $ (n^2+1) $ is divisible by 5, and hence $ n^5-n $ is divisible by $ 6\times 5=30 $.
\end{soln}
\begin{qn}[I]
    Show that $ n(n+1)(2n+1) $ is divisible by 6.
\end{qn}
\begin{soln}
    In the expression $ n(n+1)(2n+1) $, $ n $ must be of the form $ 6m $, $ 6m+1 $, $ 6m+2 $.\\
    Now,\\
    when $ n=6m $, $ n(n+1)(2n+1)= 6m(6m+1)(12m+1) $ which is divisible by 6.\\
    when $ n=6m+1 $, $ n(n+1)(2n+1)= (6m+1)(6m+2)(12m+2+1)=(6m+1)(m+1)(4m+1)\cdot2\cdot3 $ which is divisible by 6.\\
    when $ n=6m+2 $, $ n(n+1)(2n+1)= (6m+2)(6m+3)(12m+5)=3\cdot2\cdot(3m+1)(2m+1)(12m+5)$ which is divisible by 6.\\

    Thus, $ n(n+1)(2n+1) $ is divisible by 6.
\end{soln}
\end{document}