\documentclass[../main-sheet.tex]{subfiles}
\usepackage{../style}
\graphicspath{ {../img/} }
\backgroundsetup{contents={}}
\begin{document}
\chapter{Numerical Solution of Non-Linear Systems of Equations: Fixed Point Iteration}
\section{Fixed Points For Functions of Several Variables}
A system of nonlinear equations has the form
\begin{equation}
    \begin{rcases}
        f_1(x_1,x_2,\dots, x_n)=0\quad\\
    f_2(x_1,x_2,\dots, x_n)=0\\
    \qquad\quad \qquad\vdots \\
    f_n(x_1,x_2,\dots, x_n)=0    
    \end{rcases}
    \label{eq:fixed1}
\end{equation}
where each function \(f_i\) can be thought of as mapping a vector \(x=(x_1, x_2, \dots, x_n)^t\) of the \(n-\)dimensional space \(\R^n\) into the real line \(\R\).

This system of \(n\) nonlinear equations in \(n\) unknowns can alternatively be represented by defining a function \(F\) mapping \(\R^n\) into \(\R^n\) by
\[
    F(x_1,x_2,\dots,x_n)=\left( f_1(x_1,x_2,\dots,x_n), f_2(x_1,x_2,\dots,x_n),\dots,f_n(x_1,x_2,\dots,x_n) \right)^t
\]
If the vector notation is used to represent the variables \(x_1,x_2,\dots,x_n\), then the system \eqref{eq:fixed1} assumes the form
\[F(x)=0\]
The functions \(f_1,f_2,\dots,f_n\) are called coordinate functions of \(F\).
\begin{defn}
    A function \(\mathbf{G} \) from \(D\subseteq \R^n\) into \(\R^n\) has a fixed point at \(\mathbf{P}\in D\) if \(G(\mathbf{P})=\mathbf{P}\).
\end{defn}
\begin{thm}
    \label{thm:fixedA}
    Let \(f\) be a function from \(D\subset \R\) into \(\R\) and \(\mathbf{x}_0\in D\). Suppose that all the partial derivatives
    of \(f\) exist and constants \(\delta>0\) and \(K>0\) exist so that whenever \(\left\|\mathbf{x}-\mathbf{x}_0\right\|<\delta\) and \(\mathbf{x}\in D\),
    we have
    \[
        \left\vert\frac{\partial f(\mathbf{x})}{\partial x_j}\right\vert\leq K \qquad \text{ for each } j=1,2,\dots,n
        \]
        Then \(f\) is continuous at \(\mathbf{x}_0\).
    \end{thm}
\begin{thm}
    \label{thm:fixedB}
    Let \(D=\set{(x_1,x_2,\dots,x_n)^t:a_i\leq x_i\leq b_i \text{ for each }i=1,2,3,\dots,n}\) for some collection of constants \(a_1,a_2,\dots,a_n\) and \(b_1,b_2,\dots,b_n\). Suppose \(\mathbf{G}\) is a continuous function from \(D\subset \R^n\) into \(\R^n\) with the property that \(\mathbf{G}(\mathbf{x})\in D\) whenever \(\mathbf{x}\in D\). Then \(\mathbf{G}\) has a fixed point in \(D\).
\end{thm}

Suppose, in addition, that \(G\) has continuous partial derivatives and a constant \(k<1\) exists with \(\abs{\frac{\partial g_i(\mathbf{x})}{\partial x_j}}\leq \frac{k}{n}\) whenever \(\mathbf{x}\in D\) for each \(j=1,2,\dots,n\) and each component function \(g_i\). Then the sequence \(\set{x^{(k)}}^\infty_{k=0}\) defined by an arbitrarily selected \(\mathbf{x}^{(0)}\in D\) and generated by
\[
    \mathbf{x}^{(k)}=G\left( \mathbf{x}^{(k-1)} \right)\qquad \text{ for each }k\geq 1
\]
converges to the unique fixed point \(\mathbf{p}\in D\) and 
\begin{equation}
    \left\|\mathbf{x}^{(k)}-\mathbf{p}\right\|_\infty\leq\frac{k^n}{1-k}\left\|\mathbf{x}^{(1)}-\mathbf{x}^{(0)}\right\|_\infty
    \label{eq:fixed2}
\end{equation}
\begin{ex}
    Consider the non linear system
    \begin{align*}
        &3x_1-\cos(x_2x_3)-\frac{1}{2}=0\\
        &{x_1}^2-81(x_2+0.1)^2+\sin x_3+1.06=0\\
        &e^{-x_1x_2}+20x_3+\frac{10\pi -3}{3}=0
    \end{align*}
    if the \(i\)th equation is solved for \(x_i\), the system is changed into the fixed-point problem
    \begin{equation}
        \begin{rcases}
            x_1=\frac{1}{3}\cos(x_2x_3)+\frac{1}{6}\\
            x_2=\frac{1}{9}\sqrt{x_1^2+\sin x_3+1.06}-0.1\qquad\\
            x_3=-\frac{1}{20}e^{-x_1x_2}-\frac{10\pi -3}{60}
        \end{rcases}
        \label{eq:fixed4}
    \end{equation}
    Let \(\mathbf{G}:\R^3\to\R^3\) be defined by \(\mathbf{G}(\mathbf{x})=\left( g_1(\mathbf{x}),g_2(\mathbf{x}),g_3(\mathbf{x}) \right)^t\), where 
    \begin{align*}
        g_1(x_1,x_2,x_3)&=\frac{1}{3}\cos(x_2x_3)+\frac{1}{6}\\
        g_2(x_1,x_2,x_3)&=\frac{1}{9}\left(x_1^2+\sin x_3+1.06\right)^{\frac{1}{2}}-0.1\\
        g_3(x_1,x_2,x_3)&=-\frac{1}{20}e^{-x_1x_2}-\frac{10\pi -3}{60}
    \end{align*}
    Now, we will show that \(\mathbf{G}\) has a unique fixed point in \(D=\set{(x_1,x_2,x_3)^t:-1\leq x_i\leq1;\text{ for each } i=1,2,3}\) by theorems \ref{thm:fixedA} and \ref{thm:fixedB}.\\
    For \(\mathbf{x}=(x_1,x_2,x_3)^t\) in \(D\).
    \begin{align*}
        \abs{g_1(x_1,x_2,x_3)}&\leq \frac{1}{3}\abs{\cos(x_2x_3)}+\frac{1}{6}\leq 0.5\\
        \abs{g_2(x_1,x_2,x_3)}&\leq \abs{\frac{1}{9}\left(x_1^2+\sin x_3+1.06\right)^{\frac{1}{2}}-0.1}\\
        &\leq \frac{1}{9}\sqrt{1+\sin 1+1.06}-0.1\,<\,0.09\\
        \intertext{and}
        \abs{g_3(x_1,x_2,x_3)}&=\frac{1}{20}e^{-x_1x_2}+\frac{10\pi -3}{60}\\
        &\leq \frac{1}{20}e+\frac{10\pi -3}{60}\,<\,0.61
    \end{align*}
    So, \(-1\leq g_i(x_1,x_2,x_3)\leq 1\) for each \(i=1,2,3\). Thus, \(\mathbf{G}(\mathbf{x})\in D\) whenever \(\mathbf{x}\in D\).\\
    To find the bounds for partial derivatives on \(D\), we have
    \[
        \abs{\frac{\partial g_1}{\partial x_1}}=0,\quad\abs{\frac{\partial g_2}{\partial x_2}}=0,\text{ and }\;\;\;\abs{\frac{\partial g_3}{\partial x_3}}=0
    \]
    as well as 
    \begin{align*}
        \abs{\frac{\partial g_1}{\partial x_2}}&\leq \frac{1}{3}\abs{x_3} \abs{\sin x_2x_3}\leq\frac{1}{3}\sin 1< 0.281\\
        \abs{\frac{\partial g_1}{\partial x_3}}&\leq \frac{1}{3}\abs{x_2} \abs{\sin x_2x_3}\leq\frac{1}{3}\sin 1< 0.281\\
        \abs{\frac{\partial g_2}{\partial x_1}}&= \frac{\abs{x_1}}{9\sqrt{{x_1}^2+\sin x_3+1.06}}<\frac{1}{9\sqrt{0.218}}< 0.238\\
        \abs{\frac{\partial g_2}{\partial x_3}}&= \frac{\abs{\cos x_3}}{18\sqrt{{x_1}^2+\sin x_3+1.06}}<\frac{1}{18\sqrt{0.218}}< 0.119\\
        \abs{\frac{\partial g_3}{\partial x_1}}&= \frac{\abs{x_2}}{20}e^{-x_1x_2}\leq \frac{1}{20}e<0.14\\
        \intertext{and}
        \abs{\frac{\partial g_3}{\partial x_2}}&= \frac{\abs{x_1}}{20}e^{-x_1x_2}\leq \frac{1}{20}e<0.14
    \end{align*}
    Since the partial derivatives of \(g_1\), \(g_2\) and \(g_3\) are bounded on \(D\), theorem \ref{thm:fixedA} implies that these functions are continuous on \(D\). Consequently, \(G\) is continuous on \(D\). Moreover, for every \(\mathbf{x}\in D\)
    \[
        \abs{\frac{\partial g_i (\mathbf{x})}{\partial x_j}}\leq 0.281\qquad \text{ for each }i=1,2,3 \text{ and }j=1,2,3
    \]
    and the condition \(\abs{\frac{\partial g_i (\mathbf{x})}{\partial x_j}}\leq \frac{K}{n}\) holds with \(K=3\times 0.281=0.843\).


    To approximate the fixed point \(\mathbf{p}\), we chose \(\mathbf{x}^{(0)}=(0.1,0.1,-0.1)^t\), the sequence of vectors generated by
    \begin{align*}
        x_1^{(k)} &= \frac{1}{3}\cos x_2^{(k-1)}x_3^{(k-1)}+\frac{1}{6}\\
        x_2^{(k)} &= \frac{1}{9}\sqrt{\left(x_1^{(k-1)}\right)^2+\sin x_3^{(k-1)}+1.06}-0.1\\
        x_3^{(k)} &= -\frac{1}{20}e^{-x_1^{(k-1)} x_2^{(k-1)}}-\frac{10\pi -3}{60}
    \end{align*}
    converges to the unique solution of the given non-linear system.

    The results are listed in the table below. The sequences are generated until \(\left\|\mathbf{x}^{(k)}-\mathbf{x}^{(k-1)}\right\|_\infty<10^{-5}\)
    \begin{table}[H]
        \centering
        \begin{tabular}{clllc}
            \toprule
            \(k\) & \multicolumn{1}{c}{\(x_1^{(k)}\)} & \multicolumn{1}{c}{\(x_2^{(k)}\)} & \multicolumn{1}{c}{\(x_3^{(k)}\)} & \(\left\|\mathbf{x}^{(k)}-\mathbf{x}^{(k-1)}\right\|_\infty\)\\\midrule
            0 & 0.10000000 & 0.10000000 & -0.10000000 & \\
            1 & 0.49998333 & 0.00944115 & -0.52310127 & 0.423 \\
            2 & 0.49999593 & 0.00002557 & -0.52336331 & \(9.4\times 10^{-3}\) \\
            3 & 0.50000000 & 0.00001234 & -0.52359814 & \(2.3\times 10^{-4}\) \\
            4 & 0.50000000 & 0.00000003 & -0.52359847 & \(1.2\times 10^{-5}\) \\
            5 & 0.50000000 & 0.00000002 & -0.52359877 & \(3.1\times 10^{-7}\) \\
            \bottomrule
        \end{tabular}
    \end{table}
    using the error bound formula 
    \(
        \left\|\mathbf{x}^{(k)}-\mathbf{p}\right\|_\infty\leq\frac{K^n}{1-K}\left\|\mathbf{x}^{(1)}-\mathbf{x}^{(0)}\right\|_\infty
    \) with \(k=0.843\) gives 
    \[
        \left\|\mathbf{x}^{(5)}-\mathbf{p}\right\|_\infty\leq\frac{(0.843)^n}{1-0.843}(0.423)<1.15
    \]
    which does not indicate the true accuracy of \(\mathbf{x}^{(5)}\) because of the inaccurate initial approximation. The actual solution is
    \[
        \mathbf{p}=\left( 0.5,0,\frac{-\pi}{6} \right)^t\approx \left( 0.5,0,-0.5235987757\right)^t
    \]
    So the true error is \(\left\|\mathbf{x}^{(5)}-\mathbf{p}\right\|_\infty\leq2\times 10^{-8}\).\\


    To accelerate the convergence of the fixed-point iteration, we can use the Gauss-Seidel method, we have
    \begin{align*}
        x_1^{(k)} &= \frac{1}{3}\cos x_2^{(k-1)}x_3^{(k-1)}+\frac{1}{6}\\
        x_2^{(k)} &= \frac{1}{9}\left(\left(x_1^{(k)}\right)^2+\sin x_3^{(k-1)}+1.06\right)^{\frac{1}{2}}-0.1\\
        x_3^{(k)} &= -\frac{1}{20}e^{-x_1^{(k)} x_2^{(k)}}-\frac{10\pi -3}{60}
    \end{align*}
    The results are represented in the table below, with the approximation \(\mathbf{x}^{(0)}=(0.1,0.1,-0.1)^t\).
    \begin{table}[H]
        \centering
        \begin{tabular}{clllc}
            \toprule
            \(k\) & \multicolumn{1}{c}{\(x_1^{(k)}\)} & \multicolumn{1}{c}{\(x_2^{(k)}\)} & \multicolumn{1}{c}{\(x_3^{(k)}\)} & \(\left\|\mathbf{x}^{(k)}-\mathbf{x}^{(k-1)}\right\|_\infty\)\\\midrule
            0 & 0.10000000 & 0.10000000 & -0.10000000 & \\
            1 & 0.49998333 & 0.02222979 & -0.52304613 & 0.423 \\
            2 & 0.49997747 & 0.00002815 & -0.52359807 & \(2.2\times 10^{-2}\) \\
            3 & 0.50000000 & 0.00000004 & -0.52359877 & \(2.8\times 10^{-5}\) \\
            4 & 0.50000000 & 0.00000000 & -0.52359877 & \(1.2\times 10^{-8}\) \\
            \bottomrule
        \end{tabular}
    \end{table}
    \emph{Comment}: The iterate \(\mathbf{x}^{(4)}\) is accurate within \(10^{-7}\) in the \(l_\infty\) norm; so the convergence is accelerated with Gauss-Seidel method.
\end{ex}
\end{document}