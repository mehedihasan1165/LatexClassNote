\documentclass[../main-sheet.tex]{subfiles}
\usepackage{../style}
\graphicspath{ {../img/} }
\backgroundsetup{contents={}}
\begin{document}
\chapter{System of IVPs}
\section{Runge-Kutta Order 4}
\begin{prob}
    Consider the 2nd order IVP
    \[
        y''-2y'+2y=e^{2t}\sin t;\qquad 0\leq t\leq 1
    \]
    with \(y(0)=-0.4\), \(y'(0)=-0.6\).
\end{prob}
\begin{soln}
    Let \(\begin{aligned}[t]
        u_1(t)&=y(t)\\
        \text{and } y'(t)&=u_1^{'}(t)=u_2(t)\\
        \therefore\;y^{''}(t)&=u_1^{''}(t)=u_2^{'}(t)
    \end{aligned}
    \)\\
    So, \(\displaystyle u_1^{'}(t)=\ddt{u_1}=u_2(t)\)
    \begin{align*}
        &u_2^{'}(t)-2u_2(t)+2u_1(t)=e^{2t}\sin t\\
        \Rightarrow\;\;&u_2^{'}(t)=2u_2(t)-2u_1(t)+e^{2t}\sin t
    \end{align*}
    So,
    \begin{align*}
        &f_1=u_1^{'}=\ddt{u_1}=u_2\\
        \Rightarrow\;\;&u_1^{'}=f_1(t,u_1,u_2)\\
        &u_2^{'}=\ddt{u_2}=e^{2t}\sin t-2u_1+2u_2\\
        \Rightarrow\;\;&u_2^{'}=f_2(t,u_1,u_2)
    \end{align*} 
    with \(\begin{aligned}[t]
        &u_1(0)=-0.4
        \Rightarrow\;\;&u_1^0=-0.4
    \end{aligned}\) and \(\begin{aligned}[t]
        &u_2(0)=-0.6
        \Rightarrow\;\;&u_2^0=-0.6
    \end{aligned}\)\\
    Using 4th order Runge-Kutta with \(h=0.1\),
    \begin{align*}
        k_1&=hf_1(t_0,\,u_1^0,\,u_2^0)\\
        &=0.1f_1(0,\,-0.4,\,-0.6)\\
        &=0.1\times [(-0.6)]\\
        &=-0.06\\
        l_1&=hf_2(t_0,\,u_1^0,\,u_2^0)\\
        &=0.1f_1(0,\,-0.4,\,-0.6)\\
        &=0.1\times [e^{2\times 0}\sin \,0-2\times(-0.4)+2\times(-0.6)]\\
        &=-0.04
    \end{align*}
    \begin{align*}
        k_2&=hf_1\left(t_0+\frac{h}{2},\,u_1^0+\frac{k_1}{2},\,u_2^0+\frac{l_1}{2}\right)\\
        &=hf_1\left(0+\frac{0.1}{2},\,-0.4+\frac{-0.06}{2},\,-0.6+\frac{-0.04}{2}\right)\\
        &=hf_1\left(\frac{0.1}{2},\,-0.4-\frac{0.06}{2},\,-0.6-\frac{0.04}{2}\right)\\
        &=h\left(-0.6-\frac{0.04}{2}\right)\\
        &=-0.062\\
        l_2&=hf_2\left(t_0+\frac{h}{2},\,u_1^0+\frac{k_1}{2},\,u_2^0+\frac{l_1}{2}\right)\\
        &=hf_2\left(0.05,\,-0.4+\frac{-0.06}{2},\,-0.6+\frac{-0.04}{2}\right)\\
        &=hf_2\left(0.05,\,-0.43,\,-0.62\right)\\
        &=0.1\times\left[e^{2\times 0.05}\sin \,(0.05)-2\times(-0.43)+2\times(-0.62)\right]\\
        &=-0.032476
    \end{align*}
    \begin{align*}
        k_3&=hf_1\left(t_0+\frac{h}{2},\,u_1^0+\frac{k_2}{2},\,u_2^0+\frac{l_2}{2}\right)\\
        &=hf_1\left(0.05,\,-0.4+\frac{-0.062}{2},\,-0.6+\frac{-0.032476}{2}\right)\\
        &=h\left(-0.6-\frac{0.032476}{2}\right)\\
        &=-0.06162381\\
        l_3&=hf_2\left(t_0+\frac{h}{2},\,u_1^0+\frac{k_2}{2},\,u_2^0+\frac{l_2}{2}\right)\\
        &=hf_2\left(0.05,\,-0.4+\frac{-0.062}{2},\,-0.6+\frac{-0.032476}{2}\right)\\
        &=hf_2\left(0.05,\,-0.431,\,-0.616238\right)\\
        &=0.1\times\left[e^{2\times 0.05}\sin \,(0.05)-2\times(-0.431)+2\times(-0.616238)\right]\\
        &=-0.03152409
    \end{align*}
    \begin{align*}
        k_4&=hf_1\left(t_0+h,\,u_1^0+k_3,\,u_2^0+l_3\right)\\
        &=hf_1\left(0+0.1,\,-0.4-0.06162381,\,-0.6-0.03152409\right)\\
        &=h\left(-0.6-0.03152409\right)\\
        &=-0.063152409\\
        l_4&=hf_2\left(t_0+h,\,u_1^0+k_3,\,u_2^0+l_3\right)\\
        &=hf_2\left(0.1,\,-0.46162381,\,-0.63152409\right)\\
        &=0.1\times\left[e^{2\times 0.1}\sin \,(0.1)-2\times(-0.46162381)+2\times(-0.63152409)\right]\\
        &=-0.02178637
    \end{align*}
    So,
    \begin{align*}
        u_1^1&=u_1^0+\frac{1}{6}(k_1+2k_2+2k_3+k_4)=-0.4617333\\
        u_2^1&=u_2^0+\frac{1}{6}(l_1+2l_2+2l_3+l_4)=-0.63163124
    \end{align*}
    Hence, \[u_1^1=y(0.1)=-0.4617333\]
    and \[u_2^1=y'(0.1)=-0.63163124\]
    Actual Solution: \(y(t)=0.2e^{2t}(\sin \,t-2\cos\,t)\)
\end{soln}
\# \(\displaystyle \ddt{y}=f(t,y); \qquad y(t_0)=y_0\)\\
RK4:
\(k_1=hf(t_i,y_i),\qquad k_2=hf\left( t_i+\frac{h}{2},\,y_i+\frac{k_1}{2} \right),\qquad k_3=hf\left( t_i+\frac{h}{2},\,y_i+\frac{k_2}{2} \right),\qquad k_4=hf(t_{i+1},y_i+k_3)\)
\[
    \therefore\;\;y_{i+1}=y_i+\frac{1}{6}(k_1+2k_2+2k_3+k_4)
\]
\# \begin{align*}
    \ddt{u_1}&=f(t,u_1,u_2,\dots,u_m)\\
    \ddt{u_2}&=f(t,u_1,u_2,\dots,u_m)\\
    \vdots&\\
    \ddt{u_m}&=f(t,u_1,u_2,\dots,u_m)
\end{align*}
for \(a\leq t\leq b\) with initial condition [Here \(a=t_0\)].
\begin{align*}
    &u_1(a)=\alpha_1,\quad u_2(a)=\alpha_2,\quad \dots,\quad u_m(a)=\alpha_m\\ 
    \Rightarrow\;\;&u_1^0=\alpha_1,\quad u_2^0=\alpha_2,\quad \dots,\quad u_m^0=\alpha_m {\footnotemark}
\end{align*}
\footnotetext{{\(\begin{aligned}[t]
    &u_1(a)=\alpha_1\\
    \Rightarrow\;\;&u_1(t_0)=\alpha_1\\
    \Rightarrow\;\;&u_1^0=\alpha_1
\end{aligned}\)}}
Consider, \(u_1^j,\,u_2^j,\dots,u_m^j\) have been computed. We obtain, \(u_1^{j+1},\,u_2^{j+1},\dots,u_m^{j+1}\) by first calculating
\begin{align*}
    k_{1,i}&=hf_i(t_i,u_1^j,u_2^j,\dots,u_m^j) &\text{ for each } i=1,2,\dots,m\\
    k_{2,i}&=hf_i\left(t_i+\frac{h}{2},\,u_1^j+\frac{k_{1,1}}{2},\,u_2^j+\frac{k_{1,2}}{2},\dots,u_m^j+\frac{k_{1,m}}{2}\right) &\text{ for each } i=1,2,\dots,m\\
    k_{3,i}&=hf_i\left(t_i+\frac{h}{2},\,u_1^j+\frac{k_{2,1}}{2},\,u_2^j+\frac{k_{2,2}}{2},\dots,u_m^j+\frac{k_{2,m}}{2}\right) &\text{ for each } i=1,2,\dots,m\\
    k_{4,i}&=hf_i\left(t_i+h,\,u_1^j+k_{3,1},\,u_2^j+k_{3,2},\dots,u_m^j+k_{3,m}\right) &\text{ for each } i=1,2,\dots,m
\end{align*}
and then
\[
    u_i^{j+1}=u_i^j+\frac{1}{6}\left( k_{1,i}+2k_{2,i}+2k_{3,i} +k_{4,i}\right)\qquad \text{ for each }i=1,\,2,\,\dots,m \text{ and } j=0,\,1,\,2,\,\dots
\]
\begin{note}
    \(k_{1,1},\,k_{1,2},\,k_{1,3},\dots,k_{1,m}\) must be computed before any of the terms of the form \(k_{2,1},\,k_{2,2},\,k_{2,3},\dots,k_{2,m}\) i.e., \(k_{2,i}\). 
    In general, each \(k_{l,1},\,k_{l,2},\,k_{l,3},\dots,k_{l,m}\) must be computed before any of the expressions \(k_{l+1,i}\).
\end{note}
\end{document}