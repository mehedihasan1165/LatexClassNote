\documentclass[../main-sheet.tex]{subfiles}
\usepackage{../style}
\graphicspath{ {../img/} }
\backgroundsetup{contents={}}
\begin{document}
\chapter{Newton's Method For Non-Linear Systems of Equations}
\section{Newton's Method}
\[
    \mathbf{x}^{(k)}=\mathbf{G}\left( \mathbf{x}^{(k-1)} \right)=\mathbf{x}^{(k-1)}-J\left( \mathbf{x}^{(k-1)} \right)^{-1}\mathbf{F}\left( \mathbf{x}^{(k-1)} \right)
\]
where
\[
    J(\mathbf{x}^{(k)})=\begin{bmatrix}
        \frac{\partial f_1(\mathbf{x})}{\partial x_1} & \frac{\partial f_1(\mathbf{x})}{\partial x_2} &\dots&\frac{\partial f_1(\mathbf{x})}{\partial x_n} \\[3pt]
        \frac{\partial f_2(\mathbf{x})}{\partial x_1} & \frac{\partial f_2(\mathbf{x})}{\partial x_2} &\dots&\frac{\partial f_2(\mathbf{x})}{\partial x_n} \\[3pt]
        \vdots & \vdots & \ddots & \vdots\\
        \frac{\partial f_n(\mathbf{x})}{\partial x_1} & \frac{\partial f_n(\mathbf{x})}{\partial x_2} &\dots&\frac{\partial f_n(\mathbf{x})}{\partial x_n} \\
    \end{bmatrix}
\]
\[
    \mathbf{G}\left( \mathbf{x} \right)=\mathbf{x}-J\left( \mathbf{x}\right)^{-1}\mathbf{F}\left( \mathbf{x} \right)
\]
\begin{ex}
    \begin{align*}
        &3x_1-\cos(x_2x_3)-\frac{1}{2}=0\\
        &{x_1}^2-81(x_2+0.1)^2+\sin x_3+1.06=0\\
        &e^{-x_1x_2}+20x_3+\frac{10\pi -3}{3}=0
    \end{align*}
    So,
    \[
        J(x_1,x_2,x_3)=\begin{bmatrix}
            3 & x_3 \sin x_2x_3 & x_2 \sin x_2 x_3\\
            2x_1 & -162(x_2+0.1) & \cos x_3\\
            -x_2e^{-x_1x_2} & -x_1e^{-x_1x_2} & 20
        \end{bmatrix}
        \quad\text{ and }\quad \begin{bmatrix}
            x_1^{(k)}\\
            x_2^{(k)}\\
            x_3^{(k)}
        \end{bmatrix}=\begin{bmatrix}
            x_1^{(k-1)}\\
            x_2^{(k-1)}\\
            x_3^{(k-1)}
        \end{bmatrix}+\begin{bmatrix}
            y_1^{(k-1)}\\
            y_2^{(k-1)}\\
            y_3^{(k-1)}
        \end{bmatrix}
    \]
    where 
    \[
        \begin{bmatrix}
            y_1^{(k-1)}\\
            y_2^{(k-1)}\\
            y_3^{(k-1)}
        \end{bmatrix}=-\left( J\left( x_1^{(k-1)},x_2^{(k-1)},x_3^{(k-1)} \right) \right)^{-1}\mathbf{F}\left( x_1^{(k-1)},x_2^{(k-1)},x_3^{(k-1)} \right)
    \]
    Thus at the \(k\)th step, the linear system \(J\left( \mathbf{x}^{(k-1)} \right)\mathbf{y}^{(k-1)}=-\mathbf{F}\left( \mathbf{x}^{(k-1)} \right)\) must be solved, where,
    \[
        J(\mathbf{x}^{(k-1)})=\begin{bmatrix}
            3 & x_3^{(k-1)} \sin x_2^{(k-1)}x_3^{(k-1)} & x_2^{(k-1)} \sin x_2^{(k-1)} x_3^{(k-1)}\\
            2x_1^{(k-1)} & -162(x_2^{(k-1)}+0.1) & \cos x_3^{(k-1)}\\
            -x_2^{(k-1)}e^{-x_1^{(k-1)}x_2^{(k-1)}} & -x_1^{(k-1)}e^{-x_1^{(k-1)}x_2^{(k-1)}} & 20
        \end{bmatrix},\qquad
        \mathbf{y}^{(k-1)}=\begin{bmatrix}
            y_1^{(k-1)}\\
            y_2^{(k-1)}\\
            y_3^{(k-1)}
        \end{bmatrix}\]
        and\[
        \mathbf{F}\left( x^{(k-1)} \right)=\begin{bmatrix}
            3x_1^{(k-1)}-\cos\left(x_2^{(k-1)}x_3^{(k-1)}\right)-\frac{1}{2}\\
        \left(x_1^{(k-1)}\right)^2-81\left(x_2^{(k-1)}+0.1\right)^2+\sin x_3^{(k-1)}+1.06\\
        e^{-x_1^{(k-1)}x_2^{(k-1)}}+20x_3^{(k-1)}+\frac{10\pi -3}{3}
        \end{bmatrix}
    \]
    Let the initial approximation be \(\mathbf{x}^{(0)}=(0.1,0.1,-0.1)^t\). The results using this iterative procedure are shown in the table below.
    \begin{table}[H]
        \centering
        \begin{tabular}{clllc}
            \toprule
            \(k\) & \multicolumn{1}{c}{\(x_1^{(k)}\)} & \multicolumn{1}{c}{\(x_2^{(k)}\)} & \multicolumn{1}{c}{\(x_3^{(k)}\)} & \(\left\|\mathbf{x}^{(k)}-\mathbf{x}^{(k-1)}\right\|_\infty\)\\\midrule
            0 & 0.10000000 & 0.10000000 & -0.10000000 & \\
            1 & 0.50003702 & 0.01946686 & -0.52152047 & 0.422 \\
            2 & 0.50004593 & 0.00158859 & -0.52355711 & \(1.79\times 10^{-2}\) \\
            3 & 0.50000034 & 0.00001244 & -0.52359845 & \(1.58\times 10^{-3}\) \\
            4 & 0.50000000 & 0.00000000 & -0.52359877 & \(1.24\times 10^{-5}\) \\
            5 & 0.50000000 & 0.00000000 & -0.52359877 & 0 \\
            \bottomrule
        \end{tabular}
    \end{table}
\end{ex}
Newton's method can converges very rapidly with comparison of fixed point iteration.
\begin{prob}
    Use Newton's and Quasi-Newton's method to find a solution to the following non-linear systems with the given initial approximation. Iterate until \(\left\|\mathbf{x}^{(k)}-\mathbf{x}^{(k-1)}\right\|_\infty<10^{-6}\)
    \begin{enumerate}
        \item \begin{align*}
            &3x_1^2-x_2^2=0\\
            &3x_1x_2^2-x_1^3-1=0\\
            &\mathbf{x}^{(0)}=(1,1)^t
        \end{align*}
        \item \begin{align*}
            &\log \left( x_1^2+x_2^2 \right)-\sin(x_1x_2)=\log 2+\log \pi\\
            &e^{x_1-x_2}+\cos (x_1x_2)=0\\
            &\mathbf{x}^{(0)}=(2,2)^t
        \end{align*}
        \item \begin{align*}
            &x_1^2+x_2-37=0\\
            &x_1-x_2^2-5=0\\
            &x_1+x_2+x_3-3=0\\
            &\mathbf{x}^{(0)}=(0,0,0)^t
        \end{align*}
        \item \begin{align*}
            &x^2-y^2=4\\
            &x^2+y^2=16\\
            &\mathbf{x}^{(0)}=(2\sqrt{2},2\sqrt{2})^t
        \end{align*}
    \end{enumerate}
    See book of Sastry for problem 4.
\end{prob}
\end{document}