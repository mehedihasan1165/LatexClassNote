\documentclass[../main-sheet.tex]{subfiles}
\usepackage{../style}
\graphicspath{ {../img/} }
\backgroundsetup{contents={}}
\begin{document}
\begin{defn}
    The IVP 
    \begin{equation}
        \ddt{y}=f(t,y),\quad a\leq t\leq b,\quad y(a)=\alpha
        \label{eq:ini1}
    \end{equation}
    is said to be a well-posed problem if:
    \begin{enumerate}
        \item A unique solution, \(y(t)\), to the problem exists;
        \item For any \(\varepsilon>0\), there exists a positive constant \(k(\varepsilon)\) with the property that, whenever \(\abs{\varepsilon_0}<\varepsilon\) and \(\delta(t)\) is continuous with \(\abs{\delta(t)}<\varepsilon\) on \([a,b]\), a unique solution \(z(t)\), to the problem
        \begin{equation}
            \ddt{z}=f(t,z)+\delta(t),\quad a\leq t<b, \quad z(a)=\alpha+\varepsilon_0 \label{eq:ini2}
        \end{equation}
        exists with \(\abs{z(t)-y(t)}< k(\varepsilon)\) for all \(a\leq t\leq b\).
    \end{enumerate}
\end{defn}

The problem specified by \eqref{eq:ini2} is called a \emph{perturbed} problem associated with the original problem \eqref{eq:ini1}.
\begin{thm}
    \label{thm:ivp2}
    Suppose \(D=\set{(t,y):a\leq t\leq b,\;-\infty<y<\infty}\).
    If \(f\) is continuous and satisfies a Lipschitz condition in the variable \(y\) on the set \(D\), then the IVP
    \[
        \ddt{y}=f(t,y),\quad a\leq t\leq b,\quad y(a)=\alpha
        \]
        is well-posed.
    \end{thm}
    \begin{ex}
        Let \(D=\set{(t,y):0\leq t\leq 1,\;-\infty<y<\infty}\) and consider the IVP 
        \[
            \ddt{y}=y-t^2+1,\quad 0\leq t\leq 2,\quad y(0)=0.5
            \]
            Since
            \[\abs{\frac{\partial (y-t^2+1)}{\partial y}}=\abs{1}=1\]
            Theorem \ref{thm:ivp1} implies that \(f(t,y)=y-t^2+1\) satisfies a Lipschitz condition on \(D\) with Lipschitz constant 1. Since \(f\) is continuous on \(D\), theorem \ref{thm:ivp2} implies that the problem is well-posed.
        \end{ex}
        \section{Euler's Method}
        
        The objective of this method is to obtain an approximation to the well-posed initial value problem
        \begin{equation}
            \ddt{y}=f(t,y),\quad a\leq t\leq b,\quad y(a)=\alpha
            \label{eq:euler1}
        \end{equation}
        \begin{note}
            A continuous approximation to the solution \(y(t)\) will not be obtained, but approximation to \(y(t)\) will be generated at various values, called \emph{mesh/grid} points, in the interval \([a,b]\).
            Once the approximate solution is obtained at the points, the approximate solution at other points in the interval can be obtained by interpolation.
        \end{note}

        We first make the stipulation that the mesh points are equally distributed throughout the interval \([a,b]\).
        This condition is ensured by choosing a positive integer \(N\) and selecting the mesh points \(\set{t_0,t_1,t_2,\dots,t_N}\), where \(t_i=a+ih\) for each \(i=0,1,2,\dots,N\).
        The step size is \(h=(b-a)/N\).

        Suppose that \(y(t)\), the unique solution to \eqref{eq:euler1} has two continuous derivatives on \([a,b]\), so that for each \(i=0,1,2,\dots,N-1\), we have by Taylor's theorem,
        \[
            y(t_{i+1})=y(t_i)+(t_{i+1}-t_i) y^{'}(t_i)+\frac{(t_{i+1}-t_i)^2}{2!}y^{''}(\xi_i)
        \]
        for some number \(\xi_i\) in \((t_{i+1},t_i)\).
        If \(h=t_{i+1}-t_i\), then,
        \[
            y(t_{i+1})=y(t_i)+h y^{'}(t_i)+\frac{h^2}{2!}y^{''}(\xi_i)
        \]
        and since \(y(t)\) satisfies the differential equation \eqref{eq:euler1},
        \begin{equation}
            y(t_{i+1})=y(t_i)+h f(t_i,y(t_i))+\frac{h^2}{2!}y^{''}(\xi_i)
            \label{eq:euler2}
        \end{equation}
        Euler's method constructs \(w_i\approx y(t_i)\) for each \(i=1,2,\dots,N\) by deleting the remainder term.\\
        Thus,
        \begin{equation}
            \begin{rcases}
                w_0=\alpha\\
                w_{i+1}=w_i+hf(t_i,w_i)\quad
            \end{rcases}
            \label{eq:euler3}
        \end{equation}
        Equation \eqref{eq:euler3} is called the difference equation associated with Euler's method.
        \begin{note}
            Euler's method can be obtained by integration method also. For reference read Sastry.
        \end{note}
        \underline{Alternatively}:
        \[
            \ddt{y}=f(t,y),\quad a\leq t\leq b,\quad y(a)=\alpha
        \]
        In numerical methods, we determine a number \(y_i\) which is an approximation to the value of the solution \(y(t)\) at the point \(t_i\).
        The set of numbers \(\set{y_i}\) i.e., \(y_0\), \(y_1\), \(y_2,\dots,y_N\) is the numerical solution of the IVP.
        The numbers \(\set{y_i}\) are determined from a set of algebraic equations called the difference equations.
        There are many difference approximations possible for a given differential equation.
        Let us develop expressions for the first derivative in terms of the forward, backward and central difference operators.
        We assume that the function \(y(t)\) may be expanded in a Taylor series in the closed interval \(t-h\leq t\leq t+h\) [where \(t_i=t_0+ih\); \(i=0,1,2,\dots,N\)].
        We have,
        \begin{equation}
            y(t\pm h)=y(t)\pm \frac{h}{1!}y^{'}(t)\pm \frac{h^2}{2!}y^{''}(t)\pm \dots\pm (-1)^p \frac{h^p}{p!}y^{(p)}(t)\pm \dots \label{eq:euler4}
        \end{equation}
        We then have
        \begin{align}
            &\frac{y(t+h)-y(t)}{h}=y'(t)+\frac{h}{2}y''(t)+O(h^2)\notag\\
            \Rightarrow\;&\frac{\Delta y(t)}{h}=\ddt{y}+O(h)\label{eq:euler5}
        \end{align}
        where the notation \(O(h)\) means that the first term neglected is of order \(h\).\\
        Similarly, we obtain,
        \begin{align}
            \frac{\nabla y(t)}{h}&=\ddt{y}+O(h)\label{eq:euler6}\\
            \intertext{and}
            \frac{\delta y(t)}{h}&=\ddt{y}+O(h^2)\label{eq:euler7}
        \end{align}
        A difference approximation to \(y'(t)\) at \(t=t_i\) is obtained by neglecting the error term.\\
        We have,
        \begin{subequations}
            \begin{align}
            y^{'}(t_i) & \approx \frac{y_{i+1}-y_i}{h} \label{eq:euler8a}\\
            y^{'}(t_i) & \approx \frac{y_{i}-y_{i-1}}{h} \label{eq:euler8b}\\
            y^{'}(t_i) & \approx \frac{y_{i+1}-y_{i-1}}{2h} \label{eq:euler8c}
            \end{align}
            \label{eq:euler8}
        \end{subequations}
        We use the approximations \eqref{eq:euler8} for \(y'(t)\) in the given differential equation at the mesh point \(t_i\).
        This gives
        \begin{subequations}
            \begin{align}
            \frac{y_{i+1}-y_i}{h}&=f(t_i,y_i) \label{eq:euler9a}\\
            \frac{y_{i}-y_{i-1}}{h}&=f(t_i,y_i) \label{eq:euler9b}\\
            \frac{y_{i+1}-y_{i-1}}{2h}&=f(t_i,y_i) \label{eq:euler9c}
            \end{align}
            \label{eq:euler9}
        \end{subequations}
        \begin{note}
            The equations \eqref{eq:euler9} are called difference equation because they are considered as a relation between differences of an unknown function \(y_i\).
        \end{note}
\end{document}