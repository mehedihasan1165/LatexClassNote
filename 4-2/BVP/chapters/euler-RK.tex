\documentclass[../main-sheet.tex]{subfiles}
\usepackage{../style}
\graphicspath{ {../img/} }
\backgroundsetup{contents={}}
\begin{document}
\section{Mid-Point Method}
\begin{align*}
    w_0&=\alpha\\
    w_{i+1}&=w_i+h f\left( t_i+\frac{h}{2},\,w_i+\frac{h}{2}f(t_i,w_i) \right) \quad \text{for each } i=0,1,2,\dots,N-1
\end{align*}
\begin{center}
    \begin{minipage}[c]{.4\textwidth}
        \begin{align*}
            w_0&=\alpha\\
            k_1&=f(t_i,\,w_i)\\
            k_2&=f\left(t_i+\frac{h}{2},\,w_i+\frac{h}{2}k_1\right)\\
            w_{i+1}&=w_i+hk_2
        \end{align*}
    \end{minipage}
    \begin{minipage}[c]{.4\textwidth}
        \centering
        \begin{tabular}{c|cc}
            0 & & \\[.5 em]
            \(\frac{1}{2}\) & \(\frac{1}{2}\) & \\[.5 em]
            \hline
            \Tstrut & 0 & 1 \\
        \end{tabular}
    \end{minipage}
\end{center}
\section{Modified Euler Method}
\begin{align*}
    w_0&=\alpha\\
    w_{i+1}&=w_i+\frac{h}{2} \left[ f(t_i,\,w_i)+f(t_{i+1},\,w_i+hf(t_i,w_i)) \right] \quad \text{for each } i=0,1,2,\dots,N-1
\end{align*}
\begin{center}
    \begin{minipage}[c]{.4\textwidth}
        \begin{align*}
            w_0&=\alpha\\
            k_1&=f(t_i,\,w_i)\\
            k_2&=f\left(t_i+h,\,w_i+hk_1\right)\\
            w_{i+1}&=w_i+\frac{h}{2}\left( k_1+k_2 \right)
        \end{align*}
    \end{minipage}
    \begin{minipage}[c]{.4\textwidth}
        \centering
        \begin{tabular}{c|cc}
            0 & & \\[.5 em]
            \(1\) & \(1\) & \\
            \hline
            \Tstrut & \(\frac{1}{2}\) & \(\frac{1}{2}\) \\
        \end{tabular}
    \end{minipage}
\end{center}
\section{Heun's Method}
\begin{align*}
    w_0 & = \alpha\\
    w_{i+1} & = w_i + \frac{h}{4} \left[ f(t_i,\,w_i) + 3\left(f\left(t_{i} + \frac{2h}{3},\,w_i+\frac{2h}{3}f\left(t_i+\frac{h}{3},\,w_i+\frac{h}{3}f(t_i,\,w_i)\right)\right)\right) \right] \quad \text{for each } i=0,1,2,\dots,N-1
\end{align*}
\begin{center}
    \begin{minipage}[c]{.4\textwidth}
        \begin{align*}
            w_0&=\alpha\\
            k_1&=f(t_i,\,w_i)\\
            k_2&=f\left(t_i+\frac{2h}{3},\,w_i+\frac{2h}{3}k_1\right)\\
            w_{i+1}&=w_i+\frac{h}{4}\left( k_1+3k_2 \right)
        \end{align*}
    \end{minipage}
    \begin{minipage}[c]{.4\textwidth}
        \centering
        \begin{tabular}{c|cc}
            0 & & \\[.5 em]
            \(\frac{2}{3}\) & \(\frac{2}{3}\) & \\[0.5em]
            \hline
            \Tstrut & \(\frac{1}{4}\) & \(\frac{3}{4}\) \\
        \end{tabular}
    \end{minipage}
\end{center}
\begin{rem}
    Heun's method is also known as \emph{Ralston's} Method.\\
    See appendix \ref{butch} for using Butcher table.
\end{rem}
\begin{ex}
    IVP:
    \[
        y'=y-t^2+1,\qquad 0\leq t\leq 2,\;\;y(0)=0.5
    \]
    with \(N=10\), \(h=0.2\), \(t_i=0.2i\), \(w_0=0.5\).\\
    The differential equations are
    \begin{align*}
        \text{Midpoint }&: w_{i+1}=1.22w_i-0.0088i^2-0.008i+0.218\\
        \text{Modified Euler }&: w_{i+1}=1.22w_i-0.0088i^2-0.008i+0.216\\
        \text{Heun's }&: w_{i+1}=1.22w_i-0.0088i^2-0.008i+0.217\dot{3}
    \end{align*}
    for each \(i=0,1,2,\dots,9\) the following table lists the results of these calculations.
    \begin{table}[H]
        \centering
        \begin{tabular}{cccc@{}c@{}ccc}
            \toprule
            \(t_i\) & \(y(t_i)\) & Midpoint Method & Error & {Modified Euler Method} & Error& Heun's Method & Error\\\midrule
            0.0 & 0.5000000 & 0.5000000 & 0 & 0.5000000 & 0 & 0.5000000 & 0\\
            0.2 & 0.8292986 & 0.8280000 & 0.0012986 & 0.8260000 & 0.0032986 & 0.8273333 & 0.0019653\\
            0.4 & 1.2140877 & 1.2113600 & 0.0027277 & 1.2069200 & 0.0071677 & 1.2098800 & 0.0042077\\
            0.6 & 1.6489406 & 1.6446592 & 0.0042814 & 1.6372424 & 0.0116982 & 1.6421869 & 0.0067537\\
            0.8 & 2.1272295 & 2.1212842 & 0.0059453 & 2.1102357 & 0.0169938 & 2.1176014 & 0.0096281\\
            1.0 & 2.6408591 & 2.6331668 & 0.0076923 & 2.6176876 & 0.0231715 & 2.6280070 & 0.0128521\\
            1.2 & 3.1799415 & 3.1704634 & 0.0094781 & 3.1495789 & 0.0303627 & 3.1635019 & 0.0164396\\
            1.4 & 3.7324000 & 3.7211654 & 0.0112346 & 3.6936862 & 0.0387138 & 3.7120057 & 0.0203944\\
            1.6 & 4.2834838 & 4.2706218 & 0.0128620 & 4.2350972 & 0.0483866 & 4.2587802 & 0.0247035\\
            1.8 & 4.8151763 & 4.8009586 & 0.0142177 & 4.7556185 & 0.0595577 & 4.7858452 & 0.0293310\\
            2.0 & 5.3054720 & 5.2903695 & 0.0151025 & 5.2330546 & 0.0724173 & 5.2712645 & 0.0342074\\
            \bottomrule
        \end{tabular}
    \end{table}
    \begin{note}
        Details calculations of midpoint method are included in appendix \ref{app}.
    \end{note}
\end{ex}
\section{Runge-Kutta Method of Order Four}
\begin{align*}
    w_0&=\alpha\\
    k_1&=hf(t_i,\,w_i)\\
    k_2&=hf\left(t_i+\frac{h}{2},\,w_i+\frac{1}{2}k_1\right)\\
    k_3&=hf\left(t_i+\frac{h}{2},\,w_i+\frac{1}{2}k_2\right)\\
    k_4&=hf\left(t_{i+1},\,w_i+k_3\right)\\
    w_{i+1}&=w_i+\frac{1}{6}(k_1+2k_2+2k_3+k_4) \quad \text{for each } i=0,1,2,\dots,N-1
\end{align*}
\begin{note}
    This method has local truncation error \(O(h^4)\), provided the solution \(y(t)\) has five continuous derivatives.
\end{note}
\begin{ex}
    Solve the IVP:
    \[
        y'=y-t^2+1,\qquad 0\leq t\leq 2,\;\;y(0)=0.5
    \]
    with \(N=10\), \(h=0.2\), \(t_i=0.2i\), gives the results and errors listed in the following table.
    \begin{table}[H]
        \centering
        \begin{tabular}{cccc}
            \toprule
                & Exact      & Runge-Kutta Order Four          & Error          \\
            \(t_i\) & \(y_i=y(t_i)\)     &   \(w_i\)        &   \(\abs{y_i-w_i}\)        \\\midrule
            0.0 & 0.5000000  & 0.5000000 & 0.0000000 \\
            0.2 & 0.8292986  & 0.8292933 & 0.0000053 \\
            0.4 & 1.2 140877 & 1.2140762 & 0.0000114 \\
            0.6 & 1 .6489406 & 1.6489220 & 0.0000186 \\
            0.8 & 2.1272295  & 2.1272027 & 0.0000269 \\
            1.0 & 2.6408591  & 2.6408227 & 0.0000364 \\
            1.2 & 3.1799415  & 3.1798942 & 0.0000474 \\
            1.4 & 3.7324000  & 3.7323401 & 0.0000599 \\
            1.6 & 4.2834838  & 4.2834095 & 0.0000743 \\
            1.8 & 4.8151763  & 4.8150857 & 0.0000906 \\
            2.0 & 5.3054720  & 5.3053630 & 0.0001089 \\ \bottomrule
            \end{tabular}
    \end{table}
    \begin{note}
        The main computational effort in applying the Runge-Kutta methods is the evaluation of \(f\).
        In the second-order methods, the local truncation error is \(O(h^2)\), and the cost is two functional evaluation per step.
        The Runge-Kutta method of order four requires four evaluation per step and the local truncation error is \(O(h^4)\).
    \end{note}
\end{ex}
\begin{ex}
    IVP
    \[
        y'=y-t^2+1,\qquad 0\leq t\leq 2,\;\;y(0)=0.5
    \]

    Euler's method \(h=0.025\), the midpoint method with \(h=0.05\) and the Runge-Kutta forth order method with \(h=0.1\) are compared at the mesh points \(0.1\), \(0.2\), \(0.3\), \(0.4\) and \(0.5\).
    Each of these techniques requires 20 functional evaluation to determine the values listed in the following table to approximate \(y(0.5)\).
    In this example, the fourth-order method is clearly superior.
    \begin{table}[H]
        \centering
        \begin{tabular}{cccccc}
            \toprule
                &            & Euler      & Modified Euler & Runge-Kutta Order Four &\\
            \(t_i\) & Exact  & \(h=0.025\)& \(h=0.05\) & \(h=0.1\) &  \\\midrule
            0.0 & 0.5000000  & 0.5000000  & 0.5000000  & 0.5000000 &  \\
            0.1 & 0.6574145  & 0.6554982  & 0.6573085  & 0.6574144 &  \\
            0.2 & 0.8292986  & 0.8253385  & 0.8290778  & 0.8292983 &  \\
            0.3 & 1.0150706  & 1.0089334  & 1.0147254  & l.0150701 &  \\
            0.4 & 1.2140877  & 1.2056345  & 1.2136079  & 1.2140869 &  \\
            0.5 & 1.4256394  & 1.4147264  & 1.4250141  & 1.4256384 &  \\\bottomrule
            \end{tabular}
    \end{table}
    \begin{note}
        Include the error table for each case.
    \end{note}
\end{ex}
\subsection{Advantages of Runge-Kutta Method}
\begin{enumerate}
    \item They are easy to implement.
    \item They are stable.
\end{enumerate}
\subsection{Disadvantages of Runge-Kutta Method}
\begin{enumerate}
    \item They require relatively large computer time.
    \item Error estimation are not easy to be done.
    \item Do not work well for stiff differential equations.
    \item In particular, they are not good for systems of differential equations with a mix of fast and slow state dynamics.
\end{enumerate}
\emph{H.W.}
\begin{enumerate}
    \item \(y'=-y\), \(y(0)=1\) with \(h=0.01\) find \(y(0.04)\) by Euler's method.
    \item \(y'=t^2+y\), \(y(0)=1\) find \(y(0.1)\) with \(h=0.05\)  by modified  Euler's method.
    \item \(y'=y-t\), \(y(0)=2\) find \(y(0.1)\), \(y(0.2)\)  by Runge-Kutta order two and four. [See Sastry's Book]
    \item \(\ddt{y}=1+y^2\), \(y(0)=0\) find \(y(0.2)\), \(y(0.4)\), \(y(0.6)\) with \(h=0.2\) by Runge-Kutta order four.
    \item \({y}'=3t+\frac{1}{2}y\), \(y(0)=1\) by Euler, modified Euler and fourth order Runge-Kutta method. [See Sastry's Book]
\end{enumerate}

\end{document}