\documentclass[../main-sheet.tex]{subfiles}
\usepackage{../style}
\graphicspath{ {../img/} }
\backgroundsetup{contents={}}
\begin{document}
% \chapter{System of IVPs}
\section{Improved Euler/Modified Euler}
\begin{align*}
    y_1&=y_0+h\left[ \frac{f(x_0,y_0)+f(x_0+h,y_0+hf(x_0,y_0))}{2} \right]\\
    y_{n+1}&=y_n+h\left[ \frac{f(x_n,y_n)+f(x_n+h,y_n+hf(x_n,y_n))}{2} \right]
\end{align*}
\section{Runge-Kutta Method Of Order 4}
IVP: \(\ddx{y}=f(x,y)\); \(y(x_0)=y_0\)\\
\[
    y_{i+1}=y_1+\frac{1}{6}(k_1+2k_2+2k_3+k_4);\quad i=0,1,2,3,\dots
\]
where,
\begin{align*}
    k_1&=hf(x_i,y_i)\\
    k_2&=hf(x_i+h/2,y_i+k_1/2)\\
    k_3&=hf(x_i+h/2,y_i+k_2/2)\\
    k_4&=hf(x_i+h,y_i+k_3)
\end{align*}
\begin{ex}
    Solve \(\ddx{y}=3x+y^2\); \(y(1)=1.2\) with RK4 at \(x=1.1\).\\
    

    Here \(f(x,y)=3x+y^2\), \(x_0=1\), \(y_0=1.2\) and \(x_1=x_0+h\Rightarrow 1.1=1.h\Rightarrow h=0.1\).
    \begin{note}
        We cane choose \(h=0.05\) then \(x_1=x_0+h=1+0.05=1.05\), \(x_2=x_1+h=1.05+0.05=1.1\)\\
        \(x_n=x_0+nh\); \(n=0,1,2,\dots\)
    \end{note}
    \begin{align*}
        k_1&=hf(x_0,y_0)\\
        &=h(3x_0+y_0^2)\\
        &=0.1(3.1+1.2^2)\\
        &=0.444
    \end{align*}
    or
    \begin{align*}
        k_1&=hf(x_0,y_0)=hf(1,1.2)\\
        &=0.1\times (3\times 1+1\cdot2^2)=0.444
    \end{align*}
    \begin{align*}
        k_2&=hf(x_0+h/2,y_0+k_1/2)=hf(1+0.1/2,1.2+0.4444)\\
        &=0.1\times (3\times (1+0.1/2)+(1.2+0.444/2)^2)=0.5172\\
        k_3&=hf(x_0+h/2,y_0+k_2/2)=hf(1+0.1/2,1.2+0.5172/2)\\
        &=0.1\times (3\times (1+0.1/2)+(1.2+0.5172/2)^2)=0.5278\\
        k_4&=hf(x_0+h,y_0+k_3)=hf(1+0.1,\,1.2+0.5278)\\
        &=0.1\times (3\times (1+0.1)+(1.2+0.5278)^2)=0.6285
    \end{align*}
    \begin{align*}
        \therefore y_1&=y(1.1)=y_0+\frac{1}{6}(k_1+2k_2+2k_3+k_4)\\
        &=1.2+.5271=1.7271
    \end{align*}
\end{ex}
\section{Modified Euler Method}
\begin{ex}
    \[\ddx{y}=f(x,y)=x+y,\qquad y(0)=1\]
    with \(h=0.2\) carry out 2 
\end{ex}
\begin{soln}
    Given, \(\ddx{y}=x+y\) with \(x_0=0\), \(y_0=1\), \(h=0.2\)\\
    \underline{Predictor formula:}\\
    \begin{equation}
        y_n^p=y_{n-1}h[f(x_{n-1},y_{n-1})] \label{eq:mod1}
    \end{equation}
    \(x_n=x_0+nh\); \(h=\) step length\\
    \underline{Corrector formula:}\\
    \begin{equation}
        y_n^c=y_{n-1}+\frac{h}{2}[f(x_{n-1},y_{n-1}+f(x_{n},y_{n}^{p}))]\label{eq:mod2}
    \end{equation}
    Form \eqref{eq:mod1}, \(n=1\)
    \begin{align*}
        y_1^p&=y_0+hf(x_0,y_0)\\
        &=1+0.2 f(0,1)\\
        &=1.2
    \end{align*}
    Form \eqref{eq:mod2}, \(n=1\)
    \begin{align*}
        y_1^c&=y_0+\frac{h}{2}[f(x_{0},y_{0}+f(x_{1},y_{1}^{p}))]\\
        &=1+\frac{0.2}{2}[ f(0,1)+f(0.2,1.2)]\\
        &=1.24
    \end{align*}
    Now,
    \begin{align*}
        y_1^{c_1}&=1+\frac{0.2}{2}\left[ f(0,1)+f(0.2,1.24) \right]\\
        &=1.244\\
        y_1^{c_2}&=1+\frac{0.2}{2}\left[ f(0,1)+f(0.2,1.244) \right]\\
        &=1.244
    \end{align*}
    So \(y_1^{c_2}=y(0.2)=1.244\)
\end{soln}
\section{Adam-Bashforth Predictor and Corrector Method}
\begin{equation}
    y_4^p=y_3+\frac{h}{24}\left[ 55f_3-59f_2+37f_1-9f_0 \right]\label{eq:mod3}
\end{equation}
\begin{equation}
    y_4^c=y_3+\frac{h}{24}\left[ 9f_4^p+19f_3-5f_2+f_1 \right]\label{eq:mod4}
\end{equation}
\begin{ex}
    Given \(\ddx{y}=x^2(1+y)\) by Adam-Bashforth method using \(y(1)=1\), \(y(1.1)=1.233\), \(y(1.2)=1.548\), \(y(1.3)=1.979 \), find \(y(1.4)\).
\end{ex}
\begin{soln}
    Given \(\ddx{y}=x^2(1+y)\)
    \begin{table}[H]
        \centering
        \begin{tabular}{ccl}
            \toprule 
            \(x\) & \(y\)&\(f=x^2(1+y)\)\\\midrule
            \(x_0=1\) & \(y_0=1\)&\(f_0=x_0^2(1+y_0)=1^2(1+1)=2\)\\
            \(x_1=1.1\) & \(y_1=1.233\)&\(f_1=x_1^2(1+y_1)=1.1^2(1+1.233)=2.70193\)\\
            \(x_2=1.2\) & \(y_2=1.548\)&\(f_2=x_2^2(1+y_2)=1.2^2(1+1.548)=3.69912\)\\
            \(x_3=1.3\) & \(y_3=1.979\)&\(f_3=x_3^2(1+y_3)=1.3^2(1+1.979)=5.0345\)\\
            \(x_4=1.4\) & \(y_4^p=2.5273\)&\(f_4^p=x_4^2(1+y_4^p)=1.4^2(1+2.5273)=7.0017\)\\
            & \(y_c=2.5749\)&\\\bottomrule
        \end{tabular}
    \end{table}
    From \eqref{eq:mod3},
    \begin{align*}
        y_4^p&=y_3+\frac{h}{24}[55f_3-59f_2+37f_1-9f_0]\\
        &=1.979+\frac{0.1}{24}[55(5.03451)-59(3.69912)+37(2.70193)-9(2)]\\
        &=2.5723
    \end{align*}
    From \eqref{eq:mod4},
    \begin{align*}
        y_4^c&=y_3+\frac{h}{24}[9f_4^p+19f_3-5f_2+f_1]\\
        &=1.979+\frac{0.1}{24}[9(7.0017)+19(5.03451)-5(3.69912)+2(2.70193)]\\
        &=2.5749
    \end{align*}
\end{soln}
\begin{note}
    If the value of \(y_4^c\) need to correct up to required decimal point, then we need to follow the following steps:
    \[f_4^p=x_4^2(1+y_4^c)=(1.4)^2(1+2.5749)=...\]
    Then
    \[y_4^c=y_3+\frac{h}{24}[9f_4^p+19f_3-5f_2+f_1]=\dots\]
    using \(y_4^c\)\\
    Continue this process until get required accuracy.
\end{note}
\section{Adams-Moulton or Modified Adams Methods}
\begin{itemize}
    \item \(y'=\ddx{y}=f(x,y)\) given
    \item Three starting values of \(y\) (i.e., \(y_1,\,y_2,\,y_3\)) will be given, if not then use Picard, Taylor, Euler or Runge-Kutta method to find these values.
    \item Find the corresponding values of \(y'=f(x,y)\)
    \item Step size \(h\) will also be given
\end{itemize}
\end{document}