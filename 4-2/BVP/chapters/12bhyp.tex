\documentclass[../main-sheet.tex]{subfiles}
\usepackage{../style}
\graphicspath{ {../img/} }
\backgroundsetup{contents={}}
\begin{document}
\section{Method For Numerical Integration Along A Characteristic}
Let \(U\) be specified on the initial curve \(\Gamma\) which must not be a characteristic curve.

\begin{center}
    \begin{tikzpicture}
        \draw[use Hobby shortcut,thick] (-1,1).. (1,1) ..(3,0);
        \node at (3.2,0){$\Gamma$};
        \node at (2.2,3.2){$C$};
        \draw[use Hobby shortcut,thick] (1,1)..(1,2)..(2,3);
        \draw[fill=black] (1,1)circle (.1);
        \node at (1,.5) {$R(x_R,y_R)$};
        \node at (2,2) {$P(x_p,y_p)$};
        \draw[fill=black] (1,2)circle (.1);
        \node at (1,-.5){fig: 1};
        \end{tikzpicture}
\end{center}
Let \(R(x_R,y_R)\) be a point on \(\Gamma\) and \(P(x_p,y_p)\) be a point on the characteristic curve \(C\) through \(R\) such that \(x_p-x_R\) is small (fig-1). The difference equation for the characteristic is
\begin{equation}
    a\dy=b\dx \label{eq:hypb1}
\end{equation}
which gives either \(\dy\) or \(\dx\) when the other quantities are known.

The differential equation for the solution along a characteristic is either
\begin{equation}
    a\D U=c\dx\qquad \text{or, }\qquad b\D U=c\dy\label{eq:hypb2}
\end{equation}
which gives \(\D U\) for known \(\dx\) or \(\dy\) and known \(a\), \(b\) and \(c\).

Denote a first approximation to \(U\) by \(u^{(1)}\) a second approximation by \(u^{(2)}\) etc.

\underline{First approximations:} Assume that \(x_p\) is known. Then by the equation \eqref{eq:hypb1}, we have
\[a_R\set{y_p^{(1)}-y_R}=b_R(x_p-x_R)\]
gives a first approximation \(y_p^{(1)}\) to \(y_p\) and by \eqref{eq:hypb2}, we will get,
\[a_R\set{u_p^{(1)}-u_R}=c_R(x_p-x_R)\]
gives \(u_p^{(1)}\).

\underline{Second and Subsequent approximations:} Replace the coefficient \(a\), \(b\) and \(c\) by known mean values over the arc \(R\). Then
\[\frac{1}{2}\left( a_R+a_p^{(1)} \right)\left( y_p^{(2)}-y_R \right)=\frac{1}{2}\left( b_R+b_p^{(1)} \right)(x_p-x_R)\]
gives \(y_p^{(2)}\)\\
and
\[\frac{1}{2}\left( a_R+a_p^{(1)} \right)\left( U_p^{(2)}-U_R \right)=\frac{1}{2}\left( C_R+C_p^{(1)} \right)(x_p-x_R)\]
gives \(u_p^{(2)}\).

This second procedure can be repeated iteratively until successive iterates agree to a specified number of decimal places.
\begin{ex}
    The function \(U\) satisfies the equation
    \[\sqrt{x}\frac{\partial U}{\partial x}+U\frac{\partial U}{\partial y}=-U^2\]
    and the condition \(U=1\) on \(y=0\), \(0<x<\infty\).
    
    Show that the Cartesian equation of the characteristic through the point \(R(x_R,0)\), \(x_R>0\) is \(y=\log(2\sqrt{x}+1-2\sqrt{x_R})\). Use a finite difference method to calculate first approximation to the solution and to the value of \(y\) at the point \(P(1.1,y)\), \(y>0\), on the characteristic through the point \(R(1,0)\).
    
    Calculate a second approximation to these values by an iterative method. Compare the results with those given by the analytical formulae for \(y\) and \(U\).
\end{ex}
\begin{soln}
    Given,
    \begin{equation}
        \sqrt{x}\frac{\partial U}{\partial x}+U\frac{\partial U}{\partial y}=-U^2\label{eq:hypb1.1}
    \end{equation}
    comapring with \(a\frac{\partial U}{\partial x}+b\frac{\partial U}{\partial y}=c\) we have \(a=\sqrt{x}\), \(b=U\) and \(c=-U^2\).\\
    Hence
    \begin{equation}
        \frac{\dx}{ \sqrt{x}}=\frac{\dy}{U}=\frac{\D u}{-U^2}\label{eq:hypb1.2}
    \end{equation}
    From,
    \begin{align*}
        &\frac{\dy}{U}=\frac{\D u}{-U^2}\\
        \Rightarrow\;\;&\dy=-\frac{\D u}{U}\\
        \Rightarrow\;\;&y=-\log A{U}
    \end{align*}
    As \(U=1\) at \((x_R,0)\) then \(A=1 \) and so
    \begin{equation}
        y=\log\left( \frac{1}{U} \right)\label{eq:hypb1.3}
    \end{equation}
    Similarly, from 
    \begin{align*}
        &\frac{\dx}{\sqrt{x}}=\frac{\D u}{-U^2}\\
        \Rightarrow\;\;&2\sqrt{x}=\frac{1}{U}+B
    \end{align*}
    As \(U=1\) at \((x_R,0)\), \(B=2\sqrt{x_R}-1\).\\
    Therefore
    \begin{equation}
        \frac{1}{U}=2\sqrt{x}+1-2\sqrt{x_R}\label{eq:hypb1.4}
    \end{equation}
    Hence 
    \begin{equation}
        y=\log(2\sqrt{x}-2\sqrt{x_R}+1)\label{eq:hypb1.5}
    \end{equation}
    which is the required Cartesian equation. Now from \eqref{eq:hypb1.3} \(U=e^{-y}\) and from \eqref{eq:hypb1.4}
    \[U=\frac{1}{2\sqrt{x}+1-2\sqrt{x_R}}\]
    \underline{First approximation at \(P(1.1,y)\), \((y>0)\):}
    We have,
    \begin{align*}
        &\frac{\dx}{\sqrt{x}}=\frac{\dy}{U}\\
        \Rightarrow\;\;&{\sqrt{x}}\dy=U\dx
    \end{align*}
    \begin{center}
        \begin{tikzpicture}
            \draw[use Hobby shortcut,thick] (-1,1).. (1,1) ..(3,0);
            \node at (3.2,0){$\Gamma$};
            \node at (2.2,3.2){$C$};
            \draw[use Hobby shortcut,thick] (1,1)..(1,2)..(2,3);
            \draw[fill=black] (1,1)circle (.1);
            \node at (.75,.25) {$R(x_R,y_R)=(1,0)$};
            \node at (3.5,2) {$P(x_p,y_p)=(1.1,y),\, y>0$};
            \draw[fill=black] (1,2)circle (.1);
            % \node at (1,-.5){fig: 1};
        \end{tikzpicture}
    \end{center}
    Hence,
    \begin{align*}
        &\sqrt{x_R}(y_p^{(1)}-y_R)=U_R(x_p-x_R)\\
        \Rightarrow\;\;&\sqrt{1}(y_p^{(1)}-0)=1(1.1-1)\qquad [\because x_R=1,\,\,U_R=1]\\
        \Rightarrow\;\;&y_p^{(1)}=0.1
    \end{align*}
    Again,
    \begin{align*}
        &\frac{\dx}{\sqrt{x}}=-\frac{\D U}{U^2}\\
        \Rightarrow\;\;&{\sqrt{x}}\D U=-U^2\dx\\
        \Rightarrow\;\;&\sqrt{x_R}(U_p^{(1)}-U_R)=-U_R^2(x_p-x_R)\\
        \Rightarrow\;\;&\sqrt{1}(U_p^{(1)}-1)=(-1)^2(1.1-1)\\
        \Rightarrow\;\;&U_p^{(1)}=0.9
    \end{align*}
    \underline{Second Approximation:} Using average values for the coefficients,
    \begin{align*}
        &\frac{1}{2}\left( \sqrt{x_R}+\sqrt{x_p} \right)\left( y_p^{(2)}-y_R \right)=\frac{1}{2}\left( U_R+U_p^{(1)} \right)\left( x_p-x_R \right)\\
        \Rightarrow\;\;&\frac{1}{2}\left( \sqrt{1}+\sqrt{1.1} \right)\left( y_p^{(2)}-0 \right)=\frac{1}{2}\left( 1+0.9\right)\left( 1.1-1.0 \right)\\
        \Rightarrow\;\;& y_p^{(2)}=0.19
    \end{align*}
    and 
    \begin{align*}
        &\frac{1}{2}\left( \sqrt{x_R}+\sqrt{x_p} \right)\left( U_p^{(2)}-U_R \right)=\frac{1}{2}\left( U_R^2+(U_p^{(1)})^2 \right)\left( x_p-x_R \right)\\
        \Rightarrow\;\;&\frac{1}{2}\left( \sqrt{1}+\sqrt{1.1} \right)\left( U_p^{(2)}-1 \right)=\frac{1}{2}\left( 1^2+0.9^2\right)\left( 1.1-1.0 \right)\\
        \Rightarrow\;\;& U_p^{(2)}=0.9117
    \end{align*}
    \underline{Analytical Value:} By equation \eqref{eq:hypb1.5}, we have
    \begin{align*}
        y_p&=\log (2\sqrt{x}-2\sqrt{x_R}+1)\\
        &=\log (2\sqrt{1.1}-2\sqrt{1}+1)\\
        &=0.0931\\
        \intertext{and }
        U_p&=\frac{1}{2\sqrt{x}-2\sqrt{x_R}+1}\\
        &=\frac{1}{2\sqrt{1.1}-2\sqrt{1}+1}\\
        &=0.9111
    \end{align*}
\end{soln}
\begin{note}
    \underline{Characteristic Curves and Equations:}
    \[
        au_x+bu_y=c;\quad x,y\to \text{ independent variable},\quad u=u(x,y)\to \text{ dependent variable},\quad u_x=\frac{\partial u}{\partial x},\;\;u_y=\frac{\partial u}{\partial y}
    \]
    \((a,b,c)\) tangent vector to the solution integral surface at \((x,y,u)\). [Direction of \((a,b,c)\) is characteristic direction]\\
    Characteristic curve: tangent at any point, tangent direction must coincide with characteristic direction.\\
    Parametric form of characteristic curve:\\
    Let,
    \begin{align*}
        x&=x(t)\\
        y&=y(t)\\
        u&=u(t)
    \end{align*}
    `\(t\)' parameter/unknown.\\
    Tangent vector: \((\ddt{x},\ddt{y},\ddt{u})=(a,b,c)\) [must coincide]
    \begin{align}
        &\ddt{x}=a,\quad \ddt{y}=b,\quad \ddt{u}=c\notag\\
        \Rightarrow\;\;&\frac{\dx}{a}=\frac{\dy}{b}=\frac{\D u}{c}\label{eq:hypb2.1}
    \end{align}
    which is characteristic equation.\\
    \eqref{eq:hypb2.1} is a system of equation, where two independent variables, so we will get two solutions.
    \begin{align*}
        &(x,y,u)+\text{ one arbitrary constant}\\
        &(x,y,u)+\text{ another arbitrary constant}
    \end{align*}
    these are characteristic curves.\\
    From \(\frac{\dx}{a}=\frac{\dy}{b}\;\;\Rightarrow\;\;\ddx{y}=\frac{b}{a}\) which is the slope of the characteristic curve.
    \[\ddx{y}=\frac{b(x,y,u)}{a(x,y,u)}\]
\end{note}
\end{document}