\documentclass[../main-sheet.tex]{subfiles}
\usepackage{../style}
\graphicspath{ {../img/} }
\backgroundsetup{contents={}}
\begin{document}
\chapter{Newton's Method For Non-Linear Systems of Equations(Detail Calculation)}
\section{Newton Raphson's Method}
If \(f(x)=0\) is a nonlinear equation, then \(x_n=x_{n-1}-\frac{f(x)}{f'(x)}\).\\
Consider a system of nonlinear equations
\begin{align*}
            f_1(x_1,x_2,\dots, x_n)&=0\\
        f_2(x_1,x_2,\dots, x_n)&=0\qquad\qquad \mathbf{x}=G(\mathbf{x})\\
        \vdots\qquad\quad& \\
        f_n(x_1,x_2,\dots, x_n)&=0
\end{align*}
The Newton's method is
\[
    {G}\left( \mathbf{x}\right)=\mathbf{x}-J\left( \mathbf{x} \right)^{-1}\mathbf{F}\left( \mathbf{x} \right)
\]
where \(J(\mathbf{x})\) is the Jacobian matrix and it is
\[
    J(\mathbf{x}^{(k)})=\begin{bmatrix}
        \frac{\partial f_1(\mathbf{x})}{\partial x_1} & \frac{\partial f_1(\mathbf{x})}{\partial x_2} &\dots&\frac{\partial f_1(\mathbf{x})}{\partial x_n} \\[1em]
        \frac{\partial f_2(\mathbf{x})}{\partial x_1} & \frac{\partial f_2(\mathbf{x})}{\partial x_2} &\dots&\frac{\partial f_2(\mathbf{x})}{\partial x_n} \\[1em]
        \vdots & \vdots & \ddots & \vdots\\
        \frac{\partial f_n(\mathbf{x})}{\partial x_1} & \frac{\partial f_n(\mathbf{x})}{\partial x_2} &\dots&\frac{\partial f_n(\mathbf{x})}{\partial x_n} \\
    \end{bmatrix}
\]
and the functional iteration procedure evolves from selecting \(\mathbf{x}^{(0)}\) and generating for \(k\geq 1\),
\[
    \mathbf{x}^{(k)}=\mathbf{G}\left( \mathbf{x}^{(k-1)} \right)=\mathbf{x}^{(k-1)}-J\left( \mathbf{x}^{(k-1)} \right)^{-1}\mathbf{F}\left( \mathbf{x}^{(k-1)} \right)
\]\
This is called Newton's method for nonlinear system.
\begin{ex}
    \begin{align*}
        &3x_1-\cos(x_2x_3)-\frac{1}{2}=0\\
        &{x_1}^2-81(x_2+0.1)^2+\sin x_3+1.06=0\\
        &e^{-x_1x_2}+20x_3+\frac{10\pi -3}{3}=0
    \end{align*}
    So,
    \begin{align*}
        x_1=&\frac{1}{3}\cos(x_2x_3)+\frac{1}{6}=0\\
        x_2=&\frac{1}{9}\sqrt{{x_1}^2+\sin x_3+1.06}-0.1\\
        x_3=&-\frac{1}{20}e^{-x_1x_2}-\frac{10\pi -3}{60}
    \end{align*}
    We know,
    \begin{equation}
        \mathbf{x}^{(k)}=\mathbf{x}^{(k-1)}-J\left( \mathbf{x}^{(k-1)} \right)^{-1}\mathbf{F}\left( \mathbf{x}^{(k-1)} \right)
        \label{eq:newtonNL1}
    \end{equation}
    Let,
    \begin{align}
        &\mathbf{y}^{(k-1)}=-J\left( \mathbf{x}^{(k-1)} \right)^{-1}\mathbf{F}\left( \mathbf{x}^{(k-1)} \right)\notag\\
        \Rightarrow\;&J\left( \mathbf{x}^{(k-1)} \right)\mathbf{y}^{(k-1)}=-\mathbf{F}\left( \mathbf{x}^{(k-1)} \right)\label{eq:newtonNL2}\\
        \intertext{from \eqref{eq:newtonNL1}}
        \Rightarrow\;&\mathbf{x}^{(k)}=\mathbf{x}^{(k-1)}+\mathbf{y}^{(k-1)}\label{eq:newtonNL3}
    \end{align}
    where,
    \[
        \mathbf{x}=\begin{bmatrix}
            x_1\\
            x_2\\
            x_3
        \end{bmatrix},\qquad\quad
        \mathbf{y}=\begin{bmatrix}
            y_1\\
            y_2\\
            y_3
        \end{bmatrix}
    \]
    Now,
    \[
        \mathbf{F}(\mathbf{x})=\begin{bmatrix}
            f_1\\
            f_2\\
            f_3
        \end{bmatrix}=
        \begin{bmatrix}
            3x_1-\cos(x_2x_3)-\frac{1}{2}\\[1em]
        {x_1}^2-81(x_2+0.1)^2+\sin x_3+1.06\\[1em]
        e^{-x_1x_2}+20x_3+\frac{10\pi -3}{3}
        \end{bmatrix}
    \]
    So,
    \[
        \mathbf{F}(\mathbf{x}^{(k-1)})=
        \begin{bmatrix}
            3x_1^{(k-1)}-\cos(x_2^{(k-1)}x_3^{(k-1)})-\frac{1}{2}\\[1em]
       \left(x_1^{(k-1)}\right)^2-81(x_2^{(k-1)}+0.1)^2+\sin x_3^{(k-1)}+1.06\\[1em]
        e^{-x_1^{(k-1)}x_2^{(k-1)}}+20x_3^{(k-1)}+\frac{10\pi -3}{3}
        \end{bmatrix}
    \]
    and
    \[
        J(\mathbf{x})=\begin{bmatrix}
            \frac{\partial f_1}{\partial x_1} & \frac{\partial f_1}{\partial x_2} &\frac{\partial f_1}{\partial x_3} \\[1em]
            \frac{\partial f_2}{\partial x_1} & \frac{\partial f_2}{\partial x_2} &\frac{\partial f_2}{\partial x_3} \\[1em]
            \frac{\partial f_3}{\partial x_1} & \frac{\partial f_3}{\partial x_2} &\frac{\partial f_3}{\partial x_3}
        \end{bmatrix}
    \]
    So,
    \[
        J(\mathbf{x})=\begin{bmatrix}
            3 & x_3 \sin x_2x_3 & x_2 \sin x_2 x_3\\[1em]
            2x_1 & -162(x_2+0.1) & \cos x_3\\[1em]
            -x_2e^{-x_1x_2} & -x_1e^{-x_1x_2} & 20
        \end{bmatrix}
    \]
    \[
        \therefore J(\mathbf{x}^{(k-1)})=\begin{bmatrix}
            3 & x_3^{(k-1)} \sin x_2^{(k-1)}x_3^{(k-1)} & x_2^{(k-1)} \sin x_2^{(k-1)} x_3^{(k-1)}\\[1em]
            2x_1^{(k-1)} & -162(x_2^{(k-1)}+0.1) & \cos x_3^{(k-1)}\\[1em]
            -x_2^{(k-1)}e^{-x_1^{(k-1)}x_2^{(k-1)}} & -x_1^{(k-1)}e^{-x_1^{(k-1)}x_2^{(k-1)}} & 20
        \end{bmatrix}
    \]
    For \(k=1\), from \eqref{eq:newtonNL2}
    \begin{equation}
        \mathbf{x}^{(1)}=\mathbf{x}^{(0)}+\mathbf{y}^{(0)}
        \label{eq:newtonNL4}
    \end{equation}
    where,
    \begin{align*}
        &\mathbf{y}^{(0)}=\left(J\left( \mathbf{x}^{(0)} \right)\right)^{-1}\mathbf{F}\left( \mathbf{x}^{(0)} \right)\\
        \Rightarrow\; &J\left( \mathbf{x}^{(0)} \right)\mathbf{y}^{(0)}=\mathbf{F}\left( \mathbf{x}^{(0)} \right)
    \end{align*}
    Let us consider \({x}^{(0)}=(0.1,0.1,-0.1)^t\).\\So,
    \begin{align*}
        {F}({x}^{(0)})&
        {=\begin{bmatrix*}
            3x_1^{(0)}-\cos(x_2^{(0)}x_3^{(0)})-\frac{1}{2}\\[1em]
            \left(x_1^{(0)}\right)^2-81(x_2^{(0)}+0.1)^2+\sin x_3^{(0)}+1.06\\[1em]
            e^{-x_1^{(0)}x_2^{(0)}}+20x_3^{(0)}+\frac{10\pi -3}{3}
        \end{bmatrix*}}\\
        &={\begin{bmatrix*}
            0.3-\cos(-0.01)-\frac{1}{2}\\[1em]
            0.01-3.21+\sin (-0.1)+1.06\\[1em]
            e^{-0.01}-2+\frac{10\pi -3}{3}
        \end{bmatrix*}}\\
        &={\begin{bmatrix*}
            1.19995\\[1em]
            -2.269833417\\[1em]
            8.462025346
        \end{bmatrix*}}
    \end{align*}
    and
    \[
        J(x^(0))={\begin{bmatrix*}
            3& 0.000999983 & -0.00099983\\
            0.2& -32.1 & 0.995001165\\
            -0.099004984& 0.099001983 & 20
        \end{bmatrix*}}
    \]
    Now,
    \begin{align*}
        & J\left( x^{(0)} \right)y^{(0)}=-F\left( x^{(0)} \right)\\
        \Rightarrow\; & J\left( x^{(0)} \right){\begin{bmatrix}
            y_1\\
            y_2\\
            y_3
        \end{bmatrix}}=-{\begin{bmatrix*}
            1.19995\\[1em]
            -2.269833417\\[1em]
            8.462025346
        \end{bmatrix*}}
    \end{align*}
    It is a linear system of equations, using Gaussian elimination (or any other method), we get
    \[
        y^{(0)}=\begin{bmatrix}
        0.40003702\\
        -0.08053314\\
        0.12152047
        \end{bmatrix}
    \]
    \begin{align*}
        & x^{(1)}=x^{(0)}+y^{(0)}\\
        \Rightarrow\; & {\begin{bmatrix}
            x_1^{(1)}\\
            x_2^{(1)}\\
            x_3^{(1)}
        \end{bmatrix}}={\begin{bmatrix}
            0.1\\
            0.1\\
            -0.1
            \end{bmatrix}}+{\begin{bmatrix}
            0.40003702\\
            -0.08053314\\
            0.12152047
            \end{bmatrix}}\\
        \Rightarrow\; & \mathbf{x}^{(1)}= {\begin{bmatrix}
            x_1^{(1)}\\
            x_2^{(1)}\\
            x_3^{(1)}
        \end{bmatrix}}={\begin{bmatrix}
            0.50003702\\
            0.01946686\\
            -0.52152017
            \end{bmatrix}}
    \end{align*}
    \begin{note}
        Now, for \(k=2\), we have
        \begin{equation}
            {x}^{(2)}={x}^{(1)}+{y}^{(1)}
        \label{eq:newtonNL5}
    \end{equation}
    where, 
    \begin{equation}
        J\left( x^{(1)} \right)y^{(1)}=-F\left( x^{(1)} \right)
        \label{eq:newtonNL6}
        \end{equation}
        Use the results of \(x^{(1)}\), calculate \(J(x^{(1)})\), \(F\left( x^{(1)} \right)\), from \eqref{eq:newtonNL6} find \(y^{(1)}\), then put these values in \eqref{eq:newtonNL5} and then from \eqref{eq:newtonNL5} we will get \(x^{(2)}\) and then by continuing the process we will get the following results:
    \end{note}
    \begin{table}[H]
        \centering
        \begin{tabular}{clllc}
            \toprule
            \(k\) & \multicolumn{1}{c}{\(x_1^{(k)}\)} & \multicolumn{1}{c}{\(x_2^{(k)}\)} & \multicolumn{1}{c}{\(x_3^{(k)}\)} & \(\left\|\mathbf{x}^{(k)}-\mathbf{x}^{(k-1)}\right\|_\infty\)\\\midrule
            0 & 0.10000000 & 0.10000000 & -0.10000000 & \\
            1 & 0.50003702 & 0.01946686 & -0.52152077 & 0.422 \\
            2 & 0.50004593 & 0.00158859 & -0.52355711 & \(1.79\times 10^{-2}\) \\
            3 & 0.50000034 & 0.00001244 & -0.52359845 & \(1.58\times 10^{-3}\) \\
            4 & 0.50000000 & 0.00000000 & -0.52359877 & \(1.24\times 10^{-5}\) \\
            5 & 0.50000000 & 0.00000000 & -0.52359877 & 0 \\
            \bottomrule
        \end{tabular}
    \end{table}
\end{ex}
[Newton's based Techniques: N-R, Quasi-Newton method.]\\
\begin{note}\hfill
    \begin{enumerate}
        \item Newton's method converge very rapidly.
        \item But it is very difficult to determine initial solution. For known sufficiently accurate initial approximation, method converges very fast.
        \item The method is comparatively expensive.
        \item Accurate initial approximation is needed to ensure convergence.
    \end{enumerate}
\end{note}
\section{Steepest Descent Techniques}
\begin{enumerate}
    \item It converges linearly to the solution.
    \item It usually converges even for poor initial approximation.
    \item This method is used to find sufficiently accurate starting approximation for the Newton based techniques.
\end{enumerate}
\end{document}