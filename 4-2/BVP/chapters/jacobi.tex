\documentclass[../main-sheet.tex]{subfiles}
\usepackage{../style}
\graphicspath{ {../img/} }
\backgroundsetup{contents={}}
\begin{document}
\chapter{Jacobi Iterative Method For Systems}
\section{Norm}
A vector norm on \(\R^n\) is a function \(\|\cdot \|\) from \(\R^n\) into \(\R^n\) with the following properties:
\begin{enumerate}[label=(\roman*)]
    \item \(\| x\|\geq 0\forall x\in \R^n\)
    \item \(\| x\|= 0\) iff \(x=(0,0,\dots,0)\in \R^n\)
    \item \(\|\alpha x\|=\abs{\alpha} \;\|x\|\forall \alpha \in \R, x\in \R^n\)
    \item \(\|x+y\|\leq \|x\|+\|y\|, \;\forall x,y \in \R^n\)
\end{enumerate}
\begin{defn}
    The \(l_2\) and \(l_\infty\) norms for the vector \(x=(x_1,x_2,\dots x_n)^t\) are defined by
    \[
        \|x\|_2=\left\{ \sum_{i=1}^{n}\abs{{x_i}^2} \right\}^{\frac{1}{2}}=\left\{ \sum_{i=1}^{n}{x_i}^2 \right\}^{\frac{1}{2}}
    \]
    and 
    \[
        \|x\|_\infty=\max_{1\leq i\leq n}\abs{x_i}
    \]
\end{defn}
\begin{ex}
    The vector \(x=(-1,1,-2)^t\) in \(\R^3\) has norms \(\|x\|_2=\sqrt{(-1)^2+1^2+(-2)^2}=\sqrt{6}\) and \(\|x\|_\infty=\max \left\{ \abs{-1},\abs{1},\abs{-2} \right\}=2\).
\end{ex}
\begin{note}
    Let \(x=(x_1,x_2,\dots,x_n)\) and \(y=(y_1,y_2,\dots, y_n)\), then 
    \begin{align*}
        \|x+y\|_2&=\set{(x_1+y_1)^2+(x_2+y_2)^2+\dots+(x_n+y_n)^2}^\frac{1}{2}\\
        \|x+y\|_\infty&=\max_{1\leq i\leq n}\abs{x_i+y_i}\\
        \intertext{and}
        \|x-y\|_2&=\set{\sum_{i=1}^{n}\abs{x_i-y_i}^2}^\frac{1}{2}\\
        % \intertext{and}
        \|x+y\|_\infty&=\max_{1\leq i\leq n}\abs{x_i-y_i}
    \end{align*}
\end{note}
\begin{defn}
    A sequence \(\set{x^{(k)}}_{k=1}^\infty\) of vectors in \(\R^n\) is said to converge to \(x\) with respect to the norm \(\|\cdot\|\) if, given any \(\varepsilon > 0\), there exists an integer \(N(\varepsilon)\) such that
    \[\|x^{(k)} - x\| < \varepsilon, \forall k \geq N(\varepsilon)\] 
\end{defn}
\begin{thm}
    The sequence of vectors \(\{x^{(x)}\}\) converges to \(x\) in \(\R^n\) with respect to the norm \(\|\cdot\|\) if and only if \(\lim_{k\to\infty}x_i^{(k)} = X_i\), for each \(i = 1,2,\dots,n\).
\end{thm}
\begin{ex}
    Let \(x^{(k)}=\left( x_1^{(k)},x_2^{(k)},x_3^{(k)},x_4^{(k)} \right)=(1,2+1/k,3/k^2,e^{-k}\sin k)\). Since, \(\lim_{k\to \infty}1=1\), \(\lim_{k\to \infty} (2+1/k)=2\), \(\lim_{k\to \infty} 3/k^2=0\), \(\lim_{k\to \infty}e^{-k}\sin k=0\). Above theorem implies that the sequence \(\set{x^{(k)}}\) converges to \((1,2,0,0)^t\) with respect to \(\|\cdot \|_\infty\)
\end{ex}
\begin{defn}[Matrix Norm]
    A matrix norm on the set of all \(n \times n\) matrices is a real-valued function, \(\|\cdot\|\), defined on this set, satisfying for all \(n \times n\) matrices \(A\) and \(B\) and all real numbers \(\alpha\):
    \begin{enumerate}[label=(\roman*)]
        \item \(\|A\|\geq 0\)
        \item \(\| A \| = 0\), if and only if \(A\) is 0, the matrix with all 0 entries
        \item \(\|\alpha A\|=\abs{\alpha}\;\;\|A\|\)
        \item \(\|A + B\|\leq \|A\| + \|B\|\)
        \item \(||AB||\leq \|A\|\;\;\|B\|\)
    \end{enumerate}
\end{defn}


    The distance between \(n \times n\) matrices \(A\) and \(B\) with respect to this matrix norm is \(\|A-B\|\).
\begin{thm}
    If \(\|\cdot\|\) is a vector norm on \(\R^n\), then \(\|A\|=\max_{\|x\|=1}\|Ax\|\) is a matrix norm.
\end{thm}
\begin{thm}
    If \(A=(a_{ij})_{n\times n}\) is a matrix, then \(\|A\|_\infty=\max_{1\leq i\leq n} \sum_{j=1}^n \abs{a_{ij}}\).
\end{thm}
\begin{ex}
    If \(A=\begin{bmatrix}
        1 & 2 & -1\\
        0 & 3 & -1\\
        5 & -1 & 1
    \end{bmatrix}\) then \(\sum_{j=1}^3\abs{a_{1j}}=\abs{1}+\abs{2}+\abs{-1}=4\), \(\sum_{j=1}^3\abs{a_{2j}}=\abs{0}+\abs{3}+\abs{-1}=4\) and \(\sum_{j=1}^3\abs{a_{3j}}=7\).\\
    \(\therefore \|A\|_{\infty}=\max \set{4,4,7}=7\).
\end{ex}
\section{Iterative Techniques For Solving Linear System}
\subsection{Jacobi Iterative Method}
For solving a linear system \(AX=B\) starts with an initial approximation \(X^{(0)}\) to the solution \(X\) and generates a sequence of vectors \(\set{X^{(k)}}_{k=0}^\infty\) that converges to \(X\). Iterative techniques involve a process that converts the system \(AX=B\) into an equivalent system of the form \(X=TX+C\) for some fixed matrix \(T\) and vector \(C\). After the initial vector \(X^{(0)}\) is selected, the sequence of approximate solution vectors is generated by computing \(X^{(k)}=TX^{(k-1)}+C\) for each \(k=1,2,3,\dots\).
\begin{note}
    \begin{enumerate}
        \item Iterative techniques are seldom used for solving linear systems of small dimension.
        \item For a large system with a high percent of zero entries, these techniques are efficient in terms of both computer storage and computational time. Such systems arise frequently in circuit analysis and in the numerical solution of boundary=value problem and partial-differential equations.
    \end{enumerate}
\end{note}
\begin{ex}
    The linear system \(Ax=b\) given by
    \[\systeme{
        10x_1-x_2+2x_3=6,
        -x_1+11x_2-x_3+3x_4=25,
        2x_1-x_2+10x_3-x_4=-11,
        3x_2-x_3+8x_4=15
    }\]
    has the unique solution \(x=(1,2,-1,1)^t\). Reduce the above system into the form \(x=Tx+c\), we obtain,
    \begin{alignat*}{5}
        x_1 & {}={} &  & \phantom{+} \frac{1}{10}x_2 & {}-{} \frac{1}{5}x_3 & & {}+{} \frac{3}{5}\\
        x_2 & {}={} & \frac{1}{11}x_1 & & {}+{} \frac{1}{11}x_3& {}-{} \frac{3}{11}x_4 & {}+{} \frac{25}{11}\\
        x_3 & {}={} & {}-{} \frac{1}{5}x_1 & {}+{} \frac{1}{10}x_2 & & {}+{} \frac{1}{10}x_4 & {}-{} \frac{11}{10}\\
        x_4 & {}={} & & {}-{} \frac{3}{8}x_2 & {}+{} \frac{1}{8}x_3 & & {}+{} \frac{15}{8}
    \end{alignat*}
    To write this in the form \(x=Tx+c\) we use,
    \[T=\begin{bmatrix}
        0 & 1/10 & -1/5& 0\\
        1/11 & 0 & 1/11& -3/11\\
        -1/5 & 1/10 & 0& 1/10\\
        0 & -3/8 & 1/8& 0
    \end{bmatrix}\quad \text{and }\quad C=\begin{bmatrix}
        3/5\\25/11\\-11/10\\15/8
    \end{bmatrix},\quad x=\begin{bmatrix}
        x_1\\x_2\\x_3\\x_4
    \end{bmatrix}\]
    For an initial approximation, let \(x^{(0)}=(0,0,0,0)^t\). Then \(x^{(1)}\) is given by
    \begin{alignat*}{5}
        x_1^{(1)} & {}={} &  & \phantom{+} \frac{1}{10}x_2^{(0)} & {}-{} \frac{1}{5}x_3^{(0)} & & {}+{} \frac{3}{5}=0.6000\\
        x_2^{(1)} & {}={} & \frac{1}{11}x_1^{(0)} & & {}+{} \frac{1}{11}x_3^{(0)}& {}-{} \frac{3}{11}x_4^{(0)} & {}+{} \frac{25}{11}=2.2727\\
        x_3^{(1)} & {}={} & {}-{} \frac{1}{5}x_1^{(0)} & {}+{} \frac{1}{10}x_2^{(0)} & & {}+{} \frac{1}{10}x_4^{(0)} & {}-{} \frac{11}{10}=-1.1000\\
        x_4^{(1)} & {}={} & & {}-{} \frac{3}{8}x_2^{(0)} & {}+{} \frac{1}{8}x_3^{(0)} & & {}+{} \frac{15}{8}=1.8750
    \end{alignat*}
    Additional iterates \(x^{(k)}=(x_1^{(k)},x_2^{(k)},x_3^{(k)},x_4^{(k)})\) are generated in a similar manner and are presented in the following table.
    \begin{table}[H]
        \begin{tabular}{cccccccccccc}
        \toprule
        \(k\)  & 0      & 1       & 2       & 3       & 4       & 5       & 6       & 7       & 8       & 9       & 10      \\ \midrule
        \(x_1^{(k)}\) & 0.0000  & 0.6000  & 1.0473  & 0.9326  & 1.0152  & 0.9890  & 1.0032  & 0.9981  & 1.0006  & 0.9997  & 1.0001  \\
        \(x_2^{(k)}\) & 0.0000 & 2.2727  & 1.7159  & 2.053   & 1.9537 & 2.0114 & 1.9922 & 2.0023  & 1.9987 & 2.0004  & 1.9998 \\
        \(x_3^{(k)}\) & 0.0000 & -1.1000 & -0.8052 & -1.0493 & -0.9681 & -1.0103 & -0.9945 & -1.0020 & -0.9990 & -1.0004 & -0.9998 \\
        \(x_4^{(k)}\) & 0.0000 & 1.8750  & 0.8852  & 1.1309  & 0.9739  & 1.0214  & 0.9944  & 1.0036 & 0.9989  & 1.0006 & 0.9998  \\ \bottomrule
        \end{tabular}
        \end{table}
        Iterations are stopped based on the criterion
        \[\frac{\|x^{(10)}-x^{(9)}\|_\infty}{\|x^{(10)}\|_\infty}=\frac{8.0\times 10^{-4}}{1.9998}<10^{-3}\]
        or,
        \[\|x^{(10)}-x\|_\infty=0.0002<10^{-3}\]
        or,
        \[\frac{\|x^{(k)}-x^{(k-1)}\|_\infty}{\|x^{(k)}\|_\infty}<\varepsilon;\qquad \varepsilon=\text{ Tolerance}\]

\end{ex}
\end{document}