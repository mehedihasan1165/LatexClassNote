\documentclass[../main-sheet.tex]{subfiles}
\usepackage{../style}
\graphicspath{ {../img/} }
\backgroundsetup{contents={}}
\begin{document}
\begin{note}
    The order of a difference equation is the number of intervals separating the largest and the smallest arguments of the dependent variable.
    Thus, the difference equation \eqref{eq:euler9a} and \eqref{eq:euler9b} are of first order and the difference equation \eqref{eq:euler9c} is of second order.
    The methods \eqref{eq:euler9a}, \eqref{eq:euler9b} are called single step methods and the method \eqref{eq:euler9c} is called a two-step or multistep method.
\end{note}
\begin{note}
    The approximate values \(y_i\) will contain errors.
\end{note}
\begin{defn}
    A method is convergent if as more grid points are taken or step size is decreased, the numerical solution converges to the exact solution, in the absence of round off errors.
\end{defn}
\begin{defn}
    A method is stable if the effect of any single fixed round off errors is bounded, independent of the number of mesh points.
\end{defn}
\begin{ex}
    Suppose Euler's method is used to approximate the solution to the IVP:
    \[
        y'=y-t^2+1,\qquad 0\leq t\leq 2,\;\;y(0)=0.5
    \]
    with \(N=10\), then \(h=0.2\), \(t_i=0.2i\), \(w_0=0.5\) and
    \begin{align*}
        w_{i+1}&=w_i+h(w_i-t^2+1)\\
        &=w_i+0.2[w_i-0.04i^2+1]\\
        &=1.2w_i-0.08i^2+0.2\quad\text{ for } i=0,1,2,\dots,9
    \end{align*}
    The exact solution is \(y(t)=(t+1)^2-0.5e^t\).
    Table 1 shows the comparison between the approximate values at \(t_i\) and the actual values.
    \begin{table}[H]
        \centering
        \begin{tabular}{cccc}
            \toprule
            \(t_i\) & \(w_i\)&\(y_i=y(t_i)\)&\(\abs{y_i-w_i}\)\\\midrule
            0.0 & 0.5000000 & 0.5000000 & 0.0000000 \\
            0.2 & 0.8000000 & 0.8292986 & 0.0292986 \\
            0.4 & 1.1520000 & 1.2140877 & 0.0620877 \\
            0.6 & 1.5504000 & 1.6489406 & 0.0985406 \\
            0.8 & 1.9884800 & 2.1272295 & 0.1387495 \\
            1.0 & 2.4581760 & 2.6408591 & 0.1826831 \\
            1.2 & 2.9498112 & 3.1799451 & 0.2301303 \\
            1.4 & 3.4517754 & 3.7324000 & 0.2806266 \\
            1.6 & 3.9501281 & 4.2834838 & 0.3333557 \\
            1.8 & 4.4281558 & 4.8151763 & 0.3870225 \\
            2.0 & 4.8657845 & 5.3054720 & 0.4396874 \\\bottomrule
        \end{tabular}
    \end{table}
    \emph{Comment:} The error grows slightly as the values of \(t\) increased.
    This controlled error growth is a consequence of stability of Euler's method.
    Although this method is not accurate enough to warrant its use in practice, it is sufficiently elementary method to analyze the error that is produced from its application.
\end{ex}
\begin{thm}
    \label{thm:3}
    Suppose \(f\) is continuous and satisfies a Lipschitz condition with the constant \(L\) on \(D=\set{(t,y):a\leq t\leq b,\;-\infty<y<\infty}\) and that a constant \(M \) exists with the property that \(\abs{y^{''}(t)}\leq M\;\forall t\in [a,b]\).

    Let \(y(t)\) denote the unique solution to the IVP 
    \[
        y'=f(t,y),\quad a\leq t\leq b,\quad y(a)=\alpha
    \]
    and \(w_0,\;w_1,\;\dots, w_N\) be the approximations generated by Euler's method for some positive integer \(N\).
    Then for each \(i=0,1,2,\dots,N\),
    \begin{equation}
        \abs{y(t_i)-w_i}\leq \frac{h M}{2L}\left[ e^{L(t_i-a)} -1\right]
        \label{eq:thm3.1}
    \end{equation}
\end{thm}
\begin{proof}
    When \(i=0\), the result is clearly true, since \(y(t_0)=w_0=\alpha\).\\
    We have,
    \begin{equation}
        y(t_{i+1})=y(t_i)+h f(t_i,y(t_i))+\frac{h^2}{2!}y^{''}(\xi_i)
        \label{eq:thm3.2}
    \end{equation}
    and 
    \begin{equation}
        w_{i+1}=w_i+h f(t_i,w_i)\quad\text{ for } i=0,1,2,\dots,N-1
        \label{eq:thm3.3}
    \end{equation}
    Using the notation, \(y_i=y(t_i)\) and \(y_{i+1}=y(t_{i+1})\),\\
    we have
    \[
        y_{i+1}-w_{i+1}=y_i-w_i+h\left[ f(t_i,y_i)-f(t_i,w_i)\right]+\frac{h^2}{2!}y^{''}(\xi_i)
        \]
        and
        \begin{equation}
            \abs{y_{i+1}-w_{i+1}}\leq \abs{y_i-w_i}+h\abs{ f(t_i,y_i)-f(t_i,w_i)}+\frac{h^2}{2!}\abs{y^{''}(\xi_i)}\label{eq:thm3.4}
        \end{equation}
        Since \(f\) satisfies a Lipschitz condition in the second variable with constant \(L\) and \(\abs{y^{''}(t)}\leq M\), we have
        \[
            \abs{y_{i+1}-w_{i+1}}\leq (1+hL)\abs{y_i-w_i}+\frac{h^2M}{2}
        \]
        \begin{lem}
            If \(s\) and \(t\) are real numbers, \(\set{a_i}_{i=0}^k\) is a sequence satisfying \(a_0\geq -t/s\) and \(a_{i+1}\leq (1+s)a_i+t\) for each \(i=0,1,2,\dots,k\), then 
            \[
                a_{i+1}\leq e^{(i+1)s}\left(a_0+\frac{t}{s}\right)-\frac{t}{s}
            \]
        \end{lem}
                
        By using the above lemma and letting \(a_j=\abs{y_j-w_j} \) for each \(j=0,1,2,\dots,N\) while \(s=hL\) and \(t=h^2M/2\), we see that
        \[
            \abs{y_{i+1}-w_{i+1}}\leq e^{(i+1)hL}\left( \abs{y_0-w_0}+\frac{h^2M}{2hL} \right)-\frac{h^2M}{2hL}
        \]
        Since \(\abs{y_0-w_0}=0\) and \((i+1)h=t_{i+1}-t_0=t_{i+1}-a\), we have
        \[
            \abs{y_{i+1}-w_{i+1}}\leq \frac{h M}{2L}\left[ e^{L(t_{ i+1}-a)} -1\right]
        \]
        for each \(i=0,1,2,\dots, N-1\).
    \end{proof}
    \begin{ex}
        Consider the IVP:
        \[
            y'=y-t^2+1,\quad 0\leq t\leq 2,\;\;y(0)=0.5
        \]
        We have, \(\pardy{f}=1\) for all \(y\), so \(L=1\).\\
        The exact solution is \(y(t)=(t+1)^2-\frac{1}{2}e^t\), so \(y^{''}(t)=2-0.5e^t\) and \(\abs{y^{''}(t)}\leq 0.5e^2-2\) for all \(t\in [0,2]\).\\
        Using the inequality in the error bound for Euler's  method with \(h=0.2\), \(L=1\) and \(M=0.5e^2-2\) gives the error bound
        \[
            \abs{y_i-w_i}\leq 0.1 (0.5e^2-2)(e^{t_i}-1)
        \]
        The following table lists the actual error found in this example together with this error bound.
        \begin{table}[H]
            \centering
            \begin{tabular}{ccccccccccc}
                \toprule
                \(t_i\) & 0.2 & 0.4 & 0.6 & 0.8 & 1.0 & 1.2 & 1.4 & 1.6 & 1.8 & 2.0\\\midrule
                Actual Error & 0.02930 & 0.06209 & 0.09854 & 0.13875 & 0.18268 & 0.23013 & 0.28063 & 0.33336 & 0.38702 & 0.43969\\
                Error Bound & 0.03752 & 0.08334 & 0.13931 & 0.20767 & 0.29117 & 0.39315 & 0.51771 & 0.66985 & 0.85568 & 1.08264\\
                \bottomrule
            \end{tabular}
        \end{table}
    \end{ex}
        \emph{Comment:} Even though the true bound for the second derivative of the solution was used, the error bound was considerably larger than the actual error.

    The principal importance of the error bound formula given in theorem \ref{thm:3} is that the bound depends linearly on the step size \(h\). As \(h\) becomes smaller, more calculations are necessary and more round off error is expected.
    \subsection{Advantages of Euler's Method}
    \begin{enumerate}
        \item Euler's method is simple and direct.
        \item This method can be used for nonlinear initial value problems.
    \end{enumerate}
    \subsection{Disadvantages of Euler's Method}
    \begin{enumerate}
        \item It is less accurate and numerically unstable.
        \item Approximation error is proportional to the step size \(h\).
        Hence, good approximation is obtained with a very small \(h\).
        This requires a larger number of time discretization leading to a larger computation time.
        \item Usually applicable to explicit differential equations.
    \end{enumerate}
    \blfootnote{
        \begin{enumerate}
            \item Most DEs can not be solved exactly.
            \item Use the definition of derivative to create a difference equation.
            \item Develop numerical method to solve DEs.
            \begin{enumerate}[label=(\roman*)]
                \item Euler's method
                \item Improved Euler's method.
            \end{enumerate}
        \end{enumerate}
    }
\end{document}