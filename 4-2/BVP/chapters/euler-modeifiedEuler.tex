\documentclass[../main-sheet.tex]{subfiles}
\usepackage{../style}
\graphicspath{ {../img/} }
\backgroundsetup{contents={}}
\begin{document}
\section{Euler Method}
\[
    \ddx{y}=f(x,y);\quad y(x_0)=y_0
\]
\begin{figure}[H]
    \centering
    \begin{minipage}[c]{.45\textwidth}
        \import{../tikz/}{euler.tikz}
    \end{minipage}
    \begin{minipage}[c]{.45\textwidth}
        \import{../tikz/}{euler-2.tikz}
    \end{minipage}
\end{figure}
\begin{align*}
    &\tan\;\theta=\frac{y_1-y_0}{x_1-x_0}\\
    \Rightarrow\;&\tan\;\theta=\frac{y_1-y_0}{h}\\
    \Rightarrow\;&f(x_0,y_0)=\frac{y_1-y_0}{h}\\
    \Rightarrow\;&y_1=y_0+hf(x_0,y_0)
\end{align*}
Similarly,
\[y_2=y_1+hf(x_1,y_1)\]
Generally,
\[y_{i+1}=y_i+hf(x_i,y_i)\]
which is Euler's method.
\section{Modified Euler Method}
\begin{align*}
    \ddx{y}&=f(x,y);\qquad y(x_0)=y_0\\
    \text{or, } \ddt{y}&=f(t,y);\qquad y(t_0)=y_0\\
\end{align*}
\begin{figure}[H]
    \centering
    \import{../tikz/}{modifiedEuler.tikz}
\end{figure}
\[
    y_{n+1}=y_n+\frac{\Delta t}{2}\left[ f(t_n,y_n)+f(t_n+\Delta t,y_{n+1})\right]
\]
Predictor-Corrector Method:
\begin{align*}
    y_{n+1}^{p}&=y_n+\Delta t f(t_n,y_n)\\
    \therefore\;\;y_{n+1}&=y_n+\frac{\Delta t}{2}\left[ f(t_n,y_n)+f(t_n+\Delta t,y_{n+1}^p)\right]
\end{align*}
which is called Modified Euler method.\\
This is also called 2nd-order Runge-Kutta method.
\section{Runge-Kutta Method (RK2) (Two stage)}
\begin{align*}
    k_1&=\Delta t f(t_n,y_n)\\
    k_2&=\Delta t f(t_n+\Delta t,y_n+k_1)\\
    \therefore\;\;y_{n+1}&=y_n+\frac{1}{2}(k_1+k_2)
\end{align*}
\begin{ex}
    Modified Euler Method:\\
    
    
    First approximation:
    \[
        y_1^{(1)}=y_0+\frac{h}{2}\left[ f(x_0,y_0)+f(x_1,y_{1})\right]
    \]
    where
    \[
        y_1=y_0+hf(x_0,y_0)
    \]
    Now,
    \begin{align*}
        y_1^{(2)}&=y_0+\frac{h}{2}\left[ f(x_0,y_0)+f(x_1,y_{1}^{(1)})\right]\\
        y_1^{(3)}&=y_0+\frac{h}{2}\left[ f(x_0,y_0)+f(x_1,y_{1}^{(2)})\right]\\
        \vdots &
    \end{align*}
    Let us consider \(y_1^{(n+1)}=y_1^{(n)}\), then \(y_1=y_1^{(n+1)}\).
    
    
    Second approximation: \(x_1=x_0+h\)
    \[
        y_2^{(1)}=y_1+\frac{h}{2}\left[ f(x_1,y_1)+f(x_2,y_{2})\right]
    \]
    where
    \[
        y_2=y_1+hf(x_1,y_1)
    \]
    Now,
    \begin{align*}
        y_2^{(2)}&=y_1+\frac{h}{2}\left[ f(x_1,y_1)+f(x_2,y_{2}^{(1)})\right]\\
        y_2^{(3)}&=y_1+\frac{h}{2}\left[ f(x_1,y_1)+f(x_2,y_{2}^{(2)})\right]\\
        \vdots &
    \end{align*}

    Generally,
    \[
        y_{i+1}^{(n+1)}=y_i+\frac{h}{2}\left[ f(x_i,y_i)+f(x_{i+1},y_{i+1}^{(n)})\right]
    \]
    where
    \[
        y_{i+1}^{(n)}=y_i+hf(x_i,y_i)
    \]
\end{ex}
\begin{prob}
    \(\ddx{y}=x+y\), with \(y(0)=1\) for \(x=0.1\) taking \(h=0.05\)
\end{prob}
\begin{soln}
    1st Iteration\\

    1st approximation:
    \begin{equation}
        y_{1}^{(1)}=y_0+\frac{h}{2}\left[ f(x_0,y_0)+f(x_{1},y_{1})\right]
        \label{eq:probme1.1}
    \end{equation}
    where,
    \begin{align*}
        y_1&=y_0+hf(x_0,y_0)\\
        &=1+0.05(0+1)\\
        &=1.05
    \end{align*}
    Now from \eqref{eq:probme1.1}
    \begin{align*}
        y_1^{(1)}&=1+\frac{0.05}{2}\left[ 0+1+0.05+1.05\right]\\
        &=1.0525
    \end{align*}
    Again,
    \begin{align*}
        y_1^{(2)}&=y_0+\frac{h}{2}\left[ f(x_0,y_0)+f(x_{1},y_{1}^{(1)})\right]\\
        &=1+\frac{0.05}{2}\left[ 0+1+0.05+1.0525\right]\\
        &=1.0526
    \end{align*}
    [Note: \(f(x_1,y_1^{(1)})=x_1+y_1^{(1)}=0.05+1.0525\)]
    \begin{align*}
        y_1^{(3)}&=y_0+\frac{h}{2}\left[ f(x_0,y_0)+f(x_{1},y_{1}^{(2)})\right]\\
        &=1+\frac{0.05}{2}\left[ 0+1+0.05+1.0526\right]\\
        &=1.0526
    \end{align*}
    \(f(x_1,y_1^{(2)})=x_1+y_1^{(2)}=0.05+1.0526\)\\
    So \(y=1.0526\) at \(x=0.05\)\\
    
    
    2nd Iteration: \(x_0=0.05\), \(y_0=1.0526\), \(h=0.05\), \(x_1=x_0+h=0.05+0.05=0.10\)\\
    \begin{equation}
        \therefore\;\;y_{1}^{(1)}=y_0+\frac{h}{2}\left[ f(x_0,y_0)+f(x_{1},y_{1})\right]
        \label{eq:probme1.2}
    \end{equation}
    where,
    \begin{align*}
        y_1&=y_0+hf(x_0,y_0)\\
        &=1.0526+0.05f(0.05,1.0526)\\
        &=1.1077
    \end{align*}
    Now from \eqref{eq:probme1.2}
    \begin{align*}
        y_1^{(1)}&=1.0526+\frac{0.05}{2}\left[ f(0.05,1.0526)+f(0.1,1.1077)\right]\\
        &=1.0526+\frac{0.05}{2}\left[ 0.05+1.0526+0.1+1.1077\right]\\
        &=1.1103
    \end{align*}
    Again,
    \begin{align*}
        y_1^{(2)}&=y_0+\frac{h}{2}\left[ f(x_0,y_0)+f(x_{1},y_{1}^{(1)})\right]\\
        &=1.0526+\frac{0.05}{2}\left[ f(0.05,1.0526)+f(0.1,1.1103)\right]\\
        &=1.0526+\frac{0.05}{2}\left[ 0.05+1.0526+0.1+1.1103\right]\\
        &=1.1104
    \end{align*}
    \begin{align*}
        y_1^{(3)}&=y_0+\frac{h}{2}\left[ f(x_0,y_0)+f(x_{1},y_{1}^{(2)})\right]\\
        &=1.0526+\frac{0.05}{2}\left[ f(0.05,1.0526)+f(0.1,1.1104)\right]\\
        &=1.0526+\frac{0.05}{2}\left[ 0.05+1.0526+0.1+1.1104\right]\\
        &=1.1104
    \end{align*}
    So \(y=1.1104\) at \(x=0.1\)
\end{soln}
\end{document}