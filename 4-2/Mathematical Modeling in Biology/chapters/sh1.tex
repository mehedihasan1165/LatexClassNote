\documentclass[../main-sheet.tex]{subfiles}
\usepackage{../style}
\graphicspath{ {../img/} }
\backgroundsetup{contents={}}
\begin{document}
\begin{prob}
    What is mathematical model? Discuss the steps in building a good mathematical model. Discuss briefly the limitations of mathematical modelling.
\end{prob}
\begin{soln}
    \underline{Mathematical model:} A differential equation that describes some physical process is often called a mathematical model of the process.\\
    \underline{Steps in building a good mathematical model:} The following steps are necessary for building a good mathematical model.
    \begin{enumerate}[label=(\roman*)]
        \item To identify the independent and dependent variables. The independent variable is often time.
        \item To choose the limits of measurement for each variable.
        \item To articulate the basic principle that underlies or governs the problem.
        \item To express the principle or law in terms of the variables.
        \item To make sure that each term in the equation has the same physical units.
    \end{enumerate}
    \underline{Limitations of mathematical modeling:}

    There are still an equally large or even a larger number of situations which have not yet been mathematically modeled either because the situations are sufficiently complex or because mathematical models formed are mathematically intractable.\\
    However, successful guidelines are not available for choosing the number of parameter and of estimating the values of these parameters. Mathematical modelling of large systems presents its own special problems.
\end{soln}
\begin{prob}
    Discuss the mathematical model for population. What are the necessity and techniques of mathematical model?
\end{prob}
\begin{soln}
    A differential equation that describes some physical process is often called a mathematical model of the process.

    It is necessary first to formulate the appropriate differential equation that describes or models, the problem being investigated. Sometimes it is difficult to construct a satisfactory model.\\

    The following are necessity and techniques of mathematical model.
    \begin{enumerate}[label=(\roman*)]
        \item Identify the dependent and independent variables. The independent variable is often time.
        \item Choose the units of measurement for each variable.
        \item Articulate the basic principle that underlies or governs the problem.
        \item Express the principle or law in terms of the variables.
        \item Make sure that each term in the equation has same physical units.
    \end{enumerate}
\end{soln}
\begin{prob}
    Discuss Malthusian model for single species population growth.\\
    Discuss the limiting behavior of the model as \(t\to \infty\). Comment on the appropriateness of the model and suggest some improvement.\\
    or,\\
    Describe the mathematical model of a single species population. Comment on the plausibility of the model and suggest some improvement.
\end{prob}
\begin{soln}
    Let \(N(t)\) be the population of species at time \(t\). Then the rate of change,
    \begin{equation}
        \ddt{N}=\text{births }-\text{deaths }+\text{migration}
        \label{eq:1.1}
    \end{equation}
    is a conservation equation for the population.

    The simplest model has no migration and the birth and death terms are proportional to \(N\).\\
    Thus, \eqref{eq:1.1} takes the form,
    \begin{align}
        \ddt{N}&=(b-d)N\notag\\
        &=rN\qquad\text{Where } r=b-d\notag\\
       \Rightarrow \frac{\D N}{N}&=r\D t\notag\\
       \Rightarrow\ln N&=rt+\ln c\quad [\text{Integrating}]\notag\\
       \Rightarrow N&=c e^{rt}\label{eq:1.2}
    \end{align}
    Let the initial condition, \(N=N_0\) at \(t=t_0\)\\
    From \eqref{eq:1.2} we have,
    \begin{align*}
        &N(0)=ce^{rt_0}\\
        \Rightarrow &c=N_0 e^{-rt_0}
    \end{align*}
        \[\therefore N=N_0 e^{r(t-t_0)}\]
    This is the Malthusian model for population growth.

    \emph{Behaviour of the solution:}\\
    For different values of \(r\), we obtain three different solution of Malthusian model.\\


    \underline{Case 1:} If \(r<0\), then \(\lim_{t\to\infty}N(t)=0\). This is called the case of extinction decays (exponentially).\\
    
    
    \underline{Case 2:} If \(r=0\), then \(N(t)=N_0\). The population remains unchanged. (constant)\\


    \underline{Case 3:} If \(r>0\), then \(\lim_{t\to\infty}N(t)=\infty\). The population grows exponentially. (unlimited growth)
    \begin{figure}[H]
        \centering
        \import{../tikz/}{model-malthusian.tikz}
    \end{figure}
    Taking \(t_0=0\), then the solution reduces to
    \[N(t)=N_0e^{rt}\]
    In the absence of birth (\(b=0\)), the population is by death at rated and consequently the average life span of a number of this problem \(\frac{1}{d}\).
    
    If \(b>0\), then the average number of offspring over the lifetime of an average individual under the Malthusian model would be \(\frac{b}{d}\).
    
    If the ratio is greater than \(1\), then birth exceeds death and the population explodes.
    
    If the ratio is less than 1, then death exceeds birth and the population dies out.
\end{soln}
\begin{prob}
    Describe the Verhulst logistic growth model for the dynamics of a single species population. Find the complete solution of the model. Discuss the behavior of the population as \(t\to \infty\). Discuss the stability of the equilibrium states of the logistic model.\\
    Or, Describe the single species discrete time population model. Show that according to this model the population tends to be a constant (carrying capacity).
\end{prob}
\begin{soln}
    \emph{1st part:} For the dynamics of a single species population Verhulst suggested the model
    \[\ddt{N}=rN\left( 1-\frac{N}{K} \right)\]
    where \(N=N(t)=\) population of single species at time \(t\) and \(r\) and \(K\) are constants. Here, \(r\left( 1-\frac{N}{K} \right)\) is the per capita birth rate. The constant \(K\) is the carrying capacity of the environment.\\
    
    \emph{2nd part:} We have,
    \begin{align}
        &\ddt{N}=rN\left( 1-\frac{N}{K} \right)\notag\\
        \Rightarrow\; &\frac{k\D N}{N(K-N)}=r\D t\notag\\
        \Rightarrow\; &\left(\frac{1}{N}+\frac{1}{K-N}\right)\D N=r\D t\notag\\
        \intertext{Integrating both sides, we get}
        \Rightarrow\; &\log N-\log(K-N)=rt+\log c\notag\\
        \Rightarrow\; &\log \frac{N}{K-N}=\log c e^{rt}\qquad [\because \log e^{rt}=rt]\notag\\
        \Rightarrow\; &\frac{N}{K-N}=c e^{rt}\label{eq:ver1.1}
    \end{align}
    Using the initial condition, \(N=N_0\) at \(t=0\)
    \begin{align*}
        &\frac{N_0}{K-N_0}=ce^0\\
        \Rightarrow\; &c=\frac{N_0}{K-N_0}\\
        \intertext{Equation \eqref{eq:ver1.1} becomes,}
        \Rightarrow\; &\frac{N}{K-N}=\frac{N_0}{K-N_0}e^{rt}\\
        \Rightarrow\; &N(K-N_0)=N_0(K-N)e^{rt}\\
        \Rightarrow\; &N(K-N_0+N_0e^{rt})=N_0Ke^{rt}\\
        \Rightarrow\; & N=\frac{N_0Ke^{rt}}{(K-N_0)+N_0e^{rt}}
    \end{align*}
    \begin{equation}
        \therefore\;N=\frac{N_0K}{N_0+(K-N_0)e^{-rt}} \label{eq:ver1.2}
    \end{equation}
    
    \emph{3rd part:} From \eqref{eq:ver1.2} we have
    \[N(t)=\frac{N_0K}{N_0+(K-N_0)e^{-rt}}\]
    when, \(t\to \infty\), then,
    \begin{align*}
        & N(t)\to\frac{KN_0}{N_0+(K-N_0)\cdot 0}\\
        \Rightarrow\;& N(t)\to\frac{KN_0}{N_0}\\
        \Rightarrow\;& N(t)\to K
    \end{align*}
    This is the limiting behavior of the model as \(t\to\infty\).
    
    \emph{4th part:} \underline{For stability and equilibrium:}\\
    For stability and equilibrium,\\
    we have,
    \begin{align*}
        &\ddt{N}=0\\
        \Rightarrow\; &rN\left( 1-\frac{N}{K} \right)=0\\
        \Rightarrow\; &N=0\quad\text{or,}\;\;\left( 1-\frac{N}{K} \right)=0\\
        &\qquad\;\;\Rightarrow\;\frac{N}{K}=1\\
        &\qquad\;\;\Rightarrow\;N=K
    \end{align*}
    Thus there are two equilibrium states \(N=0\) and \(N=K\).
    \begin{enumerate}[label=(\roman*)]
        \item \(N=0\) is unstable linearization about it gives \(\ddt{N}\approx rN\) and so \(N\) grows exponentially from any small initial values.
        \item For linearization about \(N=k\),
        we put \(N=k+\varepsilon\) with \(\abs{\varepsilon}\) small so that,
        \begin{align*}
            &\ddt{(k+\varepsilon)}=r(k+\varepsilon)\left( 1-\frac{(k+\varepsilon)}{k} \right)\\
            \Rightarrow\;&\ddt{\varepsilon}=r(k+\varepsilon)\left( \frac{(-\varepsilon)}{k} \right)\\
            \Rightarrow\;&\ddt{\varepsilon}\approx -r\varepsilon\quad \text{to the first order}\\
            \Rightarrow\;&\ddt{(N-k)}\approx -r(N-k)
        \end{align*}
        which gives \(N\to k\) as \(t\to \infty\).\\
        Hence, \(N=k\) is stable.
    \end{enumerate}
    \begin{figure}[H]
        \centering
        \import{../tikz/}{model-stab.tikz}
    \end{figure}
\end{soln}
\begin{prob}
    A population grows according to the equation
    \[\ddt{N}=1-e^{-r\left( 1-\frac{N}{k} \right)}\]
    where \(r\) and \(k\) are positive constant.
    \begin{enumerate}[label=(\roman*)]
        \item Determine the equilibrium population size.
        \item Decide whether the equilibrium is stable, unstable or neutral.
    \end{enumerate}
\end{prob}
\begin{soln}
    The given equation is
    \[\ddt{N}=1-e^{-r\left( 1-\frac{N}{k} \right)},\qquad r,k>0\]
    \begin{enumerate}[label=(\roman*)]
        \item For equilibrium, we have
        \begin{align*}
            &\ddt{N}=0\\
            \Rightarrow\; &1-e^{-r\left( 1-\frac{N}{k} \right)}=0\\
            \Rightarrow\; & e^{-r\left( 1-\frac{N}{k} \right)}=1\\
            \Rightarrow\; & e^{-r\left( 1-\frac{N}{k} \right)}=1=e^0\\
            \Rightarrow\; & -r\left( 1-\frac{N}{k} \right)=0\\
            \Rightarrow\; & 1-\frac{N}{k}=0\\
            \Rightarrow\; & \frac{N}{k}=1\\
            \Rightarrow\; & N=k
        \end{align*}
        Thus the equilibrium population size is \(N=N(t)=k=\) constant.
        \item We linearize the given equation by putting \(N=k+\varepsilon\) with \(\abs{\varepsilon}\ll 1\).\\
        We have
    \end{enumerate}
        \begin{align*}
            &\ddt{}[k+\varepsilon]=1-e^{-r\left( 1-\frac{k+\varepsilon}{k} \right)}\\
            \Rightarrow\; &\ddt{\varepsilon}=1-e^{-r\left( \frac{-\varepsilon}{k} \right)}\\
            \Rightarrow\; &\ddt{\varepsilon}=1-e^{\left( \frac{r\varepsilon}{k} \right)}\\
            \Rightarrow\; &\ddt{\varepsilon}=1-\left[ 1+\frac{r \varepsilon/k}{1!}+\frac{(r \varepsilon/k)^2}{2!}+\dots \right]\\
            \Rightarrow\; &\ddt{\varepsilon}\approx -\frac{r\varepsilon}{k}\qquad \text{as }\abs{\varepsilon}\ll 1\\
            \Rightarrow\; &\ddt{(N-k)}\approx -\frac{r}{k}(N-k)
        \end{align*}
        This shows the population decreases.\\
        Hence, the equilibrium \(N=k\) is stable.
\end{soln}
\begin{prob}
    Discuss the continuous/ discontinuous
    \[u_{t+1}=u_t \exp [r(1-u_t)]\qquad 0<r<1\]
    Discuss its total qualitative behavior.
\end{prob}
\begin{soln}
    We have, the given discontinuous model.
    \begin{equation}
        u_{t+1}=u_t \exp [r(1-u_t)]\qquad 0<r<1\label{eq:6.1}
    \end{equation}
    For the steady states, we have, \(u_{t+1}=u_t=u^{*}\) and \eqref{eq:6.1} becomes
    \begin{align*}
        & u^{*}=u^{*} \exp [r(1-u^{*})]\\    
        \Rightarrow\; & u^{*}[1- \exp \{r(1-u^{*})\}]=0\\    
        \Rightarrow\; & u^{*}=0\quad \text{or} \quad 1- \exp \{r(1-u^{*})\}=0\\    
        & \qquad\qquad\Rightarrow\; \exp \{r(1-u^{*})\}=e^0\quad [\because e^0=1]\\    
        & \qquad\qquad\Rightarrow\; r(1-u^{*})=0\\    
        & \qquad\qquad\Rightarrow\; (1-u^{*})=0,\;\;r>0\\    
        & \qquad\qquad\Rightarrow\; u^{*}=1    
    \end{align*}
    The steady states are \(u^{*}=0\), \(u^{*}=1\)\\
    Here,
    \begin{align*}
        &u_{t+1}=f(u_t;r)=u_t \exp\left\{ r(1-u_t) \right\}\\
        &\qquad\qquad\qquad\;=u_t e^{ r(1-u_t)}\\
        \Rightarrow\; &u_t=e^{r(1-u_t) }+u_t e^{ r(1-u_t)}(-1)r\\
        \Rightarrow\; &(u_t)=(1-ru_t)e^{r(1-u_t) }
    \end{align*}
    The corresponding eigenvalues are, \(u^{*}=0,1\), then
    \begin{align*}
        \lambda=f'(u^{*}=0)&=(1-u^{*})e^{r(1-u^{*}) }\\
        &=(1-0)e^{r(1-0) }\\
        &=e^{r }>0\quad [\because r>0]
    \end{align*}
    when, \(u^{*}=1\), then,
    \begin{align*}
        \lambda&=f'(u^{*}=1)\\
        &=(1-r)e^{r(1-1) }\\
        &=(1-r) e^{0 }>0\\
        &=(1-r)
    \end{align*}
    Hence, \(u^{*}=0\) is unstable and \(u^{*}=1\) is stable for \(0<r<2\).
\end{soln}
\begin{prob}
    Suppose that a certain population obey the Verhulst model with intrinsic growth rate \(r\) and carrying capacity \(K\). Find a complete solution of the model. Obtain a formula for the time \(t\) where the population size \(p(t)=\beta K\), given that initial population \(p(0)=\alpha K\) with \(0<\alpha<\beta<1\).
\end{prob}
\begin{soln}
    The considered population obeys the Verhulst model with intrinsic growth rate \(r\) and carrying capacity \(k\). So the model is,
    \begin{equation}
        \ddt{N}=rN\left( 1-\frac{N}{K} \right) \label{eq:7.1}
    \end{equation}
    where, \(N=N(t)\) is the population size at time \(t\) and \(r\) is the intrinsic growth and \(k\) is the carrying capacity.\\
    From \eqref{eq:7.1}
    \begin{align*}
        &\frac{\D N}{N(K-N)}=\frac{r}{K} \D t\\
        \Rightarrow\; &\frac{1}{K}\left( \frac{1}{N}+\frac{1}{K-N} \right) \D N=\frac{r}{K} \D t\\
        \intertext{Taking integration,}
        \Rightarrow\; &\frac{1}{K}\left\{ \ln {N}+\ln(K-N) \right\} =\frac{r}{K}t+A\quad\text{ [Where \(A\) is an arbitrary constant]}\\
        \Rightarrow\; &\frac{1}{K}\ln \left( \frac{N}{K-N} \right) =\frac{1}{K} rt+A\numberthis \label{eq:7.2}
    \end{align*}
    If initially at \(t=0\), \(N=N(0)=N_0\), then 
    \[\frac{1}{K}\ln \left( \frac{N_0}{K-N_0} \right) =A\]
    putting the value of \(A\) in \eqref{eq:7.2}, we get
    \begin{align*}
        &\frac{1}{K}\ln \left( \frac{N}{K-N}\times\frac{K-N_0}{N_0}\right) =\frac{1}{K} rt\\
        \Rightarrow\; &\frac{KN-NN_0}{KN_0-NN_0}=e^{rt}\\
        \Rightarrow\; &KN-NN_0=KN_0e^{rt}-NN_0e^{rt}\\
        \Rightarrow\; &KN-NN_0+NN_0e^{rt}=KN_0e^{rt}\\
        \Rightarrow\; &N(K-N_0+N_0e^{rt})=KN_0e^{rt}\\
        \Rightarrow\; &N=\frac{KN_0e^{rt}}{K-N_0+N_0e^{rt}}\\
        \therefore\; &N(t)=\frac{KN_0}{N_0+(K-N_0)e^{-rt}}\numberthis\label{eq:7.3}
    \end{align*}
    Which is the complete solution of the model.\\

    \emph{2nd part:} Let \(N(t)=P(t)\). If \(P(0)=\alpha K\), then \eqref{eq:7.3} gets,
    \begin{align*}
        \alpha K&=\frac{kP_0}{P_0+(K-P_0)e^0}\\
        &=\frac{kP_0}{P_0+K-P_0}\\
        &=\frac{kP_0}{K}\\
        &=P_0\\
        \therefore\;\alpha\,K&=P_0\numberthis \label{eq:7.4}
    \end{align*}
    Thus from equation \eqref{eq:7.3}
    \begin{align*}
        &P(t)=\frac{K\,P_0}{P_0+(K-P_0)e^{-rt}}\\
        \Rightarrow\;&\beta\,K=\frac{K\,P_0}{P_0+(K-P_0)e^{-rt}}\\
        \Rightarrow\;&\beta\,K=\frac{K\,P_0}{\alpha\,K+(K-\alpha\,K)e^{-rt}}\\
        \Rightarrow\;&\beta\,\alpha\,K+(\beta\,K-\alpha\,\beta\,K)e^{-rt}=\alpha\,K\;\;[\text{dividing both sides by }K]\\
        \Rightarrow\;&e^{-rt}=\frac{\alpha\,K-\beta\,\alpha\,K}{\beta\,K-\alpha\,\beta\,K}\\
        \Rightarrow\;&e^{-rt}=\frac{K(\alpha-\beta\,\alpha)}{K(\beta-\alpha\,\beta)}\\
        \Rightarrow\;&e^{rt}=\frac{\beta-\alpha\,\beta}{\alpha-\alpha\,\beta}\\
        \Rightarrow\;&e^{rt}=\frac{\alpha\,\beta(\frac{1}{\alpha}-1)}{\alpha\,\beta(\frac{1}{\beta}-1)}\\
        \Rightarrow\;&{rt}=\ln \left\{\frac{1-\frac{1}{\alpha}}{1-\frac{1}{\beta}}\right\}\\
        \therefore\;&{t}=\frac{1}{r}\ln \left\{\frac{1-\frac{1}{\alpha}}{1-\frac{1}{\beta}}\right\}
    \end{align*}
    Which is the required formula for time \(t\).
\end{soln}
\begin{prob}
    Determine the outcome of a competition modeled by the system.
    \begin{align*}
        \ddt{x}&=x(80-3x-2y)\\
        \ddt{y}&=y(80-x-y)
    \end{align*}
\end{prob}
\begin{soln}
    Here, \(\begin{aligned}[t]
        &f(x,y)=80-3x-2y\\
        \therefore\;&fy=-2<0
    \end{aligned}\)  and \(\begin{aligned}[t]
        &g(x,y)=80-x-y\\
        \therefore\;&gx=-1<0
    \end{aligned}\)\\

    
    Since, \(fy<0\), \(gx<0\), hence the given system is competition model.\\
    For equilibrium,
    \begin{align*}
        &x(80-3x-2y)=0\\
        &y(80-x-y)=0
    \end{align*}
    A consistence equilibrium is found by solving this system.
    \begin{align}
        (80-3x-2y)=0\qquad\Rightarrow\;& 3x+2y=80\label{eq:8.2}\\
        (80-x-y)=0\qquad\Rightarrow\;& x+y=80\nonumber\\
        \therefore\;& y=80-x\label{eq:8.3}
    \end{align}
    From \eqref{eq:8.2}
    \begin{align*}
        &3x+2(80-x)=80\\
        \Rightarrow\;&3x-2x=80-160\\
        \Rightarrow\;&x=-80\\
        \therefore\;&y=80+80=160
    \end{align*}
    \(\therefore\) The coenistence equilibrium \((-80,160)\), this is not possible, because the species \(x\) can not be negative.\\
    When \(x=0\), then \(y=80\)\\
    When \(y=0\), then \(3x=80\) \(\Rightarrow\;x=\frac{80}{3}\)\\


    \(\therefore\) The equilibrium points are \((0,0)\), \((0,80)\), \((\frac{80}{3},x)\).\\

    Thus, there is no equilibrium within \(x>0,\;y>0\).\\

    Here, \(\begin{aligned}[t]
        F(x,y)&=x(80-3x-2y)\\
        G(x,y)&=y(80-x-y)
    \end{aligned}\)\\
    \begin{align*}
        \therefore\;&F_x=80-6x-2y, &F_y&=-2x\\
        &G_x=-y, &G_y&=80-x-2y
    \end{align*}


    For \((0,0)\): \(\begin{aligned}[t]
        F_x(0,0)&=80,\\
        F_y(0,0)&=0
    \end{aligned}\)\hspace{1cm}\(\begin{aligned}[t]
        G_x(0,0)&=0,\\
        G_y(0,0)&=80
    \end{aligned}\)

    % \begin{align*}
        % F_x(0,0)&=80, &F_y(0,0)&=0\\
        % G_x(0,0)&=0, &G_y(0,0)&=80
    % \end{align*}
    \(\therefore\;\) The community matrix is \(A=\begin{pmatrix}
        80&0\\
        0&80
    \end{pmatrix}\)
    \begin{align*}
        \therefore\;&\abs{A}=6400>0\\
        &\text{tra }A=80+80=160>0\\
        &\Delta =(80-80)^2+4\cdot0\cdot0=0
    \end{align*}
    So the equilibrium is unstable point \((0,0)\) is node.\\


    For \((0,80)\): \(\begin{aligned}[t]
        F_x(0,80)&=80-2\cdot80=-80,\\
        F_y(0,80)&=0
    \end{aligned}\)\hspace{1cm}\(\begin{aligned}[t]
        G_x(0,80)&=-80,\\
        G_y(0,80)&=80-0-2\cdot80=-80
    \end{aligned}\)


    \(\therefore\;\) The community matrix is \(A=\begin{pmatrix}
        -80&0\\
        -80&-80
    \end{pmatrix}\)
    \begin{align*}
        \therefore\;&\abs{A}=6400>0\\
        &\text{tra }A=-80-80=-160<0\\
        &\Delta =(-80+80)+4\cdot0\cdot0=0
    \end{align*}
    So the equilibrium is asymptotically stable node point.

    From Dulac's criterion with \(\beta(x,y)=\frac{1}{xy}\)\\
    \[\therefore\; \pardx{}\left( \frac{80-3x-2y}{y} \right)+\pardy{}\left( \frac{80-x-y}{x} \right)=\frac{-3}{y}-\frac{1}{x}<0\]
    So there is no periodic orbit, this means that every point approaches \((0,80)\).\\
    
    
    For \((\frac{80}{3},0)\): \(\begin{aligned}[t]
        F_x(\frac{80}{3},0)&=-80,\\
        F_y(\frac{80}{3},0)&=\frac{-160}{3}
    \end{aligned}\)\hspace{1cm}\(\begin{aligned}[t]
        G_x(\frac{80}{3},0)&=0,\\
        G_y(\frac{80}{3},0)&=\frac{160}{3}
    \end{aligned}\)


    \(\therefore\;\) The community matrix is \(A=\begin{pmatrix}
        -80&\frac{-160}{3}\\
        0&\frac{160}{3}
    \end{pmatrix}\)
    \begin{align*}
        \therefore\;&\abs{A}=\frac{-3200}{3}<0\\
        &\text{tr}(A)=\frac{160}{3}-80=\frac{-80}{3}<0\\
        &\Delta =(-80-\frac{160}{3})+4\cdot0<0
    \end{align*}
    So the equilibrium point is unstable saddle point and every orbit approaches \((\frac{80}{3},0)\).
\end{soln}
\begin{prob}
    The following two-dimensional NLODE has been proposed as a model for call differentiation.
    \begin{align*}
        \ddt{x}&=y-x\\
        \ddt{y}&=\frac{5x^2}{4+x^2}-y
    \end{align*}
    \begin{enumerate}[label=(\roman*)]
        \item Sketch the graph \(y=x\), \(y=\frac{5x^2}{4+x^2}\) in the positive quadrant of the \((x,y)\) plane.
        \item Determine the equilibrium points.
        \item Determine the local stability of each positive equilibrium point and classify the equilibrium points.
    \end{enumerate}
\end{prob}
\begin{soln}
    Given, \(\begin{aligned}[t]
        y&=x\\
        y&=\frac{5x^2}{4+x^2}
    \end{aligned}\)\\
    For \(y=\frac{5x^2}{4+x^2}\)
    \begin{table}[H]
        \begin{minipage}{.4\linewidth}
            \centering
            \begin{tabular}{|c|c|c|c|c|c|c|c|}
                \hline
                \(x\)&0&0.5&1&2&3&4&5\\
                \hline
                \(y\)&0&0.3&1&2.5&3.4&4&4.3\\\hline
            \end{tabular}
        \end{minipage}\quad
        \begin{minipage}{.5\linewidth}
            \begin{figure}[H]
                \centering
                \import{../tikz/}{graph-9.tikz}
            \end{figure}
        \end{minipage}
    \end{table}
    
    
    (ii) The equilibrium points are the solution of
    \begin{equation}
        y-x=0\label{eq:9.3}
    \end{equation}
    \begin{equation}
        \frac{5x^2}{4+x^2}-y=0\label{eq:9.4}
    \end{equation}
    putting \(x\) in \eqref{eq:9.4}, we get
    \begin{align*}
        &\frac{5x^2}{4+x^2}-x=0\\
        \Rightarrow\;&5x^2-4x-x^3=0\\
        \Rightarrow\;&x(x^2-5x+4)=0\\
        \Rightarrow\;&x(x^2-4x-x+4)=0\\
        \Rightarrow\;&x\{x(x-4)-1(x-4)\}=0\\
        \Rightarrow\;&x(x-4)(x-1)=0\\
        \therefore\;&x=0,\;1\;4
    \end{align*}
    So the equilibrium points are \((0,0)\), \((1,1)\), \((4,4)\).

    (iii) Here, \[F(x,y)=y-x,\qquad G(x,y)=\frac{5x^2}{4+x^2}-y\]
    \begin{align*}
        Fx&=-1, & Gx&=\frac{(4+x^2)10x-5x^2(2x)}{(4+x^2)^2}\\
        Fy&=-1, & &=\frac{40x+10x^3-10x^3}{(4+x^2)^2}\\
        & & &=\frac{40x}{(4+x^2)^2}\\
        & & Gy&-1
    \end{align*}
    The linearization at an equilibrium \((x_{\infty},y_\infty)\) is
    \begin{align*}
        u'&=Fx(x_\infty,y_\infty)u+Fy(x_\infty,y_\infty)v\\
        v'&=Gx(x_\infty,y_\infty)u+Gy(x_\infty,y_\infty)v
    \end{align*}
    Thus,
    \begin{align*}
        u'&=-u+v\\
        v'&=\frac{40x_\infty}{(4+x_\infty^2)^2}u-v
    \end{align*}
    For equilibrium point \((0,0)\) with linearization
    \begin{align*}
        u'&=-u+v\\
        v'&=-v
    \end{align*}
    For equilibrium point \((1,1)\) with linearization
    \begin{align*}
        u'&=-u+v\\
        v'&=\frac{40}{(4+1)^2}u-v=\frac{8}{5}u-v
    \end{align*}
    For equilibrium point \((4,1)\) with linearization
    \begin{align*}
        u'&=-u+v\\
        v'&=\frac{40\cdot 4}{(4+4)^2}u-v=\frac{2}{5}u-v
    \end{align*}
    The community matrix at the equilibrium point \((x_\infty,y_\infty)\) is,
    \begin{align*}
        A&=\begin{bmatrix}
            Fx & Fy \\
            Gx & Gy
        \end{bmatrix}\\
        &=\begin{bmatrix}
            -1 & 1 \\
            \displaystyle\frac{40x_\infty}{(4+x_\infty^2)^2} & -1
        \end{bmatrix}
    \end{align*}

    \emph{For equilibrium point \((0,0)\):}\\
    The community matrix, \(A_1=\begin{bmatrix}
        -1 & 1 \\
        0 & -1
    \end{bmatrix}=1>1\)
    \(\therefore \, \abs{A_1}=1>0\) and \(\text{tr }(A)=-1-1=-2<0\)\\
    The critical point \((0,0)\) is asymptotically stable.\\
    The characteristic equation is,
    \begin{align*}
        &\lambda^2-(-1-1)\lambda+1-0=0\\
        \Rightarrow\;&\lambda^2+2\lambda+1=0\\
        \Rightarrow\;&(\lambda+1)^2=0\\
        \Rightarrow\;&\lambda=-1,-1
    \end{align*}
    Since, the roots are real, equal and both have \(-\)ve sign. Thus, the point \((0,0)\) is an asymptotically stable node.


    \emph{For equilibrium point \((1,1)\):}
    \begin{align*}
        Fx(1,1)&=-1, &Fy(1,1)&=1\\
        Gx(1,1)&=\frac{40}{25}=\frac{8}{5}, &Gy(1,1)&=-1
    \end{align*}
    The community matrix, \(A_2=\begin{bmatrix}
        -1 & 1 \\
        \frac{8}{5} & -1
    \end{bmatrix}\)
    \(\therefore \, \abs{A_2}=\frac{-3}{5}<0\) and \(\text{tr }(A_2)=-1-1=-2<0\)\\
    Hence, \((1,1)\) is unstable saddle point.\\
    The characteristic equation is,
    \begin{align*}
        &\lambda^2-5(-1-1)\lambda+1-\frac{8}{5}=0\\
        \Rightarrow\;&5\lambda^2+10\lambda-3=0\\
        \Rightarrow\;&\lambda=\frac{-10\pm \sqrt{100+60}}{2\cdot5}\\
        \Rightarrow\;&\lambda=\frac{-10\pm \sqrt{10}}{10}
    \end{align*}
    Since, the roots are real, unequal and opposite sign. Thus, the equilibrium point \((0,0)\) is an unstable node.


    \emph{For equilibrium point \((4,4)\):}
    \begin{align*}
        Fx(4,4)&=-1, &Fy(4,4)&=1\\
        Gx(4,4)&=\frac{40\cdot4}{(4+4)^2}=\frac{2}{5}, &Gy(4,4)&=-1
    \end{align*}
    The community matrix, \(A_3=\begin{bmatrix}
        -1 & 1 \\
        \frac{2}{5} & -1
    \end{bmatrix}\)
    \(\therefore \, \abs{A_3}=\frac{3}{5}>0\) and \(\text{tr }(A_3)=-1-1=-2<0\)\\
    Hence, \((4,4)\) is asymptotically stable point.\\
    The characteristic equation is,
    \begin{align*}
        &\lambda^2+2\lambda+1-\frac{2}{5}=0\\
        \Rightarrow\;&5\lambda^2+10\lambda+3=0\\
        \Rightarrow\;&\lambda=\frac{-10\pm \sqrt{100-60}}{2\cdot5}\\
        \Rightarrow\;&\lambda=\frac{-10\pm \sqrt{40}}{10}
    \end{align*}
    Since, the roots are real, unequal and both have negative sign. Thus, the equilibrium point \((4,4)\) is an asymptotically stable node.
\end{soln}
\end{document}