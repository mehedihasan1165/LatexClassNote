\documentclass[../main-sheet.tex]{subfiles}
\usepackage{../style}
\graphicspath{ {../img/} }
\backgroundsetup{contents={}}
\begin{document}
\chapter{Equilibrium and Stability}
\begin{prob}
    Define and discuss all the equilibria and linearized of the model of two interacting species. Also find the community matrix of each possible equilibria.
\end{prob}
\begin{soln}
    We will consider populations of two interacting species with population size \(x(t)\) and \(y(t)\) respectively.

    We will assume that \(x(t)\) and \(y(t)\) are continuously differentiable function of \(t\) whose derivatives are function of two population sizes at the same time.\\
    Thus our model will be systems of 1sr order differential equations
    \begin{equation}
        \begin{rcases}
            x'=F(x,y)\qquad\\
            y'=G(x,y)\qquad
        \end{rcases}
        \label{eq:sh2.1.1}
    \end{equation}
    An equilibrium is a solution \((x_\infty, y_\infty)\) of the pair of equations, \(F(x_\infty, y_\infty)=0,\,G(x_\infty, y_\infty)=0\).\\
    Thus an equilibrium is a constant solution of the system of differential equation. Geometrically, an equilibrium is point in the phase plane that is the orbit of a constant solution.
    
    If \((x_\infty, y_\infty)\) is an equilibrium, we make the change of variables, \(u=x-x_\infty\), \(v=y-y_\infty\) obtaining the system
    \begin{align*}
        u'=F(x_\infty+u, y_\infty+v)\\
        v'=G(x_\infty+u, y_\infty+v)
    \end{align*}
    Using Taylor's theorem for function of two variables, we may write
    \begin{align*}
        F(x_\infty+u, y_\infty+v)&=F(x_\infty, y_\infty)+F_x(x_\infty, y_\infty)u+F_y(x_\infty, y_\infty)v+h_1\\
        G(x_\infty+u, y_\infty+v)&=G(x_\infty, y_\infty)+G_x(x_\infty, y_\infty)u+G_y(x_\infty, y_\infty)v+h_2
    \end{align*}
    where \(h_1 \) and  \(h_2\) are functions that are small for small \(uv\) in the sense that
    \[
        \lim_{\substack{u\to0\\ v\to0}}\frac{h_1(u,v)}{\sqrt{u^2+v^2}}=\lim_{\substack{u\to0\\ v\to0}}\frac{h_2(u,v)}{\sqrt{u^2+v^2}}=0
        \]
    The linearized of the system obtained by using, \(F(x_\infty, y_\infty)=0\), \(G(x_\infty, y_\infty)=0\) and neglecting the higher order term \(h_1(u,v)\) and \(h_2(u,v)\) is defined to be the 2-D linear system.
    \begin{equation}
        \begin{rcases}
            u'=F_x(x_\infty, y_\infty)u+F_y(x_\infty, y_\infty)v\qquad\\
            v'=G_x(x_\infty, y_\infty)u+G_y(x_\infty, y_\infty)v
        \end{rcases}
        \label{eq:sh2.1.2}
    \end{equation}
    The coefficient matrix of the system \eqref{eq:sh2.1.2}
    \[\begin{pmatrix}
        F_x(x_\infty, y_\infty)&F_y(x_\infty, y_\infty)\\
        G_x(x_\infty, y_\infty)&G_y(x_\infty, y_\infty)
    \end{pmatrix}\]
    is called the community matrix of the system at the equilibrium \((x_\infty, y_\infty)\).\\
    
    Again, consider the system,
    \begin{align*}
        x'&=xf(x,y)\\
        y'&=yg(x,y)
    \end{align*}
    so that \(f(x,y)\) and \(g(x,y)\) are the per capita growth rates of the two species. The community matrix at the equilibrium then has the form
    \[\begin{pmatrix}
        x_\infty f_x(x_\infty, y_\infty)+f(x_\infty, y_\infty)&x_\infty f_y(x_\infty, y_\infty)\\
        y_\infty g_x(x_\infty, y_\infty)&y_\infty g_y(x_\infty, y_\infty)+g(x_\infty, y_\infty)
    \end{pmatrix}\]
    There are four distinct kinds of possible equilibria as follows:
    \begin{enumerate}[label=(\roman*)]
        \item \((0,0)\) the community matrix
        \[\begin{pmatrix}
            f(0,0)&0\\
            0&g(0, 0)
        \end{pmatrix}\]
        \item \((k,0)\) with \(k>0\), \(f(k,0)=0\) having community matrix
        \[\begin{pmatrix}
            kf_x(0,0)&kf_y(k,0)\\
            0&g(k, 0)
        \end{pmatrix}\]
        \item \((0,M)\) with \(M>0\), \(g(0,M)=0\) having community matrix
        \[\begin{pmatrix}
            f(0,M)&0\\
            Mg_x(0,M)&Mg_y(0,M)
        \end{pmatrix}\]
        \item \((x_\infty, y_\infty)\) with \(x_\infty>0,\,y_\infty>0\), \(f(x_\infty, y_\infty)=0\), \(g(x_\infty, y_\infty)=0\) having community matrix
        \[\begin{pmatrix}
            x_\infty f_x(x_\infty, y_\infty)&x_\infty f_y(x_\infty, y_\infty)\\
            y_\infty g_x(x_\infty, y_\infty)&y_\infty g_y(x_\infty, y_\infty)
        \end{pmatrix}\]
        The term \(F_x(x_\infty, y_\infty)\) and \(g_y(x_\infty, y_\infty)\) in the community matrix are self-regulating terms which are normally non-positive.

        The term \(f_y(x_\infty, y_\infty)\) and \(g_x(x_\infty, y_\infty)\) are interacting term.
        \begin{itemize}
            \item If both interacting terms are \(-\)ve the two species are said to be competition.
            \item If there is one \(+\)ve and one \(-\)ve interaction term, the two species are said to be in a predator-prey solution.
            \item If both interacting terms are \(+\)ve is called mutualistic.
        \end{itemize}
    \end{enumerate}
\end{soln}


An equilibrium \((x_\infty, y_\infty)\) is said to be stable if every solution \((x(t),y(t))\) with \(x(0),y(0)\) sufficiently close to the equilibrium remains close to the equilibrium for all \(t\geq0\).\\


An equilibrium \((x_\infty, y_\infty)\) is said to be asymptotically stable if it is stable and if solution with \(x(0),y(0)\) sufficiently close to the equilibrium tends to the equilibrium as \(t\to \infty\).

\begin{thm}
    If \((x_\infty, y_\infty)\) is an equilibrium of the system \(x'=F(x,y)\), \(y'=G(x,y)\) and if all eigenvalues of the coefficient of the linearized of this equilibrium have \(-\)ve real part. Specifically if,
    \begin{align*}
        &\text{tr } A(x_\infty, y_\infty)=F_x(x_\infty, y_\infty)+G_y(x_\infty, y_\infty)<0\\
        &\det A(x_\infty, y_\infty)=F_xG_y-F_yG_x >0
    \end{align*}
    then the equilibrium \((x_\infty, y_\infty)\) is asymptotically stable.
    \begin{align*}
        \Delta&=(a+d)^2-4(ad-bc)\\
        &=(a-d)^2+4(bc)<0
    \end{align*}
    \begin{enumerate}
        \item If \(\det A=(ad-bc)<0\), the origin is a saddle point.
        \item If \(\det A >0\) and tr \(A=a+d<0\), the origin is asymptotically stable, a node if \(\Delta\geq0\) and spiral point if \(\Delta<0\).
        \item If \(\det A >0\) and tr \(A>0\), the origin is unstable, a node if \(\Delta\geq0\) and spiral point if \(\Delta<0\).
        \item If \(\det A >0\) and tr \(A=0\), the origin is center.
    \end{enumerate}
\end{thm}
\begin{prob}
    Determine whether each equilibrium of the system,
    \begin{align*}
        x'&=y\\
        y'&=2(x^2-1)y-x
    \end{align*}
    is asymptotically stable or not.
\end{prob}
\begin{soln}
    The equilibrium are the solution of \(x'=0\) and \(y'=0\).\\
    Then \(y=0\) and \(2(x^2-1)y-x=0\)\\
    Thus the only equilibrium point is \((0,0)\).\\
    Now let us consider,
    \begin{align*}
        F(x,y)&=y\\
        G(x,y)&=2(x^2-1)y-x
    \end{align*}
    then \begin{align*}
        F_x&=0, & F_y&=1\\
        G_x&=2\cdot 2xy-1=4xy-1, & G_y&=2(x^2-1)
    \end{align*}
    So the community matrix is
    \[\begin{pmatrix}
        0 &1\\
        4xy-1 &2(x^2-1)
    \end{pmatrix}\]
    For equilibrium point \((0,0)\) the community matrix is,
    \[\begin{pmatrix}
        0 &1\\
        -1 &-2
    \end{pmatrix}\]
Here,
\begin{align*}
    \det A&=1>0\\
    \text{trace }A&=-2<0
\end{align*}
Thus, the equilibrium point \((0,0)\) is asymptotically stable.
\end{soln}
\begin{prob}
    Determine the qualitative behavior of solutions of the system
    \begin{align*}
        x'&=x\left( 1-\frac{x}{30} \right)-\frac{xy}{x+10}\\
        y'&=y\left( \frac{x}{x+10}-\frac{1}{3} \right)
    \end{align*}
\end{prob}
\begin{soln}
    Equilibria are the solution of the pair of equations
    \begin{align}
        &x\left( 1-\frac{x}{30} \right)-\frac{xy}{x+10}=0\label{eq:sh2.2.1}\\
        &y\left( \frac{x}{x+10}-\frac{1}{3} \right)=0\label{eq:sh2.2.2}
    \end{align}
    there is an equilibrium (0,0).\\
    If \(y=0\), then equation \eqref{eq:sh2.2.1} gives, \(x\left( 1-\frac{x}{30} \right)=0\)\\
    \[\therefore\;\;x=0\quad\text{ or }\quad 1-\frac{x}{30}=0\;\;\Rightarrow\;\;x=30\]
    \(\therefore \) The 2nd equilibrium is \((30,0)\)\\
    If \(x\neq 0\), \(y\neq 0\), we must solve,
    \begin{align}
        &1-\frac{x}{30}=\frac{xy}{x+10}\label{eq:sh2.2.3}\\
        \intertext{and}
        &\frac{x}{x+10}=\frac{1}{3}\label{eq:sh2.2.4}
    \end{align}
    From \eqref{eq:sh2.2.4},
    \begin{align*}
        &\frac{x}{x+10}=\frac{1}{3}\\
        \Rightarrow\;\;&3x=x+10\\
        \Rightarrow\;\;&2x=10\\
        \Rightarrow\;\;&x=5
    \end{align*}
    From \eqref{eq:sh2.2.3},
    \begin{align*}
        &1-\frac{5}{30}=\frac{y}{5+10}\\
        \Rightarrow\;\;&y=12.5
    \end{align*}
    \(\therefore\;\) The 3rd equilibrium is \((5,12.5)\).\\
    Let, \begin{align*}
        F(x,y)&=x\left( 1-\frac{x}{30} \right)-\frac{xy}{x+10}=\left( x-\frac{x^2}{30} \right)-\frac{xy}{x+10}\\
        G(x,y)&=y\left( \frac{x}{x+10}-\frac{1}{3} \right)
    \end{align*}
    Then,
    \begin{align*}
        F_x&=\left( 1-\frac{2x}{30} \right)-\frac{(x+10)y-xy\cdot 1}{(x+10)^2}\\
        &=\left( 1-\frac{x}{15} \right)-\frac{10y}{(x+10)^2}\\
        F_y&=-\frac{x}{x+10}\\
        G_x&=y\left[ \frac{x+10-x}{(x+10)^2} \right]=\frac{10y}{(x+10)^2}\\
        G_y&=\left( \frac{x}{x+10}-\frac{1}{3} \right)
    \end{align*}
    The community matrix,
    \[\begin{pmatrix}
        1-\frac{x}{15}-\frac{10y}{(x+10)^2} &\frac{-x}{x+10}\\
        \frac{10y}{(x+10)^2} &\frac{x}{x+10}-\frac{1}{3}
    \end{pmatrix}\]
    The community matrix at \((0,0)\) is 
    \[A=\begin{pmatrix}
        1&0\\
        0&-\frac{1}{3}
    \end{pmatrix}\qquad \det A=-\frac{1}{3}<0\]
    Since, the determinant is negative, hence the equilibrium point is unstable saddle point.

    
    The community matrix at \((30,0)\) is 
    \[B=\begin{pmatrix}
        -1&-\frac{3}{4}\\
        0&-\frac{5}{12}
    \end{pmatrix}\qquad \det B=-\frac{5}{12}<0\]
    Since, the determinant is negative, hence the equilibrium point is unstable saddle point.
    
    The community matrix at \((5,12.5)\) is 
    \[C=\begin{pmatrix}
        \frac{1}{9}&-\frac{1}{3}\\
        \frac{5}{9}&0
    \end{pmatrix}\qquad \det C=\frac{5}{27}>0\quad \text{trace }C=\frac{1}{9}>0\]
    The equilibrium point \((5,12.5)\) is unstable.\\
    Every orbit approaches a periodic orbit around the unstable equilibrium \((5,12.5)\). Thus, the two species co-exist with oscillations.
\end{soln}
\begin{prob}
    Determine the equilibrium behavior of predator-prey system model by,
    \begin{align*}
        x'&=x\left( 1-\frac{x}{30} \right)-\frac{xy}{x+10}\\
        y'&=y\left( \frac{x}{x+10}-\frac{3}{5} \right)
    \end{align*}
\end{prob}
\begin{soln}
    Equilibria are the solution of the pair of equations
    \begin{align*}
        &x\left( 1-\frac{x}{30} \right)-\frac{xy}{x+10}=0\\
        \Rightarrow\;\;&x=0,\qquad1-\frac{x}{30}-\frac{y}{x+10}=0\\
        \intertext{and}
        &y\left( \frac{x}{x+10}-\frac{1}{3} \right)=0\\
        \Rightarrow\;\;&\frac{x}{x+10}-\frac{1}{3} =0,\qquad y=0\\
        \Rightarrow\;\;&\frac{5x-3x-30}{5(x+10)}=0\\
        \Rightarrow\;\;&x=0\\
    \end{align*}
    Thus the equilibrium points are (0,0), (30,0), (15,12.5)\\
    Here, \begin{align*}
        F(x,y)&=x\left( 1-\frac{x}{30}-\frac{y}{x+10} \right)= x-\frac{x^2}{30}-\frac{xy}{x+10}\\
        G(x,y)&=y\left( \frac{x}{x+10}-\frac{3}{5} \right)
    \end{align*}
    Then,
    \begin{align*}
        F_x&=1-\frac{x}{15}-\frac{10y}{(x+10)^2}\\
        F_y&=-\frac{x}{x+10}\\
        G_x&=\frac{10y}{(x+10)^2}\\
        G_y&=\left( \frac{x}{x+10}-\frac{3}{5} \right)
    \end{align*}
    \underline{For \((0,0)\)}:\begin{align*}
        F_x(0,0)&=1, &F_y(0,0)&=0\\
        G_x(0,0)&=0, &G_y(0,0)&=-\frac{3}{5}
    \end{align*}
    The community matrix at \((0,0)\) is 
    \[A=\begin{pmatrix}
        1&0\\
        0&-\frac{3}{5}
    \end{pmatrix}\qquad \det A=-\frac{3}{5}<0\]
    Thus \((0,0)\) is unstable saddle point.\\
    
    
    \underline{For \((30,0)\)}:\begin{align*}
        F_x(30,0)&=-1, &F_y(30,0)&=-\frac{3}{4}\\
        G_x(30,0)&=0, &G_y(30,0)&=\frac{3}{20}
    \end{align*}
    The community matrix at \((30,0)\) is 
    \[A=\begin{pmatrix}
        -1&-\frac{3}{4}\\
        0&\frac{3}{20}
    \end{pmatrix}\qquad \det A=-\frac{3}{20}<0\]
    Thus \((30,0)\) is also unstable saddle point.\\

    \underline{For \((15,12.5)\)}:\begin{align*}
        F_x(15,12.5)&=-\frac{1}{5}, &F_y(15,12.5)&=-\frac{3}{5}\\
        G_x(15,12.5)&=\frac{1}{5}, &G_y(15,12.5)&=0
    \end{align*}
    The community matrix at \((15,12.5)\) is 
    \[A=\begin{pmatrix}
        -\frac{1}{5}&-\frac{3}{5}\\
        \frac{1}{5}&0
    \end{pmatrix}\qquad \det A=\frac{3}{25}>0,\text{ trace }A=-\frac{1}{5}<0\]
    Thus \((15,12.5)\) is asymptotically stability and every orbit approaches the equilibrium.
\end{soln}
\begin{prob}
    Determine the behavior as \(t\to \infty\) of solution in the 1st quadrant of the system.
    \begin{align*}
        \ddt{x}&=x(100-4x-2y)\\
        \ddt{y}&=y(60-x-y)\\
    \end{align*}
\end{prob}
\begin{soln}
    Equilibria are the solution of
    \begin{align*}
        x(100-4x-2y)&=0\\
        y(60-x-y)&=0
    \end{align*}
    then \(x=0\) and \(-4x-2y=-100\;\;\Rightarrow 4x+2y=100\)\\
    and \(y=0\) and \(-x-y=-60\;\;\Rightarrow x+y=60\)\\
    If \(y=0\) then \(4x=100\;\;\Rightarrow x=25\)\\
    If \(x=0\) then \(-y=-60\;\;\Rightarrow y=60\)\\
    then the equilibrium points are \((0,0)\), \((25,0)\), \((0,60)\).\\
    Now,
    \begin{align*}
        4x+2y&=100\\
        x+y&=60
    \end{align*}
    Solving this we get \(x=-70\), \(y=70\).\\
    Another equilibrium point is \((-10,70)\)\\
    The equilibria at \((0,0)\), \((0,60)\), \((25,0)\) but not at \((-10,70)\)\\
    Let \begin{align*}
        F(x,y)&=x(100-4x-2y)\\
        G(x,y)&=y(60-x-y)
    \end{align*}
    The community matrix is 
    \[\begin{pmatrix}
        F_x & F_y\\
        G_x & G_y
    \end{pmatrix}=\begin{pmatrix}
        100-8x-2y & -2x\\
        -y & 60+x-2y
    \end{pmatrix}\]
    \underline{For point \((0,0)\)}: The community matrix is 
    \[A=\begin{pmatrix}
        100&0\\
        0&60
    \end{pmatrix}\qquad \begin{aligned}
        \det A&=6000>0\\
        \text{trace } A&=160>0
    \end{aligned}\]
    Hence, the equilibrium point is unstable.\\


    \underline{For point \((25,0)\)}:
    \begin{align*}
        F_x&=100-8\cdot 25=-100, &F_y&=-2\times 25=-100\\
        G_x&=0, &G_y&=60-25=35
    \end{align*} 
    The community matrix is 
    \[A=\begin{pmatrix}
        -100&-50\\
        0&35
    \end{pmatrix}\qquad \begin{aligned}
        \det A&=-3500<0
    \end{aligned}\]
    Hence, the equilibrium point is unstable saddle point.\\
    
    \underline{For point \((0,60)\)}:
    \begin{align*}
        F_x&=-120, &F_y&=0\\
        G_x&=-60, &G_y&=-60
    \end{align*} 
    The community matrix is 
    \[A=\begin{pmatrix}
        -120&0\\
        -60&-60
    \end{pmatrix}\qquad \begin{aligned}
        \det A&=-1200>0\\
        \text{trace } A&=-80<0
    \end{aligned}\]
    So the equilibrium is asymptotically stable. Since \(\Delta>0\), so \((0,60)\) is a node.

    In order to show every orbit approaches \((0,60)\), we must show that there is no periodic orbits.
    \[\beta(x,y)=\frac{1}{xy},\quad \frac{\partial}{\partial x}\left( \frac{100-4x-2y}{y} \right)+\frac{\partial}{\partial y}\left( \frac{60-x-y}{x} \right)=\frac{-4}{y}-\frac{1}{x}<0\]
\end{soln}
\begin{prob}
    Determine the qualitative behavior of solution (outcome or nature) of the system.
    \begin{align*}
        x'&=x(100-4x-y)\\
        y'&=y(60-x-2y)
    \end{align*}
\end{prob}
\end{document}