\documentclass[../main-sheet.tex]{subfiles}
\usepackage{../style}
\graphicspath{ {../img/} }
\backgroundsetup{contents={}}
\begin{document}
\chapter{Liapunov's Direct Method}
\begin{defn}
    Let \(E(x,y)\) have continuous first partial derivatives at all points \((x,y)\) in a domain \(D\) containing the origin \((0,0)\).
    \begin{enumerate}
        \item The function \(E\) is called positive defined in \(D\) if \(E(0,0)=0\) and \(E(x,y)>0\) for all other points in \((x,y)\) in \(D\).
        \item The function \(E\) is called positive semi-defined in \(D\) if \(E(0,0)=0\) and \(E(x,y)\geq 0\) for all other points in \((x,y)\) in \(D\).
        \item The function \(E\) is called negative defined in \(D\) if \(E(0,0)=0\) and \(E(x,y)< 0\) for all other points in \((x,y)\) in \(D\).
        \item The function \(E\) is called negative semi-defined in \(D\) if \(E(0,0)=0\) and \(E(x,y)\leq 0\) for all other points in \((x,y)\) in \(D\).
    \end{enumerate}
\end{defn}
\section{Liapunov Function}
Consider the non-linear autonomous system,
\begin{equation}
    \begin{rcases}
        \ddt{x}=P(x,y)\quad\\
        \ddt{y}=Q(x,y)
    \end{rcases}
    \label{eq:lia1}
\end{equation}
has isolated critical point at \((0,0)\) and \(P\) and \(Q\) have continuous first partial derivatives for all \((x,y)\). If there exists a differentiable function \(E(x,y)\) such that,
\begin{enumerate}[label=(\roman*)]
    \item \(E(x,y)\) is positive defined and
    \item \(\dot{E}(x,y)\) is negative semi defined.
\end{enumerate}
Then \(E(x,y)\) is called Liapunov function for the system \eqref{eq:lia1} in \(D\).
\section{Three theorem on Liapunov}
\begin{thm}
    Consider the system,
    \begin{align*}
        \ddt{x}&=P(x,y)\\
        \ddt{y}&=Q(x,y)
    \end{align*}
    if
    \begin{enumerate}[label=(\roman*)]
        \item \((0,0)\) is an isolated critical point
        \item \(E(x,y)\) is a Liapunov function
    \end{enumerate}
    Then \((0,0)\) is called stable critical point.
\end{thm}
\begin{thm}
    Consider the system,
    \begin{align*}
        \ddt{x}&=P(x,y)\\
        \ddt{y}&=Q(x,y)
    \end{align*}
    if
    \begin{enumerate}[label=(\roman*)]
        \item \((0,0)\) is an isolated critical point
        \item \(E(x,y)\) is a Liapunov function
        \item \(\dot{E}(x,y)\) is a negative defined
    \end{enumerate}
    Then \((0,0)\) is asymptotically stable.
\end{thm}
\begin{thm}
    Consider the system,
    \begin{align*}
        \ddt{x}&=P(x,y)\\
        \ddt{y}&=Q(x,y)
    \end{align*}
    if there exists a function \(E(x,y)\) such that
    \begin{align*}
        &E(0,0)=0\\
        &E(x,y)>0\quad \text{ for } x\neq 0,\,\,y\neq0
    \end{align*}
    Then \((0,0)\) is unstable.
\end{thm}
\begin{prob}
    For what value of \(A\), \(V(x,y)=x^2+y^2\) is a Liapunov function for the system.
    \begin{align*}
        \ddt{x}&=Ax+xy^2\\
        \ddt{y}&=Ay-yx^2
    \end{align*}
    and discuss the stability of the critical point \((0,0)\).
\end{prob}
\begin{soln}
    We have,
    \begin{align*}
        \ddt{x}&=Ax+xy^2\\
        \ddt{y}&=Ay-yx^2
    \end{align*}
    Again, we have,
    \begin{align*}
        V(x,y)&=x^2+y^2\\
        \dot{V}(x,y)&=2x\dot{x}+2y\dot{y}\\
        &=2x(Ax+xy^2)+2y(Ay-yx^2)\\
        &=2Ax^2-2x^2y^2+2Ay^2-2x^2y\\
        &=2A(x^2+y^2)-4x^2y^2
    \end{align*}
    If \(V(x,y)\) is a Liapunov function then,
    \begin{align*}
        &\dot{V}(x,y)\leq 0\\
        \Rightarrow\;\;&2A(x^2+y^2)-4x^2y^2\leq 0\\
        \Rightarrow\;\;&A\leq \frac{4x^2y^2}{2(x^2+y^2)}
    \end{align*}
    Now, we observe that
    \begin{enumerate}[label=(\roman*)]
        \item \(V\) is a differentiable function of \(x\) and \(y\)
        \item \(V\) is positive defined
        \item \(\dot{V}(x,y)\leq 0\) if \(A\leq \frac{4x^2y^2}{2(x^2+y^2)}\)
    \end{enumerate}
    The critical point \((0,0)\) is stable for the given system for \(A\leq \frac{4x^2y^2}{2(x^2+y^2)}\).
\end{soln}
\begin{prob}
    For the autonomous system 
    \begin{align*}
        \ddt{x}&=-x-y-x^3\\
        \ddt{y}&=x-y-y^3
    \end{align*}
    Construct a Liapunov function of the form \(Ax^2+By^2\) where \(A\) and \(B\) are the constant and use the function to determine the stability of the trivial solution of the system.
\end{prob}
\begin{soln}
    The given autonomous system,
    \begin{align*}
        \ddt{x}&=-x-y-x^3\\
        \ddt{y}&=x-y-y^3
    \end{align*}
    Let us consider a Liapunov function is,
    \[E(x,y)=Ax^2+By^2\]
    which is differentiable function of \(x\) and \(y\).\\
    Now,
    \begin{align*}
        \dot{E}(x,y)&=2Ax\dot{x}+2By\dot{y}\\
        &=2Ax(-x-y-x^3)+2By(x-y-y^3)\\
        &=-2Ax^2-2Axy-2Ax^4+2Bxy-2By^2-2By^4\\
        &=2(Bxy-Axy)-2\left\{ A(x^2+x^4)+B(y^2+y^4) \right\}
    \end{align*}
    For Liapunov function
    \begin{align*}
        & \dot{E}(x,y)\leq 0\\
        \Rightarrow\;\; & 2(Bxy-Axy)= 0\\
        \Rightarrow\;\; & Bxy=Axy\\
        \Rightarrow\;\; & \frac{A}{B}=\frac{1}{1}\\
        \therefore\;\; & A=1\\
        \therefore\;\; & B=1
    \end{align*}
    Hence, \(E(x,y)=x^2+y^2\)\\
    The function \(E\) is defined by \(E(x,y)=x^2+y^2\) is positive defined in every domain \(D\) containing \((0,0)\).\\
    Clearly, \(E(0,0)=0\)\\
    Also, \(\dot{E}(x,y)<0\) for all \((x,y)\)\\
    Hence, \((0,0)\) is asymptotically stable point.
\end{soln}
\begin{prob}
    For the system 
    \begin{align*}
        \ddt{x}&=-x+2x^2+y^2\\
        \ddt{y}&=-y+xy
    \end{align*}
    Construct a Liapunov function of the form \(Ax^2+By^2\) where \(A\) and \(B\) are the constant and use the function to determine whether the critical point \((0,0)\) of the system is asymptotically stable or at least stable.
\end{prob}
\begin{soln}
    The given system is,
    \begin{align*}
        \ddt{x}&=-x+2x^2+y^2\\
        \ddt{y}&=-y+xy
    \end{align*}
Let us consider the Liapunov function,
\[E(x,y)=Ax^2+By^2\]
Now,
\begin{align*}
    \dot{E}(x,y)&=2Ax\dot{x}+2By\dot{y}\\
    &=2Ax(-x+2x^2+y^2)+2By(-y+xy)\\
    &=-2Ax^2+4Ax^3+2Axy^2-2By^2+2Bxy^2\\
    &=x^2(-2A+4Ax)+y^2(2Ax-2B+2Bx)
\end{align*}
For Liapunov function,
\begin{align}
    &\dot{E}(x,y)\leq0\notag\\
    \Rightarrow\;\;&-2A+4Ax=0\label{eq:liap1}\\
    \intertext{and}
    \Rightarrow\;\;&2Ax-2B+2Bx=0\label{eq:liap2}
\end{align}
From \eqref{eq:liap1} we get,
\begin{align*}
    &-2A=-4Ax\\
    \Rightarrow\;\;&1=2x\\
    \Rightarrow\;\;&x=\frac{1}{2}
\end{align*}
From \eqref{eq:liap2} we get,
\begin{align*}
    &2A\frac{1}{2}-2B+2B\frac{1}{2}=0\\
    \Rightarrow\;\;&A-2B+B=0\\
    \Rightarrow\;\;&A-B=0\\
    \Rightarrow\;\;&\frac{A}{B}=\frac{1}{1}
\end{align*}
Now, 
\begin{align*}
    \dot{E}(x,y)&=2x\dot{x}+2y\dot{y}\\
    &=2x(-x+2x^2+y^2)+2y(-y+xy)\\
    &=-2x^2+4x^3+2xy^2-2y^2-2y^2+2xy^2\\
    &=-2(x^2+y^2)+4x^3+4xy^2\\
    &=-2(x^2+y^2)+4x(x^2+y^2)
\end{align*}
Here, \(E(0,0)=0\) and \(\dot{E}(x,y)<0\)\\
Hence, \((0,0)\) is an asymptotically stable point.
\end{soln}
\begin{prob}
    Find the Liapunov function of the dynamical system
    \begin{align*}
        \ddt{x}&=-y-\frac{x}{2}-\frac{x^3}{4}\\
        \ddt{y}&=x-\frac{y}{2}-\frac{y^3}{4}
    \end{align*}
    and examine the stability of \((0,0)\).
\end{prob}
\begin{soln}
    The given system is,
    \begin{equation}
        \begin{rcases}
            \ddt{x}=-y-\frac{x}{2}-\frac{x^3}{4}\quad\\
        \ddt{y}=x-\frac{y}{2}-\frac{y^3}{4}
        \end{rcases}
        \label{eq:liapro1}
    \end{equation}
\end{soln}
\end{document}