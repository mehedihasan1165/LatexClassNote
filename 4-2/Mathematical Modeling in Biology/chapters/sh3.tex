\documentclass[../main-sheet.tex]{subfiles}
\usepackage{../style}
\graphicspath{ {../img/} }
\backgroundsetup{contents={}}
\begin{document}
\begin{prob}
    Find the equilibrium population of
    \[
        \ddt{x}=x(10-y);\quad \ddt{y}=y(10-x)
    \]
    Discuss the stability of the model and sketch some trajectories.
\end{prob}
\begin{soln}
    The equilibrium are the solution of
    \begin{align*}
        x(10-y)=0\qquad\Rightarrow\;&x=0,\quad 10-y=0\\
        &x=0,\quad y=10\\
        y(10-x)=0\qquad\Rightarrow\;&y=0,\quad 10-x=0\\
        &y=0,\quad x=10
    \end{align*}
    The equilibrium points are \((0,0)\), \((10,10)\).\\
    Here,
    \begin{align*}
        F(x,y)=x(10-y),\qquad &G(x,y)=y(10-x)\\
        F_x=10-y,\,\,F_y=-x\qquad &G_x=-y,\,\,G_y=10-x
    \end{align*}
    The community matrix at the equilibrium \((x_\infty,y_\infty)\) is,
    \[
        A=\begin{bmatrix}
        F_x & F_y\\
        G_x & G_y
    \end{bmatrix}=\begin{bmatrix}
        10-y_\infty & -x_infty\\
        -y_infty & 10-x_infty
    \end{bmatrix}
    \]
    The community matrix at \((0,0)\) is,
    \[A_1=\begin{bmatrix}
        10 & 0\\
        0 & 10
    \end{bmatrix}\]
    and the characteristic equation is,
    \begin{align*}
        &\lambda^2-(10+10)\lambda+100=0\\
        \Rightarrow\;\;&\lambda^2-20\lambda+100=0\\
        \Rightarrow\;\;&(\lambda-10)^2=0\\
        \Rightarrow\;\;&\lambda=10,10
    \end{align*}
    Since, the roots are real, equal and has positive sign.

    Hence, \((0,0)\) is an asymptotically unstable node.\\


    The community matrix at \((10,10)\) is,
    \[A_2=\begin{bmatrix}
        0 & -10\\
        -10 & 0
    \end{bmatrix},\qquad \abs{A_2}=-100<0\]
    and the characteristic equation is,
    \begin{align*}
        &\lambda^2-(0)\lambda+(0-100)=0\\
        \Rightarrow\;\;&\lambda^2-100=0\\
        \Rightarrow\;\;&\lambda=10,-10
    \end{align*}
    Since, the roots are real, unequal and has opposite sign.

    Hence, \((10,10)\) is an unstable saddle point.\\
    \emph{2nd part}\\
    Hence,
    \begin{equation}
        \ddx{y}=\frac{y(10-x)}{x(10-y)}\label{eq:q10.1}
    \end{equation}
    The isoclines of \eqref{eq:q10.1} are,
    \[\frac{y(10-x)}{x(10-y)}=c\]
    If \(c=0\), then \(y(10-x)=0\) \(\Rightarrow\;y=0,\,x=10\), \(\theta=\tan^{-1}0=0\)\\
    If \(c=\infty\), then \(x(10-y)=0\) \(\Rightarrow\;x=0,\,y=10\), \(\theta=\tan^{-1} \infty=90^{\circ}\)
    \begin{figure}[H]
        \centering
        \import{../tikz/}{q10.tikz}
    \end{figure}
\end{soln}
\section{Chemostat}
\begin{prob}
    Define and discuss the chemostat.
\end{prob}
\begin{soln}
    A chemostat is a piece of laboratory apparatus used to cultivate bacteria. It consists of a reservoir containing a nutrient, a culture vessel in which the bacteria are cultivated and an output receptacle. Nutrient is pumped from the reservoir to the culture vessel at a constant rate and the bacteria are collected in the receptacle by pumping the contents of the culture vessel at same rate. This process is called a continuous culture of bacteria, in contrast with a batch --------- in which a fixed quantity of nutrient is supplied and bacteria are harvested after a growth period.

    Let \(y\) represent the number of bacteria and \(c\) the concentration of nutrient in the chemostat, both are functions of \(t\).

    Let \(v\) be the volume of the chemostat and \(Q\) be the rate of flow into the chemostat from the nutrient to reservoir and also the rate of flow out from the chemostat. The fixed concentration of nutrient of the reservoir is a constant \(C^{(0)}\).

    We assume that the average per capita bacterial birth rate is a function \(r(c)\) of the nutrient concentration and that the rate of nutrient consumption of an individual bacterium is proportional to \(r(c)\) say \(\alpha r(c)\).

    Then the rate of change of population size is the birth rate \(r(c)y\) of bacteria minus the outflow rate \(\frac{Qy}{v}\). It is convenient to let \(q=\frac{Q}{v}\), so that this outflow rate becomes \(qy\).\\
    The rate of change of nutrient volume is the replenishment rate \(QC^{(0)}\) minus the outflow rate \(Q\) minus the consumption rate \(\alpha r(c)y\).

    This gives the pair of differential equations
    \begin{equation}
        \begin{rcases}
            \displaystyle\ddt{y}=r(c)y-qy\quad\\
            \displaystyle\ddt{cv}=Q(c^{(0)}-c) -\alpha r(c)y\quad   
        \end{rcases}
        \label{eq:chem1}
    \end{equation}
    we divide the 2nd equation by the constant \(v\) and let \(\beta=\frac{\alpha}{v}\) to give the system 
    \begin{equation}
        \begin{rcases}
            \displaystyle y'=r(c)y-qy\quad\\
            \displaystyle c'=q(c^{(0)}-c) -\beta r(c)y\quad   
        \end{rcases}
        \label{eq:chem2}
    \end{equation}
    The system \eqref{eq:chem2} describes the chemostat. Assume that the function \(r(c)\) is zero if \(c=0\) and approaches a limit when \(c\) become large.
    
    The simplest function with these properties is,
    \begin{equation}
        r(c)=\frac{ac}{c+A}\label{eq:chem3}
    \end{equation}
    where \(a\) and \(A\) are constants and this was the choice originally made by Monod.\\
    The explicit chemostat model is now,
    \begin{equation}
        \begin{rcases}
            \displaystyle y'=\frac{acy}{c+A}-qy\quad\\
            \displaystyle c'=q(c^{(0)}-c) -\frac{\beta acy}{c+A}\quad   
        \end{rcases}
        \label{eq:chem4}
    \end{equation}
    where \(a,\,A,\,q\) and \(\beta\) are constants.
    \begin{figure}[H]
        \centering
        \begin{tikzpicture}
            \node[cylinder,draw=black,shape border rotate=90,aspect=2,minimum width = 4cm,minimum height=5cm] at (0,0){};
            \node[cylinder,draw=black,shape border rotate=90,aspect=2,minimum width = 2cm,minimum height=3cm] at (5,-1){};
            \node[cylinder,draw=black,shape border rotate=90,aspect=2,minimum width = 1.2cm,minimum height=3.25cm] at (-1,-.5){};
            \node[cylinder,draw=black,shape border rotate=90,aspect=2,minimum width = .7cm,minimum height=3.25cm] at (1,-.5){};
            \draw (2,1.8)--(5.4,1.8);
            \node[rotate=90] at (-2.4,0.1) {Reservoir};
            \node[] at (5,-2.7) {Receptacle};
            \draw (2,1.2)--(4.6,1.2);
            \draw[dashed] (2,1.5)--(4.5,1.5);
            \draw[-latex] (4.5,1.5)--(4.9,1.5)--(4.9,0.4);
            \draw (4.6,1.2)--(4.6,-0.2);
            \draw (5.4,1.8)--(5.4,-0.2);
            \draw (1,1) arc [start angle=0, end angle=180,x radius=1cm, y radius=1cm];
            \draw[-latex] (-1,1)--(-1,0);
            \draw (2,1.49) arc [start angle=60, end angle=150,x radius=2cm, y radius=1.2cm];
            \end{tikzpicture}
    \end{figure}
\end{soln}
\end{document}