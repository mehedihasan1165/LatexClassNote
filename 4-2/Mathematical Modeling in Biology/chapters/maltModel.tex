\documentclass[../main-sheet.tex]{subfiles}
\usepackage{../style}
\graphicspath{ {../img/} }
\backgroundsetup{contents={}}
\begin{document}
\emph{\underline{Malthusian Model} :} The model is defined by
\[\ddt{N_t}=KN_t\]
The solution of the model is \(N_t=N_0e^{Kt}\)\\
where, \(\begin{aligned}[t]
    N_t&=\text{number of organisms present at time }t\\
    K&=\text{the average number of offspring born per organism in the population per unit time}\\
    t&=\text{the independent variable representing time}
\end{aligned}\)\\


\noindent\emph{\underline{Logistic Model} :} This model is given by equation
\[\ddt{x}=rx\left( 1-\frac{x}{k} \right)\]
where \(\begin{aligned}[t]
    \displaystyle f(x)&=r\left( 1-\frac{x}{k} \right)\\
    r&=\text{constant of proportionality}\\
    k&=\text{carrying capacity}\\
\end{aligned}\)\\
The solution of the model is \[x=\frac{K x_0 e^{rt}}{K-x_0+x_0e^{rt}}\]
\begin{prob}
    A population \(N\) grows accordingly to the differential equation
    \[\ddt{N}=KN\]
    where \(K\) is a positive constant. Determine how long it takes the population to double in size?
\end{prob}
\begin{soln}
    We have,
    \begin{align}
        &\ddt{N}=KN\notag\\
        \Rightarrow\;&\frac{\D N}{N}=K\D t\notag\\
        \intertext{Integrating,}
        \Rightarrow\;&\log N=Kt+\log c\notag\\
        \Rightarrow\;&N=ce^{Kt}\label{eq:mprob1.1}
    \end{align}
    Applying initial population, \(N=N_0\), for \(t=t_0\). From \eqref{eq:mprob1.1}
    \begin{align*}
        &N_0=ce^{Kt_0}\\
        \Rightarrow\;&c=N_0e^{-Kt_0}
    \end{align*}
    Putting the value of \(c\) in \eqref{eq:mprob1.1}, we get,
    \begin{align}
        &N=N_0e^{-kt_0}e^{kt}\notag\\
        \therefore\;&N=N_0e^{k(t-t_0)}\label{eq:mprob1.2}
    \end{align}
    Suppose at time \(t=T\), the population will be double in size i.e., \(N=2N_0\).\\
    Now from \eqref{eq:mprob1.2}
    \begin{align*}
        &2N_0=N_0e^{K(T-t_0)}\\
        \Rightarrow\;&2=e^{K(T-t_0)}\\
        \Rightarrow\;&\log 2={K(T-t_0)}\\
        \Rightarrow\;&T=\frac{1}{K}\log 2+t_0
    \end{align*}
    Which is the required time that will make population double in size.
\end{soln}
\begin{prob}
    World population was estimated to be 1550 million in 1900 and 2500 in 1950. Estimate the population of the world in the year 2000.
\end{prob}
\begin{soln}
    We know, from Malthusian model,
    \begin{equation}
        N=N_0 e^{K(t-t_0)} \label{eq:mprob2.1}
    \end{equation}
    
    
    Here, \(N=2500,\quad N_0=1550,\quad t=1950,\quad t_0=1900\)\\
    From \eqref{eq:mprob2.1}
    \begin{align*}
        &2500=1500 e^{K(1950-1900)}\\
        \Rightarrow\;&\frac{2500}{1500}= e^{50K}\\
        \Rightarrow\;&50K=\log\left( \frac{2500}{1500} \right)=.4780358\\
        \Rightarrow\;&K=0.009
    \end{align*}
    Now let \(N\) be the population in 2000.\\
    Here, \(N=?\), \(N_0=2500\), \(t=2000\), \(t_0=1950\)\\
    Now from \eqref{eq:mprob2.1}
    \begin{align*}
        N&=2500 e^{.009(2000-1950)}\\
        &=2500 e^{.009\times 50}\\
        &=2500 e^{.45}\\
        &=2500 \times 1.5683122\\
        \therefore\,N&=3920.7805 \text{ million}
    \end{align*}
\end{soln}
\begin{prob}
    Assume that the rate of change of human population of the world is proportional to the number of people present at any time and suppose that this population is increasing at the rate of \(2\%\) per year. The world population of 1978 was 4219 million. Calculate the world population of
    \begin{enumerate}[label=(\alph*)]
        \item 1950
        \item 2000
    \end{enumerate}
\end{prob}
\begin{soln}
    \begin{enumerate}[label=(\alph*)]
        \item We know, from Malthusian population model,
        \begin{equation}
            N=N_0e^{K(t-t_0)}\label{eq:mprob3.1}
        \end{equation}
        
        
        Here, \(N_0=4219,\quad t_0=1978,\quad t=1950,\quad K=\frac{2}{100}=0.02\)\\
        From \eqref{eq:mprob3.1}
        \begin{align*}
            N&=4219\times e^{0.02(1950-1978)}\\
            &=4219\times e^{0.02(-28)}\\
            &=4219\times e^{-0.56}\\
            &=4219\times 0.5712\\
            \therefore\,N&=2409 \text{ million}
        \end{align*}
        \item When \(t=2000\) then
        \begin{align*}
            N&=4219\times e^{0.02(2000-1978)}\\
            &=4219\times e^{0.02\times 22}\\
            &=4219\times e^{0.44}\\
            &=4219\times 1.5527072\\
            \therefore\,N&=6550.8717 \text{ million}
        \end{align*}
    \end{enumerate}
\end{soln}
\begin{prob}
    In a certain bacteria culture, the rate of increasing is the number of bacteria is proportional to the number of present.
    \begin{enumerate}[label=(\alph*)]
        \item If the number triples in 5 hrs, how many will be present in 10 hrs.
        \item When will the number present be 10 times the number initially presents?
    \end{enumerate}
\end{prob}
\begin{soln}
    We have the solution of Malthusian population model,
    \begin{equation}
        N=N_0e^{K(t-t_0)}\label{eq:mprob4.1}
    \end{equation}
    
    
    Initially, \(N=N_0,\quad t_0=0\)\\
    Here, \(N(5 \text{ hrs.})=3N_0\)\\
    From \eqref{eq:mprob4.1}
    \begin{align*}
        &3N_0= N_0 e^{5K};\qquad t_0=0,\;\;\;t=5 \text{ hrs.}\\
        \Rightarrow\;&3= e^{5K}\\
        \Rightarrow\;&5K= \log 3\\
        \Rightarrow\;&K= \frac{1}{5}\log 3\\
        \therefore\;&K= 0.2197224
    \end{align*}
    \begin{enumerate}[label=(\alph*)]
        \item Let in 10 hrs the bacteria will be \(P \) time i.e., \(N=PN_0\), \(t=10 \) hrs.
        From \eqref{eq:mprob4.1}
        \begin{align*}
            &PN_0=N_0 e^{.2197224\times 10}\\
            \Rightarrow\;&P=e^{2.197224}\\
            \Rightarrow\;&P=8.9999948\\
            \therefore\;&P\approx 9 \text{ times}
        \end{align*}
        \item Let at time \(t=T\), the number will be 10 times i.e., \(N=10N_0\).
        From \eqref{eq:mprob4.1}
        \begin{align*}
            &10N_0=N_0 e^{.2197224\times T}\\
            \Rightarrow\;&10=e^{2.197224}\\
            \Rightarrow\;&0.2197224\times T=\log 10\\
            \therefore\;&T=10.48\text{ hours.}
        \end{align*}
    \end{enumerate}
\end{soln}
\begin{prob}
    The population of a city increase at a rate proportional to the number of inhabitants present at any time \(t\). If the population of the city was 30,000 in 1970 and 35,000 in 1980. What will be the population in 1990 and 2000?
\end{prob}
\begin{soln}
    We know the solution of Malthusian population model,    
    \begin{equation}
        N=N_0e^{K(t-t_0)}\label{eq:mprob5.1}
    \end{equation}
    
    Here, \(N_0=30000,\quad N=35000,\quad t_0=1970,\quad t=1980\)\\
    From \eqref{eq:mprob5.1}
    \begin{align*}
        &35000=30000 \times e^{K(1980-1970)}\\
        \Rightarrow\;&e^{10K}=\frac{7}{6}\\
        \Rightarrow\;&{10K}=\log\left(\frac{7}{6}\right)\\
        \Rightarrow\;&{K}=\frac{1}{10}\log\left(\frac{7}{6}\right)\\
        \therefore\;&{K}=0.015
    \end{align*}
    Now, \(N(1990)=?\quad N_0=35000,\quad t_0=1980,\quad t=1990\)\\
    From \eqref{eq:mprob5.1}
    \begin{align*}
        N(1990)&=35000\times e^{0.015(1990-1980)}\\
        &=35000\times e^{0.015\times 10}\\
        &=35000\times e^{0.15}\\
        &=35000\times 1.1618342\\
        &=40664.198
    \end{align*}
    Now, \(N(2000)=?\quad N_0=40664.198,\quad t_0=1990,\quad t=2000\)\\
    From \eqref{eq:mprob5.1}
    \begin{align*}
        N(2000)&=40664.198\times e^{0.015(2000-1990)}\\
        &=40664.198\times e^{0.015\times 10}\\
        &=40664.198\times e^{0.15}\\
        &=40664.198\times 1.1618342\\
        &=47245.058
    \end{align*}
\end{soln}
\begin{prob}
    The population of a city satisfies the logistic law \(\ddt{N}=10^{-2}N-10^{-8}N^2\) where \(t\) is measured in years. Given that the population of the city was 100000 in 1980.
    \begin{enumerate}[label=(\alph*)]
        \item Determine the population in the year 2000.
        \item When will the population be 200000?
        \item What would be the maximum population of the city?
    \end{enumerate}
\end{prob}
\begin{soln}
    We have,
    \begin{align}
        &\ddt{N}=10^{-2}N-10^{-8}N^2\notag\\
        \Rightarrow\;&\frac{\D N}{10^{-2}N-10^{-8}N^2}=\D t\notag\\
        \Rightarrow\;&\frac{10^2\D N}{N-10^{-6}N^2}=\D t\notag\\
        \Rightarrow\;&\frac{10^2\D N}{N(1-10^{-6}N)}=\D t\notag\\
        \Rightarrow\;&10^2\left[ \frac{1}{N}+\frac{10^{-6}}{1-10^{-6}N} \right]\D N=\D t\notag\\
        \Rightarrow\;&10^2\left[ \log{N}-\log({1-10^{-6}N}) \right]=t+100\log c\qquad\text{where \(100\log c\) is constant}\notag\\
        \Rightarrow\;&\log\frac{N}{1-10^{-6}N}=\frac{t}{100}+\log c\notag\\
        \Rightarrow\;&\log\frac{N}{1-10^{-6}N}=\log c e^{\frac{t}{100}}\notag\\
        \Rightarrow\;&\frac{N}{1-10^{-6}N}=c e^{\frac{t}{100}}\label{eq:lprob1.1}\\
        \Rightarrow\;&{N}=c e^{\frac{t}{100}}-10^{-6}c e^{\frac{t}{100}}N\notag\\
        \Rightarrow\;&{N}\left( 1+10^{-6}ce^{\frac{t}{100}} \right)=c e^{\frac{t}{100}}\notag\\
        \therefore\;&{N}=\frac{c e^{\frac{t}{100}}}{1+10^{-6}ce^{\frac{t}{100}}}\label{eq:lprob1.2}
    \end{align}
    Again from \eqref{eq:lprob1.1},
    \begin{align*}
         & N=ce^{\frac{t}{100}}-10^{-6}ce^{\frac{t}{100}}N\\
        \Rightarrow\; & Ne^{\frac{-t}{100}}=c-10^{-6}Nc\\
        \Rightarrow\; & c=\frac{N\,e^{\frac{-t}{100}}}{1-10^{-6}N}\numberthis\label{eq:lprob1.3}
    \end{align*}
    using initial condition, \(t=1980\), \(N=100000\) in \eqref{eq:lprob1.3} we get,
    \begin{align*}
        c&=\frac{100000 e^{\frac{-1980}{100}}}{1-10^{-6}\times 100000}\\
        &=\frac{10^{5} e^{-19.8}}{1-10^{-6}\times 10^5}\\
        &=\frac{10^{5} e^{-19.8}}{1-10^{-1}}\\
        &=\frac{10^{6} e^{-19.8}}{9}
    \end{align*}
    putting the value of \(c\) in \eqref{eq:lprob1.2},
    \begin{align*}
        N&=\frac{\frac{10^6 e^{-19.8}}{9}e^{\frac{t}{100}}}{1+10^{-6}\frac{10^6 e^{-19.8}}{9}e^{\frac{t}{100}}}\\
        &=\frac{10^6 e^{\frac{t}{100}-19.8}}{9}\times \frac{9}{9+e^{\frac{t}{100}-19.8}}\\
        &=\frac{10^6 e^{\frac{t}{100}-19.8}}{9+e^{\frac{t}{100}-19.8}}\\
        \text{i.e., }N(t)&=\frac{10^6}{1+9e^{19.8-\frac{t}{100}}}\numberthis\label{eq:lprob1.4}
    \end{align*}
    which represents the population at any time \(t>1980\).
    \begin{enumerate}[label=(\alph*)]
        \item When \(t=2000\) then from \eqref{eq:lprob1.4}
        \begin{align*}
            N(2000)&=\frac{10^6}{1+9e^{19.8-20}}\\
            N(2000)&=\frac{10^6}{1+9e^{-.2}}\\
            N(2000)&=\frac{10^6}{8.3685768}\\
            N(2000)&=\frac{100000}{8.3685768}\\
            N(2000)&=119494.63
        \end{align*}
        \item Let at time \(t=T\), the population will be \(200000\). i.e., \(N(T)=200000=N(t)\).\\
        From \eqref{eq:lprob1.4},
        \begin{align*}
            & 200000=\frac{10^6}{1+9e^{19.8-\frac{T}{100}}}\\
            \Rightarrow\; & 200000+1800000e^{19.8-\frac{T}{100}}=1000000\\
            \Rightarrow\; & 1800000e^{19.8-\frac{T}{100}}=800000\\
            \Rightarrow\; & e^{19.8-\frac{T}{100}}=\frac{1800000}{800000}=\frac{4}{9}\\
            \Rightarrow\; & 19.8-\frac{T}{100}=\log\left(\frac{4}{9}\right)=-0.8109302\\
            \Rightarrow\; & \frac{T}{100}=19.8+0.8109302=20.61093\\
            \Rightarrow\; & T=2061.093\\
            \therefore\; & T\approx 2061
        \end{align*}
        \item \(t\to\infty\) gives the maximum population. Now from \eqref{eq:lprob1.4}
        \begin{align*}
            \lim_{t\to\infty}N(t)&=\lim_{t\to\infty}\frac{10^6}{1+9e^{19.8-\frac{t}{100}}}\\
            &=10^6\\
            &=100000
        \end{align*}
    \end{enumerate}
\end{soln}
\begin{prob}
    The population \(N\) of Natore satisfies the logistic law,
    \[
        \ddt{N}=(0.03)N-3\times 10^{-8}N^2
    \]
    where time \(t\) is measured in years.
    If the population of Natore was \(200000\) in 1980.
    What will be the population in the year 2000?
\end{prob}
\begin{soln}
    We have,
    \begin{align*}
        & \ddt{N}=(.03)N-3\times 10^{-8}N^2\\
        \Rightarrow\; & \ddt{N}=3\times 10^{-2} N-3\times 10^{-8}N^2\\
        \Rightarrow\; & \ddt{N}=3\times 10^{-2}( N-10^{-6}N^2)\\
        \Rightarrow\; & \ddt{N}=3\times 10^{-2}N( 1-10^{-6}N)\\
        \Rightarrow\; & \frac{\D N}{N( 1-10^{-6}N)}=3\times 10^{-2}\D t\\
        \Rightarrow\; & \left[\frac{1}{N}+\frac{10^{-6}}{ 1-10^{-6}N}\right]\D N=3\times 10^{-2}\D t\\
        \intertext{Integrating,}
        \Rightarrow\; & \log{N}-\log(1-10^{-6}N)=\frac{3t}{100}+\log c\\
        \Rightarrow\; & \frac{N}{1-10^{-6}N}=ce^{\frac{3t}{100}}\\
        \Rightarrow\; & N=ce^{\frac{3t}{100}}-10^{-6}ce^{\frac{3t}{100}}N\numberthis\label{eq:logi1.1}\\
        \Rightarrow\; & N\left(1+10^{-6}ce^{\frac{3t}{100}}\right)=ce^{\frac{3t}{100}}\\
        \therefore\; & N=\frac{ce^{\frac{3t}{100}}}{1+10^{-6}ce^{\frac{3t}{100}}}\numberthis\label{eq:logi1.2}\\
        \intertext{Again from \eqref{eq:logi1.1}}
        & N= ce^{\frac{3t}{100}} \left( 1-10^{-6}N \right)\\
        \Rightarrow\; & c= \frac{Ne^{\frac{3t}{100}}}{1-10^{-6}N}\numberthis\label{eq:logi1.3}
    \end{align*}
    Applying initial condition, \(N=200000=2\times 10^{5}\), \(t=1980\) in \eqref{eq:logi1.3} we get,
    \begin{align*}
        c&=\frac{2\times 10^{5} e^{-59.4}}{1-10^{-6}\times 2\times 10^{5}}\\
        &=\frac{5}{2}\times 10^{5} e^{-59.4}\\
        &=\frac{10^{6}e^{-59.4}}{4}
    \end{align*}
    putting the value of \(c\) in \eqref{eq:logi1.2}
    \begin{align*}
        N(t)&=\frac{\frac{10^{6}e^{-59.4}}{4} e^{\frac{3t}{100}}}{1+10^{-6}\frac{10^{6}e^{-59.4}}{4} e^{\frac{3t}{100}}}\\
        \therefore \;N(t)&=\frac{10^{6}e^{\left(\frac{3t}{100}-59.4\right)}}{4+e^{\frac{3t}{100}-59.4}}\numberthis\label{eq:logi1.5}
    \end{align*}
    which represents the population at any time \(t>1980\).\\
    Now, \(t=2000\)\\
    From \eqref{eq:logi1.5}
    \begin{align*}
        N(2000)&=\frac{10^{6}\left(e^{\frac{3\times 2000}{100}-59.4}\right)}{4+e^{\frac{3\times 2000}{100}-59.4}}\\
        &=\frac{10^{6}\times e^{60-59.4}}{4+e^{60-59.4}}\\
        &=312964.76\\
        \therefore\; N(2000) &\approx 312964
    \end{align*}
\end{soln}
\begin{prob}
    The human population of an island obeys the logistic law,
    \[\ddt{N}-(0.0025)N-10^{-8}N^2\]
    If the initial population of the island is 20,000 in 1980, then,
    \begin{enumerate}[label=(\roman*)]
        \item Find the population in 2000.
        \item Find the maximum ultimate population.
        \item When will the population be 40,000.
        \item Modify the model when 100 people leaves the island every year.
    \end{enumerate}
\end{prob}
\begin{soln}
    We have
    \begin{align}
        &\ddt{N}=0.0025N-10^{-8}N^2\notag\\
        \Rightarrow\;\;&\ddt{N}=\frac{25}{10000}N-10^{-8}N^2\notag\\
        \Rightarrow\;\;&\ddt{N}=\frac{N}{400}-10^{-8}N^2\notag\\
        \Rightarrow\;\;&\frac{\D N}{N\left( \frac{1}{400}-N\;10^{-8} \right)}=\D t\notag\\
        \Rightarrow\;\;&\frac{\D N}{\frac{1}{400}N\left( 1-4\times 10^{-6}N \right)}=\D t\notag\\
        \Rightarrow\;\;&\frac{\D N}{N\left( 1-4\times 10^{-6}N \right)}=\frac{\D t}{400}\notag\\
        \Rightarrow\;\;&\left[ \frac{1}{N}+\frac{4\times 10^{-6}}{1-4\times 10^{-6}N} \right] \D N=\frac{\D t}{400}\notag\\
        \Rightarrow\;\;&\log{N}-\log({1-4\times 10^{-6}N})=\frac{t}{400}+\log c\notag\\
        \Rightarrow\;\;&\frac{N}{1-4\times 10^{-6}N}=ce^{\frac{t}{400}}\notag\\
        \Rightarrow\;\;&{N}=ce^{\frac{t}{400}}-4\times 10^{-6}ce^{\frac{t}{400}} N\label{eq:prob:20.1}\\
        \Rightarrow\;\;&{N}=\frac{ce^{\frac{t}{400}}}{1+4\times 10^{-6}ce^{\frac{t}{400}}}\notag\\
        \therefore\;\;&{N}=\frac{c}{4c10^{-6}+e^{-\frac{t}{400}}}\label{eq:prob20.2}
    \end{align}
    From \eqref{eq:prob20.2},
    \begin{align}
        & ce^{\frac{t}{400}}\left( 1-4\times10^{-6}N \right)=N\notag\\
        & c=\frac{Ne^{-\frac{t}{400}}}{1-4\times10^{-6}N}\label{eq:prob20.3}
    \end{align}
    Applying initial condition \(N=20,000\) \(t=1980\) in \eqref{eq:prob20.3}
    \begin{align*}
        c&=\frac{2\times 10^4 e^{-\frac{1980}{400}}}{1-4\times10^{-6}\times 2\times 10^4}\\
        &=\frac{2\times 10^4 e^{-4.95}}{1-8\times10^{-2}}\\
        &=\frac{10^6 e^{-4.95}}{46}
    \end{align*}
    Now \eqref{eq:prob20.2} becomes,
    \begin{align}
        N(t)&=\frac{\frac{10^6 e^{-4.95}}{46}}{4\cdot \frac{10^6 e^{-4.95}}{46}10^{-6}+e^{-t/400}}\notag\\
        &=\frac{10^6 e^{-4.95}}{4e^{-4.95}+46 e^{-t/400}}\notag\\
        \therefore\;\;N(t)&=\frac{10^6}{4+46 e^{4.95-t/400}}\label{eq:prob20.4}
    \end{align}
    Which represents population at any time \(t>1980\).
    \begin{enumerate}[label=(\roman*)]
        \item When \(t=2000\), then
        \begin{align*}
            N(2000)&=\frac{10^6}{4+46 e^{4.95-5}}\\
            &=\frac{10^6}{4+46 e^{-0.05}}\\
            &=\frac{10^6}{4+43.76}\\
            &=\frac{10^6}{47.76}\\
            \therefore\;\;N(2000)&\approx \;20938
        \end{align*}
        \item The population will be maximum when \(t\to \infty\).\\ From \eqref{eq:prob20.4}
        \begin{align*}
            \lim_{t\to\infty}N(t)&=\lim_{t\to\infty}\frac{10^6}{4+46e^{4.95-t/400}}\\
            &=\frac{10^6}{4}\\
            &=250000
        \end{align*}
        \item Let at time \(t=T\), the population will be 40,000. i.e., \(N(t)=40,000\).\\ From \eqref{eq:prob20.4}
        \begin{align*}
            &40,000=\frac{10^6}{4+46e^{4.95-T/400}}\\
            \Rightarrow\;\;&4=\frac{10^2}{4+46e^{4.95-T/400}}\\
            \Rightarrow\;\;&184e^{4.95-T/400}=100-16\\
            \Rightarrow\;\;&e^{4.95-T/400}=0.4565217\\
            \Rightarrow\;\;&4.95-\frac{T}{400}=\log(0.4565217)\\
            \Rightarrow\;\;&4.95-\frac{T}{400}=-0.7841189\\
            \Rightarrow\;\;&\frac{T}{400}=5.734119\\
            \Rightarrow\;\;&T \approx\;2293
        \end{align*}
        \item If 100 peoples leaves the island every year then we can show that \(K=100\). Thus, the model will be,
        \[\ddt{N}=-100N-10^{-8}N^2\]
    \end{enumerate}
\end{soln}

\begin{prob}
    If the population of a country double in 50 years, in how many years will it triple under the assumption of the Malthusian model? What will be the population in the year 2000?
\end{prob}
\begin{soln}
    We have from Malthusian model,
    \begin{equation}
        N=N_0e^{K(t-t_0)}\label{eq:probdef2.1}
    \end{equation}
    Suppose at \(t-t_0=50\) years, the population will be double in size. i.e., \(N=2N_0\).\\
    From \eqref{eq:probdef2.1} 
    \begin{align*}
        &2N_0=N_0 e^{K\times 50}\\
        \Rightarrow\;\;& e^{50K}=2\\
        \Rightarrow\;\;& {50K}=\ln 2\\
        \Rightarrow\;\;& k=\frac{1}{50}\ln 2\\
        \therefore\;\;& k=0.01386
    \end{align*}
    For population to be tripled \(N=3N_0\), \(t-t_0=?\)\\
    From \eqref{eq:probdef2.1} 
    \begin{align*}
        &3N_0=N_0 e^{K(t-t_0)}\\
        \Rightarrow\;\;&3= e^{0.01386(t-t_0)}\\
        \Rightarrow\;\;& (t-t_0)=\frac{\ln 3}{0.01386}=79.26\text{ years}
    \end{align*}
    After 79.26 years, the population will be triple.
\end{soln}
\begin{prob}
    Use the logistic model with an assumed carrying capacity of \(100\times 10^9\) an observed population of \(5\times 10^9\) in 1986 and an observed rate of growth of \(2\%\) per year when population size is \(5 \times 10^9\) predict the population of the earth in the year 2008.
\end{prob}
\begin{soln}
    We have, from logistic model,
    \begin{equation}
        N(t)=\frac{K N_0}{N_0+(K-N_0)e^{-rt}}\label{eq:probdef3.1}
    \end{equation}
    Where, \(\begin{aligned}[t]
        k&=100\times 10^9\\
        r&=2\% = 0.02\\
        t&=2008-1968=22\\
        N_0&=5\times 10^9
    \end{aligned}\)\\

    Then from \eqref{eq:probdef3.1},
    \begin{align*}
        N(t)&=\frac{100\times 10^9\times 5\times10^9}{5\times 10^9+(100\times 10^9-5\times 10^9\times e^{-0.02\times 22})}\\
        &=\frac{500\times 10^{18}}{5\times 10^9+95\times 10^9\times e^{-0.44}}\\
        &=\frac{500\times 10^{18}}{10^9(5+95\times 0.6440)}\\
        &=7.55\times 10^9
    \end{align*}
\end{soln}
\begin{prob}
    The Pacific halibut fishery is modeled by the logistic equation with carrying capacity \(80.5\times 10^6\) measured in kilograms and intrinsic growth rate \(0.71\) per year. If the initial biomass is one fourth the carrying capacity find the biomass one year later and the time required for the biomass to grow to half the carrying capacity.
\end{prob}
\begin{soln}
    The logistic model is
    \begin{equation}
        x'=rx\left(1-\frac{x}{K}\right)\label{eq:probdef4.1}
    \end{equation}
    Given, \(\begin{aligned}[t]
        \text{Carrying capacity, } K&=80.5\times 10^6 \text{ kg}\\
        \text{Intrinsic growth rate, } r&=0.71\\
        \text{The initial biomass, } x_0&=\frac{1}{4}K\\
        &=\frac{1}{4}\times 80.5\times 10^6 \text{ kg}\\
        &=20.125\times 10^6 \text{ kg}\\
        \text{time, } x_0&=1\text{ years}\\
        \text{The biomass one year later, } x(1)&=?
    \end{aligned}\)\\


    We know, the solution of logistic model is,
    \[
        x(t)=\frac{K x_0}{x_0+(K-x_0)e^{-rt}}
        \]
        \begin{align*}
            \therefore \;x(t)&=\frac{80.5\times 10^6 \times 20.125\times 10^6}{20.125\times 10^6+(80.5\times 10^6-20.125\times 10^6)e^{-.71\times 1}}\\
            &=\frac{1620.0625\times 10^6}{20.125+(60.375)\times 0.492}\\
            &=\frac{1620.063}{49.808}\times 10^6\\
            &=32.526\times 10^6\text{ kilograms}
        \end{align*}
    Now let, after time \(t\), the biomass be \(x(t)\).
    \[
        \therefore\;x(t)=\frac{1}{2}K=\frac{1}{2}\times 10^6=40.25\times10^6\text{ kg}
    \]
    Then by solution of logistic model,
    \begin{align*}
        &x(t)=\frac{K x_0}{x_0+(K-x_0)e^{-rt}}\\
        \Rightarrow\;\;&x(t)=\frac{K x_0e^{rt}}{x_0e^{rt}+K-x_0}\\
        \Rightarrow\;\;&x\left( {x_0e^{rt}+K-x_0} \right)={K x_0e^{rt}}\\
        \Rightarrow\;\;&xx_0e^{rt}+xK-xx_0-{K x_0e^{rt}}=0\\
        \Rightarrow\;\;&e^{rt}\left(xx_0-{K x_0}\right)=x(x_0-K)\\
        \Rightarrow\;\;&e^{rt}=\frac{x(x_0-K)}{xx_0-{K x_0}}\\
        \Rightarrow\;\;&e^{rt}=\frac{x(x_0-K)}{x_0(x-K)}\\
        \Rightarrow\;\;&e^{rt}=\frac{40.25\times10^6(20.125\times10^6-80.5\times10^6)}{20.125\times10^6(40.25\times10^6-80.5\times10^6)}\\
        \Rightarrow\;\;&e^{rt}=\frac{40.25\times (-60.375)}{20.125\times (-40.25)}\\
        \Rightarrow\;\;&e^{rt}=3\\
        \Rightarrow\;\;&{rt}=\ln 3\\
        \Rightarrow\;\;&{t}=\frac{\ln 3}{r}\\
        \Rightarrow\;\;&{t}=\frac{\ln 3}{0.71}\\
        \therefore\;\;&{t}=1.547 \text{ years}
    \end{align*}
\end{soln}
\begin{prob}
    Assume that \(P(t)\) size of a population obeying an exponential growth law. \(P_1\), \(P_2\) be the values of \(P(t)\) at distinct times \(t_1\) and \(t_2\) \((t_1<t_2)\) respectively. Prove that the growth rate,
    \[r=\frac{1}{t_2-t_1}\ln\frac{P_2}{P_1}
    \]
    Determine how long it takes the population to double its size under the model.
\end{prob}
\begin{proof}
    We know that,
    \[P(t)=P_0e^{r(t-t_0)}\]
    Then, 
    \[
        P_1=P_0e^{r(t_1-t_0)}, \;\;P(t_1)=P_1
    \]
    \[
        P_2=P_0e^{r(t_2-t_0)}, \;\;P(t_2)=P_2
    \]
    \begin{align*}
        \therefore\;\;&\frac{P_2}{P_1}=\frac{P_0e^{r(t_2-t_0)}}{P_0e^{r(t_1-t_0)}}\\
        \Rightarrow\;\;&\frac{P_2}{P_1}=e^{r(t_2-t_1)}\\
        \Rightarrow\;\;&r(t_2-t_1)=\ln\frac{P_2}{P_1}\\
        \Rightarrow\;\;&r=\frac{1}{t_2-t_1}\ln\frac{P_2}{P_1}
    \end{align*}
    Let, at time \(t=T\), the population will be double in size i.e., \(P=2P_0\)
    \begin{align*}
        \therefore\;\;&P=P_0e^{r(t-t_0)}\\
        \Rightarrow\;\;&2P_0=P_0e^{r(T-t_0)}\\
        \Rightarrow\;\;&2=e^{r(T-t_0)}\\
        \Rightarrow\;\;&{r(T-t_0)}=\ln 2\\
        \Rightarrow\;\;&T=t_0+\frac{\ln 2}{r}
    \end{align*}
\end{proof}
\end{document}