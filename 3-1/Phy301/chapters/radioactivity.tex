\documentclass[../main.tex]{subfiles}
\usepackage{../style}
\graphicspath{ {../img/} }
\begin{document}
\chapter{Radioactivity}
\section{Application of Radioactivity}
\begin{enumerate}
    \item Nuclear radiation like $ \gamma $-rays have been utilized for the preservation of food. Food stuff mainly meat, poultry, fish, fruits etc are exposed to $ \gamma $ rays from cobalt-60 or caesium-137. A dose of about 2 to 5 million rads is sufficient to destroy almost all bacteria in food.
    \item Radiation is used for producing new and improved varieties of plants.
    \item Gamma radiation from cobalt-60 is used in hospitals to sterilize materials like hypodermic syringe, surgical instruments, dressings etc.
    \item Radiation can also be used as pesticide.
    \item Gamma radiation from cobalt-60, iridium-192 are used in industrial radiography i.e, for investigating  the  interiors  metallic  castings  for  detecting  any  flaws  or     defects.
    \item A carefully prepared mixture of radio thorium ($ \alpha $-emitter) with zinc sulphide exhibits a more or less permanent luminescence and is used for coating the pointers and figures of clocks and watches, for rendering visible signs in theaters and so on.
    \item Radioisotopes are used to diagnose the nature of blood circulatory disorder, defects of bone metabolism, to locate tumors, etc. Radio-sodium is used to study the circulatory disorder in blood vessels while radioactive iodine is used to study any disorder in thyroid gland. $ Tc^{99m} $ is used to study the functioning of different organs like liver, kidney and spleen under normal and diseased conditions.
\end{enumerate}
\[
    \boxed{\text{[$ Tc^{99m} $  metastable nuclear isomer of technisium-99]}}
\]
\section{Activity}
The activity of a sample of any radioactive nuclide is the rate at which the nuclei of its constituent atom decay. If $ N $ is the number of Nuclei present in the sample at a certain time, its activity $ R $ is given by
\begin{align*}
    R&=\frac{-\D N}{\D t}\\
    \text{or, }R&=\lambda N
\end{align*}
SI unit of activity is Becquerel.
\[
    1 \text{Becquerel} = 1 \text{Bq} = 1 \text{decay/s}
\]
\section{Radioactive Dating: Age of The Earth}
The age of earth is estimated from the relative abundance of the two isotopes of uranium $ U-238 $ and $ U-235 $. The present abundance ratio of $ U-235 $ and $ U-238 $ is $ 1.140 \,(0.7\% \text{ to } 99.3\%)$. The half-lives of U-235 and U-238, according to the best estimate are $ 7.07\E{8} $ years and $ 4.5\E{9} $ years respectively. Assuming that at the beginning the proportions of the two isotopes were equal, the present relative abundance of U-238 and U-235 may be expressed as
\[
    \frac{N_1}{N_2}=\frac{99.3}{0.7}=\frac{N_0e^{\lambda_1 t}}{N_0e^{\lambda_2 t}}=e^{\lambda_1 -\lambda_2 t}
\]
Where 
\[
    \lambda_1=\frac{0.693}{4.5\E{9}}y^{-1}\quad\text{and}\quad\lambda_2=\frac{0.693}{7.07\E{8}}y^{-1}
\]
Now, \begin{align*}
    \ln\left[ \frac{99.3}{0.7} \right]&=\left( \lambda_2-\lambda_1 \right)t\\
    t&=\frac{1}{\lambda_2-\lambda_1}\ln\left[ \frac{99.3}{0.7} \right]\\
    &\approx 5.93\E{9} \text{ years}\\
    &\approx 5000 \text{ million years}
\end{align*}
This values agrees nearly with that given by astronomical evidence for the age of the universe.

\section{Solve the following Problems}
\begin{enumerate}
    \item The half-life of a radioactive substance is $ 30 $ days. Calculate
    \begin{enumerate}
        \item the radioactive decay constant,
        \item the mean life,
        \item the time taken for 3/4 of the original number of atoms to disintegrate and
        \item the time for 1/8 of the original number of atoms to remain unchanged.
    \end{enumerate}
    \item The half-life of radium is $ 1620 $ years. In how many years will one gram of pure element
    \begin{enumerate}
        \item lose one centigram and
        \item be reduced to one centigram?
    \end{enumerate}
    \item 1 gram of radium is reduced by $ 2.1 \,mg $ in 5 years by	$ \alpha $-decay. Calculate the half-life of radium.
    \item The alpha decay of $ Rn-222 $ to $ Po-218 $ has half-life of $ 3.8 $ days.
    \begin{enumerate}
        \item How long does it take for $ 60\% $ of sample of radon to decay (initial mass of $ 1.0\,mg $)
        \item Find the activity of $ 1.0\,mg $ of Radon-222, whose atomic mass is $ 222u $.
    \end{enumerate}
\end{enumerate}
\end{document}
