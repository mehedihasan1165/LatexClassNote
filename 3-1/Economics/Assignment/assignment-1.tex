\documentclass[12pt]{article}
\usepackage{../style}
\graphicspath{ {../img/} }
\begin{document}
{\large Assignment 1}\\
{\large Set 1: Demand-Supply analysis}
\begin{enumerate}
	\item Distinguish between change in demand and change in quantity needed. (use figure)
	\item What happens on the current demand curve and/or supply curve for the following cases?
	      \begin{enumerate}
		      \item An increase in consumer's income
		      \item Tax levied on the production of the good increases
		      \item Expected future price of the product decreases
	      \end{enumerate}
	\item Define Market equilibrium. Suppose consumer's income increases (for normal good) and tax imposed on the production of that goods occur simultaneously. What happens to the equilibrium price and quantity if
	      \begin{enumerate}
		      \item Increase in income = increase in tax
		      \item Increase in income $ > $ increase in tax
		      \item Increase in income $ < $ increase in tax
	      \end{enumerate}
\end{enumerate}
{\large Set 2: Budget line \& Indifferent curve}
\begin{enumerate}
	\item Diagram the following budget constrains
	      \begin{enumerate}[label=\alph*.]
		      \item Income $=\$4000;\,P_X=\$50;\,P_Y=\$100$
		      \item Income $=\$3000;\,P_X=\$25;\,P_Y=\$200$
		      \item Income $=\$2000;\,P_X=\$40;\,P_Y=\$150$
	      \end{enumerate}
	      Find out the slope of each of the budget lines. Explain graphically what happens to the budget line if,
	      \begin{enumerate}[label=(\roman*)]
		      \item Income of consumer $ a $, $ b $, and $ c $ increases by $\$1000$.
		      \item Price of good $ y $ increases by $\$50$ in each of the cases
	      \end{enumerate}
	\item Explain the law of diminishing Marginal Utility using hypothetical data.
\end{enumerate}
\newpage
{\large Assignment 2}\\
{\large Set 1:}
\begin{enumerate}
	\item Suppose there are 100 million people in the civilian labor force and 90 million people employed. How many people are unemployed? What is the unemployment rate?
	\item Using the following data,
	      \indent Civilian noninstitutional population = 200 million
	      \indent Number of employed persons = 126 million
	      \indent Number of unemployed persons = 8 million
	      Compute,
	      \begin{enumerate}
		      \item The unemployment rate,
		      \item The employment rate and
		      \item The labor force participation rate.
	      \end{enumerate}
	\item Assume the market basket contains $ 10X $, $ 20Y $ and $ 45Z $. The current-year prices for goods $ X $, $ Y $ and $ Z $ are $ \$1 $, $ \$4 $ and $ \$6 $ respectively. The base-year prices are $ \$1 $, $ \$3 $ and $ \$5 $ respectively. What is the CPI in the current year?
	\item Distinguish between demand pull and push inflation.
\end{enumerate}
{\large Set 2: Money}
\begin{enumerate}
	\item Define $ M1 $ and $ M2 $.
	\item Illustrate the tools of money supply.
\end{enumerate}
\newpage
\section{Assignment 1}
\subsection{Set 1}
\begin{prob}
	Distinguish between change in demand and change in quantity needed. (use figure)
\end{prob}
\begin{soln}
	Change in quantity demanded is when demand for a commodity change due to change in its own price. In other words, the extension of demand caused by a decrease in price of the same good and contraction of demand caused by an increase in the price of the same good.
	\begin{figure}[H]
		\centering
		\vspace{-0.25cm}
		\import{../tikz/}{demand-curve.tikz}
		\vspace{-0.5cm}
	\end{figure}

	Change in quantity needed is when demand for a commodity changes due to change in other factors other than price. As a result, there is a shift in the demand curve. In other words, the decrease in demand or the backward shift in the demand curve caused by a change in factors other than the price of the good and an increase in demand or a forward shift in the demand curve caused by a change in factors other than the price of goods. Example: change in income.
	\begin{figure}[H]
		\vspace{-0.25cm}
		\centering
		\import{../tikz/}{supply-curve.tikz}
		\vspace{-1cm}
	\end{figure}
	To summarize,
	\begin{table}[H]
		\begin{tabular}{lcp{12cm}}
			Change in quantity demanded & = & A movement from one point to another point on the same demand curve caused by a change in the price of the good. \\
			Change in quantity needed   & = & shift in the demand curve
		\end{tabular}
	\end{table}
\end{soln}
\newpage
\begin{prob}
	What happens on the current demand and/or supply curve for the following cases?
	\begin{enumerate}[label=(\alph*)]
		\item An increase in consumer's income.
		\item Tax levied on the production of the good increases.
		\item Expected future price of the product decreases.
	\end{enumerate}
\end{prob}
\begin{soln}
	An increase in consumers' income results in shift in demand curve. If a consumer has increased income then demand rises for a normal good and demand falls for an inferior good.\\


	Tax levied on the production of the good increase results in shift in supply curve. Some taxes increase per-unit costs. This leads to a leftward shift in the supply curve, indication that the manufacture wants to produce and offer to sell fewer pairs of good at each price.\\


	Expected future price of the product decreasing results in change in both demand curve and supply curve. Buyers who expect the price of a good to be lower next month may wait until next month to buy it, thus decreasing the current demand for the good. If the price of a good is expected to be fewer in the future, producers nay sell more of the product today to avoid loses. Therefore, the current supply curve will shift rightward.

\end{soln}
\newpage
\begin{prob}
	Define Market equilibrium. Suppose consumer's income increases (for normal good) and tax imposed on the production of that goods occur simultaneously. What happens to the equilibrium price and quantity if
	\begin{enumerate}[label=(\alph*)]
		\item Increase in income = increase in tax
		\item Increase in income $ > $ increase in tax
		\item Increase in income $ < $ increase in tax
	\end{enumerate}
\end{prob}
\begin{soln}
	Equilibrium means `at rest'. Equilibrium in a market is the price-quantity combination from which buyers or seller do not tend to move away. Graphically, equilibrium is the intersection point of the supply and demand curve. If consumers' income increases (for normal good) and tax imposed on the production of that good occur simultaneously then the demand curve will shift in the rightward and supply curve will shift in leftward. Their shifting effects on the equilibrium price and quantity.\\


	If increase in income = increase in tax or, increasing demand = decreasing supply then the equilibrium price and quantity becomes,
	\begin{figure}[H]
		\centering
		\import{../tikz/}{equilibrium-1.tikz}
	\end{figure}
	Demand increases (the demand curve shifts rightward from $ D_1 $ to $ D_2 $ ) and supply decreases (the supply curve shifts leftward from $ S_1 $ to  $ S_2 $), the equilibrium price rises from $ P_1 $ to $ P_2 $ and equilibrium quantity remains same.\newpage


	If increase in income $ > $ increase in tax or, increasing demand $ > $ decreasing supply, then the equilibrium price becomes,
	\begin{figure}[H]
		\centering
		\import{../tikz/}{equilibrium-2.tikz}
	\end{figure}
	The equilibrium price rises from $ P_1 $ to $ P_2 $ and the equilibrium quantity rises from $ Q_1 $ to $ Q_2 $.


	If increase in income $ < $ increase in tax or, increasing demand $ < $ decreasing supply then the equilibrium price and quantity becomes,
	\begin{figure}[H]
		\centering
		\import{../tikz/}{equilibrium-3.tikz}
	\end{figure}
	the equilibrium price rises from $ P_1 $ to $ P_2 $ and the equilibrium quantity falls from $ Q_1 $ to $ Q_2 $.
\end{soln}
\newpage
\subsection{Set 2}
\begin{prob}
	Diagram the following budget constrains
	\begin{enumerate}[label=\alph*.,nolistsep]
		\item Income $=\$4000;$ $P_X=\$50;$ $P_Y=\$100$
		\item Income $ =\$3000;\, P_X=\$25;\, P_Y=\$200 $
		\item Income $=\$2000;$ $P_X=\$40;$ $P_Y=\$150$
	\end{enumerate}
	Find out the slope of each of the budget lines. Explain graphically what happens to the budget line if,
	\begin{enumerate}[label=(\roman*),nolistsep]
		\item Income of consumer $ a $, $ b $, and $ c $ increases by $\$1000$.
		\item Price of good $ Y $ increases by $\$50$ in each of the cases
	\end{enumerate}
\end{prob}
\begin{soln}
	A budget constraint represents all the combinations or bundles of two goods a person can purchase given a certain money income and prices for two goods.
	\begin{enumerate}[label=(\alph*)]
		\item Given,
		      \vspace{-.5cm}
		      \begin{table}[H]
			      \begin{tabular}{rl}
				      \hspace{2cm} Income & $= \$4000$ \\
				      $ P_X $             & $= \$50$   \\
				      $ P_Y $             & $= \$100$
			      \end{tabular}
		      \end{table}
		      \vspace{-.75cm}
		      $ \therefore $ Quantity of good $X\begin{aligned}[t]
				       & =  \frac{\text{Income}}{(\text{Price of good } X)} \\
				       & =  \frac{4000}{50}\text{ units}                    \\
				       & =80 \text{ units}
			      \end{aligned}
		      $\\
		      Again,\\
		      $ \therefore $ Quantity of good $Y\begin{aligned}[t]
				       & =\frac{\text{Income}}{(\text{Price of good }Y)} \\
				       & = \frac{4000}{100} \text{ units}                \\
				       & = 40 \text{ units}
			      \end{aligned}
		      $\\
		      Now, the budget constraint can be represented as follows,
		      \begin{figure}[H]
			      \centering
			      \import{../tikz/}{budget-1.tikz}
			      \vspace{-.25cm}
		      \end{figure}
		      In figure 1, the horizontal axis measures the quantity of the good $ X $ and the vertical axis measures the quantity of good $ Y $.\\
		      $ \therefore $ Slope of this budget line $=\frac{P_X}{P_Y}\,=\frac{50}{100}=0.5$
		\item Given,
		      \begin{table}[H]
			      \begin{tabular}{rl}
				      \hspace{2cm} Income & $= \$3000$ \\
				      $ P_X $             & $= \$25$   \\
				      $ P_Y $             & $= \$200$
			      \end{tabular}
		      \end{table}
		      $ \therefore $ Quantity of good $X\begin{aligned}[t]
				       & =\frac{\text{Income}}{(\text{Price of good }X)} \\
				       & =  \frac{3000}{25}\text{ units}                 \\
				       & =120 \text{ units}
			      \end{aligned}$\\
		      Again,\\
		      $ \therefore $ Quantity of good $Y\begin{aligned}[t]
				       & =\frac{\text{Income}}{(\text{Price of good }Y)} \\
				       & =  \frac{3000}{200}\text{ units}                \\
				       & =15 \text{ units}
			      \end{aligned}$\\
		      Now, the budget constraint can be represented as follows,
		      \begin{figure}[H]
			      \centering
			      % NOT TO SCALE
			      \import{../tikz/}{budget-2.tikz}
		      \end{figure}
		      In figure 2, the horizontal axis measures the quantity of the good $ X $ and the vertical axis measures the quantity of good $ Y $.\\
		      $ \therefore $ Slope of this budget line $\begin{aligned}[t]
				       & =  \frac{P_X}{P_Y} \\
				       & =  \frac{25}{200}  \\
				       & = 0.125
			      \end{aligned}
		      $\newpage
		\item Given,
		      \begin{table}[H]
			      \begin{tabular}{rl}
				      \hspace{2cm} Income & $= \$2000$ \\
				      $ P_X $             & $= \$40$   \\
				      $ P_Y $             & $= \$150$
			      \end{tabular}
		      \end{table}
		      $ \therefore $ Quantity of good $X\begin{aligned}[t]
				       & =\frac{\text{Income}}{(\text{Price of good }X)} \\
				       & =  \frac{2000}{40}\text{ units}                 \\
				       & =50 \text{ units}
			      \end{aligned}$\\
		      Again,\\
		      $ \therefore $ Quantity of good $Y\begin{aligned}[t]
				       & =\frac{\text{Income}}{(\text{Price of good }Y)} \\
				       & =  \frac{2000}{150}\text{ units}                \\
				       & =13.33 \text{ units}
			      \end{aligned}$\\
		      Now, the budget constraint can be represented as follows,
		      \begin{figure}[H]
			      \centering
			      % NOT TO SCALE
			      \import{../tikz/}{budget-3.tikz}
		      \end{figure}
		      In figure 3, the horizontal axis measures the quantity of the good X and the vertical axis measures the quantity of good Y.\\
		      $ \therefore $ Slope of this budget line $\begin{aligned}[t]
				       & =  \frac{P_X}{P_Y} \\
				       & =  \frac{40}{150}  \\
				       & = - 0.2667
			      \end{aligned}
		      $\newpage
	\end{enumerate}
	\begin{enumerate}[label=(\roman*)]
		\item \begin{enumerate}[label=\alph*)]
			      \item Increased by $ \$1000 $, now the income is $ =\$(4000+1000)\,=\,\$5000 $\\So,
			            \begin{table}[H]
				            \begin{tabular}{rl}
					            \hspace{2cm} Income & $= \$5000$ \\
					            $ P_X $             & $= \$50$   \\
					            $ P_Y $             & $= \$100$
				            \end{tabular}
			            \end{table}
			            $ \therefore $ Quantity of good $X\begin{aligned}[t]
					             & =\frac{\text{Income}}{(\text{Price of good }X)} \\
					             & =  \frac{5000}{50}\text{ units}                 \\
					             & = 100 \text{ units}
				            \end{aligned}$\\
			            Again,\\
			            $ \therefore $ Quantity of good $Y\begin{aligned}[t]
					             & =\frac{\text{Income}}{(\text{Price of good }Y)} \\
					             & =  \frac{5000}{100}\text{ units}                \\
					             & =50 \text{ units}
				            \end{aligned}$\\
			            Now, the budget constraint can be represented as follows,
			            \begin{figure}[H]
				            \centering
				            % NOT TO SCALE
				            \import{../tikz/}{increase-income-1.tikz}
			            \end{figure}
			            Slope of $ AB $ budget line $ \begin{aligned}[t]
					             & =\frac{40}{80} \\
					             & =0.5
				            \end{aligned}
			            $\\
			            Slope of $ CD $ budget line $\begin{aligned}
					             & =\frac{50}{100} \\
					             & =0.5            \\
				            \end{aligned}
			            $\\
			            By comparing figure 1 and 4, we can say that the budget line is shifted on the right side.\newpage
			      \item Increased by $ \$1000 $, now the income is $ =\$(3000+1000)\,=\,\$4000 $\\So,
			            \begin{table}[H]
				            \begin{tabular}{rl}
					            \hspace{2cm} Income & $= \$4000$ \\
					            $ P_X $             & $= \$25$   \\
					            $ P_Y $             & $= \$200$
				            \end{tabular}
			            \end{table}
			            $ \therefore $ Quantity of good $X\begin{aligned}[t]
					             & =\frac{\text{Income}}{(\text{Price of good }X)} \\
					             & =  \frac{4000}{25}\text{ units}                 \\
					             & = 160 \text{ units}
				            \end{aligned}$\\
			            Again,\\
			            $ \therefore $ Quantity of good $Y\begin{aligned}[t]
					             & =\frac{\text{Income}}{(\text{Price of good }Y)} \\
					             & =  \frac{4000}{200}\text{ units}                \\
					             & =20 \text{ units}
				            \end{aligned}$\\
			            Now, the budget constraint can be represented as follows,
			            \begin{figure}[H]
				            \centering
				            % NOT TO SCALE
				            \import{../tikz/}{increase-income-2.tikz}
			            \end{figure}
			            Slope of $ AB $ budget line $ \begin{aligned}[t]
					             & =\frac{15}{120} \\
					             & =0.125
				            \end{aligned}
			            $\\
			            Slope of $ CD $ budget line $\begin{aligned}
					             & =\frac{20}{160} \\
					             & =0.125          \\
				            \end{aligned}
			            $\\
			            By comparing figure 2 and 5, we can say that the budget line is shifted on the right side.\newpage
			      \item Increased by $ \$1000 $, now the income is $ =\$(2000+1000)\,=\,\$3000 $\\So,
			            \begin{table}[H]
				            \begin{tabular}{rl}
					            \hspace{2cm} Income & $= \$3000$ \\
					            $ P_X $             & $= \$40$   \\
					            $ P_Y $             & $= \$150$
				            \end{tabular}
			            \end{table}
			            $ \therefore $ Quantity of good $X\begin{aligned}[t]
					             & =\frac{\text{Income}}{(\text{Price of good }X)} \\
					             & =  \frac{3000}{40}\text{ units}                 \\
					             & = 75 \text{ units}
				            \end{aligned}$\\
			            Again,\\
			            $ \therefore $ Quantity of good $Y\begin{aligned}[t]
					             & =\frac{\text{Income}}{(\text{Price of good }Y)} \\
					             & =  \frac{3000}{150}\text{ units}                \\
					             & =20 \text{ units}
				            \end{aligned}$\\
			            Now, the budget constraint can be represented as follows,
			            \begin{figure}[H]
				            \centering
				            % NOT TO SCALE
				            \import{../tikz/}{increase-income-3.tikz}
			            \end{figure}
			            Slope of $ AB $ budget line $ \begin{aligned}[t]
					             & =\frac{13.33}{50} \\
					             & =0.267
				            \end{aligned}
			            $\\
			            Slope of $ CD $ budget line $\begin{aligned}
					             & =\frac{20}{75} \\
					             & =0.267         \\
				            \end{aligned}
			            $\\
			            By comparing figure 3 and 6, we can say that the budget line is shifted on the right side.\newpage
		      \end{enumerate}
		\item \begin{enumerate}[label=\alph*)]
			      \item The price of good $ Y $ is increased by $ \$50 $.\\So,
			            \begin{table}[H]
				            \begin{tabular}{rl}
					            \hspace{2cm} Income & $= \$4000$             \\
					            $ P_X $             & $= \$50$               \\
					            $ P_Y $             & $= \$100+\$50\,=\$150$
				            \end{tabular}
			            \end{table}
			            $ \therefore $ Quantity of good $X\begin{aligned}[t]
					             & =\frac{\text{Income}}{(\text{Price of good }X)} \\
					             & =  \frac{4000}{50}\text{ units}                 \\
					             & = 80 \text{ units}
				            \end{aligned}$\\
			            Again,\\
			            $ \therefore $ Quantity of good $Y\begin{aligned}[t]
					             & =\frac{\text{Income}}{(\text{Price of good }Y)} \\
					             & =  \frac{4000}{150}\text{ units}                \\
					             & =26.6667 \text{ units}
				            \end{aligned}$\\
			            Now, the budget constraint can be represented as follows,
			            \begin{figure}[H]
				            \centering
				            % NOT TO SCALE
				            \import{../tikz/}{increase-price-1.tikz}
			            \end{figure}
			            In figure 7, price of good $ Y $ rises from $ \$100 $ to $ \$150 $. The maximum number of units of good $ Y $ falls from $ 40 $ to $ 26.6667 $ units.\newpage
			      \item The price of good $ Y $ is increased by $ \$50 $.\\So,
			            \begin{table}[H]
				            \begin{tabular}{rl}
					            \hspace{2cm} Income & $= \$3000$             \\
					            $ P_X $             & $= \$25$               \\
					            $ P_Y $             & $= \$200+\$50\,=\$250$
				            \end{tabular}
			            \end{table}
			            $ \therefore $ Quantity of good $X\begin{aligned}[t]
					             & =\frac{\text{Income}}{(\text{Price of good }X)} \\
					             & =  \frac{3000}{25}\text{ units}                 \\
					             & = 120 \text{ units}
				            \end{aligned}$\\
			            Again,\\
			            $ \therefore $ Quantity of good $Y\begin{aligned}[t]
					             & =\frac{\text{Income}}{(\text{Price of good }Y)} \\
					             & =  \frac{3000}{250}\text{ units}                \\
					             & =12 \text{ units}
				            \end{aligned}$\\
			            Now, the budget constraint can be represented as follows,
			            \begin{figure}[H]
				            \centering
				            % NOT TO SCALE
				            \import{../tikz/}{increase-price-2.tikz}
			            \end{figure}
			            In figure 8, price of good $ Y $ rises from $ \$200 $ to $ \$250 $. The maximum number of units of good $ Y $ falls from $ 15 $ to $ 12 $ units.\newpage
			      \item The price of good $ Y $ is increased by $ \$50 $.\\So,
			            \begin{table}[H]
				            \begin{tabular}{rl}
					            \hspace{2cm} Income & $= \$2000$             \\
					            $ P_X $             & $= \$40$               \\
					            $ P_Y $             & $= \$150+\$50\,=\$200$
				            \end{tabular}
			            \end{table}
			            $ \therefore $ Quantity of good $X\begin{aligned}[t]
					             & =\frac{\text{Income}}{(\text{Price of good }X)} \\
					             & =  \frac{2000}{40}\text{ units}                 \\
					             & = 50 \text{ units}
				            \end{aligned}$\\
			            Again,\\
			            $ \therefore $ Quantity of good $Y\begin{aligned}[t]
					             & =\frac{\text{Income}}{(\text{Price of good }Y)} \\
					             & =  \frac{2000}{200}\text{ units}                \\
					             & =10 \text{ units}
				            \end{aligned}$\\
			            Now, the budget constraint can be represented as follows,
			            \begin{figure}[H]
				            \centering
				            % NOT TO SCALE
				            \import{../tikz/}{increase-price-3.tikz}
			            \end{figure}
			            In figure 9, price of good $ Y $ rises from $ \$150 $ to $ \$200 $. The maximum number of units of good $ Y $ falls from $ 13.33 $ to $ 10 $ units.
		      \end{enumerate}
	\end{enumerate}
\end{soln}
\newpage
\begin{prob}
	Explain the law of diminishing Marginal Utility using hypothetical data.
\end{prob}
\begin{soln}
	Law of diminishing marginal utility states that the utility of each successive unit goes on diminishing as more and more units of commodity is consumed when units of other commodities remain constant.\\
	\underline{Assumptions:}
	\begin{enumerate}[label=(\roman*)]
		\item Rational behavior of consumer.
		\item Cardinal measurability of utility.
		\item Utility can be measured by money and marginal utility of money remains constant.
		\item Income and mental status of consumer are assumed to be remains constant.
		\item Price of commodity, price of other commodities, and other factors that can affect the utility are assumed to be remain constant.
	\end{enumerate}
	Now let us take a look at following table and diagram:
	\begin{table}[H]
		\centering
		\begin{tabular}{|c|c|c|}
			\hline
			$ Q $ & T.U    & M.U    \\\hline
			$ 1 $ & $ 12 $ & $ 12 $ \\\hline
			$ 2 $ & $ 20 $ & $ 8 $  \\\hline
			$ 3 $ & $ 26 $ & $ 6 $  \\\hline
			$ 4 $ & $ 30 $ & $ 4 $  \\\hline
			$ 5 $ & $ 30 $ & $ 0 $  \\\hline
			$ 6 $ & $ 28 $ & $ -2 $ \\\hline
		\end{tabular}
	\end{table}
	\begin{figure}[H]
		\centering
		\import{../tikz/}{marginal-utility.tikz}
	\end{figure}
	\underline{Explanation of the law:}\\
	Initially as the consumer increases the consumption marginal utility decreases but it is positive as our example from first to fourth unit of commodity marginal utility decreases from $ 12 $ to $ 4 $. After fifth unit of consumption the marginal utility becomes zero. This represents the point of complete satisfaction. After sixth unit of consumption the marginal utility becomes $ -2 $. This means the consumer is deriving negative utility or disutility.
\end{soln}
\newpage
\section{Assignment 2}
\subsection{Set 1}
\begin{prob}
	Suppose there are 100 million people in the civilian labor force and 90 million people employed. How many people are unemployed? What is the unemployment rate?
\end{prob}
\begin{soln}Here,
	\begin{table}[H]
		\begin{tabular}{rcl}
			\hspace{3cm}People in civilian labor force & = & 100 million \\
			People employed                            & = & 90 million
		\end{tabular}
	\end{table}
	We know,
	People unemployed $\begin{aligned}[t]
			 & = \text{Civilian labor force } - \text{ People employed} \\
			 & = (100-90) \text{ million}                               \\
			 & = 10 \text{ million}
		\end{aligned}
	$\\
	The unemployment rate is the percentage of the civilian labor force that is employed. It is equal to the number of unemployed persons divided by the civilian labor force.\\
	So,\\
	Unemployment rate $ (U) \begin{aligned}[t]
			 & = \frac{\text{Number of unemployed people}}{\text{Civilian labor force}} \\
			 & = \frac{10}{100}                                                         \\
			 & = 0.1
		\end{aligned}
	$
\end{soln}
\newpage
\begin{prob}
	Using the following data,\\
	\indent Civilian noninstitutional population = 200 million\\
	\indent Number of employed persons = 126 million\\
	\indent Number of unemployed persons = 8 million\\
	Compute,
	\begin{enumerate}[label=(\alph*)]
		\item The unemployment rate,
		\item The employment rate and
		\item The labor force participation rate.
	\end{enumerate}
\end{prob}
\begin{soln}
	\begin{enumerate}[label=(\alph*)]
		\item People in civilian labor force $\begin{aligned}[t]
				       & = \text{Number of employed people} + \text{Number of unemployed people} \\
				       & = (126+8) \text{ million}                                               \\
				       & = 134 \text{ million}
			      \end{aligned}
		      $\\
		      Unemployment rate $ (U)\begin{aligned}[t]
				       & = \frac{\text{Number of unemployed people}}{\text{Civilian labor force}} \\
				       & = \frac{8}{134}                                                          \\
				       & = 0.0597
			      \end{aligned}
		      $\\
		\item Employment rate $ (E)\begin{aligned}[t]
				       & = \frac{\text{Number of employed people}}{\text{Civilian noninstutuional people}} \\
				       & = \frac{126}{200}                                                                 \\
				       & = 0.63
			      \end{aligned}
		      $
		\item Labor force participation rate $ (LFPR) \begin{aligned}[t]
				       & =\frac{\text{Civilian labor force}}{\text{Civilian noninstitutional people}} \\
				       & =\frac{134}{200}                                                             \\
				       & =0.67
			      \end{aligned} $
	\end{enumerate}
\end{soln}
\newpage
\begin{prob}
	Assume the market basket contains $ 10X $, $ 20Y $ and $ 45Z $. The current-year prices for goods $ X $, $ Y $ and $ Z $ are $ \$1 $, $ \$4 $ and $ \$6 $ respectively. The base-year prices are $ \$1 $, $ \$3 $ and $ \$5 $ respectively. What is the CPI in the current year?
\end{prob}
\begin{soln}\hfill
	\begin{table}[H]
		\centering
		\begin{tabular}{|cccrc|}
			\hline
			Market basket &            & Current year price &             & Current year expenditure \\\hline
			$ 10\,X $     & $ \times $ & $ \$1 $            & $ = $       & $ \$10 $                 \\
			$ 20\,Y $     & $ \times $ & $ \$4 $            & $ = $       & $ \$80 $                 \\
			$ 45\,Z $     & $ \times $ & $ \$6 $            & $ = $       & $ \$270 $                \\\hline
			              &            &                    & total $ = $ & $ \$360 $                \\\hline
		\end{tabular}
	\end{table}
	\begin{table}[H]
		\centering
		\begin{tabular}{|cccrc|}
			\hline
			Market basket &            & Base-year price &             & Base-year expenditure \\\hline
			$ 10\,X $     & $ \times $ & $ \$1 $         & $ = $       & $ \$10 $              \\
			$ 20\,Y $     & $ \times $ & $ \$3 $         & $ = $       & $ \$60 $              \\
			$ 45\,Z $     & $ \times $ & $ \$5 $         & $ = $       & $ \$225 $             \\\hline
			              &            &                 & total $ = $ & $ \$295 $             \\\hline
		\end{tabular}
	\end{table}
	So,\\
	C.P.I. $ \begin{aligned}[t]
			 & =\left( \frac{\text{Total dollar expenditure in current year}}{\text{Total dollar expenditure in base year}} \right)\times 100 \\
			 & =\left( \frac{360}{295} \right)\times 100                                                                                      \\
			 & =122.0339
		\end{aligned} $
\end{soln}
\newpage
\begin{prob}
	Distinguish between demand pull and cost push inflation.
\end{prob}
\begin{soln}
	Demand pull inflation is a period of inflation which arises from rapid growth in aggregate demand. It occurs when economic growth is too fast. This imbalance essentially ends up in an excessive amount of money chasing too few goods and services.
	\begin{figure}[H]
		\centering
		\import{../tikz/}{LRAS.tikz}
	\end{figure}
	Cost push inflation occurs when we experience rising prices due to costs of production and higher costs of raw materials. Cost push inflation can lead to lower economic growth and often causes a fall in living standards, though it often proves to be temporary.
	\begin{figure}[H]
		\centering
		\import{../tikz/}{SRAS.tikz}
	\end{figure}
	\underline{Key-differences between demand-pull and cost-push inflation:}
	\begin{itemize}
		\item Demand pull inflation is most likely to occur when an economy is becoming stretched and is said to be the danger of over-heating.
		\item Cost push inflation occurs when businesses respond to rising costs, by increasing their prices to protect profit margins.
		\item Demand pull inflation arises when the aggregate demand increased at a faster rate than aggregate supply.
		\item Cost push inflation is a result of an increase in the price of inputs due to the shortage of cost of production, leading to a decrease in the supply of outputs.
		\item The reason for demand pull inflation is the increase in money supply, government spending and foreign exchange rate.
		\item Cost push inflation is mainly caused by monopolistic groups of society.
	\end{itemize}
\end{soln}
\newpage
\subsection{Set 2}
\begin{prob}
	Define M1 and M2.
\end{prob}
\begin{soln}
	M1 is sometime referred as the narrow definition of the money supply or as transactions money. It is money that can be directly used for everyday transactions. M1 consists of currency held outside banks, checkable deposits and traveler's check.
	\begin{table}[H]
		\begin{tabular}{ccl}
			M1 & = & Currency held outside banks \\
			   &   & + Checkable deposits        \\
			   &   & + Traveler's checks.
		\end{tabular}
	\end{table}

	Currency includes coin and paper money.\\

	Checkable deposits are deposits on which checks can be written. There are different types of checkable accounts that pay no interest and NOW and ATS accounts which do pay interest on their balance.\\

	A traveler's check is a certified note issued by a bank that may be used by travelers as a risk-free substitute for paper money.\\

	M2 sometimes referred to as the broad definition of the money supply. M2 is made up of M1 plus saving deposits (including MMDA), small-denomination time deposits and money market mutual funds (retail).
	\begin{table}[H]
		\begin{tabular}{ccl}
			M2 & = & M1                                    \\
			   &   & + Saving deposits (including MMDA)    \\
			   &   & + small-denomination time deposits    \\
			   &   & + Money market mutual funds (retail).
		\end{tabular}
	\end{table}

	Saving deposit is an interest-earning account at a commercial bank or thrift institution. Normally, checks cannot be written on saving deposits and the funds in a saving deposits can be withdrawn without a penalty payment.\\

	Money market deposit account is an interest-earning account at a bank or thrift institution. Usually, a minimum balance is required for an MMDA. Most MMDAs offer limited check writing privileges.\\

	Time deposit is an interest-earning deposit with a specified maturity dates. Time deposits are subject to penalties for early withdrawn. Small-denomination time deposits are deposits of less than $ \$100000 $.\\

	Money market mutual fund is an interest-earning account at a mutual fund company. Usually, a minimum balance is required for MMMF account. Most MMMF accounts offer limited check-writing privileges. Only retail MMMFs are part of M2.
\end{soln}
\newpage
\begin{prob}
	Illustrate the tools of money supply.
\end{prob}
\begin{soln}
	The Fed uses three tools for controlling the money supply. They are
	\begin{enumerate}
		\item Open market operations.
		\item The required reserve ratio
		\item The discount rate
	\end{enumerate}
	\underline{Open market operations}\\
	\underline{Open market purchase}\\
	The Fed conducts an open market purchase, which is a type of open market operation. Specifically, it buys government securities from a bank. When the Fed buys securities, someone has to sell securities. Suppose the Fed buys $ \$5 $ million worth of government securities from bank XYZ. The Fed pays for the government securities by increasing the balance in bank XYZ's reserve account.\\

	\underline{Open market sales}\\
	\indent Sometimes, the Fed sells government securities to bank and others. Suppose the Fed sells $ \$5 $ million worth of government securities to bank XYZ. The Fed surrenders the securities to bank XYZ and is paid $ \$5 $ million previously deposited in bank XYZ's reserve account at the Fed.\\

	\underline{The required reserve ratio}\\
	\indent The Fed can influence the money supply by changing the required reserve ratio. We can find the maximum change in checkable deposits by using the following formula
	\[\text{Maximum change in checkable deposits }=\frac{1}{r}\times \Delta R\]
	For example, if reserves $ (R) $ increased by $ \$1000 $ and the required ratio $ (r) $ is 10 percent, then maximum change in checkable deposits is $ \$10,000 $.\\
	Now, if the Fed officials increase the required reserve ratio from 10 to 20 percent, then maximum change in checkable deposit is $ \$5,000 $. So the maximum change in checkable deposits decreases. Similarly if the reserve ratio is lowered to 5 percent then the maximum change in checkable deposits will increase.\\

	\underline{Discount rate}\\
	\indent Federal funds market is a market where banks lend reserves to one another. Usually for short periods.\\

	Federal funds rate is the interest rate in the federal funds market that banks charges one another to borrow reserves.\\

	Discount rate is the interest rate the Fed charges depository institutions that borrow reserves from it.\\

	If the discount rate was lowered so that it is below the federal funds rate then banks would go to the Fed for loans instead of going to each other. Let bank ABC gets a loan from the Fed, its reserves increases while the reserves of no other bank decrease. The result is increased reserved for the banking system as a whole. So the money supply increases. When the discount rate is raised above the federal funds rate banks will not borrow from the Fed.
\end{soln}
\end{document}